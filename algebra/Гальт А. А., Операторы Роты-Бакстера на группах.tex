\documentclass[12pt]{article}
\usepackage{hyphsubst}
\usepackage[T2A]{fontenc}
\usepackage[english,main=russian]{babel}
\usepackage[utf8]{inputenc}
\usepackage[letterpaper,top=2cm,bottom=2cm,left=2cm,right=2cm,marginparwidth=2cm]{geometry}
\usepackage{float}
\usepackage{mathtools, commath, amssymb, amsthm}
\usepackage{enumitem, tabularx,graphicx,url,xcolor,rotating,multicol,epsfig,colortbl,lipsum}

\setlist{topsep=1pt, itemsep=0em}
\setlength{\parindent}{0pt}
\setlength{\parskip}{6pt}

\usepackage{hyphenat}
\hyphenation{ма-те-ма-ти-ка вос-ста-нав-ли-вать}

\usepackage[math]{anttor}

\newenvironment{talk}[6]{%
\vskip 0pt\nopagebreak%
\vskip 0pt\nopagebreak%
\section*{#1}
\phantomsection
\addcontentsline{toc}{section}{#2. \textit{#1}}
% \addtocontents{toc}{\textit{#1}\par}
\textit{#2}\\\nopagebreak%
#3\\\nopagebreak%
\ifthenelse{\equal{#4}{}}{}{\url{#4}\\\nopagebreak}%
\ifthenelse{\equal{#5}{}}{}{Соавторы: #5\\\nopagebreak}%
\ifthenelse{\equal{#6}{}}{}{Секция: #6\\\nopagebreak}%
}

\definecolor{LovelyBrown}{HTML}{FDFCF5}

\usepackage[pdftex,
breaklinks=true,
bookmarksnumbered=true,
linktocpage=true,
linktoc=all
]{hyperref}

\begin{document}
\pagenumbering{gobble}
\pagestyle{plain}
\pagecolor{LovelyBrown}
\begin{talk}
{Операторы Роты--Бакстера на группах}
{Гальт Алексей Альбертович}
{Институт математики им. С.Л. Соболева СО РАН}
{galt84@gmail.com}
{}
{Алгебра}

Операторы Роты--Бакстера на группах ввели Л. Гуо, Х. Ланг и Ю. Шенг в работе~2021 года [1]. После выхода данной статьи уже опубликованы различные работы, посвященные данному направлению. С одной стороны, активно исследуется соответствие между операторами Роты--Баксетра на группах Ли и алгебрах Ли. С другой стороны, определение оператора Роты--Бакстера на группе было перенесено на полугруппы Клиффорда, алгебры Хопфа, решетки и другие структуры. Кроме того, была доказана глубокая взаимосвязь между операторами Роты--Бакстера на группах и структурами Хопфа--Галуа, а также косыми левыми брейсами, которые в свою очередь являются теоретико-множественными решениями уравнения Янга--Бакстера.

В 2023 году, В. Бардаков и В. Губарев показали, что все операторы Роты-Бакстера на простых спорадических группах  расщепляемы, то есть получаются из точных факторизаций групп. В докладе будет рассказано о знакопеременных и диэдральных группах, которые обладают нерасщепляемыми операторпми Роты-Бакстера.

\medskip

\begin{enumerate}
\item[{[1]}] L. Guo, H. Lang, Yu. Sheng, {\it Integration and geometrization of Rota--Baxter Lie
algebras}, Adv. Math., 387 (2021), 107834.
\end{enumerate}
\end{talk}
\end{document}