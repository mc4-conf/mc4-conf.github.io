\documentclass[12pt]{article}
\usepackage{hyphsubst}
\usepackage[T2A]{fontenc}
\usepackage[english,main=russian]{babel}
\usepackage[utf8]{inputenc}
\usepackage[letterpaper,top=2cm,bottom=2cm,left=2cm,right=2cm,marginparwidth=2cm]{geometry}
\usepackage{float}
\usepackage{mathtools, commath, amssymb, amsthm}
\usepackage{enumitem, tabularx,graphicx,url,xcolor,rotating,multicol,epsfig,colortbl,lipsum}

\setlist{topsep=1pt, itemsep=0em}
\setlength{\parindent}{0pt}
\setlength{\parskip}{6pt}

\usepackage{hyphenat}
\hyphenation{ма-те-ма-ти-ка вос-ста-нав-ли-вать}

\usepackage[math]{anttor}

\newenvironment{talk}[6]{%
\vskip 0pt\nopagebreak%
\vskip 0pt\nopagebreak%
\section*{#1}
\phantomsection
\addcontentsline{toc}{section}{#2. \textit{#1}}
% \addtocontents{toc}{\textit{#1}\par}
\textit{#2}\\\nopagebreak%
#3\\\nopagebreak%
\ifthenelse{\equal{#4}{}}{}{\url{#4}\\\nopagebreak}%
\ifthenelse{\equal{#5}{}}{}{Соавторы: #5\\\nopagebreak}%
\ifthenelse{\equal{#6}{}}{}{Секция: #6\\\nopagebreak}%
}

\definecolor{LovelyBrown}{HTML}{FDFCF5}

\usepackage[pdftex,
breaklinks=true,
bookmarksnumbered=true,
linktocpage=true,
linktoc=all
]{hyperref}

\begin{document}
\pagenumbering{gobble}
\pagestyle{plain}
\pagecolor{LovelyBrown}
\begin{talk}
{Минимальное число порождающих сопряженных инволюций, произведение которых равно 1, групп $PSp_4(q)$}
{Гвоздев Родион Игоревич}
{Сибирский федеральный университет}
{}
{Нужин Я.Н., Петруть Т.С., Соколовская А.М.}
{Алгебра} % [6] название секции

В работе G.~Malle, J.~Saxl, T.~Weigel. Generation of classical groups, Geom. Dedicata, 1994 записана следующая проблема. Для каждой конечной простой неабелевой группы $G$ найти $n_c(G)$ --- минимальное число порождающих сопряженных инволюций, произведение которых равно 1 (см. также вопрос 14.69в) в коуровской тетради). К настоящему времени вопрос решен для спорадических, знакопеременных групп и групп $PSL_n(q)$, $q$ нечетно, исключая случай $n=6$ и $q\equiv3(mod4)$.

Если $G$ --- конечная простая неабелева группа, то $n_c(G)\geq 5$, а если она еще и порождается тремя инволюциями $\alpha$, $\beta$, $\gamma$, первые две из которых перестановочны и все четыре инволюции $\alpha$, $\beta$, $\gamma$ и $\alpha\beta$ сопряжены, то $n_c(G)=5$. Доказана\medskip

\textbf{Теорема. } {\sl Группа $PSp_4(q)$ тогда и только тогда порождается тремя инволюциями $\alpha$, $\beta$, $\gamma$, первые две из которых перестановочны и все четыре инволюции $\alpha$, $\beta$, $\gamma$ и $\alpha\beta$ сопряжены, когда $q\neq2,3$.}\\

\textbf{Следствие. } {\sl 1) $n_c(PSp_4(q))=5$, при $q\neq2,3$;\\
2) $n_c(PSp_4(3))=6$;\\
3) $n_c(PSp_4(2))=10$.}\medskip
\end{talk}
\end{document}