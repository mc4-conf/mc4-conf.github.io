\documentclass[12pt]{article}
\usepackage{hyphsubst}
\usepackage[T2A]{fontenc}
\usepackage[english,main=russian]{babel}
\usepackage[utf8]{inputenc}
\usepackage[letterpaper,top=2cm,bottom=2cm,left=2cm,right=2cm,marginparwidth=2cm]{geometry}
\usepackage{float}
\usepackage{mathtools, commath, amssymb, amsthm}
\usepackage{enumitem, tabularx,graphicx,url,xcolor,rotating,multicol,epsfig,colortbl,lipsum}

\setlist{topsep=1pt, itemsep=0em}
\setlength{\parindent}{0pt}
\setlength{\parskip}{6pt}

\usepackage{hyphenat}
\hyphenation{ма-те-ма-ти-ка вос-ста-нав-ли-вать}

\usepackage[math]{anttor}

\newenvironment{talk}[6]{%
\vskip 0pt\nopagebreak%
\vskip 0pt\nopagebreak%
\section*{#1}
\phantomsection
\addcontentsline{toc}{section}{#2. \textit{#1}}
% \addtocontents{toc}{\textit{#1}\par}
\textit{#2}\\\nopagebreak%
#3\\\nopagebreak%
\ifthenelse{\equal{#4}{}}{}{\url{#4}\\\nopagebreak}%
\ifthenelse{\equal{#5}{}}{}{Соавторы: #5\\\nopagebreak}%
\ifthenelse{\equal{#6}{}}{}{Секция: #6\\\nopagebreak}%
}

\definecolor{LovelyBrown}{HTML}{FDFCF5}

\usepackage[pdftex,
breaklinks=true,
bookmarksnumbered=true,
linktocpage=true,
linktoc=all
]{hyperref}

\begin{document}
\pagenumbering{gobble}
\pagestyle{plain}
\pagecolor{LovelyBrown}
\begin{talk}
{Лямбда-структуры и степенные структуры на кольцах Гротендика многообразий с действиями конечных групп}
{Горчинский Сергей Олегович}
{Математический институт им. В.\,А.\,Стеклова РАН}
{gorchins@mi-ras.ru}
{Дёмин Данила Александрович}
{Алгебра}

Доклад основан на совместной работе с Д.\,А. Дёминым.

Существуют различные естественные структуры на коммутативных кольцах, задающиеся набором операций на них. Одним из хорошо известных примеров является лямда-структура: набор отображений \(\lambda_i\colon A\to A\), \(i\geqslant 1\), для кольца \(A\), удовлетворяющий условиям \(\lambda_1={\rm id}\), \({\lambda_n(a+b)=\sum_{i+j=n}\lambda_i(a)\lambda_j(b)}\). Другой пример даётся степенной структурой, введённой в работах С.\,М. Гусейна-Заде, И. Луенго и А. Мелле-Хернандеса. Степеннные структуры строятся по лямбда-структурам.

В алгебраической геометрии активно рассматривается кольцо Гротендика многообразий, а также его различные варианты, включая кольцо Гротендика многообразий с действиями конечных групп. Между данными кольцами определен естественный гомоморфизм, а на каждом из этих колец определена лямбда-структура в терминах (эквивариантных) симметрических степеней многообразий. Легко видеть, что гомоморфизм не коммутирует с лямбда-структурами.

Основной результат доклада заключается в том, что указанный выше гомоморфизм коммутирует с степенными структурами на кольцах Гротендика. Это получено при помощи новой общей формулы, выражающей степенную структуру в терминах комбинаторики корневых деревьев. В качестве приложения найдено существенное усиление и, в частности, новое доказательство, гипотезы Галкина--Шиндера о мотивной и категорной дзета-функциях многообразий.
\end{talk}
\end{document}