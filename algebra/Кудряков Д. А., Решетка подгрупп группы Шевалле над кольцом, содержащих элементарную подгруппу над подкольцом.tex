\documentclass[12pt]{article}
\usepackage{hyphsubst}
\usepackage[T2A]{fontenc}
\usepackage[english,main=russian]{babel}
\usepackage[utf8]{inputenc}
\usepackage[letterpaper,top=2cm,bottom=2cm,left=2cm,right=2cm,marginparwidth=2cm]{geometry}
\usepackage{float}
\usepackage{mathtools, commath, amssymb, amsthm}
\usepackage{enumitem, tabularx,graphicx,url,xcolor,rotating,multicol,epsfig,colortbl,lipsum}

\setlist{topsep=1pt, itemsep=0em}
\setlength{\parindent}{0pt}
\setlength{\parskip}{6pt}

\usepackage{hyphenat}
\hyphenation{ма-те-ма-ти-ка вос-ста-нав-ли-вать}

\usepackage[math]{anttor}

\newenvironment{talk}[6]{%
\vskip 0pt\nopagebreak%
\vskip 0pt\nopagebreak%
\section*{#1}
\phantomsection
\addcontentsline{toc}{section}{#2. \textit{#1}}
% \addtocontents{toc}{\textit{#1}\par}
\textit{#2}\\\nopagebreak%
#3\\\nopagebreak%
\ifthenelse{\equal{#4}{}}{}{\url{#4}\\\nopagebreak}%
\ifthenelse{\equal{#5}{}}{}{Соавторы: #5\\\nopagebreak}%
\ifthenelse{\equal{#6}{}}{}{Секция: #6\\\nopagebreak}%
}

\definecolor{LovelyBrown}{HTML}{FDFCF5}

\usepackage[pdftex,
breaklinks=true,
bookmarksnumbered=true,
linktocpage=true,
linktoc=all
]{hyperref}

\begin{document}
\pagenumbering{gobble}
\pagestyle{plain}
\pagecolor{LovelyBrown}
\begin{talk}
{Решетка подгрупп группы Шевалле над кольцом, содержащих элементарную подгруппу над подкольцом}
{Кудряков Дмитрий Александрович}
{Санкт-Петербургский международный математический институт имени Леонарда Эйлера (СПбГУ)}
{shorfod@gmail.com}
{Степанов Алексей Владимирович}
{Алгебра}

Обозначим через \(L(D,G)\) решетку подгрупп группы \(G\), содержащих фиксированную подгруппу \(D\). Решетка \(L(D,G)\) называется стандартной, если она разбивается в дизъюнктное объединение подрешеток (так называемых сэндвичей) \(L(F_a,N_a)\), где \(a\) пробегает некоторое множество индексов, а \(N_a\) обозначает нормализатор \(F_a\). Пусть \(G(A)\) --- группа Шевалле над кольцом \(A\), построенная по приведенной неприводимой системе корней с простыми связями. В докладе рассматривается задача стандартного описания решетки подгрупп группы \(G(A)\), содержащих элементарную подгруппу \(E(R)\) над подкольцом \(R\). При этом нижними границами  сэндвичей служат элементарные подгруппы \(E(S)\), соответствующие подкольцам \(S\), промежуточным между \(R\) и \(A\).
\end{talk}
\end{document}