\documentclass[12pt]{article}
\usepackage{hyphsubst}
\usepackage[T2A]{fontenc}
\usepackage[english,main=russian]{babel}
\usepackage[utf8]{inputenc}
\usepackage[letterpaper,top=2cm,bottom=2cm,left=2cm,right=2cm,marginparwidth=2cm]{geometry}
\usepackage{float}
\usepackage{mathtools, commath, amssymb, amsthm}
\usepackage{enumitem, tabularx,graphicx,url,xcolor,rotating,multicol,epsfig,colortbl,lipsum}

\setlist{topsep=1pt, itemsep=0em}
\setlength{\parindent}{0pt}
\setlength{\parskip}{6pt}

\usepackage{hyphenat}
\hyphenation{ма-те-ма-ти-ка вос-ста-нав-ли-вать}

\usepackage[math]{anttor}

\newenvironment{talk}[6]{%
\vskip 0pt\nopagebreak%
\vskip 0pt\nopagebreak%
\section*{#1}
\phantomsection
\addcontentsline{toc}{section}{#2. \textit{#1}}
% \addtocontents{toc}{\textit{#1}\par}
\textit{#2}\\\nopagebreak%
#3\\\nopagebreak%
\ifthenelse{\equal{#4}{}}{}{\url{#4}\\\nopagebreak}%
\ifthenelse{\equal{#5}{}}{}{Соавторы: #5\\\nopagebreak}%
\ifthenelse{\equal{#6}{}}{}{Секция: #6\\\nopagebreak}%
}

\definecolor{LovelyBrown}{HTML}{FDFCF5}

\usepackage[pdftex,
breaklinks=true,
bookmarksnumbered=true,
linktocpage=true,
linktoc=all
]{hyperref}

\begin{document}
\pagenumbering{gobble}
\pagestyle{plain}
\pagecolor{LovelyBrown}
\begin{talk}
{О группах, порожденных сопряженными элементами порядка 3, с ограничениями на 2-порожденные подгруппы}
{Мамонтов Андрей Сергеевич}
{ИМ СО РАН}
{mamontov@math.nsc.ru}
{}
{Алгебра}

На протяжении всей истории теории групп разнообразные условия конечности группы и связи между ними вызывали живой интерес исследователей.
Классическими примерами подобных условий являются периодичность и локальная конечность: они широко изучаются в многочисленных работах.
Условие периодичности по своей сути накладывает ограничение на подгруппы, порожденные одним элементом группы.
Классический пример, где ограничения накладываются на 2-порожденные подгруппы --- это группы 3-транспозиций.
Группа \(G\) называется группой \(n\)-транспозиций, если она порождается классом \(C\) сопряженных элементов порядка 2 (инволюций), таким что
порядок \(xy\) не превосходит \(n\) для любых элементов \(x\) и \(y\) из \(C\).
Группы 2-транспозиций, очевидно, абелевы и потому локально конечны.
Известно, что группы \(3\)-транспозиций локально конечны.
Вопрос о локальной конечности групп \(n\)-транспозиций для \(n>3\) открыт.
В настоящем докладе речь пойдет о группах, порожденных классом \(D\) сопряженных элементов порядка три.
Дополнительно будем считать, что любые два элемента из \(D\) порождают подгруппу являющуюся гомоморфным образом группы из \(M=\{ 3^{1+2},  SL_2(3), SL_2(5) \}\), здесь \(3^{1+2}\) --- экстраспециальная группа порядка \(27\) и периода \(3\), а \(SL\) --- специальные линейные группы.
Такие группы порождают два элемента порядка 3, инвертируемые 3-транспозициями так называемого типа \(S_4\).
\end{talk}
\end{document}