\documentclass[12pt]{article}
\usepackage{hyphsubst}
\usepackage[T2A]{fontenc}
\usepackage[english,main=russian]{babel}
\usepackage[utf8]{inputenc}
\usepackage[letterpaper,top=2cm,bottom=2cm,left=2cm,right=2cm,marginparwidth=2cm]{geometry}
\usepackage{float}
\usepackage{mathtools, commath, amssymb, amsthm}
\usepackage{enumitem, tabularx,graphicx,url,xcolor,rotating,multicol,epsfig,colortbl,lipsum}

\setlist{topsep=1pt, itemsep=0em}
\setlength{\parindent}{0pt}
\setlength{\parskip}{6pt}

\usepackage{hyphenat}
\hyphenation{ма-те-ма-ти-ка вос-ста-нав-ли-вать}

\usepackage[math]{anttor}

\newenvironment{talk}[6]{%
\vskip 0pt\nopagebreak%
\vskip 0pt\nopagebreak%
\section*{#1}
\phantomsection
\addcontentsline{toc}{section}{#2. \textit{#1}}
% \addtocontents{toc}{\textit{#1}\par}
\textit{#2}\\\nopagebreak%
#3\\\nopagebreak%
\ifthenelse{\equal{#4}{}}{}{\url{#4}\\\nopagebreak}%
\ifthenelse{\equal{#5}{}}{}{Соавторы: #5\\\nopagebreak}%
\ifthenelse{\equal{#6}{}}{}{Секция: #6\\\nopagebreak}%
}

\definecolor{LovelyBrown}{HTML}{FDFCF5}

\usepackage[pdftex,
breaklinks=true,
bookmarksnumbered=true,
linktocpage=true,
linktoc=all
]{hyperref}

\begin{document}
\pagenumbering{gobble}
\pagestyle{plain}
\pagecolor{LovelyBrown}
\begin{talk}
{Граф ортогональности некоторых классических классов колец}
{Маркова Ольга Викторовна}
{МГУ имени М.\,В. Ломоносова}
{ov_markova@mail.ru}
{}
{Алгебра}

Доклад основан на работе [О.\,В. Маркова, Д.\,Ю. Новочадов, {\it Графы ортогональности прямой суммы колец и полупростых артиновых колец}, Зап. научн. сем. ПОМИ, 514 (2022), 138-166].

Изучение алгебраических структур, базирующееся на исследовании соответствующих им графов отношений, находится в центре внимания математиков в течение нескольких десятилетий. Для колец данное направление восходит к работе Бека 1986 года.

С каждым кольцом можно связать \textit{граф ортогональности}, множеством вершин которого являются все двусторонние делители нуля кольца, и две вершины соединены ребром, если соответствующие им элементы кольца ортогональны.

Вопрос о связности и возможных значениях диаметра полностью решен для графов ортогональности коммутативных колец Андерсоном и Ливингстоном в 1999 г.
В докладе будут представлены результаты о графе отношения ортогональности на некоммутативных кольцах. Будет показано решение вопроса о наличии изолированных вершин в графах ортогональности простых и полупростых артиновых колец, первичных и полупервичных колец с двусторонними условиями Голди. Также описаны компоненты связности и поведение функции диаметра при переходе от двух произвольных колец к их прямой сумме: показано, что граф ортогональности прямой суммы колец может в общем случае состоять из изолированных вершин и одной большой связной компоненты, диаметр которой может принимать любое целое значение от 1 до 4. Найдены критерии отсутствия изолированных вершин.
\end{talk}
\end{document}