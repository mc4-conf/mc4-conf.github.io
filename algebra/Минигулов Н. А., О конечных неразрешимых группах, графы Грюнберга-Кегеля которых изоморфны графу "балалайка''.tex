\documentclass[12pt]{article}
\usepackage{hyphsubst}
\usepackage[T2A]{fontenc}
\usepackage[english,main=russian]{babel}
\usepackage[utf8]{inputenc}
\usepackage[letterpaper,top=2cm,bottom=2cm,left=2cm,right=2cm,marginparwidth=2cm]{geometry}
\usepackage{float}
\usepackage{mathtools, commath, amssymb, amsthm}
\usepackage{enumitem, tabularx,graphicx,url,xcolor,rotating,multicol,epsfig,colortbl,lipsum}

\setlist{topsep=1pt, itemsep=0em}
\setlength{\parindent}{0pt}
\setlength{\parskip}{6pt}

\usepackage{hyphenat}
\hyphenation{ма-те-ма-ти-ка вос-ста-нав-ли-вать}

\usepackage[math]{anttor}

\newenvironment{talk}[6]{%
\vskip 0pt\nopagebreak%
\vskip 0pt\nopagebreak%
\section*{#1}
\phantomsection
\addcontentsline{toc}{section}{#2. \textit{#1}}
% \addtocontents{toc}{\textit{#1}\par}
\textit{#2}\\\nopagebreak%
#3\\\nopagebreak%
\ifthenelse{\equal{#4}{}}{}{\url{#4}\\\nopagebreak}%
\ifthenelse{\equal{#5}{}}{}{Соавторы: #5\\\nopagebreak}%
\ifthenelse{\equal{#6}{}}{}{Секция: #6\\\nopagebreak}%
}

\definecolor{LovelyBrown}{HTML}{FDFCF5}

\usepackage[pdftex,
breaklinks=true,
bookmarksnumbered=true,
linktocpage=true,
linktoc=all
]{hyperref}

\begin{document}
\pagenumbering{gobble}
\pagestyle{plain}
\pagecolor{LovelyBrown}
\begin{talk}
{О конечных неразрешимых группах, графы Грюнберга--Кегеля которых изоморфны графу ``балалайка''}
{Минигулов Николай Александрович}
{Институт математики и механики им. Н.\,Н. Красовсого УрО РАН, Екатеринбург}
{n.a.minigulov@imm.uran.ru}
{А.\,С. Кондратьев}
{Алгебра}

{\it Графом Грюнберга--Кегеля} (или {\it графом простых чисел}) \(\Gamma(G)\) конечной группы \(G\) называют граф, в
котором множеством вершин является множество всех простых делителей порядка группы \(G\) и две различные вершины \(p\)
и \(q\) смежны тогда и только тогда, когда в группе \(G\) существует элемент порядка \(pq\). Графом ``балалайка'' (paw)
называют граф, который имеет ровно 4 вершины со степенями 1, 2, 2 и 3.

А.С.~Кондратьев описал конечные группы с графом Грюнберга--Кегеля как у групп \(Aut(J_2)\) (см. [1]) и \(A_{10}\) (см. [2]). Графы Грюнберга--Кегеля этих групп изоморфны как абстрактные графы графу ``балалайка''.

Нами была поставлена более общая задача: описать конечные группы, графы Грюнберга--Кегеля которых изоморфны как
абстрактные графы графу ``балалайка''. В дальнейшем будем считать, что
\(G\) --- конечная группа, граф Грюнберга--Кегеля которой как абстрактный граф изоморфен графу ``балалайка'', т. е.,
граф \(\Gamma(G)\) имеет следующий вид:
\begin{center}
\begin{picture}(60,35)
\put(10,0){\circle{4}}
\put(10,2){\line(0,1){30}}
\put(10,2){\line(4,3){20}}
\put(10,34){\circle{4}}
\put(10,32){\line(4,-3){20}}
\put(32,17){\circle{4}}
\put(34,17){\line(1,0){21}}
\put(57,17){\circle{4}}
\put(0,34){\(r\)}
\put(0,0){\(s\)}
\put(30,8){\(p\)}
\put(55,8){\(q\)}
\put(70,0){,}
\end{picture}
\end{center}
где \(r\), \(s\), \(p\) и \(q\) --- некоторые попарно различные простые числа.

В [3] мы доказали, что если группа \(G\) неразрешима, то фактор-группа \(\overline{G}=G/S(G)\) (где
\(S(G)\) --- разрешимый радикал группы \(G\)) почти проста, и класифицировали все конечные почти простые группы графы
Грюнберга--Кегеля которых изоморфны как абстрактные графы подграфу графа ``балалайка''. В [4] мы описали
конечные разрешимые группы \(G\). Также в [5, 6] мы классифицировали конечные неразрешимые группы \(G\) в
следующих трех случаях:
\begin{enumerate}[label=(\arabic*)]
\item группа \(G\) не содержит элементов порядка 6;
\item группа \(G\) содержит элемент порядка 6 и вершина \(q\) графа \(\Gamma(G)\) делит \(|S(G)|\);
\item вершина \(q\) графа \(\Gamma(G)\) меньше 5.
\end{enumerate}

В данной работе мы продолжаем изучение этой проблемы. Мы доказываем следующую теорему.

\textbf{Теорема.} {\it Пусть \(G\)~--- неразрешимая группа, \(\{r,s\}=\{2,3\}\), \(p>3\), \(q>3\) и \(q\) не делит \(|S(G)|\). Тогда
граф Грюнберга--Кегеля фактор-группы \(\overline{G}:=G/S(G)\) несвязен.}

\medskip

\begin{enumerate}
\item[{[1]}] A.~S.~Kondrat'ev, {\it Finite groups with prime graph as in the group \(Aut(J_2)\)},  Proc. Steklov Inst. Math. 283:1 (2013), 78-85.
\item[{[2]}] A.~S.~Kondrat'ev, {\it Finite groups that have the same prime graph as the group \(A_{10}\)}, Proc. Steklov Inst. Math., 285:~1 (2014), 99-107.
\item[{[3]}] A.~S.~Kondrat'ev, N.~A~Minigulov, {\it Finite almost simple groups whose Gruenberg--Kegel graphs as abstract graphs are isomorphic to subgraphs of the Gruenberg--Kegel graph of the alternating group \(A_{10}\)}, Siberian Electr. Math. Rep., 15 (2018), 1378-1382.
\item[{[4]}] A.~S.~Kondrat'ev, N.~A.~Minigulov, {\it Finite solvable groups whose Gruenberg-Kegel graphs are isomorphic to the paw}, Тр. ИММ УрО РАН., 28:2 (2022), 269-273.
\item[{[5]}] A.~S.~Kondrat'ev, N.~A.~Minigulov, {\it On finite non-solvable groups whose Gruenberg--Kegel graphs are isomorphic to the paw}, Commun. Math. Stat., 10:4 (2022), 653-667.
\item[{[6]}] A.~S.~Kondrat'ev, N.~A.~Minigulov, {\it On finite non-solvable groups whose Gruenberg–Kegel graphs are isomorphic to the paw}, Тезисы докладов международной (54-й всеросийской) молодежной школы-конференции "Современные проблемы математики и ее приложений". Екатеринбург: ИММ УрО РАН, 2023, 39.
\end{enumerate}
\end{talk}
\end{document}