\documentclass[12pt]{article}
\usepackage{hyphsubst}
\usepackage[T2A]{fontenc}
\usepackage[english,main=russian]{babel}
\usepackage[utf8]{inputenc}
\usepackage[letterpaper,top=2cm,bottom=2cm,left=2cm,right=2cm,marginparwidth=2cm]{geometry}
\usepackage{float}
\usepackage{mathtools, commath, amssymb, amsthm}
\usepackage{enumitem, tabularx,graphicx,url,xcolor,rotating,multicol,epsfig,colortbl,lipsum}

\setlist{topsep=1pt, itemsep=0em}
\setlength{\parindent}{0pt}
\setlength{\parskip}{6pt}

\usepackage{hyphenat}
\hyphenation{ма-те-ма-ти-ка вос-ста-нав-ли-вать}

\usepackage[math]{anttor}

\newenvironment{talk}[6]{%
\vskip 0pt\nopagebreak%
\vskip 0pt\nopagebreak%
\section*{#1}
\phantomsection
\addcontentsline{toc}{section}{#2. \textit{#1}}
% \addtocontents{toc}{\textit{#1}\par}
\textit{#2}\\\nopagebreak%
#3\\\nopagebreak%
\ifthenelse{\equal{#4}{}}{}{\url{#4}\\\nopagebreak}%
\ifthenelse{\equal{#5}{}}{}{Соавторы: #5\\\nopagebreak}%
\ifthenelse{\equal{#6}{}}{}{Секция: #6\\\nopagebreak}%
}

\definecolor{LovelyBrown}{HTML}{FDFCF5}

\usepackage[pdftex,
breaklinks=true,
bookmarksnumbered=true,
linktocpage=true,
linktoc=all
]{hyperref}

\begin{document}
\pagenumbering{gobble}
\pagestyle{plain}
\pagecolor{LovelyBrown}
\begin{talk}
{k-замыкания групп подстановок}
{Скресанов Савелий Вячеславович}
{Новосибирский государственный университет; Институт математики им. С.\,Л. Соболева СО РАН}
{skresan@math.nsc.ru}
{}
{Алгебра}

Группа подстановок \(G\) на конечном множестве \(\Omega\) также действует
естественным образом на декартовой степени \(\Omega^k\),  \(k \geq 1\).
Наибольшая группа подстановок на  \(\Omega\), имеющая такие же орбиты на \(\Omega^k\), что и \(G\), называется \(k\)-замыканием группы \(G\). Это понятие,
предложенное Х.~Виландом в связи с изучением порядков примитивных групп
подстановок, нашло впоследствии многочисленные применения для изучения
автоморфизмов комбинаторных структур --- например, группа автоморфизмов любого
графа будет совпадать со своим \(2\)-замыканием. Особо важным оказалось изучение
абстрактных свойств групп, сохраняемых при \(k\)-замыканиях. В докладе планируется
рассказать о недавних продвижениях в этом направлении, а также затронуть
алгоритмические вопросы вычисления \(k\)-замыканий и их связь с проблемой
изоморфизма графов из теории сложности.

\medskip

Исследование выполнено за счет гранта Российского научного фонда \textnumero 24-11-00127, \url{https://rscf.ru/project/24-11-00127/}.
\end{talk}
\end{document}