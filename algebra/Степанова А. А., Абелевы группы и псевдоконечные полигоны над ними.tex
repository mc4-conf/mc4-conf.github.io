\documentclass[12pt]{article}
\usepackage{hyphsubst}
\usepackage[T2A]{fontenc}
\usepackage[english,main=russian]{babel}
\usepackage[utf8]{inputenc}
\usepackage[letterpaper,top=2cm,bottom=2cm,left=2cm,right=2cm,marginparwidth=2cm]{geometry}
\usepackage{float}
\usepackage{mathtools, commath, amssymb, amsthm}
\usepackage{enumitem, tabularx,graphicx,url,xcolor,rotating,multicol,epsfig,colortbl,lipsum}

\setlist{topsep=1pt, itemsep=0em}
\setlength{\parindent}{0pt}
\setlength{\parskip}{6pt}

\usepackage{hyphenat}
\hyphenation{ма-те-ма-ти-ка вос-ста-нав-ли-вать}

\usepackage[math]{anttor}

\newenvironment{talk}[6]{%
\vskip 0pt\nopagebreak%
\vskip 0pt\nopagebreak%
\section*{#1}
\phantomsection
\addcontentsline{toc}{section}{#2. \textit{#1}}
% \addtocontents{toc}{\textit{#1}\par}
\textit{#2}\\\nopagebreak%
#3\\\nopagebreak%
\ifthenelse{\equal{#4}{}}{}{\url{#4}\\\nopagebreak}%
\ifthenelse{\equal{#5}{}}{}{Соавторы: #5\\\nopagebreak}%
\ifthenelse{\equal{#6}{}}{}{Секция: #6\\\nopagebreak}%
}

\definecolor{LovelyBrown}{HTML}{FDFCF5}

\usepackage[pdftex,
breaklinks=true,
bookmarksnumbered=true,
linktocpage=true,
linktoc=all
]{hyperref}

\begin{document}
\pagenumbering{gobble}
\pagestyle{plain}
\pagecolor{LovelyBrown}
\begin{talk}
{Абелевы группы и псевдоконечные полигоны над ними}
{Степанова Алена Андреевна}
{Дальневосточный федеральный университет}
{stepanova.aa@dvfu.ru}
{С.\,Г. Чеканов}
{Алгебра}

Теория моделей псевдоконечных структур --- активно развивающаяся в последнее время область математики. Это направление исследований, благодаря теореме Лося, тесно связано с теорией конечных моделей. Структура \(\mathfrak{M}\) языка \(L\) псевдоконечна, если любое предложение языка \(L\), истинное в \(\mathfrak{M}\), истинно в некоторой конечной структуре языка \(L\). В работах [1-6] изучаются вопросы строения псевдоконечных структур (полей, групп, колец, графов, унаров, полигон над моноидом).

Доклад посвящен исследованию \(T\)-псевдоконечных полигонов над группой \(G\), где \(T\) --- теория всех полигонов  над \(G\). Полигон \(_GA\) называется \(T\)-псевдоконечным, если  любое предложение, истинное в \(_GA\), истинно в некотором конечном полигоне над \(G\). Доказано, что класс всех полигонов над конечнопорожденной абелевой группой, представимых в виде копроизведения полигонов \(_GG/S_i\), \(1\le i\le n\), где  \(S_1,\ldots,S_n\) --- фиксированные подгруппы группы \(G\), \(T\)-псевдоконечен.

\medskip

\begin{enumerate}
\item[{[1]}] Z. Chatzidakis. Notes on the model theory of finite and pseudo-finite fields.\\{\tt http://www.logique.jussieu.fr/\~zoe/papiers/Helsinki.pdf}.
\item[{[2]}] D. Macpherson, {\it  Model theory of finite and pseudofinite groups}, Arch. Math. Logic, 57:1-2 (2018), 159-184.
\item[{[3]}] R. Bello-Aguirre, Model theory of finite and pseudofinite rings, PhD thesis, University of Leeds, 2016.
\item[{[4]}] N. D.  Markhabatov, {\it Approximations of Acyclic Graphs}, Известия Иркутского государственного университета. Серия «Математика», 40 (2022), 104–111.
\item[{[5]}] E.L. Efremov, A.A. Stepanova, S.G. Chekanov, {\it Pseudofinite unars}, Algebra Logic (in press).
\item [{[6]}] E.L. Efremov, A.A. Stepanova, S.G. Chekanov, {\it Pseudofinite S-acts}, Siberian Electronic Mathematical Reports,  V. 21:1 (2024), p. 271-276.
\end{enumerate}
\end{talk}
\end{document}