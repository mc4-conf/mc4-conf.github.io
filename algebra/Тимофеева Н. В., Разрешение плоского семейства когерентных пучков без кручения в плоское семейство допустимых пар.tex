\documentclass[12pt]{article}
\usepackage{hyphsubst}
\usepackage[T2A]{fontenc}
\usepackage[english,main=russian]{babel}
\usepackage[utf8]{inputenc}
\usepackage[letterpaper,top=2cm,bottom=2cm,left=2cm,right=2cm,marginparwidth=2cm]{geometry}
\usepackage{float}
\usepackage{mathtools, commath, amssymb, amsthm}
\usepackage{enumitem, tabularx,graphicx,url,xcolor,rotating,multicol,epsfig,colortbl,lipsum}

\setlist{topsep=1pt, itemsep=0em}
\setlength{\parindent}{0pt}
\setlength{\parskip}{6pt}

\usepackage{hyphenat}
\hyphenation{ма-те-ма-ти-ка вос-ста-нав-ли-вать}

\usepackage[math]{anttor}

\newenvironment{talk}[6]{%
\vskip 0pt\nopagebreak%
\vskip 0pt\nopagebreak%
\section*{#1}
\phantomsection
\addcontentsline{toc}{section}{#2. \textit{#1}}
% \addtocontents{toc}{\textit{#1}\par}
\textit{#2}\\\nopagebreak%
#3\\\nopagebreak%
\ifthenelse{\equal{#4}{}}{}{\url{#4}\\\nopagebreak}%
\ifthenelse{\equal{#5}{}}{}{Соавторы: #5\\\nopagebreak}%
\ifthenelse{\equal{#6}{}}{}{Секция: #6\\\nopagebreak}%
}

\definecolor{LovelyBrown}{HTML}{FDFCF5}

\usepackage[pdftex,
breaklinks=true,
bookmarksnumbered=true,
linktocpage=true,
linktoc=all
]{hyperref}

\begin{document}
\pagenumbering{gobble}
\pagestyle{plain}
\pagecolor{LovelyBrown}
\begin{talk}
{Разрешение плоского семейства когерентных пучков без кручения в плоское семейство допустимых пар и~пространство модулей допустимых пар в размерности~\(\ge 2\)}
{Тимофеева Надежда Владимировна}
{Ярославский госуниверситет им. П.\,Г.Демидова, Центр интегрируемых систем}
{ntimofeeva@list.ru}
{}
{Алгебра}

В докладе будут рассмотрены следующие вопросы:
\begin{enumerate}
\item[(a)] Преобразование единичного когерентного алгебраического пучка \(E\), имеющего ранг \(r\) и полином Гильберта \(rp(t)\) на неособом проективном алгебраическом многообразии \((S,L)\) размерности \(d\ge 2\) (\(L\) -- обильный обратимый пучок) в допустимую пару \(((\tilde S, \tilde L), \tilde E)\) (\((\tilde S, \tilde L)\) -- проективная алгебраическая схема определённого вида, \(\tilde E\) -- локально свободный пучок того же ранга \(r\) и с тем же полиномом Гильберта \(rp(t)\)) [1];
\item[(b)] Преобразование плоского семейства когерентных алгебраических пучков в плоское семейство допустимых пар, послойно сводящееся к преобразованию п.1;
\item[(c)] Понятия стабильности (полустабильности) допустимой пары \(((\tilde S, \tilde L), \tilde E)\) и их связь со стабильностью (полустабильностью) когерентного пучка \(E\), разрешением которого получена эта пара [1];
\item[(d)] Индуцированный морфизм пространства (алгебраической схемы) модулей полустабильных допустимых пар на классическое пространство модулей Гизекера -- Маруямы когерентных пучков без кручения.
\end{enumerate}

С обоснованием и происхождением рассматриваемых задач также можно ознакомиться по работе [1].

\medskip

\begin{enumerate}
\item[{[1]}] Н. В. Тимофеева, {\it Стабильность и эквивалентность допустимых пар произвольной размерности для компактификации пространства модулей стабильных векторных расслоений}, ТМФ, 212:1 (2022),  109–128.
\end{enumerate}
\end{talk}
\end{document}