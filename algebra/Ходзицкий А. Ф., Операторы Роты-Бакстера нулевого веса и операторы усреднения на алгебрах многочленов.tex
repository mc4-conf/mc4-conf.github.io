\documentclass[12pt]{article}
\usepackage{hyphsubst}
\usepackage[T2A]{fontenc}
\usepackage[english,main=russian]{babel}
\usepackage[utf8]{inputenc}
\usepackage[letterpaper,top=2cm,bottom=2cm,left=2cm,right=2cm,marginparwidth=2cm]{geometry}
\usepackage{float}
\usepackage{mathtools, commath, amssymb, amsthm}
\usepackage{enumitem, tabularx,graphicx,url,xcolor,rotating,multicol,epsfig,colortbl,lipsum}

\setlist{topsep=1pt, itemsep=0em}
\setlength{\parindent}{0pt}
\setlength{\parskip}{6pt}

\usepackage{hyphenat}
\hyphenation{ма-те-ма-ти-ка вос-ста-нав-ли-вать}

\usepackage[math]{anttor}

\newenvironment{talk}[6]{%
\vskip 0pt\nopagebreak%
\vskip 0pt\nopagebreak%
\section*{#1}
\phantomsection
\addcontentsline{toc}{section}{#2. \textit{#1}}
% \addtocontents{toc}{\textit{#1}\par}
\textit{#2}\\\nopagebreak%
#3\\\nopagebreak%
\ifthenelse{\equal{#4}{}}{}{\url{#4}\\\nopagebreak}%
\ifthenelse{\equal{#5}{}}{}{Соавторы: #5\\\nopagebreak}%
\ifthenelse{\equal{#6}{}}{}{Секция: #6\\\nopagebreak}%
}

\definecolor{LovelyBrown}{HTML}{FDFCF5}

\usepackage[pdftex,
breaklinks=true,
bookmarksnumbered=true,
linktocpage=true,
linktoc=all
]{hyperref}

\begin{document}
\pagenumbering{gobble}
\pagestyle{plain}
\pagecolor{LovelyBrown}
\begin{talk}
{Операторы Роты--Бакстера нулевого веса и операторы усреднения на алгебрах многочленов}
{Ходзицкий Артем Федорович}
{НГУ, Новосибирск}
{a.khodzitskii@g.nsu.ru}
{}
{Алгебра}

Пусть \(A\) --- алгебра над полем \(F\).
Линейный оператор \(T\) на \(A\) называется оператором усреднения,
если выполнены соотношения
\(T(a)T(b) = T(T(a)b) = T(a T(b))\)
для всех \(a,b\in A\).
Линейный оператор \(R\) на \(A\)
называется оператором Роты--Бакстера, если
\[R(a)R(b) = R\big(R(a)b + a R(b) + \lambda a b\big)\]
выполнено для всех \(a,b \in A\).
Здесь \(\lambda\in F\) --- фиксированный скаляр, вес оператора~\(R\).

Линейный оператор \(L\) на алгебре многочленов называется мономиальным,
если для любого монома \(t\) найдутся
моном \(z_t\) и скаляр \(\alpha_t\) такие, что \(L(t) = \alpha_t z_t\).
Мономиальные операторы Роты--Бакстера на \(F[x]\) были введены в~[1]
и описаны на \(F[x]\) в~[2].

В работе~[3] найдена взаимосвязь между
операторами Роты--Бакстера и операторами усреднения.
В этой работе был описан класс операторов Роты--Бакстера ненулевого веса,
построенных по гомоморфным операторам усреднения на \(F[x,y]\).
Мы классифицировали операторы Роты--Бакстера нулевого веса,
построенные по операторам усреднения
с~линейными функциями в степенях мономов из образа на \(F[x,y]\).

\medskip

\begin{enumerate}
\item[{[1]}] L. Guo, M. Rosenkranz, and S.H. Zheng.
\textit{Rota---Baxter operators on the polynomial algebras,~integration~and~averaging operators,~Pacific~J.\,Math.\,(2)~275~(2015),\,481--507.}
\item[{[2]}] H. Yu.
\textit{Classification of monomial Rota---Baxter operators on \(k[x]\),
J. Algebra Appl. 15 (2016), 1650087.}
\item[{[3]}] A. Khodzitskii,
\textit{Monomial Rota---Baxter Operators of Nonzero Weight on \(F[x, y]\) Coming from Averaging Operators,
Mediterr. J. Math. 20 (2023), No~251.}
\end{enumerate}
\end{talk}
\end{document}