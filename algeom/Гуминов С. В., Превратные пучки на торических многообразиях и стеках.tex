\documentclass[12pt]{article}
\usepackage{hyphsubst}
\usepackage[T2A]{fontenc}
\usepackage[english,main=russian]{babel}
\usepackage[utf8]{inputenc}
\usepackage[letterpaper,top=2cm,bottom=2cm,left=2cm,right=2cm,marginparwidth=2cm]{geometry}
\usepackage{float}
\usepackage{mathtools, commath, amssymb, amsthm}
\usepackage{enumitem, tabularx,graphicx,url,xcolor,rotating,multicol,epsfig,colortbl,lipsum}

\setlist{topsep=1pt, itemsep=0em}
\setlength{\parindent}{0pt}
\setlength{\parskip}{6pt}

\usepackage{hyphenat}
\hyphenation{ма-те-ма-ти-ка вос-ста-нав-ли-вать}

\usepackage[math]{anttor}

\newenvironment{talk}[6]{%
\vskip 0pt\nopagebreak%
\vskip 0pt\nopagebreak%
\section*{#1}
\phantomsection
\addcontentsline{toc}{section}{#2. \textit{#1}}
% \addtocontents{toc}{\textit{#1}\par}
\textit{#2}\\\nopagebreak%
#3\\\nopagebreak%
\ifthenelse{\equal{#4}{}}{}{\url{#4}\\\nopagebreak}%
\ifthenelse{\equal{#5}{}}{}{Соавторы: #5\\\nopagebreak}%
\ifthenelse{\equal{#6}{}}{}{Секция: #6\\\nopagebreak}%
}

\definecolor{LovelyBrown}{HTML}{FDFCF5}

\usepackage[pdftex,
breaklinks=true,
bookmarksnumbered=true,
linktocpage=true,
linktoc=all
]{hyperref}

\begin{document}
\pagenumbering{gobble}
\pagestyle{plain}
\pagecolor{LovelyBrown}
\begin{talk}
{Превратные пучки на торических многообразиях и стеках}
{Гуминов Сергей Владимирович}
{МФТИ, ВШЭ}
{sergey.guminov@gmail.com}
{А.\,И. Бондал}
{Алгебраическая геометрия}

Явное описание категорий превратных пучков остаётся важной задачей топологии алгебраических многообразий. Несмотря на это, эта задача решена для крайне скромного числа многообразий большой размерности. Доклад будет посвящен превратным пучкам на торических многообразиях. Оказывается, категория превратных пучков на гладком торическом многообразии \(X\) с веером \(\Sigma\), на котором действует тор \(T\), эквивалентна категории конечномерных модулей над алгеброй \(A(\Sigma)\), которая является конечным расширением кольца функций двойственного тора \(T^\vee\). В докладе я сформулирую этот результат, скажу несколько слов о его (потенциальной) связи с когерентно-конструктивным соответствием, а также, если позволит время, поясню, чем отличается случай негладкого торического многообразия \(X\).
\end{talk}
\end{document}