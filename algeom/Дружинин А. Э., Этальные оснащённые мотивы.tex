\documentclass[12pt]{article}
\usepackage{hyphsubst}
\usepackage[T2A]{fontenc}
\usepackage[english,main=russian]{babel}
\usepackage[utf8]{inputenc}
\usepackage[letterpaper,top=2cm,bottom=2cm,left=2cm,right=2cm,marginparwidth=2cm]{geometry}
\usepackage{float}
\usepackage{mathtools, commath, amssymb, amsthm}
\usepackage{enumitem, tabularx,graphicx,url,xcolor,rotating,multicol,epsfig,colortbl,lipsum}

\setlist{topsep=1pt, itemsep=0em}
\setlength{\parindent}{0pt}
\setlength{\parskip}{6pt}

\usepackage{hyphenat}
\hyphenation{ма-те-ма-ти-ка вос-ста-нав-ли-вать}

\usepackage[math]{anttor}

\newenvironment{talk}[6]{%
\vskip 0pt\nopagebreak%
\vskip 0pt\nopagebreak%
\section*{#1}
\phantomsection
\addcontentsline{toc}{section}{#2. \textit{#1}}
% \addtocontents{toc}{\textit{#1}\par}
\textit{#2}\\\nopagebreak%
#3\\\nopagebreak%
\ifthenelse{\equal{#4}{}}{}{\url{#4}\\\nopagebreak}%
\ifthenelse{\equal{#5}{}}{}{Соавторы: #5\\\nopagebreak}%
\ifthenelse{\equal{#6}{}}{}{Секция: #6\\\nopagebreak}%
}

\definecolor{LovelyBrown}{HTML}{FDFCF5}

\usepackage[pdftex,
breaklinks=true,
bookmarksnumbered=true,
linktocpage=true,
linktoc=all
]{hyperref}

\begin{document}
\pagenumbering{gobble}
\pagestyle{plain}
\pagecolor{LovelyBrown}
\begin{talk}
{Этальные оснащённые мотивы}
{Дружинин Андрей Эдуардович}
{СПбГУ МКН, лаборатория им. Чебышёва; ПОМИ РАН, лаборатория Алгебры и теории чисел}
{andrei.druzh@gmail.com}
{Ola Sande}
{Алгебраическая геометрия}

Теорема сравнения для мотивных и классических стабильных гомотопических групп, доказанная М. Левином, утверждает изоморфизм упомянутых групп
\[\pi^{\mathbb A^1,\mathrm{Nis}}_{i,0}(X)\simeq \pi_{i}(X(\mathbb C)),\]
для всякой гладкой схемы \(X\) над \(\mathbb C\) и топологического пространства её комплексных точек.

В первой части доклада будет рассказано о двух обобщениях или аналогах функтора реализации Бетти
\[X\mapsto X(\mathbb C)\]
для полей положительной характеристики:
1) с помощью \(\infty\)-категорной теории топосов Гротендика, построенной Дж. Лури, и топологичесих моделей Д. Исаксена, построение которых в \(\infty\)-категорном контексте выполнено М. Айуа,
и
2) с помощью теорем жёсткости Воеводского-Суслина, Дж. Айуба и Т. Бахмана,
а также
об обобщении указанного выше изоморфизма на алгебраически замкнутые базовые поля произвольной характеорискики, и целого числа \(n\) обратимого в базовом поле, доказанном М. Заргаром.

Во второй части доклада будет рассказано альтернативное доказательство указанного выше изоморфизма над \(\mathbb C\) и его обобщения для произвольного базового поля, основанное на теории оснащённых мотивов, построенной оригинально Г. Гаркушей и И. Паниным для Нисневич локальных \(\mathbb A^1\)-мотивных \(\mathbb P^1\)-спектров, в сочетании с её переном на гиперполные этальные \(n\)-пополненные \(\mathbb A^1\)-мотивные \(\mathbb P^1\)-спектры, выполненном автором в соавторстве с Ула Санде.
\end{talk}
\end{document}