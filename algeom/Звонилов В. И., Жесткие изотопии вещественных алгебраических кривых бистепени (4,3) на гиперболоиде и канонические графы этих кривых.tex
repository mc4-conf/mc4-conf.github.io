\documentclass[12pt]{article}
\usepackage{hyphsubst}
\usepackage[T2A]{fontenc}
\usepackage[english,main=russian]{babel}
\usepackage[utf8]{inputenc}
\usepackage[letterpaper,top=2cm,bottom=2cm,left=2cm,right=2cm,marginparwidth=2cm]{geometry}
\usepackage{float}
\usepackage{mathtools, commath, amssymb, amsthm}
\usepackage{enumitem, tabularx,graphicx,url,xcolor,rotating,multicol,epsfig,colortbl,lipsum}

\setlist{topsep=1pt, itemsep=0em}
\setlength{\parindent}{0pt}
\setlength{\parskip}{6pt}

\usepackage{hyphenat}
\hyphenation{ма-те-ма-ти-ка вос-ста-нав-ли-вать}

\usepackage[math]{anttor}

\newenvironment{talk}[6]{%
\vskip 0pt\nopagebreak%
\vskip 0pt\nopagebreak%
\section*{#1}
\phantomsection
\addcontentsline{toc}{section}{#2. \textit{#1}}
% \addtocontents{toc}{\textit{#1}\par}
\textit{#2}\\\nopagebreak%
#3\\\nopagebreak%
\ifthenelse{\equal{#4}{}}{}{\url{#4}\\\nopagebreak}%
\ifthenelse{\equal{#5}{}}{}{Соавторы: #5\\\nopagebreak}%
\ifthenelse{\equal{#6}{}}{}{Секция: #6\\\nopagebreak}%
}

\definecolor{LovelyBrown}{HTML}{FDFCF5}

\usepackage[pdftex,
breaklinks=true,
bookmarksnumbered=true,
linktocpage=true,
linktoc=all
]{hyperref}

\begin{document}
\pagenumbering{gobble}
\pagestyle{plain}
\pagecolor{LovelyBrown}
\begin{talk}
{Жесткие изотопии вещественных алгебраических кривых бистепени (4,3) на гиперболоиде и канонические графы этих кривых}
{Звонилов Виктор Иванович}
{Национальный исследовательский Нижегородский государственный университет им. Н.\,И. Лобачевского}
{zvonilov@itmm.unn.ru}
{}
{Алгебраическая геометрия}

Жесткой изотопией вещественных алгебраических кривых некоторого класса
называется путь в пространстве  кривых этого класса. Данное исследование завершает
жесткую изотопическую классификацию неособых вещественных алгебраических кривых
бистепени (4,3) на гиперболоиде, начатую автором в более ранних работах [1], [2]. Перечислены также компоненты
связности пространства вещественных алгебраических кривых бистепени (4,3),
имеющих единственную невырожденную двойную точку или точку возврата. Основным техническим средством являются графы 	 вещественных тригональных кривых на поверхностях Хирцебруха.

\medskip

\begin{enumerate}
\item[{[1]}]
В. И. Звонилов, {\it Жесткая изотопическая классификация вещественных алгебраических кривых бистепени (4,3) на гиперболоиде}, Вестник Сыктывкарского университета,  Сер. 1, вып. 3 (1999), 81-88.
\item[{[2]}] В. И. Звонилов, {\it Жесткая изотопическая классификация вещественных алгебраических кривых бистепени (4,3) на гиперболоиде. Приложение}, Вестник Сыктывкарского университета,  Сер. 1, вып. 5 (2003), 239-242.
\end{enumerate}
\end{talk}
\end{document}