\documentclass[12pt]{article}
\usepackage{hyphsubst}
\usepackage[T2A]{fontenc}
\usepackage[english,main=russian]{babel}
\usepackage[utf8]{inputenc}
\usepackage[letterpaper,top=2cm,bottom=2cm,left=2cm,right=2cm,marginparwidth=2cm]{geometry}
\usepackage{float}
\usepackage{mathtools, commath, amssymb, amsthm}
\usepackage{enumitem, tabularx,graphicx,url,xcolor,rotating,multicol,epsfig,colortbl,lipsum}

\setlist{topsep=1pt, itemsep=0em}
\setlength{\parindent}{0pt}
\setlength{\parskip}{6pt}

\usepackage{hyphenat}
\hyphenation{ма-те-ма-ти-ка вос-ста-нав-ли-вать}

\usepackage[math]{anttor}

\newenvironment{talk}[6]{%
\vskip 0pt\nopagebreak%
\vskip 0pt\nopagebreak%
\section*{#1}
\phantomsection
\addcontentsline{toc}{section}{#2. \textit{#1}}
% \addtocontents{toc}{\textit{#1}\par}
\textit{#2}\\\nopagebreak%
#3\\\nopagebreak%
\ifthenelse{\equal{#4}{}}{}{\url{#4}\\\nopagebreak}%
\ifthenelse{\equal{#5}{}}{}{Соавторы: #5\\\nopagebreak}%
\ifthenelse{\equal{#6}{}}{}{Секция: #6\\\nopagebreak}%
}

\definecolor{LovelyBrown}{HTML}{FDFCF5}

\usepackage[pdftex,
breaklinks=true,
bookmarksnumbered=true,
linktocpage=true,
linktoc=all
]{hyperref}

\begin{document}
\pagenumbering{gobble}
\pagestyle{plain}
\pagecolor{LovelyBrown}
\begin{talk}
{Характеры и коприсоединённые орбиты унипотентных групп}
{Игнатьев Михаил Викторович}
{НИУ ВШЭ}
{mihail.ignatev@gmail.com}
{Михаил Венчаков, Алексей Петухов}
{Алгебраическая геометрия}

Основным инструментом в теории представлений нильпотентных групп и алгебр Ли является метод орбит, созданный А.А. Кирилловым в 1962 году. Коротко говоря, он гласит, что неприводимые представления (для групп) и их аннуляторы в универсальной обёртывающей алгебре (для алгебр Ли) находятся во взаимно однозначном соответствии с орбитами коприсоединённого представления данной алгебраической группы. Для унипотентных групп над конечными полями метод орбит описывает обычные их конечномерные комплексные представления.

Увы, полная классификация орбит даже для группы унитреугольных матриц является дикой задачей. С другой стороны, для максимальных унипотентных подгрупп в группах Шевалле над конечными полями удаётся достаточно хорошо описать и изучить различные важные классы орбит --- больших и малых размерностей, ассоциированных с расстановками ладей и др. Я расскажу о методах, позволяющих изучать такие орбиты и соответствующие им неприводимые характеры, и сформулирую недавние результаты и открытые вопросы.
\end{talk}
\end{document}