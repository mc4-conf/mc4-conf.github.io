\documentclass[12pt]{article}
\usepackage{hyphsubst}
\usepackage[T2A]{fontenc}
\usepackage[english,main=russian]{babel}
\usepackage[utf8]{inputenc}
\usepackage[letterpaper,top=2cm,bottom=2cm,left=2cm,right=2cm,marginparwidth=2cm]{geometry}
\usepackage{float}
\usepackage{mathtools, commath, amssymb, amsthm}
\usepackage{enumitem, tabularx,graphicx,url,xcolor,rotating,multicol,epsfig,colortbl,lipsum}

\setlist{topsep=1pt, itemsep=0em}
\setlength{\parindent}{0pt}
\setlength{\parskip}{6pt}

\usepackage{hyphenat}
\hyphenation{ма-те-ма-ти-ка вос-ста-нав-ли-вать}

\usepackage[math]{anttor}

\newenvironment{talk}[6]{%
\vskip 0pt\nopagebreak%
\vskip 0pt\nopagebreak%
\section*{#1}
\phantomsection
\addcontentsline{toc}{section}{#2. \textit{#1}}
% \addtocontents{toc}{\textit{#1}\par}
\textit{#2}\\\nopagebreak%
#3\\\nopagebreak%
\ifthenelse{\equal{#4}{}}{}{\url{#4}\\\nopagebreak}%
\ifthenelse{\equal{#5}{}}{}{Соавторы: #5\\\nopagebreak}%
\ifthenelse{\equal{#6}{}}{}{Секция: #6\\\nopagebreak}%
}

\definecolor{LovelyBrown}{HTML}{FDFCF5}

\usepackage[pdftex,
breaklinks=true,
bookmarksnumbered=true,
linktocpage=true,
linktoc=all
]{hyperref}

\begin{document}
\pagenumbering{gobble}
\pagestyle{plain}
\pagecolor{LovelyBrown}
\begin{talk}
{Бирациональные инварианты отображений, сохраняющих объем}
{Логинов Константин Валерьевич}
{МИАН}
{loginov@mi-ras.ru}
{}
{Алгебраическая геометрия}

Одной из основных задач бирациональной геометрии является классификация алгебраических многообразий с точностью до бирациональной эквивалентности. Уточняя эту задачу, можно классифицировать алгебраические многообразия с дополнительной структурой, например, рассматривая многообразия с фиксированной (голоморфной) формой объема. При этом естественно рассматривать формы объема, имеющие полюса не более чем первого порядка.

Группа классов эквивалентности многообразий с такой формой называется группой Бернсайда. Эта группа хороша тем, что в ней принимают значения некоторые естественные инварианты бирациональных отображеннией, сохраняющих форму объема на данном многообразии. Мы определим и изучим эти инварианты (иногда называемые ``мотивными инвариантами'') для групп бирациональных автоморфизмов проективного пространства со ``стандартной'' торически-инвариантной формой. Мы покажем, что такие группы не являются простыми в любой размерности, начиная с четырех, а также что они не могут порождаться псевдо-регуляризуемыми элементами.
Этот результат можно рассматривать как обобщение аналогичной теоремы для классической группы Кремоны, то есть группы бирациональных автоморфизмов проективного пространства.
\end{talk}
\end{document}