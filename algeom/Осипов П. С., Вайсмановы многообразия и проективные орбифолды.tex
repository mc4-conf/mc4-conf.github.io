\documentclass[12pt]{article}
\usepackage{hyphsubst}
\usepackage[T2A]{fontenc}
\usepackage[english,main=russian]{babel}
\usepackage[utf8]{inputenc}
\usepackage[letterpaper,top=2cm,bottom=2cm,left=2cm,right=2cm,marginparwidth=2cm]{geometry}
\usepackage{float}
\usepackage{mathtools, commath, amssymb, amsthm}
\usepackage{enumitem, tabularx,graphicx,url,xcolor,rotating,multicol,epsfig,colortbl,lipsum}

\setlist{topsep=1pt, itemsep=0em}
\setlength{\parindent}{0pt}
\setlength{\parskip}{6pt}

\usepackage{hyphenat}
\hyphenation{ма-те-ма-ти-ка вос-ста-нав-ли-вать}

\usepackage[math]{anttor}

\newenvironment{talk}[6]{%
\vskip 0pt\nopagebreak%
\vskip 0pt\nopagebreak%
\section*{#1}
\phantomsection
\addcontentsline{toc}{section}{#2. \textit{#1}}
% \addtocontents{toc}{\textit{#1}\par}
\textit{#2}\\\nopagebreak%
#3\\\nopagebreak%
\ifthenelse{\equal{#4}{}}{}{\url{#4}\\\nopagebreak}%
\ifthenelse{\equal{#5}{}}{}{Соавторы: #5\\\nopagebreak}%
\ifthenelse{\equal{#6}{}}{}{Секция: #6\\\nopagebreak}%
}

\definecolor{LovelyBrown}{HTML}{FDFCF5}

\usepackage[pdftex,
breaklinks=true,
bookmarksnumbered=true,
linktocpage=true,
linktoc=all
]{hyperref}

\begin{document}
\pagenumbering{gobble}
\pagestyle{plain}
\pagecolor{LovelyBrown}
\begin{talk}
{Вайсмановы многообразия и проективные орбифолды}
{Осипов Павел Сергеевич}
{НИУ ВШЭ}
{pavos3001@gmail.com}
{}
{Алгебраическая геометрия}

Пусть \(M\) --- компактное комплексное многообразие с локально конформно кэлеровой метрикой \(g\). Тогда поднятие метрики \(g\) на универсальную накрывающую \(\tilde M\) глобально конформно эквивалентно кэлеровой метрике \(\check g\). При этом фундаментальная группа \(\pi(M)\) действует на \((\tilde M,\check g)\) гомотетиями. Если многообразие \((\tilde M,\check g)\) изометрично риманову конусу \(\left(S\times \mathbb{R}^{>0}, t^2g_S+dt^2\right)\), то локально конформно кэлерово многообразие \(M\) называется вайсмановым. Существует множество эквивалентных определений вайсмановых многообразий, и все они основаны на римановой геометрии, но оказывается, что компактные вайсмановы многообразия имеют алгебраическую природу.

Рассмотрим проективный орбифолд \(X\) и обильное линейное расслоение \(L\) на нём. Обозначим за \(\text{Tot}^\circ(L)\) пространство ненулевых векторов \(L\). Пусть \(\varphi\) --- автоморфизм \((X,L)\), удовлетворяющий условию \(\forall v\in L  \ \  |\varphi (v)|=\lambda |v|\) с фиксированной константой \(\lambda>1\). Тогда \(\text{Tot}^\circ(L)/\varphi\) --- компактное вайсманово многообразие. Более того, любое компактное вайсманово многообразие получается таким образом.
\end{talk}
\end{document}