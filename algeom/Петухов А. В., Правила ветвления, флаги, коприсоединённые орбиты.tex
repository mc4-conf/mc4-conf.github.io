\documentclass[12pt]{article}
\usepackage{hyphsubst}
\usepackage[T2A]{fontenc}
\usepackage[english,main=russian]{babel}
\usepackage[utf8]{inputenc}
\usepackage[letterpaper,top=2cm,bottom=2cm,left=2cm,right=2cm,marginparwidth=2cm]{geometry}
\usepackage{float}
\usepackage{mathtools, commath, amssymb, amsthm}
\usepackage{enumitem, tabularx,graphicx,url,xcolor,rotating,multicol,epsfig,colortbl,lipsum}

\setlist{topsep=1pt, itemsep=0em}
\setlength{\parindent}{0pt}
\setlength{\parskip}{6pt}

\usepackage{hyphenat}
\hyphenation{ма-те-ма-ти-ка вос-ста-нав-ли-вать}

\usepackage[math]{anttor}

\newenvironment{talk}[6]{%
\vskip 0pt\nopagebreak%
\vskip 0pt\nopagebreak%
\section*{#1}
\phantomsection
\addcontentsline{toc}{section}{#2. \textit{#1}}
% \addtocontents{toc}{\textit{#1}\par}
\textit{#2}\\\nopagebreak%
#3\\\nopagebreak%
\ifthenelse{\equal{#4}{}}{}{\url{#4}\\\nopagebreak}%
\ifthenelse{\equal{#5}{}}{}{Соавторы: #5\\\nopagebreak}%
\ifthenelse{\equal{#6}{}}{}{Секция: #6\\\nopagebreak}%
}

\definecolor{LovelyBrown}{HTML}{FDFCF5}

\usepackage[pdftex,
breaklinks=true,
bookmarksnumbered=true,
linktocpage=true,
linktoc=all
]{hyperref}

\begin{document}
\pagenumbering{gobble}
\pagestyle{plain}
\pagecolor{LovelyBrown}
\begin{talk}
{Правила ветвления, флаги, коприсоединённые орбиты}
{Петухов Алексей Владимирович}
{ИППИ РАН имени А.\,А. Харкевича}
{alex--2@yandex.ru}
{Р.\,С. Авдеев}
{Алгебраическая Геометрия}

Пусть \(G\) --- некоторая, редуктивная группа над \(\mathbb C\), а \(H\) --- её редуктивная
подгруппа. Любое конечномерное представление \(G\) является суммой неприводимых \(G\)-
представлений, а ограничение конечномерного неприводимого \(G\)-представления \(V\) на
\(H\) (обозначаемое \(V|_H\)) является прямой суммой неприводимых \(H\)-представлений.
Параметры таких разложений \(V|_H\) называются {\it правилами ветвления}. Мы будем
говорить, что \(G\)-модуль \(V\) имеет {\it простой \(H\)-спектр}, если в ограничении \(V|_H\)
неприводимые \(H\)-слагаемые не повторяются. Описание правил ветвления, в том числе
ветвлений с простым спектром, является интересной и важной задачей.
В своём докладе я хотел поговорить о том, как, используя теорему Бореля--Вейля,
перефразировать вопрос об описании классов неприводимых \(G\)-представлений с простым
\(H\)-спектром, связанных с \(G\)-флагами, в вопрос об описании \(H\)-сферических действий
на многообразиях \(G\)-флагов. Используя этот подход, мы с Романом Авдеевым получили
описание таких серий правил ветвления в терминах некоторых свободных решёток,
образующие которых соответствуют дивизорам в подходящем многообразии \(G\)-флагов,
являющимся стабильными относительно действия некоторой Борелевской подгруппы группы
\(H\). Такие классы \(G\)-представлений с простым \(H\)-спектром соответствуют \(H\)-сферическим действиям на многообразиях \(G\)-флагов. В конце доклада я хочу обсудить
идею класификации таких наборов (\(G\), \(G\)-флаги; \(H\)-сферическое действие), используя
идеи связанные с симплектической геометрией, отображением моментов,
коприсоединёнными орбитами, а также какие-то простые факты о сферических
многообразиях.

\medskip

\begin{enumerate}
\item[{[1]}] Р. С. Авдеев, А. В. Петухов, {\it Сферические действия на
многообразиях флагов}, Матем. сб. 205 (2014), No 9, 3–48; arXiv:1401.1777.
\item[{[2]}] R. Avdeev, A. Petukhov, {\it Branching rules related to spherical actions on flag
varieties}, Algebr. Represent. Theory 23 (2020), no. 3, 541--581; arXiv:1711.09801.
\item[{[3]}] A. V. Petukhov, {\it Bounded reductive subalgebras of sl(n)}, Transform. Groups
16 (2011), no. 4, 1173--1182; arXiv:1007.1338.
\end{enumerate}
\end{talk}
\end{document}