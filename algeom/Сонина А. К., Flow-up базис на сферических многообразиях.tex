\documentclass[12pt]{article}
\usepackage{hyphsubst}
\usepackage[T2A]{fontenc}
\usepackage[english,main=russian]{babel}
\usepackage[utf8]{inputenc}
\usepackage[letterpaper,top=2cm,bottom=2cm,left=2cm,right=2cm,marginparwidth=2cm]{geometry}
\usepackage{float}
\usepackage{mathtools, commath, amssymb, amsthm}
\usepackage{enumitem, tabularx,graphicx,url,xcolor,rotating,multicol,epsfig,colortbl,lipsum}

\setlist{topsep=1pt, itemsep=0em}
\setlength{\parindent}{0pt}
\setlength{\parskip}{6pt}

\usepackage{hyphenat}
\hyphenation{ма-те-ма-ти-ка вос-ста-нав-ли-вать}

\usepackage[math]{anttor}

\newenvironment{talk}[6]{%
\vskip 0pt\nopagebreak%
\vskip 0pt\nopagebreak%
\section*{#1}
\phantomsection
\addcontentsline{toc}{section}{#2. \textit{#1}}
% \addtocontents{toc}{\textit{#1}\par}
\textit{#2}\\\nopagebreak%
#3\\\nopagebreak%
\ifthenelse{\equal{#4}{}}{}{\url{#4}\\\nopagebreak}%
\ifthenelse{\equal{#5}{}}{}{Соавторы: #5\\\nopagebreak}%
\ifthenelse{\equal{#6}{}}{}{Секция: #6\\\nopagebreak}%
}

\definecolor{LovelyBrown}{HTML}{FDFCF5}

\usepackage[pdftex,
breaklinks=true,
bookmarksnumbered=true,
linktocpage=true,
linktoc=all
]{hyperref}

\begin{document}
\pagenumbering{gobble}
\pagestyle{plain}
\pagecolor{LovelyBrown}
\begin{talk}
{Flow-up базис на сферических многообразиях}
{Сонина Александра Константиновна }
{ПОМИ РАН}
{sasha-sonina@mail.ru}
{Виктор Петров}
{Алгебраическая геометрия}

По многообразию с действием редуктивной группы можно построить GKM-граф: вершины будут неподвижными точками под действием максимального тора редуктивной группы $G$, ребра будут получатся из инвариантных кривых (на самом деле все эти кривые будут изоморфны $\mathbb{P}^1$). Вместе с каждой инвариантной кривой у нас также будет появляться характер --- его будем записывать как метку на ребре. Также на каждом ребре будет задаваться ориентация, вместе с которой появится частичный порядок на вершинах графа.

В вершинах графа будем записывать многочлены от характеров тора. Такая расстановка правильная если разность двух многочленов в вершинах делится на характер на ребре, а также выполняются некоторые квадратичные соотношения.

Будем называть систему правильных расстановок многочленов {\itshape flow-up базисом}, если

\begin{itemize}
\item [1] Эта система расстановок --- базис (т.е. эта система линейно независима и любая правильная расстановка является линейной комбинацией данных с коэффициентами --- полиномами от корней системы $\Phi$)
\item [2] Для всех вершин $w$ существует элемент системы расстановок такой, что в этой расстановке в вершине $w$ стоит произведение меток ребер, выходящих из этой вершины, а ненулевые элементы могут стоять только в вершинах $v$ с $v\geq w$.
\end{itemize}

Благодаря локализации Ботта можно построить инъективное отображение из $CH^*_T(X)$ в множество всех правильных расстановок.

Пусть $G$ --- редуктивная группа и $B \subset G$ --- Борелевская подгруппа. $G-$многообразие $X$ будем называть {\itshape сферическим}, если в $X$ есть плотная $B-$орбита.

В нашей работе мы показали, что у любого сферического многообразия существует flow-up базис, в частности отображение из $CH^*_T(X) \otimes \mathbb{Q}$ в правильные расстановки над $\mathbb{Q}$ это изоморфизм.

\medskip

\begin{enumerate}
\item[{[1]}] Henry July {\itshape Algebraic cobodism of spherical varieties} Thèse de doctorat
\item[{[2]}] V.Guillemin and C.Zara {\itshape One-skeleta betti numbers and equivarient cohomology} \\ arXiv:math/9903051v2 [math.DG] 26 Jul 2000
 \item[{[3]}] M. Brion. {\itshape Equivariant Chow groups for torus actions.} Transformation Groups, Vol. 2, No. 3, 1997, pp. 225-267.
\end{enumerate}
\end{talk}
\end{document}