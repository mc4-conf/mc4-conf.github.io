\documentclass[12pt]{article}
\usepackage{hyphsubst}
\usepackage[T2A]{fontenc}
\usepackage[english,main=russian]{babel}
\usepackage[utf8]{inputenc}
\usepackage[letterpaper,top=2cm,bottom=2cm,left=2cm,right=2cm,marginparwidth=2cm]{geometry}
\usepackage{float}
\usepackage{mathtools, commath, amssymb, amsthm}
\usepackage{enumitem, tabularx,graphicx,url,xcolor,rotating,multicol,epsfig,colortbl,lipsum}

\setlist{topsep=1pt, itemsep=0em}
\setlength{\parindent}{0pt}
\setlength{\parskip}{6pt}

\usepackage{hyphenat}
\hyphenation{ма-те-ма-ти-ка вос-ста-нав-ли-вать}

\usepackage[math]{anttor}

\newenvironment{talk}[6]{%
\vskip 0pt\nopagebreak%
\vskip 0pt\nopagebreak%
\section*{#1}
\phantomsection
\addcontentsline{toc}{section}{#2. \textit{#1}}
% \addtocontents{toc}{\textit{#1}\par}
\textit{#2}\\\nopagebreak%
#3\\\nopagebreak%
\ifthenelse{\equal{#4}{}}{}{\url{#4}\\\nopagebreak}%
\ifthenelse{\equal{#5}{}}{}{Соавторы: #5\\\nopagebreak}%
\ifthenelse{\equal{#6}{}}{}{Секция: #6\\\nopagebreak}%
}

\definecolor{LovelyBrown}{HTML}{FDFCF5}

\usepackage[pdftex,
breaklinks=true,
bookmarksnumbered=true,
linktocpage=true,
linktoc=all
]{hyperref}

\begin{document}
\pagenumbering{gobble}
\pagestyle{plain}
\pagecolor{LovelyBrown}
\begin{talk}
{Короткие \(SL_2\)-структуры на алгебрах Ли и лиевских модулях}
{Стасенко Роман Олегович}
{НИУ ВШЭ, Московский центр фундаментальной и прикладной математики}
{theromestasenko@yandex.ru}
{}
{Алгебраическая геометрия}

Пусть \(S\) --- произвольная редуктивная алгебраическая группа. Назовем \(S\)-структурой на алгебре Ли \(\mathfrak{g}\)  гомоморфизм \(\Phi:S\rightarrow \operatorname{Aut}(\mathfrak{g})\). \(S\)-структуры ранее излучались различными авторами, в том числе Э.\,Б. Винбергом.

В докладе рассматриваются \(SL_2\)-структуры. \(SL_2\)-структуру назовем короткой, если представление \(\Phi\) группы \(SL_2\) разлагается на неприводимые представления размерностей 1, 2 и 3.
Если рассматривать неприводимые представления размерностей только 1 и 3, то получится известная конструкция Титса-Кантора-Кехера, устанавливающая взаимно-однозначное соответствие между  простыми йордановыми алгебрами и простыми алгебрами Ли определенного вида.

Аналогично теореме Титса–Кантора–Кехера в случае коротких \(SL_2\)-структур можно установить взаимно-однозначное соответствие между простыми алгебрами Ли с такой структурой и так называемыми простыми симплектическими структурами Ли-Йордана.

Пусть на алгебре Ли \(\mathfrak{g}\) задана \(SL_2\)-структура и отображение \(\rho:\mathfrak{g}\rightarrow\mathfrak{gl}(U)\) --- линейное представление. Гомомофизм \(\Psi:S\rightarrow GL(U)\) называется  \(SL_2\)-структурой на лиевском \(\mathfrak{g}\)-модуле \(U\), если
\[\Psi(s)\rho(\xi)u = \rho(\Phi(s)\xi)\Psi(s)u,\quad\forall s\in S, \xi\in\mathfrak{g}, u\in U.\]
Подобная конструкция имеет интересные приложения к теории представлений йордановых алгебр, о которых будет рассказано в докладе. Также в докладе будет представлена полная классификация неприводимых коротких \(\mathfrak{g}\)-модулей для простых алгебр Ли.
\end{talk}
\end{document}