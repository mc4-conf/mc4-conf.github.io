\documentclass[12pt]{article}
\usepackage{hyphsubst}
\usepackage[T2A]{fontenc}
\usepackage[english,main=russian]{babel}
\usepackage[utf8]{inputenc}
\usepackage[letterpaper,top=2cm,bottom=2cm,left=2cm,right=2cm,marginparwidth=2cm]{geometry}
\usepackage{float}
\usepackage{mathtools, commath, amssymb, amsthm}
\usepackage{enumitem, tabularx,graphicx,url,xcolor,rotating,multicol,epsfig,colortbl,lipsum}

\setlist{topsep=1pt, itemsep=0em}
\setlength{\parindent}{0pt}
\setlength{\parskip}{6pt}

\usepackage{hyphenat}
\hyphenation{ма-те-ма-ти-ка вос-ста-нав-ли-вать}

\usepackage[math]{anttor}

\newenvironment{talk}[6]{%
\vskip 0pt\nopagebreak%
\vskip 0pt\nopagebreak%
\section*{#1}
\phantomsection
\addcontentsline{toc}{section}{#2. \textit{#1}}
% \addtocontents{toc}{\textit{#1}\par}
\textit{#2}\\\nopagebreak%
#3\\\nopagebreak%
\ifthenelse{\equal{#4}{}}{}{\url{#4}\\\nopagebreak}%
\ifthenelse{\equal{#5}{}}{}{Соавторы: #5\\\nopagebreak}%
\ifthenelse{\equal{#6}{}}{}{Секция: #6\\\nopagebreak}%
}

\definecolor{LovelyBrown}{HTML}{FDFCF5}

\usepackage[pdftex,
breaklinks=true,
bookmarksnumbered=true,
linktocpage=true,
linktoc=all
]{hyperref}

\begin{document}
\pagenumbering{gobble}
\pagestyle{plain}
\pagecolor{LovelyBrown}
\begin{talk}
{Специальная геометрия Бора--Зоммерфельда}
{Тюрин Николай Андреевич}
{ЛТФ ОИЯИ (Дубна)}
{}
{}
{Алгебраическая геометрия}

Ю.И. Манин когда-то определил Зеркальную Симметрию как некоторую двойственность между комплексной и симплектической геометриями кэлеровых многообразий. Каждое кэлерово многообразие по самому своему определению обладает обеими ``природами'' --- комплексной и симплектической, и каждая из них порождает свой ``внутренний мир'': комплексные подмногообразия и голоморфные расслоения в первом случае и лагранжевы подмногообразия во втором. Каждый из этих внутренних миров определяет дополнительный набор инвариантов, и кэлеровы многообразия понимаются как зеркальные партнеры если  их инварианты зеркально соотносятся друг ко другу.

Однако две эти природы в сущности очень разные: комплексная предполагает конечномерность вариаций  и деформаций объектов (комплексных подмногообразий, голоморфных расслоений etc),
в то время как симплектическая является очень гибкой, и любое лагранжево подмногообразие допускает континуальное пространство деформаций. Поэтому естественным образом возникает задача
введения подходящих условий, позволяющих получать некоторые конечномерные пространства модулей. С 90-ых годов очень популярной конструкцией является специальная лагранжева геометрия,
предложенная Н. Хитчином. Однако рамки применения этой геометрии достаточно узки: условие SpLag можно вводить только на многообразиях Калаби--Яу.

Специальная геометрия Бора--Зоммерфельда возникла как программа, позволяющая по произвольному компактному односвязному алгебраическому многообразию \(X\)
построить конечномерные многообразия модулей, элементами которых являются классы подмногообразий, лагранжевых относительно кэлеровой формы метрики Ходжа.
Исходными для построения таких многообразий, см. [1], являются стандартные в Геометрическом квантовании данные: расслоение и связность предквантования \((L, a)\). Далее мы используем программу ALAG, предложенную в 1999 году А.\,Н. Тюриным и А.\,Л. Городенцевым, которая строит бесконечномерное многообразие модулей \({\cal B}_S\) лагранжевых бор --- зоммерфельдовых подмногообразий фиксированного топологического типа. Затем в прямом произведении \(\mathbb{P}(\Gamma(M, L)) \times {\cal B}_S\) строится цикл инциденции \({\cal U}_{SBS}\) пар, удовлетворяющих
условию специальности. Оказывается, что каноническая проекция \(q: {\cal U}_{SBS} \to \mathbb{P}(\Gamma(X, L))\) имеет дискретные слои, откуда получаем первое грубое определение
конечномерного многообразия модулей: поскольку в пространстве \(\Gamma(X, L)\) всех гладких сечений расслоения предквантования \(L\) имеется конечномерное подпространство \(H^0(X, L)\)
голоморфных сечений, то возможно взять прообраз \(q^{-1}(\mathbb{P} (H^0(X, L))\), что должно дать конечномерный объект. Однако такое прямолинейное определение не приводит
к разумному результату: как было установлено в [2], для голоморфного сечения \(\alpha \in H^0(X,L)\) лагранжево подмногообразие \(S \subset X\) является специальным если и только если
оно содержится в скелете Вейнстейна \(W(X \backslash D_{\alpha})\) дополнения к дивизору нулей \(D_{\alpha} = \{ \alpha = 0 \} \subset X\). Но как было установлено вейнстейнов скелет
почти всегда не содержит гладких компонент, так что прямолинейное определение приводит к тривиальному результату. Проблема разрешается вариацией параметров, имеющихся в нашем определении:
в работе [3] было предложено использовать деформацию связности предквантования \(a\) для определения многообразия модулей \({\cal M}_{SBS}\) специальных бор - зоммерфельдовых лагранжевых подмногообразий --- конечномерный объект в лагранжевой геометрии алгебраических многообразий. Как было доказано, это многообразие модулей обладает комплексной структурой, а в построенных примерах было обнаружно, что оно может иметь вид ``алгебраическое многообразие минус обильный дивизор'', откуда возникла естественная гипотеза о том, что \({\cal M}_{SBS}\) алгебраично
всегда.

\medskip

\begin{enumerate}
\item[{[1]}] Н. А. Тюрин, {\it Специальные бор–зоммерфельдовы лагранжевы подмногообразия}, Изв. РАН. Сер. матем., 80:6 (2016),  274–293;
\item [{[2]}] Н. А. Тюрин, {\it Специальные бор–зоммерфельдовы лагранжевы подмногообразия в алгебраических многообразиях}, Изв. РАН. Сер. матем., 82:3 (2018),  170–191;
\item [{[3]}] Н. А. Тюрин, {\it Специальная геометрия Бора–Зоммерфельда: вариации}, Изв. РАН. Сер. матем., 87:3 (2023),  184–205.
\end{enumerate}
\end{talk}
\end{document}