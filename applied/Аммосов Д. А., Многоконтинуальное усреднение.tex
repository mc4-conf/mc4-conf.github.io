\documentclass[12pt]{article}
\usepackage{hyphsubst}
\usepackage[T2A]{fontenc}
\usepackage[english,main=russian]{babel}
\usepackage[utf8]{inputenc}
\usepackage[letterpaper,top=2cm,bottom=2cm,left=2cm,right=2cm,marginparwidth=2cm]{geometry}
\usepackage{float}
\usepackage{mathtools, commath, amssymb, amsthm}
\usepackage{enumitem, tabularx,graphicx,url,xcolor,rotating,multicol,epsfig,colortbl,lipsum}

\setlist{topsep=1pt, itemsep=0em}
\setlength{\parindent}{0pt}
\setlength{\parskip}{6pt}

\usepackage{hyphenat}
\hyphenation{ма-те-ма-ти-ка вос-ста-нав-ли-вать}

\usepackage[math]{anttor}

\newenvironment{talk}[6]{%
\vskip 0pt\nopagebreak%
\vskip 0pt\nopagebreak%
\section*{#1}
\phantomsection
\addcontentsline{toc}{section}{#2. \textit{#1}}
% \addtocontents{toc}{\textit{#1}\par}
\textit{#2}\\\nopagebreak%
#3\\\nopagebreak%
\ifthenelse{\equal{#4}{}}{}{\url{#4}\\\nopagebreak}%
\ifthenelse{\equal{#5}{}}{}{Соавторы: #5\\\nopagebreak}%
\ifthenelse{\equal{#6}{}}{}{Секция: #6\\\nopagebreak}%
}

\definecolor{LovelyBrown}{HTML}{FDFCF5}

\usepackage[pdftex,
breaklinks=true,
bookmarksnumbered=true,
linktocpage=true,
linktoc=all
]{hyperref}

\begin{document}
\pagenumbering{gobble}
\pagestyle{plain}
\pagecolor{LovelyBrown}
\begin{talk}
{Многоконтинуальное усреднение}
{Аммосов Дмитрий Андреевич}
{Институт математики и информатики, Северо-Восточный федеральный университет им. М.\,К. Аммосова}
{dmitryammosov@gmail.com}
{}
{Прикладная математика и математическое моделирование}

Многие прикладные задачи являются многомасштабными по своей природе, с неоднородностями на различных масштабах и высоким контрастом. Для решения таких задач часто применяют методы усреднения, в которых вычисляются эффективные свойства в каждой макромасштабной точке. Однако в ряде случаев, особенно при высоком контрасте, этого оказывается недостаточно, поскольку требуется больше усредненных коэффициентов для точного моделирования.

В данном докладе рассматривается применение нового метода многоконтинуального усреднения для задач с высоким контрастом и без разделения масштабов. Данный метод представляет собой гибкий и в то же время строгий подход для вывода многоконтинуальных моделей. Основная идея метода заключается в построении специальных задач на ячейках в расширенных представительных элементах с ограничениями для учета различных эффектов. Решение задач на ячейках позволяет получить разложение искомой функции по континуумам, а затем с помощью некоторых операций вычислить эффективные свойства и вывести соответствующую многоконтинуальную модель.
\end{talk}
\end{document}