\documentclass[12pt]{article}
\usepackage{hyphsubst}
\usepackage[T2A]{fontenc}
\usepackage[english,main=russian]{babel}
\usepackage[utf8]{inputenc}
\usepackage[letterpaper,top=2cm,bottom=2cm,left=2cm,right=2cm,marginparwidth=2cm]{geometry}
\usepackage{float}
\usepackage{mathtools, commath, amssymb, amsthm}
\usepackage{enumitem, tabularx,graphicx,url,xcolor,rotating,multicol,epsfig,colortbl,lipsum}

\setlist{topsep=1pt, itemsep=0em}
\setlength{\parindent}{0pt}
\setlength{\parskip}{6pt}

\usepackage{hyphenat}
\hyphenation{ма-те-ма-ти-ка вос-ста-нав-ли-вать}

\usepackage[math]{anttor}

\newenvironment{talk}[6]{%
\vskip 0pt\nopagebreak%
\vskip 0pt\nopagebreak%
\section*{#1}
\phantomsection
\addcontentsline{toc}{section}{#2. \textit{#1}}
% \addtocontents{toc}{\textit{#1}\par}
\textit{#2}\\\nopagebreak%
#3\\\nopagebreak%
\ifthenelse{\equal{#4}{}}{}{\url{#4}\\\nopagebreak}%
\ifthenelse{\equal{#5}{}}{}{Соавторы: #5\\\nopagebreak}%
\ifthenelse{\equal{#6}{}}{}{Секция: #6\\\nopagebreak}%
}

\definecolor{LovelyBrown}{HTML}{FDFCF5}

\usepackage[pdftex,
breaklinks=true,
bookmarksnumbered=true,
linktocpage=true,
linktoc=all
]{hyperref}

\begin{document}
\pagenumbering{gobble}
\pagestyle{plain}
\pagecolor{LovelyBrown}
\begin{talk}
{Моделирование воздействия вихревых структур на сверхзвуковое обтекания крыла}
{Борисов Виталий Евгеньевич}
{ИПМ им. М.\,В. Келдыша РАН}
{borisov@keldysh.ru}
{Константиновская Т.\,В., Луцкий А.\,Е.}
{Прикладная математика и математическое моделирование}

В работе представлены результаты численного исследования сверхзвукового обтекания тандема крыльев (пара: крыло--генератор и основное крыло) с углом атаки 20\(^\circ\), актуальность которого связана, в частности, со сложностями проведения натурных экспериментов [1,2]. Расчеты проводились для прямоугольных в плане крыльев с острыми кромками и ромбовидным основанием. Рассматривались две конфигурации тандема, отличающиеся полуразмахом первого по потоку крыла-генератора, которое составляло половину либо равнялось полуразмаху основного крыла. Для численного расчета использовалась система осредненных по Рейнольдсу и Фавру нестационарных уравнений Навье--Стокса (URANS) с моделью турбулентности Спаларта--Аллмараса. Расчеты проводились на гибридной суперкомпьютерной системе К-60 [3] с помощью авторского программного комплекса ARES для расчета трехмерных турбулентных течений вязкого сжимаемого газа. Показано развитие взаимодействия вихревых структур в зависимости от полуразмаха крыла-генератора, а также изменение зоны обратного течения на подветренной стороне основного крыла. Проведено сравнение с результатами обтекания крыла в невозмущенном набегающем потоке.

\medskip

\begin{enumerate}
\item[{[1]}] Гайфуллин А.М. Вихревые течения. --М.: Наука, 2015. 319 с.
\item[{[2]}] Borisov~V.E., Davydov~A.A., Konstantinovskaya~T.V., Lutsky~A.E., Shevchenko~A.M., Shmakov~A.S. Numerical and experimental investigation of a supersonic vortex wake at a wide distance from the wing // AIP Conference Proceedings. 2018. 2027, 030120.
\item[{[3]}] Вычислительный комплекс K-60. [Электронный ресурс]. \\ URL: https://www.kiam.ru/MVS/resourses/k60.html
\end{enumerate}
\end{talk}
\end{document}