\documentclass[12pt]{article}
\usepackage{hyphsubst}
\usepackage[T2A]{fontenc}
\usepackage[english,main=russian]{babel}
\usepackage[utf8]{inputenc}
\usepackage[letterpaper,top=2cm,bottom=2cm,left=2cm,right=2cm,marginparwidth=2cm]{geometry}
\usepackage{float}
\usepackage{mathtools, commath, amssymb, amsthm}
\usepackage{enumitem, tabularx,graphicx,url,xcolor,rotating,multicol,epsfig,colortbl,lipsum}

\setlist{topsep=1pt, itemsep=0em}
\setlength{\parindent}{0pt}
\setlength{\parskip}{6pt}

\usepackage{hyphenat}
\hyphenation{ма-те-ма-ти-ка вос-ста-нав-ли-вать}

\usepackage[math]{anttor}

\newenvironment{talk}[6]{%
\vskip 0pt\nopagebreak%
\vskip 0pt\nopagebreak%
\section*{#1}
\phantomsection
\addcontentsline{toc}{section}{#2. \textit{#1}}
% \addtocontents{toc}{\textit{#1}\par}
\textit{#2}\\\nopagebreak%
#3\\\nopagebreak%
\ifthenelse{\equal{#4}{}}{}{\url{#4}\\\nopagebreak}%
\ifthenelse{\equal{#5}{}}{}{Соавторы: #5\\\nopagebreak}%
\ifthenelse{\equal{#6}{}}{}{Секция: #6\\\nopagebreak}%
}

\definecolor{LovelyBrown}{HTML}{FDFCF5}

\usepackage[pdftex,
breaklinks=true,
bookmarksnumbered=true,
linktocpage=true,
linktoc=all
]{hyperref}

\begin{document}
\pagenumbering{gobble}
\pagestyle{plain}
\pagecolor{LovelyBrown}
\begin{talk}
{Cтруктура внутренних волн в расчетах движения неоднородной
жидкости с использованием численной
модели ROMS}
{Володько Ольга Станиславовна}
{Институт вычислительного моделирования СО РАН, обособленное подразделение ФИЦ КНЦ СО РАН}
{olga.pitalskaya@gmail.com}
{Мальцев Е.\,Д.}
{Прикладная математика и математическое моделирование}

Течения и внутренние волны в озерах в основном вызываются ветровыми воздействиями. Понимание пространственной структуры обеспечивает основу для понимания последующих физических, химических и биологических процессов.
Но, как правило, натурные измерения гидрофизических характеристик (скорости течения, температуры и солености воды), могут быть проведены только в нескольких конкретных точках. При проведении численных расчетов мы имеем значения гидрофизических характеристик в каждой точке разностной сетки и можем на основании этих данных определить горизонтальную структуру внутренних волн. В настоящей работе на основе данных численных расчетов, полученных с использованием численной модели ROMS (Regional Oceanic Modeling System), определены время возникновения внутренних волн в зависимости от направления и силы ветра, характер изменения возвышения свободной поверхности и изоповерхностей температуры. Для интерпретации полученных в расчетах результатов был проведен переход от \(\sigma\)-координат к декартовым, что позволило идентифицировать наиболее длинные волны как одноузловые сейши. С применением линейной модели трехмерного течения двухслойной жидкости проведена оценка длины вращающейся сейши.
\end{talk}
\end{document}