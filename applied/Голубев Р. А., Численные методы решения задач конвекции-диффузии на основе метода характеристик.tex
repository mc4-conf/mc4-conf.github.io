\documentclass[12pt]{article}
\usepackage{hyphsubst}
\usepackage[T2A]{fontenc}
\usepackage[english,main=russian]{babel}
\usepackage[utf8]{inputenc}
\usepackage[letterpaper,top=2cm,bottom=2cm,left=2cm,right=2cm,marginparwidth=2cm]{geometry}
\usepackage{float}
\usepackage{mathtools, commath, amssymb, amsthm}
\usepackage{enumitem, tabularx,graphicx,url,xcolor,rotating,multicol,epsfig,colortbl,lipsum}

\setlist{topsep=1pt, itemsep=0em}
\setlength{\parindent}{0pt}
\setlength{\parskip}{6pt}

\usepackage{hyphenat}
\hyphenation{ма-те-ма-ти-ка вос-ста-нав-ли-вать}

\usepackage[math]{anttor}

\newenvironment{talk}[6]{%
\vskip 0pt\nopagebreak%
\vskip 0pt\nopagebreak%
\section*{#1}
\phantomsection
\addcontentsline{toc}{section}{#2. \textit{#1}}
% \addtocontents{toc}{\textit{#1}\par}
\textit{#2}\\\nopagebreak%
#3\\\nopagebreak%
\ifthenelse{\equal{#4}{}}{}{\url{#4}\\\nopagebreak}%
\ifthenelse{\equal{#5}{}}{}{Соавторы: #5\\\nopagebreak}%
\ifthenelse{\equal{#6}{}}{}{Секция: #6\\\nopagebreak}%
}

\definecolor{LovelyBrown}{HTML}{FDFCF5}

\usepackage[pdftex,
breaklinks=true,
bookmarksnumbered=true,
linktocpage=true,
linktoc=all
]{hyperref}

\begin{document}
\pagenumbering{gobble}
\pagestyle{plain}
\pagecolor{LovelyBrown}
\begin{talk}
{Численные методы решения задач конвекции-диф\-фу\-зии на основе метода характеристик}
{Голубев Роман Андреевич}
{Институт вычислительного моделирования СО РАН}
{rgolubev@icm.krasn.ru}
{Шайдуров Владимир Викторович}
{Прикладная математика и математическое моделирования}

В работе предлагаются разностные схемы для решения одномерного и двумерного уравнения конвекции-диффузии с оператором конвекции в недивергентной форме. Здесь в качестве материальной производной по направлению "потока" принят оператор переноса, для аппроксимации которого используется метод характеристик. Для эллиптической же части применяются методы конечных разностей. Для построения разностных схем рассматриваются два подхода, называемые полулагранжевыми: эйлерово-лагранжев и лагранжево-эйлеров. Первый из них реализуется на равномерной пространственно-временной сетке. Второй, лагранжево-эйлеров, реализуется на неравномерной пространственной сетке, получаемой путем пересечения характеристических кривых, выпущенных из равномерно расположенных узлов в
начальный момент времени, с равномерно расположенными временными слоями.
Для построенных разностных схем обоснованы порядки точности, а также проведены вычислительные эксперименты.

\medskip

Работа поддержана Красноярским математическим центром, финансируемым Минобрнауки РФ в рамках мероприятий по созданию и развитию региональных НОМЦ (Соглашение 075-02-2024-1378).
\end{talk}
\end{document}