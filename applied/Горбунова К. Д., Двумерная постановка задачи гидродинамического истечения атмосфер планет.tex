\documentclass[12pt]{article}
\usepackage{hyphsubst}
\usepackage[T2A]{fontenc}
\usepackage[english,main=russian]{babel}
\usepackage[utf8]{inputenc}
\usepackage[letterpaper,top=2cm,bottom=2cm,left=2cm,right=2cm,marginparwidth=2cm]{geometry}
\usepackage{float}
\usepackage{mathtools, commath, amssymb, amsthm}
\usepackage{enumitem, tabularx,graphicx,url,xcolor,rotating,multicol,epsfig,colortbl,lipsum}

\setlist{topsep=1pt, itemsep=0em}
\setlength{\parindent}{0pt}
\setlength{\parskip}{6pt}

\usepackage{hyphenat}
\hyphenation{ма-те-ма-ти-ка вос-ста-нав-ли-вать}

\usepackage[math]{anttor}

\newenvironment{talk}[6]{%
\vskip 0pt\nopagebreak%
\vskip 0pt\nopagebreak%
\section*{#1}
\phantomsection
\addcontentsline{toc}{section}{#2. \textit{#1}}
% \addtocontents{toc}{\textit{#1}\par}
\textit{#2}\\\nopagebreak%
#3\\\nopagebreak%
\ifthenelse{\equal{#4}{}}{}{\url{#4}\\\nopagebreak}%
\ifthenelse{\equal{#5}{}}{}{Соавторы: #5\\\nopagebreak}%
\ifthenelse{\equal{#6}{}}{}{Секция: #6\\\nopagebreak}%
}

\definecolor{LovelyBrown}{HTML}{FDFCF5}

\usepackage[pdftex,
breaklinks=true,
bookmarksnumbered=true,
linktocpage=true,
linktoc=all
]{hyperref}

\begin{document}
\pagenumbering{gobble}
\pagestyle{plain}
\pagecolor{LovelyBrown}
\begin{talk}
{Двумерная постановка задачи гидродинамического истечения атмосфер планет}
{Горбунова Ксения Дмитриевна}
{Институт вычислительного моделирования СО РАН}
{gorbunova.kd@icm.krasn.ru}
{Н.\,В. Еркаев}
{Прикладная математика и математическое моделирование}

Рассмотрена двумерная задача о нестационарном истечении верхних слоев атмосферы планеты в результате нагрева жестким ультрафиолетовым излучением Звезды. В отличие от одномерной постановки, учитывается особенность распространения и поглощения ультрафиолетового излучения и добавляется меридиональная составляющая скорости, которая становится больше с увеличением угла отклонения от центральной оси, направленной на родительскую звезду.

Для расчетов использовались физические характеристики теплого мини-Нептуна TOI-421c и его родительской звезды~[1], компактная схема типа Мак-Кормака и метод Рунге-Кутты четвертого порядка~[2]. Сравнение с одномерными результатами, полученными ранее~[3,4], показало, что они значительно завышают общий расход газа, в связи с этим был подобран параметр для одномерной постановки, позволяющий получить более реалистичную оценку потери массы атмосферы.

\medskip

Работа поддержана Красноярским математическим центром, финансируемым Минобрнауки РФ в рамках мероприятий по созданию и развитию региональных НОМЦ (Соглашение 075-02-2024-1378).

\begin{enumerate}
\item[{[1]}] Carleo, I., Gandolfi, D., et al., {\it The multiplanet system TOI-421: A warm Neptune and a super puffy Mini-Neptune transiting a G9 V star in a visual binary}, The Astronomical Journal, 160 (2020), 114--137.
\item[{[2]}] JavanNezhad, R., Meshkatee, A.H., et al., {\it High-order compact MacCormack scheme for two-dimensional compressible and non-hydrostatic equations of the atmosphere}, Dynamics of Atmospheres and Oceans, 75 (2016), 102--117.
\item[{[3]}] Еркаев Н.В., Горбунова К.Д., {\it Компактная разностная схема для гидродинамической модели истечения атмосфер планет}, Вычислительные технологии, 29 № 1 (2024), 5--17.
\item[{[4]}] Erkaev, N.V., Gorbunova, K.D., {\it Magnetic Barrier in Front of Exoplanets Interacting with Stellar Wind}, Springer, 2022.
\end{enumerate}
\end{talk}
\end{document}