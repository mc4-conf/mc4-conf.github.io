\documentclass[12pt]{article}
\usepackage{hyphsubst}
\usepackage[T2A]{fontenc}
\usepackage[english,main=russian]{babel}
\usepackage[utf8]{inputenc}
\usepackage[letterpaper,top=2cm,bottom=2cm,left=2cm,right=2cm,marginparwidth=2cm]{geometry}
\usepackage{float}
\usepackage{mathtools, commath, amssymb, amsthm}
\usepackage{enumitem, tabularx,graphicx,url,xcolor,rotating,multicol,epsfig,colortbl,lipsum}

\setlist{topsep=1pt, itemsep=0em}
\setlength{\parindent}{0pt}
\setlength{\parskip}{6pt}

\usepackage{hyphenat}
\hyphenation{ма-те-ма-ти-ка вос-ста-нав-ли-вать}

\usepackage[math]{anttor}

\newenvironment{talk}[6]{%
\vskip 0pt\nopagebreak%
\vskip 0pt\nopagebreak%
\section*{#1}
\phantomsection
\addcontentsline{toc}{section}{#2. \textit{#1}}
% \addtocontents{toc}{\textit{#1}\par}
\textit{#2}\\\nopagebreak%
#3\\\nopagebreak%
\ifthenelse{\equal{#4}{}}{}{\url{#4}\\\nopagebreak}%
\ifthenelse{\equal{#5}{}}{}{Соавторы: #5\\\nopagebreak}%
\ifthenelse{\equal{#6}{}}{}{Секция: #6\\\nopagebreak}%
}

\definecolor{LovelyBrown}{HTML}{FDFCF5}

\usepackage[pdftex,
breaklinks=true,
bookmarksnumbered=true,
linktocpage=true,
linktoc=all
]{hyperref}

\begin{document}
\pagenumbering{gobble}
\pagestyle{plain}
\pagecolor{LovelyBrown}
\begin{talk}
{Сравнительный анализ численных методов решения обратной задачи определения источника акустических волн}
{Губер Алексей Владимирович}
{МЦМУ в Академгородке}
{alexej.guber@yandex.ru}
{Шишленин Максим Александрович}
{Прикладная математика и математическое моделирование}

В работе исследована обратная задача определения источника волн в двумерном случае.

Рассмотрим прямую задачу для уравнения акустики в области \(\Omega=\{(x,y): x\in(0,L_x), y\in(0,L_y) \}\):
\[u_{tt} = \text{div}{(c^2(x, y) \nabla u)},\qquad (x, y) \in\Omega, \quad t\in (0, T),\]
\[u|_{t = 0} = q(x, y), \qquad u_t|_{t = 0} = 0,\]
\[u|_{\partial \Omega} = 0.\]
Подобные задачи возникают во многих приложениях. Например, в задачах распространения волны цунами \(c(x,y)=\sqrt{gh(x, y)}\) --- скорость распространения волн, \(h(x,y)\) глубина океана, \(g=9.81\) м/с\(^2\) ускорение свободного падения [1, 2].

Обратная задача состоит в определении функции \(q(x, y)\) по дополнительной информации [3]:
\[u(x_n, y_n, t) = f_n (t), \qquad n=\overline{1, N}.\]
Здесь \((x_n, y_n)\) --- расположение приемников, \(N\) --- количество приёмников.

В операторной форме обратная задача формулируется в виде \(Aq = f\).

Проведён сравнительный анализ таких численных методов решения данной задачи, как матричный метод (с использованием Tensor-Train разложения), нейронные сети PINN, градиентный метод.

\medskip

Работа выполнена при поддержке Математического Центра в Академгородке, соглашение с Министерством науки и высшего образования Российской Федерации № 075-15-2022-281.

\begin{enumerate}
\item[{[1]}] В. М. Кайстренко. Обратная задача на определение источника цунами, Сб.: Волны цунами. Труды САХКНИИ, 1972. Вып.29.С.82-92.
\item[{[2]}] Воронина Т.А. Определение пространественнго распределения источников колебаний по дистанционным измерениям в конечном числе точек, СибЖВМ.2004.-Т.7, №3. С.203--211.
\item[{[3]}] М. А. Шишленин. Матричный метод в задачах определения источников колебаний, Сиб. электрон. матем. изв., 11 (2014), C.161--C.171.
\end{enumerate}
\end{talk}
\end{document}