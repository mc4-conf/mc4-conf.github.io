\documentclass[12pt]{article}
\usepackage{hyphsubst}
\usepackage[T2A]{fontenc}
\usepackage[english,main=russian]{babel}
\usepackage[utf8]{inputenc}
\usepackage[letterpaper,top=2cm,bottom=2cm,left=2cm,right=2cm,marginparwidth=2cm]{geometry}
\usepackage{float}
\usepackage{mathtools, commath, amssymb, amsthm}
\usepackage{enumitem, tabularx,graphicx,url,xcolor,rotating,multicol,epsfig,colortbl,lipsum}

\setlist{topsep=1pt, itemsep=0em}
\setlength{\parindent}{0pt}
\setlength{\parskip}{6pt}

\usepackage{hyphenat}
\hyphenation{ма-те-ма-ти-ка вос-ста-нав-ли-вать}

\usepackage[math]{anttor}

\newenvironment{talk}[6]{%
\vskip 0pt\nopagebreak%
\vskip 0pt\nopagebreak%
\section*{#1}
\phantomsection
\addcontentsline{toc}{section}{#2. \textit{#1}}
% \addtocontents{toc}{\textit{#1}\par}
\textit{#2}\\\nopagebreak%
#3\\\nopagebreak%
\ifthenelse{\equal{#4}{}}{}{\url{#4}\\\nopagebreak}%
\ifthenelse{\equal{#5}{}}{}{Соавторы: #5\\\nopagebreak}%
\ifthenelse{\equal{#6}{}}{}{Секция: #6\\\nopagebreak}%
}

\definecolor{LovelyBrown}{HTML}{FDFCF5}

\usepackage[pdftex,
breaklinks=true,
bookmarksnumbered=true,
linktocpage=true,
linktoc=all
]{hyperref}

\begin{document}
\pagenumbering{gobble}
\pagestyle{plain}
\pagecolor{LovelyBrown}
\begin{talk}
{Создание программного комплекса квантово-хи\-ми\-чес\-ких расчетов, обладающих повышенной точностью и быстродействием}
{Даньшин Артем Александрович}
{НИЦ ``Курчатовский институт''}
{danshin_aa@nrcki.ru}
{А.\,А. Ковалишин}
{Прикладная математика и математическое моделирование}

Существующие методы квантовой химии обладают большой вычислительной сложностью в большинстве приложений в химии, физике конденсированного состояния, биохимии и фармакологии, что ограничивает их область применимости. Поэтому необходимо развивать новые численные методы и математические модели, которые позволят на порядки ускорить вычисления, не теряя в точности. В докладе рассматриваются методы квантового Монте-Карло, Хартри-Фока, пост-Хартри-Фока и теории функционала плотности как с точки зрения численной реализации [1], так и методологических аспектов [2, 3]. Представленные результаты легли в основу программного комплекса, предназначенного для расчета структуры и свойств многоэлектронных систем.

\medskip

\begin{enumerate}
\item[{[1]}] A. A. Danshin, A. A. Kovalishin, {\it Operator Spectrum Transformation in Hartree–Fock and Kohn–Sham Equations}, Doklady Mathematics, 107 (2023), 17–20.
\item[{[2]}] A. A. Danshin, A. A. Kovalishin, M. I. Gurevich, {\it Approach to determine nodal surfaces of some \(s\)-electron systems}, Physical Review E, 108 (2023), 015305.
\item[{[3]}] A. A. Danshin, A. A. Kovalishin, {\it High-Performance Computing in Solving the Electron Correlation Problem}, Lecture Notes in Computer Science, 13708 (2022), 140-151.
\end{enumerate}
\end{talk}
\end{document}