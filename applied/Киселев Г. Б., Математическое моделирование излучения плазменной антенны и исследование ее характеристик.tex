\documentclass[12pt]{article}
\usepackage{hyphsubst}
\usepackage[T2A]{fontenc}
\usepackage[english,main=russian]{babel}
\usepackage[utf8]{inputenc}
\usepackage[letterpaper,top=2cm,bottom=2cm,left=2cm,right=2cm,marginparwidth=2cm]{geometry}
\usepackage{float}
\usepackage{mathtools, commath, amssymb, amsthm}
\usepackage{enumitem, tabularx,graphicx,url,xcolor,rotating,multicol,epsfig,colortbl,lipsum}

\setlist{topsep=1pt, itemsep=0em}
\setlength{\parindent}{0pt}
\setlength{\parskip}{6pt}

\usepackage{hyphenat}
\hyphenation{ма-те-ма-ти-ка вос-ста-нав-ли-вать}

\usepackage[math]{anttor}

\newenvironment{talk}[6]{%
\vskip 0pt\nopagebreak%
\vskip 0pt\nopagebreak%
\section*{#1}
\phantomsection
\addcontentsline{toc}{section}{#2. \textit{#1}}
% \addtocontents{toc}{\textit{#1}\par}
\textit{#2}\\\nopagebreak%
#3\\\nopagebreak%
\ifthenelse{\equal{#4}{}}{}{\url{#4}\\\nopagebreak}%
\ifthenelse{\equal{#5}{}}{}{Соавторы: #5\\\nopagebreak}%
\ifthenelse{\equal{#6}{}}{}{Секция: #6\\\nopagebreak}%
}

\definecolor{LovelyBrown}{HTML}{FDFCF5}

\usepackage[pdftex,
breaklinks=true,
bookmarksnumbered=true,
linktocpage=true,
linktoc=all
]{hyperref}

\begin{document}
\pagenumbering{gobble}
\pagestyle{plain}
\pagecolor{LovelyBrown}
\begin{talk}
{Математическое моделирование излучения плазменной антенны и исследование ее характеристик методом поверхностного резонанса в тлеющем разряде}
{Киселев Глеб Борисович}
{Казанский (Приволжский) федеральный университет, институт физики}
{kiselev.gleb.97@gmail.com}
{Желтухин Виктор Семенович, Шемахин Александр Юрьевич}
{Прикладная математика и математическое моделирование}

Работа включает две основные части. Первая посвящена исследованию длин волн электромагнитных полей, генерируемых плазменной антенной в зависимости от давления. Исследование проводилось путем моделирования в среде Comsol Multiphysics тлеющего разряда вдоль трубки (одномерная постановка). По полученным распределениям тока в плазме была полученная диаграмма направленности и пространственное распределение электромагнитного поля (решение уравнений Максвелла в двумерной постановке) [1].

Вторая часть работы посвящена исследованию неинвазивного метода измерения электронной плотности приповерхностного слоя ртутного тлеющего разряда, который основан на определении резонансных частот между самим слоем разряда и частотой источника. Полученные данные были верифицированы с помощью численной модели, построенной в среде Comsol Multiphysics, и эмиссионной спектроскопией [2].

\medskip

\begin{enumerate}
\item[{[1]}] Terentev, T.N., Kiselev, G.B., Shemakhin, A.Y. et al. {\it Influence of Pressure on Plasma Antenna Resonance Wavelength}. High Energy Chem 58, 190–193 (2024).
\item[{[2]}] Shemakhin Y. A. et al., {\it Spectral studies of inductively coupled plasma characteristics of low pressure discharges for two configurations of vacuum chambers}, Journal of Physics: Conference Series. – IOP Publishing, 2022. – V. 2270. – №. 1. – p. 012007.
\end{enumerate}
\end{talk}
\end{document}