\documentclass[12pt]{article}
\usepackage{hyphsubst}
\usepackage[T2A]{fontenc}
\usepackage[english,main=russian]{babel}
\usepackage[utf8]{inputenc}
\usepackage[letterpaper,top=2cm,bottom=2cm,left=2cm,right=2cm,marginparwidth=2cm]{geometry}
\usepackage{float}
\usepackage{mathtools, commath, amssymb, amsthm}
\usepackage{enumitem, tabularx,graphicx,url,xcolor,rotating,multicol,epsfig,colortbl,lipsum}

\setlist{topsep=1pt, itemsep=0em}
\setlength{\parindent}{0pt}
\setlength{\parskip}{6pt}

\usepackage{hyphenat}
\hyphenation{ма-те-ма-ти-ка вос-ста-нав-ли-вать}

\usepackage[math]{anttor}

\newenvironment{talk}[6]{%
\vskip 0pt\nopagebreak%
\vskip 0pt\nopagebreak%
\section*{#1}
\phantomsection
\addcontentsline{toc}{section}{#2. \textit{#1}}
% \addtocontents{toc}{\textit{#1}\par}
\textit{#2}\\\nopagebreak%
#3\\\nopagebreak%
\ifthenelse{\equal{#4}{}}{}{\url{#4}\\\nopagebreak}%
\ifthenelse{\equal{#5}{}}{}{Соавторы: #5\\\nopagebreak}%
\ifthenelse{\equal{#6}{}}{}{Секция: #6\\\nopagebreak}%
}

\definecolor{LovelyBrown}{HTML}{FDFCF5}

\usepackage[pdftex,
breaklinks=true,
bookmarksnumbered=true,
linktocpage=true,
linktoc=all
]{hyperref}

\begin{document}
\pagenumbering{gobble}
\pagestyle{plain}
\pagecolor{LovelyBrown}
\begin{talk}
{Применение PINN в SIR модели игры среднего поля}
{Неверов Андрей Вячеславович}
{Институт математики им. Соболева СО РАН}
{a.neverov@g.nsu.ru}
{Криворотько Ольга Игоревна}
{Прикладная математика и математическое моделирование}

Рассматривается пространственная эпидемиологическая SIR модель, в которой люди распределены в некотором населенном пункте и стремятся не стать инфицированными. Для реализации взаимодействия большого населения в условиях эпидемии применен подход игр среднего поля~[1], характеризующийся совместным решением систем уравнений в частных производных типа Колмогорова-Фоккера-Планка и Гамильтона-Якоби-Беллмана.

Для численной реализации математического моделирования распространения эпидемии в популяции с учетом оптимального управления применяется метод машинного обучения, а именно физически информированные нейронные сети (PINN) с различными модификациями~[2]. Рассматривается возможность решения коэффициентных обратных задач, где информация вводится в виде дополнительных уравнений.

\medskip

Работа выполнена в рамках государственного задания Института математики им. С.\,Л. Соболева СО РАН, проект FWNF-2024-0002 ``Обратные некорректные задачи и машинное обучение в биологических, социально-экономических и экологических процессах''.

\begin{enumerate}
\item[{[1]}] V. Petrakova, O. Krivorotko, Mean Field Optimal Control Problem for Predicting the Spread of Viral Infections, {\it 19th International Asian School-Seminar on Optimization Problems of Complex Systems (OPCS)}, (2023), 79-84.
\item[{[2]}] M. Raissi, P. Perdikaris, G.E. Karniadakis,
Physics-informed neural networks: A deep learning framework for solving forward and inverse problems involving nonlinear partial differential equations,
{\it Journal of Computational Physics},
378
(2019),
686-707.
\end{enumerate}
\end{talk}
\end{document}