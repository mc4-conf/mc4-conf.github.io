\documentclass[12pt]{article}
\usepackage{hyphsubst}
\usepackage[T2A]{fontenc}
\usepackage[english,main=russian]{babel}
\usepackage[utf8]{inputenc}
\usepackage[letterpaper,top=2cm,bottom=2cm,left=2cm,right=2cm,marginparwidth=2cm]{geometry}
\usepackage{float}
\usepackage{mathtools, commath, amssymb, amsthm}
\usepackage{enumitem, tabularx,graphicx,url,xcolor,rotating,multicol,epsfig,colortbl,lipsum}

\setlist{topsep=1pt, itemsep=0em}
\setlength{\parindent}{0pt}
\setlength{\parskip}{6pt}

\usepackage{hyphenat}
\hyphenation{ма-те-ма-ти-ка вос-ста-нав-ли-вать}

\usepackage[math]{anttor}

\newenvironment{talk}[6]{%
\vskip 0pt\nopagebreak%
\vskip 0pt\nopagebreak%
\section*{#1}
\phantomsection
\addcontentsline{toc}{section}{#2. \textit{#1}}
% \addtocontents{toc}{\textit{#1}\par}
\textit{#2}\\\nopagebreak%
#3\\\nopagebreak%
\ifthenelse{\equal{#4}{}}{}{\url{#4}\\\nopagebreak}%
\ifthenelse{\equal{#5}{}}{}{Соавторы: #5\\\nopagebreak}%
\ifthenelse{\equal{#6}{}}{}{Секция: #6\\\nopagebreak}%
}

\definecolor{LovelyBrown}{HTML}{FDFCF5}

\usepackage[pdftex,
breaklinks=true,
bookmarksnumbered=true,
linktocpage=true,
linktoc=all
]{hyperref}

\begin{document}
\pagenumbering{gobble}
\pagestyle{plain}
\pagecolor{LovelyBrown}
\begin{talk}
{Моделирование вариативного течения инфекции на основе гибридных уравнений с запаздыванием}
{Переварюха Андрей Юрьевич}
{Санкт-Петербургский Федеральный исследовательский центр РАН}
{madelf@rambler.ru}
{}
{Прикладная математика и математическое моделирование} % [6] название секции

Обсудим гибридные модели с вероятностной компонентой для сценариев развития ситуации инвазионного процесса в биосистеме с адаптивным сопротивлением. Представим несколько аспектов запаздывания. Частный случай инвазии с неопределенно запаздывающим ответом это иммунный ответ на коронавирус, который может быть или сильным или медленно возникающим по целому ряду не полностью детерменированных факторов.
Нами предложено включение в модель возмущенного равномерно распределенной на $[0,1]$ $\sigma$ репродуктивного запаздывания $x(t-\tau\times\sigma)$ c целью получить варианты поведения траектории, которые соответствуют динамике концентрации вирионов при различных сценариях развития инфекции в организме. Варианты развития отличаются от быстрого выздоровления, до летального варианта. Наиболее сложный для моделирования сценарий хронизации после острой фазы. В предложенной нами модели получен вариант хронизации без необходимости дальнейшего увеличения $r$, $H=1/3K$:
$$
\frac{dN}{dt}=rN\left(1-\frac{N(t-\tau\times\sigma)}{\mathcal{K}}\right)\left(H-N(t-\gamma)\right), \gamma<\tau. \eqno(1)
$$
Используем в новой форме модели вместо квадратичной зависимости логарифмическую форму регуляции:
$$
\frac{dN}{dt}=r\ln\left(\frac{\mathcal{K}}{N(t-\tau)}\right) - \mathcal{Q}N(t-\nu), \eqno(2)
$$
B таком варианте уравнения с внешним воздействием биотической среды дополнение модели фактором противодействия с отдельным запаздыванием изменит качественный характер решения для описания острой фазы инфекции.
\end{talk}
\end{document}