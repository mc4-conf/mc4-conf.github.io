\documentclass[12pt]{article}
\usepackage{hyphsubst}
\usepackage[T2A]{fontenc}
\usepackage[english,main=russian]{babel}
\usepackage[utf8]{inputenc}
\usepackage[letterpaper,top=2cm,bottom=2cm,left=2cm,right=2cm,marginparwidth=2cm]{geometry}
\usepackage{float}
\usepackage{mathtools, commath, amssymb, amsthm}
\usepackage{enumitem, tabularx,graphicx,url,xcolor,rotating,multicol,epsfig,colortbl,lipsum}

\setlist{topsep=1pt, itemsep=0em}
\setlength{\parindent}{0pt}
\setlength{\parskip}{6pt}

\usepackage{hyphenat}
\hyphenation{ма-те-ма-ти-ка вос-ста-нав-ли-вать}

\usepackage[math]{anttor}

\newenvironment{talk}[6]{%
\vskip 0pt\nopagebreak%
\vskip 0pt\nopagebreak%
\section*{#1}
\phantomsection
\addcontentsline{toc}{section}{#2. \textit{#1}}
% \addtocontents{toc}{\textit{#1}\par}
\textit{#2}\\\nopagebreak%
#3\\\nopagebreak%
\ifthenelse{\equal{#4}{}}{}{\url{#4}\\\nopagebreak}%
\ifthenelse{\equal{#5}{}}{}{Соавторы: #5\\\nopagebreak}%
\ifthenelse{\equal{#6}{}}{}{Секция: #6\\\nopagebreak}%
}

\definecolor{LovelyBrown}{HTML}{FDFCF5}

\usepackage[pdftex,
breaklinks=true,
bookmarksnumbered=true,
linktocpage=true,
linktoc=all
]{hyperref}

\begin{document}
\pagenumbering{gobble}
\pagestyle{plain}
\pagecolor{LovelyBrown}
\begin{talk}
{Метод типа Ландена для вычисления функций Вейерштрасса}
{Смирнов Матвей Станиславович}
{Институт вычислительной математики им. Г.\,И. Марчука РАН}
{matsmir98@gmail.com}
{}
{Прикладная математика и математическое моделирование}

Метод Ландена широко известен в контексте эллиптических интегралов и функций Якоби. Его суть состоит в замене модулярных параметров, соответствующих удвоению одного из периодов. Повторение этой процедуры приводит к эффективному вычислительному методу, в особенности, потому что эллиптический модуль при этом сходится квадратично быстро. В отличие от функций Якоби, функции Вейерштрасса традиционно мало применяются в вычислительной практике. Данный доклад будет посвящен новому методу вычисления функций Вейерштрасса, основанному на идеях, аналогичных методу Ландена. Полученный метод также устойчив и имеет квадратичную сходимость. Будет показано как на этом пути можно получить эффективный метод вычисления периодов и отображения Абеля для данных инвариантов Вейерштрасса эллиптической кривой.
\end{talk}
\end{document}