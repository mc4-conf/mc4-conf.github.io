\documentclass[12pt]{article}
\usepackage{hyphsubst}
\usepackage[T2A]{fontenc}
\usepackage[english,main=russian]{babel}
\usepackage[utf8]{inputenc}
\usepackage[letterpaper,top=2cm,bottom=2cm,left=2cm,right=2cm,marginparwidth=2cm]{geometry}
\usepackage{float}
\usepackage{mathtools, commath, amssymb, amsthm}
\usepackage{enumitem, tabularx,graphicx,url,xcolor,rotating,multicol,epsfig,colortbl,lipsum}

\setlist{topsep=1pt, itemsep=0em}
\setlength{\parindent}{0pt}
\setlength{\parskip}{6pt}

\usepackage{hyphenat}
\hyphenation{ма-те-ма-ти-ка вос-ста-нав-ли-вать}

\usepackage[math]{anttor}

\newenvironment{talk}[6]{%
\vskip 0pt\nopagebreak%
\vskip 0pt\nopagebreak%
\section*{#1}
\phantomsection
\addcontentsline{toc}{section}{#2. \textit{#1}}
% \addtocontents{toc}{\textit{#1}\par}
\textit{#2}\\\nopagebreak%
#3\\\nopagebreak%
\ifthenelse{\equal{#4}{}}{}{\url{#4}\\\nopagebreak}%
\ifthenelse{\equal{#5}{}}{}{Соавторы: #5\\\nopagebreak}%
\ifthenelse{\equal{#6}{}}{}{Секция: #6\\\nopagebreak}%
}

\definecolor{LovelyBrown}{HTML}{FDFCF5}

\usepackage[pdftex,
breaklinks=true,
bookmarksnumbered=true,
linktocpage=true,
linktoc=all
]{hyperref}

\begin{document}
\pagenumbering{gobble}
\pagestyle{plain}
\pagecolor{LovelyBrown}
\begin{talk}
{Моделирование динамики планктонного сообщества озера Байкал}
{Сурнин Павел Сергеевич}
{Институт математики им. С.\,Л. Соболева СО РАН, Математический центр в Академгородке}
{p.surnin@internet.ru}
{}
{Прикладная математика и математическое моделирование}

Планктонное сообщество играет значимую роль в формировании качества воды поверхностных источников, используемых для систем хозяйственно-питьевого водоснабжения. Важно отметить, что \(70\%\) производимого кислорода вырабатывается фитопланктоном~[1]. Моделирование сезонных особенностей является основной задачей в понимание развития фитопланктона и прогнозирование возможных последствий, связанных с поступлением в воду продуктов жизнедеятельности и отмирания клеток фитопланктона.

В докладе будет приведен анализ данных планктонного сообщества озера Байкал. На основе математической модели [2], состоящей из системы нелинейных уравнений реакции-диффузии, проведено моделирование динамики распределения биомассы зоопланктона и фитопланктона, а также концентрации кислорода в зависимости от глубины. Приведен сравнительный анализ решения прямой задачи конечно-разностным методом и методом конечных элементов. Приведены постановки обратных задач.

\medskip

Работа выполнена при поддержке Математического Центра в Академгородке, соглашение с Министерством науки и высшего образования Российской Федерации № 075-15-2022-281.

\begin{enumerate}
\item[{[1]}] R. Lindsey, S. Michon, {\it What are Phytoplankton?}, NASA, Web, 22 (2011).
\item[{[2]}] Y. Sekerci, S. Petrovskii, {\it Mathematical Modelling of Plankton-Oxygen Dynamics Under the Climate Change}, Bulletin of mathematical biology, 11 (2015)
\end{enumerate}
\end{talk}
\end{document}