\documentclass[12pt]{article}
\usepackage{hyphsubst}
\usepackage[T2A]{fontenc}
\usepackage[english,main=russian]{babel}
\usepackage[utf8]{inputenc}
\usepackage[letterpaper,top=2cm,bottom=2cm,left=2cm,right=2cm,marginparwidth=2cm]{geometry}
\usepackage{float}
\usepackage{mathtools, commath, amssymb, amsthm}
\usepackage{enumitem, tabularx,graphicx,url,xcolor,rotating,multicol,epsfig,colortbl,lipsum}

\setlist{topsep=1pt, itemsep=0em}
\setlength{\parindent}{0pt}
\setlength{\parskip}{6pt}

\usepackage{hyphenat}
\hyphenation{ма-те-ма-ти-ка вос-ста-нав-ли-вать}

\usepackage[math]{anttor}

\newenvironment{talk}[6]{%
\vskip 0pt\nopagebreak%
\vskip 0pt\nopagebreak%
\section*{#1}
\phantomsection
\addcontentsline{toc}{section}{#2. \textit{#1}}
% \addtocontents{toc}{\textit{#1}\par}
\textit{#2}\\\nopagebreak%
#3\\\nopagebreak%
\ifthenelse{\equal{#4}{}}{}{\url{#4}\\\nopagebreak}%
\ifthenelse{\equal{#5}{}}{}{Соавторы: #5\\\nopagebreak}%
\ifthenelse{\equal{#6}{}}{}{Секция: #6\\\nopagebreak}%
}

\definecolor{LovelyBrown}{HTML}{FDFCF5}

\usepackage[pdftex,
breaklinks=true,
bookmarksnumbered=true,
linktocpage=true,
linktoc=all
]{hyperref}

\begin{document}
\pagenumbering{gobble}
\pagestyle{plain}
\pagecolor{LovelyBrown}
\begin{talk}
{Молекулярно-динамическое моделирование взаимодействия газовой смеси с графеновыми мембранами}
{Челнокова Анна Сергеевна}
{Региональный научно-образовательный математический центр Томского государственного университета}
{smolina-nyuta@mail.ru}
{Бубенчиков Алексей Михайлович}
{Прикладная математика и математическое моделирование}

Наноструктуры на основе углерода, такие как графен, углеродные нанотрубки и фуллерены, привлекли широкое внимание исследователей по всему миру благодаря своим уникальным свойствам. В наномасштабе одним из устоявшихся подходов к изучению подобных структур является молекулярно-динамическое моделирование. Оно особенно полезно для количественной оценки основных взаимодействий и динамических процессов, определяющих коэффициенты адсорбции или диффузии.

В настоящее время в качестве перспективного фильтрующего материала рассматриваются графеноподобные мембраны, и существует необходимость в разработке теоретических подходов для изучения диффузии и сорбции, которые включают межчастичные взаимодействия для предоставления точной информации о массопереносе.

В докладе будет представлена математическая модель взаимодействия компонент газовой смеси He, Ar и Xe с графеновыми листами, в том числе имеющими дефекты. Силы взаимодействия описаны с использованием потенциалов Леннарда-Джонса и Бреннера второго рода. Приведено сравнение коэффициентов проницаемости различных газовых компонент через графеновые листы с дефектами с применением вышеуказанных потенциалов. Представлены оценки температуры газовой смеси и графеновой мембраны.
\end{talk}
\end{document}