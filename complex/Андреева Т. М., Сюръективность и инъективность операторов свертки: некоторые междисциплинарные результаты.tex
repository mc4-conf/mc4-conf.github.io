\documentclass[12pt]{article}
\usepackage{hyphsubst}
\usepackage[T2A]{fontenc}
\usepackage[english,main=russian]{babel}
\usepackage[utf8]{inputenc}
\usepackage[letterpaper,top=2cm,bottom=2cm,left=2cm,right=2cm,marginparwidth=2cm]{geometry}
\usepackage{float}
\usepackage{mathtools, commath, amssymb, amsthm}
\usepackage{enumitem, tabularx,graphicx,url,xcolor,rotating,multicol,epsfig,colortbl,lipsum}

\setlist{topsep=1pt, itemsep=0em}
\setlength{\parindent}{0pt}
\setlength{\parskip}{6pt}

\usepackage{hyphenat}
\hyphenation{ма-те-ма-ти-ка вос-ста-нав-ли-вать}

\usepackage[math]{anttor}

\newenvironment{talk}[6]{%
\vskip 0pt\nopagebreak%
\vskip 0pt\nopagebreak%
\section*{#1}
\phantomsection
\addcontentsline{toc}{section}{#2. \textit{#1}}
% \addtocontents{toc}{\textit{#1}\par}
\textit{#2}\\\nopagebreak%
#3\\\nopagebreak%
\ifthenelse{\equal{#4}{}}{}{\url{#4}\\\nopagebreak}%
\ifthenelse{\equal{#5}{}}{}{Соавторы: #5\\\nopagebreak}%
\ifthenelse{\equal{#6}{}}{}{Секция: #6\\\nopagebreak}%
}

\definecolor{LovelyBrown}{HTML}{FDFCF5}

\usepackage[pdftex,
breaklinks=true,
bookmarksnumbered=true,
linktocpage=true,
linktoc=all
]{hyperref}

\begin{document}
\pagenumbering{gobble}
\pagestyle{plain}
\pagecolor{LovelyBrown}
\begin{talk}
{Сюръективность и инъективность операторов свертки: некоторые междисциплинарные результаты}
{Андреева Татьяна Михайловна}
{Южный федеральный университет}
{metzi@yandex.ru}
{}
{Комплексный анализ}

Пусть \(G\) --- выпуклая ограниченная область комплексной плоскости \(\mathbb{C}\), \(H\left(G\right)\) --- пространство всех функций, голоморфных в  \(G\), \(V=\left(v_n\right)_{n=1}^\infty\) --- последовательность возврастающих функций, непрерывных в \(G\). Обозначим \(VH\left(G\right):=\bigcup_{n=1}^{\infty}{H_{v_n}\left(G\right)}\), где
\begin{equation*}
\begin{array}{cc}
H_{v_n}(G):= \Big\{ f\in H(G): ||f||_{v_n}:=\sup\limits_{z\in G} \cfrac{|f(z)|}{e^{v_n(z)}}  \Big\}, & n\in\mathbb{N}.
\end{array}
\end{equation*}
В работе рассматриваются решения уравнения свертки \(\mu\ast f=g\), где \(\mu:VH\left(G+K\right)\rightarrow VH\left(G\right)\) --- аналитический функционал с носителем в выпуклом компакте \(K\).

Основные результаты основаны на условиях сюръективности обозначенных операторов [1, 2]. Рассматриваются некоторые междисциплинарные вопросы, которые сводятся к качественным свойствам описанных в докладе отображений.

\medskip

\begin{enumerate}
\item[{[1]}] {Abanin A.~V, Andreeva T.~M.}{On the surjectivity of the convolution operator in spaces of holomorphic functions of a prescribed growth~/\!/ Vladikavkaz. Mat. Zh.---2018.---20(2).~---Pp.~3--15.}
\item[{[2]}] {Abanin A.~V, Andreeva T.~M.}{Analytic Description of the Spaces Dual to Spaces of Holomorphic Functions of Given Growth on Caratheodory Domains~/\!/ Mat. Zametki.---2018.--- 104:3.~---Pp. 323--335.}
\end{enumerate}
\end{talk}
\end{document}