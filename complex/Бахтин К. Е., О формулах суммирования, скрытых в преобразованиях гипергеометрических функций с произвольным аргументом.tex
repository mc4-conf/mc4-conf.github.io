\documentclass[12pt]{article}
\usepackage{hyphsubst}
\usepackage[T2A]{fontenc}
\usepackage[english,main=russian]{babel}
\usepackage[utf8]{inputenc}
\usepackage[letterpaper,top=2cm,bottom=2cm,left=2cm,right=2cm,marginparwidth=2cm]{geometry}
\usepackage{float}
\usepackage{mathtools, commath, amssymb, amsthm}
\usepackage{enumitem, tabularx,graphicx,url,xcolor,rotating,multicol,epsfig,colortbl,lipsum}

\setlist{topsep=1pt, itemsep=0em}
\setlength{\parindent}{0pt}
\setlength{\parskip}{6pt}

\usepackage{hyphenat}
\hyphenation{ма-те-ма-ти-ка вос-ста-нав-ли-вать}

\usepackage[math]{anttor}

\newenvironment{talk}[6]{%
\vskip 0pt\nopagebreak%
\vskip 0pt\nopagebreak%
\section*{#1}
\phantomsection
\addcontentsline{toc}{section}{#2. \textit{#1}}
% \addtocontents{toc}{\textit{#1}\par}
\textit{#2}\\\nopagebreak%
#3\\\nopagebreak%
\ifthenelse{\equal{#4}{}}{}{\url{#4}\\\nopagebreak}%
\ifthenelse{\equal{#5}{}}{}{Соавторы: #5\\\nopagebreak}%
\ifthenelse{\equal{#6}{}}{}{Секция: #6\\\nopagebreak}%
}

\definecolor{LovelyBrown}{HTML}{FDFCF5}

\usepackage[pdftex,
breaklinks=true,
bookmarksnumbered=true,
linktocpage=true,
linktoc=all
]{hyperref}

\begin{document}
\pagenumbering{gobble}
\pagestyle{plain}
\pagecolor{LovelyBrown}
\begin{talk}
{О формулах суммирования, скрытых в преобразованиях гипергеометрических функций с произвольным аргументом}
{Бахтин Кирилл Евгеньевич}
{РНОМЦ ``Дальневосточный центр математических исследований'', ДВФУ}
{bakhtin.ke@dvfu.ru}
{Прилепкина Е.\,Г.}
{Комплексный анализ}

Хорошо известны преобразования Эйлера-Пфаффа для гипергеометрической функции Гаусса. В последствии было открыто несколько преобразований подобного типа, общий вид которых можно записать в форме
\[F\left(\left.\begin{matrix}\mathbf{a}\\\mathbf{b}\end{matrix}\right\vert Mx^w\right) =
V(1-x)^{\lambda}  F\left(\left.\begin{matrix} \mathbf{c}\\\mathbf{d}\end{matrix}\right\vert \frac{Dx^u}{(1-x)^v}\right),\ \ x\in G,\]
где \(F\) --- обобщенная гипергеометрическая функция, \(\mathbf{c}, \mathbf{d}\), \(V\) и \(\lambda\) функции, зависящие от векторов \(\mathbf{a}, \mathbf{b},\) \(w,u \in \mathbb{N},\) \(v \in \mathbb{Z}, M, D\) --- некоторые константы, и  \(G\) является  областью комплексной плоскости \({\mathbb C}_x\).
Очевидно, что если некоторые функции \(f(x),\ g(x),\ h(x)\) раскладываются в области \(G\) в степенные ряды, то в преобразовании  \(f(x)=g(x)h(x),\ x\in G,\) скрыта некоторая формула суммирования. В данном докладе мы показываем, что преобразования типа Эйлера--Пфаффа основаны на формулах суммирования для конечных гипергеометрических функций.
Обсуждаются также обобщения известной формулы суммирования Карлсона-Минтона.

\medskip

Работа выполнена в Дальневосточном центре математических исследований при финансовой поддержке Минобрнауки России, соглашение  №075-02-2024-1440 от 28 февраля 2024 года по реализации программ развития региональных научно-образовательных математических центров.
\end{talk}
\end{document}