\documentclass[12pt]{article}
\usepackage{hyphsubst}
\usepackage[T2A]{fontenc}
\usepackage[english,main=russian]{babel}
\usepackage[utf8]{inputenc}
\usepackage[letterpaper,top=2cm,bottom=2cm,left=2cm,right=2cm,marginparwidth=2cm]{geometry}
\usepackage{float}
\usepackage{mathtools, commath, amssymb, amsthm}
\usepackage{enumitem, tabularx,graphicx,url,xcolor,rotating,multicol,epsfig,colortbl,lipsum}

\setlist{topsep=1pt, itemsep=0em}
\setlength{\parindent}{0pt}
\setlength{\parskip}{6pt}

\usepackage{hyphenat}
\hyphenation{ма-те-ма-ти-ка вос-ста-нав-ли-вать}

\usepackage[math]{anttor}

\newenvironment{talk}[6]{%
\vskip 0pt\nopagebreak%
\vskip 0pt\nopagebreak%
\section*{#1}
\phantomsection
\addcontentsline{toc}{section}{#2. \textit{#1}}
% \addtocontents{toc}{\textit{#1}\par}
\textit{#2}\\\nopagebreak%
#3\\\nopagebreak%
\ifthenelse{\equal{#4}{}}{}{\url{#4}\\\nopagebreak}%
\ifthenelse{\equal{#5}{}}{}{Соавторы: #5\\\nopagebreak}%
\ifthenelse{\equal{#6}{}}{}{Секция: #6\\\nopagebreak}%
}

\definecolor{LovelyBrown}{HTML}{FDFCF5}

\usepackage[pdftex,
breaklinks=true,
bookmarksnumbered=true,
linktocpage=true,
linktoc=all
]{hyperref}

\begin{document}
\pagenumbering{gobble}
\pagestyle{plain}
\pagecolor{LovelyBrown}
\begin{talk}
{Решение уравнения Абловица-Ладика с помощью алгоритма Шура}
{Губкин Павел Васильевич}
{ПОМИ РАН}
{gubkinpavel@pdmi.ras.ru}
{Бессонов Роман Викторович}
{Комплексный анализ}

Рассмотрим аналитическую функцию \(f\), действующую из единичного круга \(\mathbb{D}\) в себя. Для нее можно определить последовательность функций \(f_0 = f, f_1, f_2, \ldots \) удовлетворяющих соотношениям
\begin{equation*}
zf_{n+1} = \frac{f_n - f_n(0)}{1 - \overline{f_n(0)}f_n}, \qquad n \ge 0,
\end{equation*}
при этом каждая из функций \(f_n\) будет снова действовать из \(\mathbb{D}\) в \(\mathbb{D}\). Алгоритмом Шура называется построение последовательности  \(\{f_n(0)\}_{n\ge 0}\) по функции \(f\), такая последовательность может быть любой последовательностью комплексных чисел из единичного круга. На докладе мы обсудим вопросы, связанные с устойчивостью алгоритма Шура и то, как это построение позволяет решать дифференциальное уравнение Абловица-Ладика
\begin{equation*}
\frac{\partial}{\partial t}q(t,n) = i\bigl(1 - |q(t,n)|^2\bigr)\bigl(q(t,n-1) + q(t,n+1)\bigr), \qquad n\in \mathbb{Z}.
\end{equation*}

Доклад основан на результатах, полученных в работе [1].

\medskip

\begin{enumerate}
\item[{[1]}] Bessonov R. V., Gubkin P. V. Stability of Schur's iterates and fast solution of the discrete integrable NLS //arXiv preprint {\tt arXiv:2402.02434}. – 2024.
\end{enumerate}
\end{talk}
\end{document}