\documentclass[12pt]{article}
\usepackage{hyphsubst}
\usepackage[T2A]{fontenc}
\usepackage[english,main=russian]{babel}
\usepackage[utf8]{inputenc}
\usepackage[letterpaper,top=2cm,bottom=2cm,left=2cm,right=2cm,marginparwidth=2cm]{geometry}
\usepackage{float}
\usepackage{mathtools, commath, amssymb, amsthm}
\usepackage{enumitem, tabularx,graphicx,url,xcolor,rotating,multicol,epsfig,colortbl,lipsum}

\setlist{topsep=1pt, itemsep=0em}
\setlength{\parindent}{0pt}
\setlength{\parskip}{6pt}

\usepackage{hyphenat}
\hyphenation{ма-те-ма-ти-ка вос-ста-нав-ли-вать}

\usepackage[math]{anttor}

\newenvironment{talk}[6]{%
\vskip 0pt\nopagebreak%
\vskip 0pt\nopagebreak%
\section*{#1}
\phantomsection
\addcontentsline{toc}{section}{#2. \textit{#1}}
% \addtocontents{toc}{\textit{#1}\par}
\textit{#2}\\\nopagebreak%
#3\\\nopagebreak%
\ifthenelse{\equal{#4}{}}{}{\url{#4}\\\nopagebreak}%
\ifthenelse{\equal{#5}{}}{}{Соавторы: #5\\\nopagebreak}%
\ifthenelse{\equal{#6}{}}{}{Секция: #6\\\nopagebreak}%
}

\definecolor{LovelyBrown}{HTML}{FDFCF5}

\usepackage[pdftex,
breaklinks=true,
bookmarksnumbered=true,
linktocpage=true,
linktoc=all
]{hyperref}

\begin{document}
\pagenumbering{gobble}
\pagestyle{plain}
\pagecolor{LovelyBrown}
\begin{talk}
{Доминирующие множества для модельных пространств}
{Дубцов Евгений Сергеевич}
{ПОМИ РАН}
{dubtsov@pdmi.ras.ru}
{А.\,Б. Александров}
{Комплексный анализ}

Пусть \(\mathcal{D}\subset \mathbb{C}^n\) обозначает ограниченную симметричную область
и \(\mathrm{b}\mathcal{D}\) --- граница Шилова области \(\mathcal{D}\).
Пусть \(\beta\) --- единственная нормированная положительная радоновская мера, заданная на границе
\(\mathrm{b}\mathcal{D}\) и инвариантная относительно всех линейных автоморфизмов
области  \(\mathcal{D}\).
Далее, пусть \(I\) --- внутренняя функция, заданная на области \(\mathcal{D}\).
Измеримое множество \(E\subset \mathrm{b}\mathcal{D}\) называется \textit{доминирующим}
для большого модельного множества
\(H^2 \ominus I H^2\), если \(\beta(E) < 1\) и
\[\|f\|_{H^2}^2 \le C\int_E |f|^2\, d\beta\]
для всех \(f\in H^2 \ominus I H^2\).
В докладе получен результат о композициях с внутренними функциями,
из которого, в частности, следует, что доминирующие множества существуют
для каждого пространства \(H^2 \ominus I H^2\).

\medskip

Исследование выполнено за счёт гранта Российского научного фонда
№24-11-00087.
\end{talk}
\end{document}