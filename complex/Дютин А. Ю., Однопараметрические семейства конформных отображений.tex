\documentclass[12pt]{article}
\usepackage{hyphsubst}
\usepackage[T2A]{fontenc}
\usepackage[english,main=russian]{babel}
\usepackage[utf8]{inputenc}
\usepackage[letterpaper,top=2cm,bottom=2cm,left=2cm,right=2cm,marginparwidth=2cm]{geometry}
\usepackage{float}
\usepackage{mathtools, commath, amssymb, amsthm}
\usepackage{enumitem, tabularx,graphicx,url,xcolor,rotating,multicol,epsfig,colortbl,lipsum}

\setlist{topsep=1pt, itemsep=0em}
\setlength{\parindent}{0pt}
\setlength{\parskip}{6pt}

\usepackage{hyphenat}
\hyphenation{ма-те-ма-ти-ка вос-ста-нав-ли-вать}

\usepackage[math]{anttor}

\newenvironment{talk}[6]{%
\vskip 0pt\nopagebreak%
\vskip 0pt\nopagebreak%
\section*{#1}
\phantomsection
\addcontentsline{toc}{section}{#2. \textit{#1}}
% \addtocontents{toc}{\textit{#1}\par}
\textit{#2}\\\nopagebreak%
#3\\\nopagebreak%
\ifthenelse{\equal{#4}{}}{}{\url{#4}\\\nopagebreak}%
\ifthenelse{\equal{#5}{}}{}{Соавторы: #5\\\nopagebreak}%
\ifthenelse{\equal{#6}{}}{}{Секция: #6\\\nopagebreak}%
}

\definecolor{LovelyBrown}{HTML}{FDFCF5}

\usepackage[pdftex,
breaklinks=true,
bookmarksnumbered=true,
linktocpage=true,
linktoc=all
]{hyperref}

\begin{document}
\pagenumbering{gobble}
\pagestyle{plain}
\pagecolor{LovelyBrown}
\begin{talk}
{Однопараметрические семейства конформных отображений}
{Дютин Андрей Юрьевич}
{Институт математики и механики им. Н.\,И. Лобачевского Казанского (Приволжского) федерального университета}
{dyutin.andrei2016@yandex.ru}
{}
{Комплексный анализ}

Получен приближённый метод нахождения конформного отображения концентрического кольца на~произвольную двусвязную многоугольную область. Этот метод основан на идеях, связанных с~параметрическим методом Левнера--Комацу. Мы рассматриваем гладкие однопараметрические семейства конформных отображений \(\mathcal{F}(z,t)\) концентрических колец на~двусвязные многоугольные области \(\mathcal{D}(t)\), которые получаются из~фиксированной двусвязной многоугольной области \(\mathcal{D}\) проведением конечного числа прямолинейных или в~общем случае полигональных разрезов переменной длины; при этом мы не~требуем монотонности семейства областей~\(\mathcal{D}(t)\). В~интегральное представление для конформных отображений \(\mathcal{F}(z,t)\) входят неизвестные (акцессорные) параметры.  Мы находим дифференциальное уравнение в~частных производных, которому удовлетворяют такие семейства конформных отображений, и~выводим из~него систему дифференциальных уравнений, описывающих динамику акцессорных параметров при~изменении параметра \(t\) и~динамику конформного модуля данной двусвязной области в~зависимости от~параметра~\(t\). Отметим, что в~правые части полученной системы ОДУ входят функции, которые являются скоростями движения концевых точек разрезов. Это позволяет полностью контролировать динамику разрезов, в~частности, добиваться их согласованного изменения в~случае, если в~области \(\mathcal{D}\) проводится более одного разреза. Рассмотрены примеры, иллюстрирующие эффективность предложенного метода.
\end{talk}
\end{document}