\documentclass[12pt]{article}
\usepackage{hyphsubst}
\usepackage[T2A]{fontenc}
\usepackage[english,main=russian]{babel}
\usepackage[utf8]{inputenc}
\usepackage[letterpaper,top=2cm,bottom=2cm,left=2cm,right=2cm,marginparwidth=2cm]{geometry}
\usepackage{float}
\usepackage{mathtools, commath, amssymb, amsthm}
\usepackage{enumitem, tabularx,graphicx,url,xcolor,rotating,multicol,epsfig,colortbl,lipsum}

\setlist{topsep=1pt, itemsep=0em}
\setlength{\parindent}{0pt}
\setlength{\parskip}{6pt}

\usepackage{hyphenat}
\hyphenation{ма-те-ма-ти-ка вос-ста-нав-ли-вать}

\usepackage[math]{anttor}

\newenvironment{talk}[6]{%
\vskip 0pt\nopagebreak%
\vskip 0pt\nopagebreak%
\section*{#1}
\phantomsection
\addcontentsline{toc}{section}{#2. \textit{#1}}
% \addtocontents{toc}{\textit{#1}\par}
\textit{#2}\\\nopagebreak%
#3\\\nopagebreak%
\ifthenelse{\equal{#4}{}}{}{\url{#4}\\\nopagebreak}%
\ifthenelse{\equal{#5}{}}{}{Соавторы: #5\\\nopagebreak}%
\ifthenelse{\equal{#6}{}}{}{Секция: #6\\\nopagebreak}%
}

\definecolor{LovelyBrown}{HTML}{FDFCF5}

\usepackage[pdftex,
breaklinks=true,
bookmarksnumbered=true,
linktocpage=true,
linktoc=all
]{hyperref}

\begin{document}
\pagenumbering{gobble}
\pagestyle{plain}
\pagecolor{LovelyBrown}
\begin{talk}
{Обратная задача по резонансам для энергозависимых потенциалов}
{Коротяев Евгений Леонидович}
{Матмех СПбГУ, С. Петербург}
{korotyaev@gmail.com}
{A. Mantile, Д. Мокеев}
{Комплексный анализ}

Мы рассматриваем уравнения Шрёдингера с потенциалами, зависящими от
энергии и имеющими  компактный носитель, на полупрямой. Сначала мы
получаем оценки числа собственных значений и резонансов для наших
комплекснозначных потенциалов. Затем мы рассмотриваем специальный
класс энергозависимых уравнений Шрёдингера без собственных значений.
Здесь мы решаем обратную задачу по резонансам и описываем множества
изо-резонансных потенциалов. Наша стратегия заключается в
использовании соответствия между уравнениями Шрёдингера и Дирака на
полупрямой. В качестве побочного результата мы описываем аналогичные
множества для оператора Дирака и показываем, что задача рассеяния
для уравнения Шредингера  или оператора Дирака с произвольным
граничным условием могут сводится к задаче рассеяния с условием
Дирихле.

Доклад основан на статье [1].

\medskip

\begin{enumerate}
\item[{[1]}] Korotyaev, E.; Mantile, A.; Mokeev, D.
Inverse resonance problems for energy-dependent potentials on the
half-line, SIAM Journal on Mathematical Analysis, 56(2024), №2,
2115-2148.
\end{enumerate}
\end{talk}
\end{document}