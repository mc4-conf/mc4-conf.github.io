\documentclass[12pt]{article}
\usepackage{hyphsubst}
\usepackage[T2A]{fontenc}
\usepackage[english,main=russian]{babel}
\usepackage[utf8]{inputenc}
\usepackage[letterpaper,top=2cm,bottom=2cm,left=2cm,right=2cm,marginparwidth=2cm]{geometry}
\usepackage{float}
\usepackage{mathtools, commath, amssymb, amsthm}
\usepackage{enumitem, tabularx,graphicx,url,xcolor,rotating,multicol,epsfig,colortbl,lipsum}

\setlist{topsep=1pt, itemsep=0em}
\setlength{\parindent}{0pt}
\setlength{\parskip}{6pt}

\usepackage{hyphenat}
\hyphenation{ма-те-ма-ти-ка вос-ста-нав-ли-вать}

\usepackage[math]{anttor}

\newenvironment{talk}[6]{%
\vskip 0pt\nopagebreak%
\vskip 0pt\nopagebreak%
\section*{#1}
\phantomsection
\addcontentsline{toc}{section}{#2. \textit{#1}}
% \addtocontents{toc}{\textit{#1}\par}
\textit{#2}\\\nopagebreak%
#3\\\nopagebreak%
\ifthenelse{\equal{#4}{}}{}{\url{#4}\\\nopagebreak}%
\ifthenelse{\equal{#5}{}}{}{Соавторы: #5\\\nopagebreak}%
\ifthenelse{\equal{#6}{}}{}{Секция: #6\\\nopagebreak}%
}

\definecolor{LovelyBrown}{HTML}{FDFCF5}

\usepackage[pdftex,
breaklinks=true,
bookmarksnumbered=true,
linktocpage=true,
linktoc=all
]{hyperref}

\begin{document}
\pagenumbering{gobble}
\pagestyle{plain}
\pagecolor{LovelyBrown}
\begin{talk}
{О некоторых экстремальных задачах для целых функций экспоненциального типа}
{Манов Анатолий Дмитриевич}
{Донецкий государственный университет}
{manov.ad@ro.ru}
{}
{Комплексный анализ}

В работе рассматривается экстремальная задача, связанная с множеством непрерывных положительно определённых функций на \(\mathbb{R}^n\), носитель которых содержится в замкнутом шаре радиуса \(r>0\), а значение в нуле фиксировано (класс \(\mathfrak{F}_r(\mathbb{R}^n)\)).

При фиксированном \(r>0\) требуется найти точную верхнюю грань функционала специального вида на множестве \(\mathfrak{F}_r(\mathbb{R}^n)\).

Получено общее решение данной задачи. Как следствие, получены новые точные неравенства для производных целых функций экспоненциального сферического типа \(\leqslant r\).
\end{talk}
\end{document}