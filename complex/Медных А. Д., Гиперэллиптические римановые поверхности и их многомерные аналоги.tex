\documentclass[12pt]{article}
\usepackage{hyphsubst}
\usepackage[T2A]{fontenc}
\usepackage[english,main=russian]{babel}
\usepackage[utf8]{inputenc}
\usepackage[letterpaper,top=2cm,bottom=2cm,left=2cm,right=2cm,marginparwidth=2cm]{geometry}
\usepackage{float}
\usepackage{mathtools, commath, amssymb, amsthm}
\usepackage{enumitem, tabularx,graphicx,url,xcolor,rotating,multicol,epsfig,colortbl,lipsum}

\setlist{topsep=1pt, itemsep=0em}
\setlength{\parindent}{0pt}
\setlength{\parskip}{6pt}

\usepackage{hyphenat}
\hyphenation{ма-те-ма-ти-ка вос-ста-нав-ли-вать}

\usepackage[math]{anttor}

\newenvironment{talk}[6]{%
\vskip 0pt\nopagebreak%
\vskip 0pt\nopagebreak%
\section*{#1}
\phantomsection
\addcontentsline{toc}{section}{#2. \textit{#1}}
% \addtocontents{toc}{\textit{#1}\par}
\textit{#2}\\\nopagebreak%
#3\\\nopagebreak%
\ifthenelse{\equal{#4}{}}{}{\url{#4}\\\nopagebreak}%
\ifthenelse{\equal{#5}{}}{}{Соавторы: #5\\\nopagebreak}%
\ifthenelse{\equal{#6}{}}{}{Секция: #6\\\nopagebreak}%
}

\definecolor{LovelyBrown}{HTML}{FDFCF5}

\usepackage[pdftex,
breaklinks=true,
bookmarksnumbered=true,
linktocpage=true,
linktoc=all
]{hyperref}

\begin{document}
\pagenumbering{gobble}
\pagestyle{plain}
\pagecolor{LovelyBrown}
\begin{talk}
{Гиперэллиптические римановые поверхности и их многомерные аналоги}
{Медных Александр Дмитриевич}
{Институт математики им. С.\,Л. Соболева СО РАН}
{smedn@mail.ru}
{}
{Комплексный анализ}

Основная цель настоящего доклада --- введение в теорию гиперэллиптических поверхностей и их трехмерных и одномерных аналогов. Доклад содержит конструктивное описание гиперэллиптических многообразий в случае малой размерности и их основных свойств. Будут приведены дискретные и многомерные варианты теорем Фаркаша, Кра и Акколы, хорошо известными в комплексном анализе. Также мы даем частичный ответ на задачу де Франшиса, поставленную в 1913 году. Точнее, мы даем структурное описание всех голоморфных отображений между римановыми поверхностями рода три и рода два. В результате получаем точную верхнюю оценку размера множества таких отображений. Результаты получены совместно с И.\,А. Медных.

\medskip

\begin{enumerate}
\item[{[1]}] Mednykh, A. D., Mednykh, I.A., {\it The equivalence classes of holomorphic mappings of genus 3 Riemann surfaces onto genus 2 Riemann surfaces}, Sib. Math. J., 57:6 (2016), 1055–1065.
\item[{[2]}]Mednykh, I. A., {\it Discrete analogs of Farkas and Accola’s theorems on hyperelliptic coverings of a Riemann surface of genus 2}, Math. Notes, 96:1 (2014), 84–94.
\item[{[3]}]Mednykh, I. A., {\it On the sharp upper bound for the number of holomorphic mappings of Riemann surfaces of low genus}, Sib. Math. J., 53:2 (2012), 259–273.
\item[{[4]}]Mednykh, I. A., {\it Classification up to equivalence of the holomorphic mappings of Riemann surfaces of low genus}, Sib. Math. J., 51:6 (2010), 1091–1103.
\item[{[5]}]A.D. Mednykh, M. Reni, {\it Twofold unbranched coverings of genus two 3-manifolds are hyperelliptic}, Isr. J. Math., 123 (2001), 149–155.
\end{enumerate}
\end{talk}
\end{document}