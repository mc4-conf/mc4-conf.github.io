\documentclass[12pt]{article}
\usepackage{hyphsubst}
\usepackage[T2A]{fontenc}
\usepackage[english,main=russian]{babel}
\usepackage[utf8]{inputenc}
\usepackage[letterpaper,top=2cm,bottom=2cm,left=2cm,right=2cm,marginparwidth=2cm]{geometry}
\usepackage{float}
\usepackage{mathtools, commath, amssymb, amsthm}
\usepackage{enumitem, tabularx,graphicx,url,xcolor,rotating,multicol,epsfig,colortbl,lipsum}

\setlist{topsep=1pt, itemsep=0em}
\setlength{\parindent}{0pt}
\setlength{\parskip}{6pt}

\usepackage{hyphenat}
\hyphenation{ма-те-ма-ти-ка вос-ста-нав-ли-вать}

\usepackage[math]{anttor}

\newenvironment{talk}[6]{%
\vskip 0pt\nopagebreak%
\vskip 0pt\nopagebreak%
\section*{#1}
\phantomsection
\addcontentsline{toc}{section}{#2. \textit{#1}}
% \addtocontents{toc}{\textit{#1}\par}
\textit{#2}\\\nopagebreak%
#3\\\nopagebreak%
\ifthenelse{\equal{#4}{}}{}{\url{#4}\\\nopagebreak}%
\ifthenelse{\equal{#5}{}}{}{Соавторы: #5\\\nopagebreak}%
\ifthenelse{\equal{#6}{}}{}{Секция: #6\\\nopagebreak}%
}

\definecolor{LovelyBrown}{HTML}{FDFCF5}

\usepackage[pdftex,
breaklinks=true,
bookmarksnumbered=true,
linktocpage=true,
linktoc=all
]{hyperref}

\begin{document}
\pagenumbering{gobble}
\pagestyle{plain}
\pagecolor{LovelyBrown}
\begin{talk}
{Структурная теорема для характерестических полиномов Лапласа циркулянтных графов}
{Медных Илья Александрович}
{Институт математики им. С.Л. Соболева СО РАН}
{ilyamednykh@mail.ru}
{}
{Комплексный анализ}

Данной доклад посвящен изучению характеристических полиномов матрицы Лапласа циркулянтных графов. Показано, что он представляется в виде конечного произведения алгебраических функций, вычисленных в корнях линейной комбинации полиномов Чебышева. Важным следствием полученного результата является свойство периодичности характеристических полиномов, вычисленных в предписанных целых числах. Также доказано, что с точностью до явно указанных линейных множителей, характеристические полиномы циркулянтных графов всегда являются полными квадратами. Указанный подход может быть применен для описания характерестических полиномов различных бесконечных семейств таких как тэта-графы,  гантельные графы, bug-графы и другие графы с малым числом вершин большой валентности.
\end{talk}
\end{document}