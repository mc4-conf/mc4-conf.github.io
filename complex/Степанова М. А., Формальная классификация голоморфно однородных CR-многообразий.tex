\documentclass[12pt]{article}
\usepackage{hyphsubst}
\usepackage[T2A]{fontenc}
\usepackage[english,main=russian]{babel}
\usepackage[utf8]{inputenc}
\usepackage[letterpaper,top=2cm,bottom=2cm,left=2cm,right=2cm,marginparwidth=2cm]{geometry}
\usepackage{float}
\usepackage{mathtools, commath, amssymb, amsthm}
\usepackage{enumitem, tabularx,graphicx,url,xcolor,rotating,multicol,epsfig,colortbl,lipsum}

\setlist{topsep=1pt, itemsep=0em}
\setlength{\parindent}{0pt}
\setlength{\parskip}{6pt}

\usepackage{hyphenat}
\hyphenation{ма-те-ма-ти-ка вос-ста-нав-ли-вать}

\usepackage[math]{anttor}

\newenvironment{talk}[6]{%
\vskip 0pt\nopagebreak%
\vskip 0pt\nopagebreak%
\section*{#1}
\phantomsection
\addcontentsline{toc}{section}{#2. \textit{#1}}
% \addtocontents{toc}{\textit{#1}\par}
\textit{#2}\\\nopagebreak%
#3\\\nopagebreak%
\ifthenelse{\equal{#4}{}}{}{\url{#4}\\\nopagebreak}%
\ifthenelse{\equal{#5}{}}{}{Соавторы: #5\\\nopagebreak}%
\ifthenelse{\equal{#6}{}}{}{Секция: #6\\\nopagebreak}%
}

\definecolor{LovelyBrown}{HTML}{FDFCF5}

\usepackage[pdftex,
breaklinks=true,
bookmarksnumbered=true,
linktocpage=true,
linktoc=all
]{hyperref}

\begin{document}
\pagenumbering{gobble}
\pagestyle{plain}
\pagecolor{LovelyBrown}
\begin{talk}
{Формальная классификация голоморфно однородных CR-многообразий}
{Степанова Мария Александровна}
{Математический институт им. В.\,А. Стеклова Российской академии наук}
{step_masha@mail.ru}
{}
{Комплексный анализ}

Мы покажем, что с точностью до формальной эквивалентности число параметров, задающих голоморфно однородное многообразие фиксированного CR-типа \((n,K)\), конечно (здесь \(n\) --- CR-размерность, \(K\) --- коразмерность). Этого естественно ожидать, исходя из имеющихся списков голоморфно однородных многообразий малых размерностей: списки состоят из семейств уравнений, зависящих лишь от конечного числа параметров (см. список литературы). Вопрос о конечности числа параметров возникает естественно еще и потому, что все имеющиеся классификации основаны на классификациях алгебр Ли малых размерностей (размерность алгебры должна быть равна размерности многообразия). А число параметров, задающих алгебры Ли фиксированной размерности, конечно (в качестве параметров можно выбрать структурные константы алгебры). Трудность состоит в том, что у одной алгебры может быть несколько неэквивалентных реализаций в виде векторных полей.

Также мы приведем оценку на число параметров, зависящую только от \(n\) и \(K\).

\medskip

Исследование выполнено за счет гранта Российского научного фонда № 24-11-00196, {\tt https://rscf.ru/project/24-11-00196/}.

\begin{enumerate}
\item[{[1]}] E. Cartan, ``Sur la g\'{e}om\'{e}trie pseudoconforme des hypersurfaces de l'espace de deux variables complexes'', Ann. Math. Pura Appl, 11:4 (1932), 17–90.
\item[{[2]}] V.K. Beloshapka, I.G. Kossovskiy, ``Classification Of Homogeneous CR-Manifolds In Dimension 4'', Journal of Mathematical Analysis and Application, 374 (2011) 655-672.
\item[{[3]}] А. В. Лобода, ``Голоморфно однородные вещественные гиперповерхности в C3'', Тр. ММО, 81, № 2, МЦНМО, М., 2020, 205–280.
\end{enumerate}
\end{talk}
\end{document}