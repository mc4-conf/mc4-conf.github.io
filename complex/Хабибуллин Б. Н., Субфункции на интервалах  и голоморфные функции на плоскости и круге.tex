\documentclass[12pt]{article}
\usepackage{hyphsubst}
\usepackage[T2A]{fontenc}
\usepackage[english,main=russian]{babel}
\usepackage[utf8]{inputenc}
\usepackage[letterpaper,top=2cm,bottom=2cm,left=2cm,right=2cm,marginparwidth=2cm]{geometry}
\usepackage{float}
\usepackage{mathtools, commath, amssymb, amsthm}
\usepackage{enumitem, tabularx,graphicx,url,xcolor,rotating,multicol,epsfig,colortbl,lipsum}

\setlist{topsep=1pt, itemsep=0em}
\setlength{\parindent}{0pt}
\setlength{\parskip}{6pt}

\usepackage{hyphenat}
\hyphenation{ма-те-ма-ти-ка вос-ста-нав-ли-вать}

\usepackage[math]{anttor}

\newenvironment{talk}[6]{%
\vskip 0pt\nopagebreak%
\vskip 0pt\nopagebreak%
\section*{#1}
\phantomsection
\addcontentsline{toc}{section}{#2. \textit{#1}}
% \addtocontents{toc}{\textit{#1}\par}
\textit{#2}\\\nopagebreak%
#3\\\nopagebreak%
\ifthenelse{\equal{#4}{}}{}{\url{#4}\\\nopagebreak}%
\ifthenelse{\equal{#5}{}}{}{Соавторы: #5\\\nopagebreak}%
\ifthenelse{\equal{#6}{}}{}{Секция: #6\\\nopagebreak}%
}

\definecolor{LovelyBrown}{HTML}{FDFCF5}

\usepackage[pdftex,
breaklinks=true,
bookmarksnumbered=true,
linktocpage=true,
linktoc=all
]{hyperref}

\begin{document}
\pagenumbering{gobble}
\pagestyle{plain}
\pagecolor{LovelyBrown}
\begin{talk}
{Субфункции на интервалах  и голоморфные функции на плоскости и круге}
{Хабибуллин Булат Нурмиевич}
{Институт математики с ВЦ Уфимского федерального исследовательского центра РАН}
{khabib-bulat@mail.ru}
{Р.\,Р. Мурясов}
{Комплексный анализ}

Пусть \(W\) --- класс непрерывных функций на интервале \(I\) вещественной оси \(\mathbb R\) со значениями в \(\mathbb R\). Функцию \(f\colon I\to \mathbb R\), следуя Ж.~Валирону и Э.~Бекенбаху, 1932--37 гг., называем \(W\)-субфункцией, если для  любых двух пар чисел  \(c_1, c_2\in W\) и функций \(w_1,w_2\in W\) и любого отрезка \([a,b]\subseteq I\) из неравенств \(f(x)\leqslant c_1w_1(x)+c_2w_2(x)\) при \(x:=a,b\) следует то же неравенство при вcех \(x\in [a,b]\). Планируется сначала дать обзор по различным классам таких \(W\)-субфункций, дополненный новыми результатами по ним.  Вторая часть доклада будет содержать применения различных классов  \(W\)-субфункций к голоморным функциям в круге и на комплексной плоскости \(\mathbb C\). Основную роль при этом будут играть  как давно используемые классы \(p\)-тригонометрически выпуклых функций при  \(p\geqslant 0\), так и недавно привлечённые нами к этим вопросам \(p\)-степенно и  \(p\)-гиперболически  выпуклые функции, а также иные  классы  \(W\)-субфункций. Классы субгармонических функций с разделёнными переменными, построенные как произведения субфункций, применяются нами к исследованию распределений корней голоморфных в круге и целых функций, аппроксимации экспоненциальными и более общими системами целых функций в разнообразных пространствах функций на компактах и областях   в \(\mathbb C\), к построению новых шкал роста целых или голоморфных в единичном круге функций.
\end{talk}
\end{document}