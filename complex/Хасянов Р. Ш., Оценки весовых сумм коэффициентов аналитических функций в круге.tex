\documentclass[12pt]{article}
\usepackage{hyphsubst}
\usepackage[T2A]{fontenc}
\usepackage[english,main=russian]{babel}
\usepackage[utf8]{inputenc}
\usepackage[letterpaper,top=2cm,bottom=2cm,left=2cm,right=2cm,marginparwidth=2cm]{geometry}
\usepackage{float}
\usepackage{mathtools, commath, amssymb, amsthm}
\usepackage{enumitem, tabularx,graphicx,url,xcolor,rotating,multicol,epsfig,colortbl,lipsum}

\setlist{topsep=1pt, itemsep=0em}
\setlength{\parindent}{0pt}
\setlength{\parskip}{6pt}

\usepackage{hyphenat}
\hyphenation{ма-те-ма-ти-ка вос-ста-нав-ли-вать}

\usepackage[math]{anttor}

\newenvironment{talk}[6]{%
\vskip 0pt\nopagebreak%
\vskip 0pt\nopagebreak%
\section*{#1}
\phantomsection
\addcontentsline{toc}{section}{#2. \textit{#1}}
% \addtocontents{toc}{\textit{#1}\par}
\textit{#2}\\\nopagebreak%
#3\\\nopagebreak%
\ifthenelse{\equal{#4}{}}{}{\url{#4}\\\nopagebreak}%
\ifthenelse{\equal{#5}{}}{}{Соавторы: #5\\\nopagebreak}%
\ifthenelse{\equal{#6}{}}{}{Секция: #6\\\nopagebreak}%
}

\definecolor{LovelyBrown}{HTML}{FDFCF5}

\usepackage[pdftex,
breaklinks=true,
bookmarksnumbered=true,
linktocpage=true,
linktoc=all
]{hyperref}

\begin{document}
\pagenumbering{gobble}
\pagestyle{plain}
\pagecolor{LovelyBrown}
\begin{talk}
{Оценки весовых сумм коэффициентов аналитических функций в круге}
{Хасянов Рамис Шавкятович}
{Санкт-Петербургский государственный университет}
{st070255@student.spbu.ru}
{}
{Комплексный анализ}

В докладе будет рассказано об оценках следующих функционалов в классах аналитических функций в круге:
\[\sum_{n\ge m}c_n|a_n|^2r^{2n} \:\: \text{и} \:\: \sum_{n\ge m}c_n|a_n|r^n, \quad 0\le r<1, \: m\ge 0.\]
Частными случаями этих сумм являются функционал площади, вторая норма на окружности  и мажорантный ряд.
Мы развиваем метод И.Р. Каюмова и С. Поннусами [2], используя в оценках результат Э. Райха [3], который обобщает теорему Голузина о мажорации подчинённых функций [4]. Сначала мы докажем общую теорему, после чего сформулируем важные следствия, например, мы докажем точную оценку площади образа круга радиуса \(r\) под действием ограниченной в единичном круге функции.

\medskip

\begin{enumerate}
\item[{[1]}] Khasyanov R., {\it The Bohr radius and the Hadamard convolution operator}, J. Math. Anal. Appl., 127782, (2024).
\item[{[2]}] Kayumov I., Ponnusamy S., {\it Bohr’s inequalities for the analytic functions with lacunary series and harmonic functions}, J. Math. Anal. Appl.,  V.465, No.2, (2018), 857–871.
\item[{[3]}] Reich E., {\it An inequality for subordinate analytic functions}, Pacific J. Math., V.4, No.2, (1954), 259–274.
\item[{[4]}]  Голузин Г. М., {\it О мажорации подчинённых функций}, Матем. сб., Т.29, №71, (1951), 209–224.
\end{enumerate}
\end{talk}
\end{document}