\documentclass[12pt]{article}
\usepackage{hyphsubst}
\usepackage[T2A]{fontenc}
\usepackage[english,main=russian]{babel}
\usepackage[utf8]{inputenc}
\usepackage[letterpaper,top=2cm,bottom=2cm,left=2cm,right=2cm,marginparwidth=2cm]{geometry}
\usepackage{float}
\usepackage{mathtools, commath, amssymb, amsthm}
\usepackage{enumitem, tabularx,graphicx,url,xcolor,rotating,multicol,epsfig,colortbl,lipsum}

\setlist{topsep=1pt, itemsep=0em}
\setlength{\parindent}{0pt}
\setlength{\parskip}{6pt}

\usepackage{hyphenat}
\hyphenation{ма-те-ма-ти-ка вос-ста-нав-ли-вать}

\usepackage[math]{anttor}

\newenvironment{talk}[6]{%
\vskip 0pt\nopagebreak%
\vskip 0pt\nopagebreak%
\section*{#1}
\phantomsection
\addcontentsline{toc}{section}{#2. \textit{#1}}
% \addtocontents{toc}{\textit{#1}\par}
\textit{#2}\\\nopagebreak%
#3\\\nopagebreak%
\ifthenelse{\equal{#4}{}}{}{\url{#4}\\\nopagebreak}%
\ifthenelse{\equal{#5}{}}{}{Соавторы: #5\\\nopagebreak}%
\ifthenelse{\equal{#6}{}}{}{Секция: #6\\\nopagebreak}%
}

\definecolor{LovelyBrown}{HTML}{FDFCF5}

\usepackage[pdftex,
breaklinks=true,
bookmarksnumbered=true,
linktocpage=true,
linktoc=all
]{hyperref}

\begin{document}
\pagenumbering{gobble}
\pagestyle{plain}
\pagecolor{LovelyBrown}
\begin{talk}
{Наилучшее приближение функций в пространстве Харди и точные значения \(n\)-поперечников некоторых классов аналитических функций}
{Шабозов Мирганд Шабозович}
{Таджикский национальный университет}
{shabozov@mail.ru}
{}
{Комплексный анализ}

В докладе излагаются решение экстремальных задач наилучших полиномиальных
приближений аналитических в круге \(U_{R}:=\{z\in\mathbb{C}: |z|<R\}\)
функций, принадлежащих пространству Харди \(H_{q,R}:=H_{q}(U_{R}), \
1\le q\le\infty.\)

Пусть \(H^{(r)}_{q,R}:=\bigl\{f\in H_{q,R}:\|f^{(r)}\|_{q,R}<\infty\bigr\}\).
Найдены точные неравенства между наилучшим полиномиальным приближением
функций \(f\in H_{q,\rho}^{(r)}\) \((r\in\mathbb{Z}_{+},\, 1\le q\le\infty,\,0<\rho< R)\)
и усредненным модулем гладкости угловых граничных значений производных \(r\)-го порядка \(f^{(r)}\in
H_{q,R}.\) Для класса \(W^{(r)}_{q,R}(\Phi)\) функций \(f\in H_{q,R}^{(r)},\)
для которых при любых \(k\in\mathbb{N},\) \(r\in\mathbb{Z}_{+},\) \(k>r\) усредненные
модули гладкости граничных значений производной \(r\)-го порядка \(f^{(r)},\)
мажорируемые в системе точек \(\{\pi/(2k)\}_{k\in\mathbb{N}},\) заданной мажорантой
\(\Phi\), вычислены точные значения различных \(n\)-поперечников в норме
пространства \(H_{q,\rho}\) \((1\le q\le\infty,\, 0<\rho<R).\)
\end{talk}
\end{document}