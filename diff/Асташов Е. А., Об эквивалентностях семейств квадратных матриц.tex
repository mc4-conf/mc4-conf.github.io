\documentclass[12pt]{article}
\usepackage{hyphsubst}
\usepackage[T2A]{fontenc}
\usepackage[english,main=russian]{babel}
\usepackage[utf8]{inputenc}
\usepackage[letterpaper,top=2cm,bottom=2cm,left=2cm,right=2cm,marginparwidth=2cm]{geometry}
\usepackage{float}
\usepackage{mathtools, commath, amssymb, amsthm}
\usepackage{enumitem, tabularx,graphicx,url,xcolor,rotating,multicol,epsfig,colortbl,lipsum}

\setlist{topsep=1pt, itemsep=0em}
\setlength{\parindent}{0pt}
\setlength{\parskip}{6pt}

\usepackage{hyphenat}
\hyphenation{ма-те-ма-ти-ка вос-ста-нав-ли-вать}

\usepackage[math]{anttor}

\newenvironment{talk}[6]{%
\vskip 0pt\nopagebreak%
\vskip 0pt\nopagebreak%
\section*{#1}
\phantomsection
\addcontentsline{toc}{section}{#2. \textit{#1}}
% \addtocontents{toc}{\textit{#1}\par}
\textit{#2}\\\nopagebreak%
#3\\\nopagebreak%
\ifthenelse{\equal{#4}{}}{}{\url{#4}\\\nopagebreak}%
\ifthenelse{\equal{#5}{}}{}{Соавторы: #5\\\nopagebreak}%
\ifthenelse{\equal{#6}{}}{}{Секция: #6\\\nopagebreak}%
}

\definecolor{LovelyBrown}{HTML}{FDFCF5}

\usepackage[pdftex,
breaklinks=true,
bookmarksnumbered=true,
linktocpage=true,
linktoc=all
]{hyperref}

\begin{document}
\pagenumbering{gobble}
\pagestyle{plain}
\pagecolor{LovelyBrown}
\begin{talk}
{Об эквивалентностях семейств квадратных матриц}
{Асташов Евгений Александрович}
{МГУ имени М.\,В. Ломоносова, механико-математический факультет}
{ast-ea@yandex.ru}
{Н.\,Т. Абдрахманова, А.\,В. Терентьев}
{Дифференциальные уравнения и динамические системы}

В работах [1-4] рассматриваются задачи классификации простых аналитических семейств квадратных, симметричных и кососимметричных матриц --- общего вида, а также с ограничениями типа (не)чётности по совокупности параметров. Такие матрицы можно рассматривать как матрицы линейных отображений/операторов либо как матрицы (косо)симметричных билинейных форм.

Доклад будет посвящен обсуждению различных отношений эквивалентности матричных семейств и соответствующих понятий простых ростков (т. е. ростков с конечным числом примыкающих орбит), а также сопоставлению решений соответствующих задач классификации.

\medskip

\begin{enumerate}
\item[{[1]}] J.\,W. Bruce, F. Tari, {\it On Families of Square Matrices}, Cadernos de Mathematica, 3 (2002), 217--242.
\item[{[2]}] J.\,W. Bruce, {\it On Families of Symmetric Matrices}, Moscow mathematical journal, 3 (2003), 335--360.
\item[{[3]}] G.\,J. Haslinger, {\it Families of Skew-symmetric Matrices}, Ph. D. thesis, University of Liverpool, 2001.
\item[{[4]}] N.\,T. Abdrakhmanova, E.\,A. Astashov, {\it Simple germs of skew-symmetric matrix families with oddness or evenness properties}, Journal of Mathematical Sciences, 270 (2023),  625-639.
\end{enumerate}
\end{talk}
\end{document}