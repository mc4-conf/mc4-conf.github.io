\documentclass[12pt]{article}
\usepackage{hyphsubst}
\usepackage[T2A]{fontenc}
\usepackage[english,main=russian]{babel}
\usepackage[utf8]{inputenc}
\usepackage[letterpaper,top=2cm,bottom=2cm,left=2cm,right=2cm,marginparwidth=2cm]{geometry}
\usepackage{float}
\usepackage{mathtools, commath, amssymb, amsthm}
\usepackage{enumitem, tabularx,graphicx,url,xcolor,rotating,multicol,epsfig,colortbl,lipsum}

\setlist{topsep=1pt, itemsep=0em}
\setlength{\parindent}{0pt}
\setlength{\parskip}{6pt}

\usepackage{hyphenat}
\hyphenation{ма-те-ма-ти-ка вос-ста-нав-ли-вать}

\usepackage[math]{anttor}

\newenvironment{talk}[6]{%
\vskip 0pt\nopagebreak%
\vskip 0pt\nopagebreak%
\section*{#1}
\phantomsection
\addcontentsline{toc}{section}{#2. \textit{#1}}
% \addtocontents{toc}{\textit{#1}\par}
\textit{#2}\\\nopagebreak%
#3\\\nopagebreak%
\ifthenelse{\equal{#4}{}}{}{\url{#4}\\\nopagebreak}%
\ifthenelse{\equal{#5}{}}{}{Соавторы: #5\\\nopagebreak}%
\ifthenelse{\equal{#6}{}}{}{Секция: #6\\\nopagebreak}%
}

\definecolor{LovelyBrown}{HTML}{FDFCF5}

\usepackage[pdftex,
breaklinks=true,
bookmarksnumbered=true,
linktocpage=true,
linktoc=all
]{hyperref}

\begin{document}
\pagenumbering{gobble}
\pagestyle{plain}
\pagecolor{LovelyBrown}
\begin{talk}
{О сингулярных решениях нелинейных дифференциальных уравнений типа Эмдена--Фаулера высокого порядка и их асимптотических свойствах}
{Асташова Ирина Викторовна}
{МГУ имени М.\,В.Ломоносова, РЭУ имени Г.\,В. Плеханова}
{ast.diffiety@gmail.com}
{}
{Дифференциальные уравнения и динамические системы}

Обсуждается существование и асимптотическое поведение сингулярных решений уравнения
\[y^{(n)}=p(x,y,y',\ldots,y^{(n-1)})|y|^k\ {\rm sign}\,y, \eqno(1)\]
где \(n\ge2\), \(k>0, k\neq 1\),
\(p\) --- положительная непрерывная функция, удовлетворяющая условию Липшица по последним \(n\) переменным.

Продолжено исследование  при \(k>1\) гипотезы И. Кигурадзе [1, Problem 16.4] о степенном асимптотическом поведении всех таких решений, имеющих ``blow-up'' в некоторой конечной точке.
Показано, что для слабо нелинейных уравнений эта гипотеза справедлива, а для сильно нелинейных уравнений степенное поведение таких решений становится нетипичным, если порядок уравнения \(n \geq 12\). Полученные результаты дополняют и расширяют результаты работы [2].

Будет обсуждаться вопрос о качественном и асимптотическом поведении сингулярных решений этого уравнения при \(0<k<1\).

\medskip

Работа выполнена при частичной поддержке РНФ (Проект 20-11-20272).

\begin{enumerate}
\item[{[1]}]  I.~T. Kiguradze, T.~A. Chanturia,
{\it Asymptotic properties of solutions of nonautonomous ordinary
differential equations.} Klu\-wer Aca\-de\-mic Publishers,
Dordrecht-Boston-London, 1993.
\item[{[2]}]Astashova I. {\it Atypicality of power-law solutions to Emden--Fowler type higher order equations} // St. Petersburg Mathematical Journal. {\bf 31} (2020), pp. 297--311.
\end{enumerate}
\end{talk}
\end{document}