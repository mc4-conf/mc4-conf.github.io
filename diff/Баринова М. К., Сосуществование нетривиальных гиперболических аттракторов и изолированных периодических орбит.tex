\documentclass[12pt]{article}
\usepackage{hyphsubst}
\usepackage[T2A]{fontenc}
\usepackage[english,main=russian]{babel}
\usepackage[utf8]{inputenc}
\usepackage[letterpaper,top=2cm,bottom=2cm,left=2cm,right=2cm,marginparwidth=2cm]{geometry}
\usepackage{float}
\usepackage{mathtools, commath, amssymb, amsthm}
\usepackage{enumitem, tabularx,graphicx,url,xcolor,rotating,multicol,epsfig,colortbl,lipsum}

\setlist{topsep=1pt, itemsep=0em}
\setlength{\parindent}{0pt}
\setlength{\parskip}{6pt}

\usepackage{hyphenat}
\hyphenation{ма-те-ма-ти-ка вос-ста-нав-ли-вать}

\usepackage[math]{anttor}

\newenvironment{talk}[6]{%
\vskip 0pt\nopagebreak%
\vskip 0pt\nopagebreak%
\section*{#1}
\phantomsection
\addcontentsline{toc}{section}{#2. \textit{#1}}
% \addtocontents{toc}{\textit{#1}\par}
\textit{#2}\\\nopagebreak%
#3\\\nopagebreak%
\ifthenelse{\equal{#4}{}}{}{\url{#4}\\\nopagebreak}%
\ifthenelse{\equal{#5}{}}{}{Соавторы: #5\\\nopagebreak}%
\ifthenelse{\equal{#6}{}}{}{Секция: #6\\\nopagebreak}%
}

\definecolor{LovelyBrown}{HTML}{FDFCF5}

\usepackage[pdftex,
breaklinks=true,
bookmarksnumbered=true,
linktocpage=true,
linktoc=all
]{hyperref}

\begin{document}
\pagenumbering{gobble}
\pagestyle{plain}
\pagecolor{LovelyBrown}
\begin{talk}
{Сосуществование нетривиальных гиперболических аттракторов и изолированных периодических орбит}
{Баринова Марина Константиновна}
{НИУ ВШЭ, Нижний Новгород}
{mkbarinova@yandex.ru}
{}
{Дифференциальные уравнения и динамические системы}

Из результатов А. Брауна 2010 года известно, что собственные нетривиальные гиперболические аттракторы \(\Omega\)-устойчивых 3-диффеоморфизмов могут быть лишь двух типов: растягивающиеся аттракторы (одномерные и двумерные, ориентируемые и неориентируемые), топологическая размерность которых совпадает с размерностью неустойчивых многообразий точек аттрактора, и двумерные Аносовские торы --- ручно вложенные 2-торы, ограничение диффеоморфизма на которые сопряжено с гиперболическим автоморфизмом тора. В докладе будут приведены результаты работы [1], в которой было показано, что если все нетривиальные множества \(\Omega\)-усточивого диффеоморфизма являются аттракторами, то они не могут быть Аносовскими торами и одномерными ориентируемыми растягивающимися аттракторами. Также были получены нижние оценки на количество изолированных периодических орбит для диффеоморфизмов, все нетривиальные базисные множества которых являются двумерными аттракторами [2].

\medskip

Исследование осуществлено в рамках Программы фундаментальных исследований НИУ ВШЭ.

\begin{enumerate}
\item[{[1]}] M.K. Barinova, O.V. Pochinka, E.I. Yakovlev, {\it On a structure of non-wandering set of an $\Omega$-stable 3-diffeomorphism possessing a hyperbolic attractor}, Discrete and Continuous Dynamical Systems, 44-1 (2024), 1-17.
\item[{[2]}] M. Barinova, {\it On isolated periodic points of diffeomorphisms with expanding attractors of codimension 1}, Cornell University. Series math "arxiv.org", 2024.
\end{enumerate}
\end{talk}
\end{document}