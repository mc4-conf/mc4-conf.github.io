\documentclass[12pt]{article}
\usepackage{hyphsubst}
\usepackage[T2A]{fontenc}
\usepackage[english,main=russian]{babel}
\usepackage[utf8]{inputenc}
\usepackage[letterpaper,top=2cm,bottom=2cm,left=2cm,right=2cm,marginparwidth=2cm]{geometry}
\usepackage{float}
\usepackage{mathtools, commath, amssymb, amsthm}
\usepackage{enumitem, tabularx,graphicx,url,xcolor,rotating,multicol,epsfig,colortbl,lipsum}

\setlist{topsep=1pt, itemsep=0em}
\setlength{\parindent}{0pt}
\setlength{\parskip}{6pt}

\usepackage{hyphenat}
\hyphenation{ма-те-ма-ти-ка вос-ста-нав-ли-вать}

\usepackage[math]{anttor}

\newenvironment{talk}[6]{%
\vskip 0pt\nopagebreak%
\vskip 0pt\nopagebreak%
\section*{#1}
\phantomsection
\addcontentsline{toc}{section}{#2. \textit{#1}}
% \addtocontents{toc}{\textit{#1}\par}
\textit{#2}\\\nopagebreak%
#3\\\nopagebreak%
\ifthenelse{\equal{#4}{}}{}{\url{#4}\\\nopagebreak}%
\ifthenelse{\equal{#5}{}}{}{Соавторы: #5\\\nopagebreak}%
\ifthenelse{\equal{#6}{}}{}{Секция: #6\\\nopagebreak}%
}

\definecolor{LovelyBrown}{HTML}{FDFCF5}

\usepackage[pdftex,
breaklinks=true,
bookmarksnumbered=true,
linktocpage=true,
linktoc=all
]{hyperref}

\begin{document}
\pagenumbering{gobble}
\pagestyle{plain}
\pagecolor{LovelyBrown}
\begin{talk}
{Динамика велосипеда: регулярные и хаотические траектории}
{Бизяев Иван Алексеевич}
{Уральский математический центр, Удмуртский государственный университет}
{bizyaevtheory@gmail.com}
{Бердникова А.\,С.}
{Дифференциальные уравнения и динамические системы}

Рассмотрена задача о движении по инерции роликового велосипеда на горизонтальной плоскости.
В этой модели велосипед представляет собой связку двух твердых тел (рама и руль), в которой каждое тело опирается на горизонтальную плоскость лезвием или роликом,
которое препятствует движению тела в фиксированном направлении [1, 2].
Получена математическая модель, описывающая динамику данной неголономной модели велосипеда.
Она сводится к анализу пяти нелинейных дифференциальных уравнений, описывающих эволюцию поступательной скорости точки контакта руля, двух компонент угловых скоростей и углов поворота руля относительно рамы и наклона рамы.
Проанализирован размер области начальных условий в фазовом пространстве исходной нелинейной редуцированной системы, из которой траектории асимптотически стремятся к прямолинейному движению (бассейн притяжения).
Показано, что при некотором выборе параметров существует достаточно большие начальные значения для углов поворота руля и наклона рамы, при которых велосипед возвращается к прямолинейному движению.
Найдены условия, в которых прямолинейное движение теряет устойчивость после бифуркации Андронова-Хопфа. В результате  в редуцированной системе возникает устойчивое периодическое решение.

\medskip

\begin{enumerate}
\item[{[1]}] J. D. G. Kooijman, J. P. Meijaard, J. M. Papadopoulos, A. Ruina,  A. L. Schwab, {\it A Bicycle Can
Be Self-Stable without Gyroscopic or Caster Effects}, Science, 2011, vol. 332, no. 6027, pp. 339–342.
\item[{[2]}] R. S. Hand, {\it Comparisons and Stability Analysis of Linearized Equations of Motion for a Basic Bicycle
Model}, Master’s Thesis, Ithaca, N.Y., Cornell Univ., 1988, 200 pp.
\end{enumerate}
\end{talk}
\end{document}