\documentclass[12pt]{article}
\usepackage{hyphsubst}
\usepackage[T2A]{fontenc}
\usepackage[english,main=russian]{babel}
\usepackage[utf8]{inputenc}
\usepackage[letterpaper,top=2cm,bottom=2cm,left=2cm,right=2cm,marginparwidth=2cm]{geometry}
\usepackage{float}
\usepackage{mathtools, commath, amssymb, amsthm}
\usepackage{enumitem, tabularx,graphicx,url,xcolor,rotating,multicol,epsfig,colortbl,lipsum}

\setlist{topsep=1pt, itemsep=0em}
\setlength{\parindent}{0pt}
\setlength{\parskip}{6pt}

\usepackage{hyphenat}
\hyphenation{ма-те-ма-ти-ка вос-ста-нав-ли-вать}

\usepackage[math]{anttor}

\newenvironment{talk}[6]{%
\vskip 0pt\nopagebreak%
\vskip 0pt\nopagebreak%
\section*{#1}
\phantomsection
\addcontentsline{toc}{section}{#2. \textit{#1}}
% \addtocontents{toc}{\textit{#1}\par}
\textit{#2}\\\nopagebreak%
#3\\\nopagebreak%
\ifthenelse{\equal{#4}{}}{}{\url{#4}\\\nopagebreak}%
\ifthenelse{\equal{#5}{}}{}{Соавторы: #5\\\nopagebreak}%
\ifthenelse{\equal{#6}{}}{}{Секция: #6\\\nopagebreak}%
}

\definecolor{LovelyBrown}{HTML}{FDFCF5}

\usepackage[pdftex,
breaklinks=true,
bookmarksnumbered=true,
linktocpage=true,
linktoc=all
]{hyperref}

\begin{document}
\pagenumbering{gobble}
\pagestyle{plain}
\pagecolor{LovelyBrown}
\begin{talk}
{Инвариантные множества диффеоморфизмов с гомоклиническими точками}
{Васильева Екатерина Викторовна}
{Санкт-Петербургский государственный университет}
{e.v.vasilieva@spbu.ru}
{}
{Дифференциальные уравнения и динамические системы}

Рассматривается диффеоморфизм \(f\) плоскости в себя с неподвижной гиперболической точкой в начале координат и нетрансверсальной гомоклинической к ней точкой. Предполагается, что собственные числа матрицы \(D f(0)\)  \(\lambda\), \(\mu\)  положительны и \(\lambda{\mu}<1\).

В работах [1], [2] изучалась окрестность нетрансверсальной гомоклинической точки, в предположении, что касание  устойчивого и неустойчивого многообразия в гомолинической точке является касанием конечного порядка. Из этих работ следует, что в окрестности гомоклинической точки может лежать  бесконечное множество устойчивых двухобходных и трехобходных периодических точек. Существование бесконечного множества устойчивых перидических траекторий зависит от  значения величины \((-\ln \lambda)(\ln \mu)^{-1}\).

Предполагается, что касание устойчивого многообразия с неустойчивым в гомоклинической точке не является касанием конечного порядка. Пример двумерного диффеоморфизма с таким касанием устойчивого многообразия с неустойчивым приведен в [3]. Известно [4], что в произвольной окрестности гомоклинической точки может лежать бесконечное множество однообходных устойчивых периодических точек исходного диффеоморфизма, причем характеристические показатели этих точек отделены от нуля. В этом случае существование бесконечного множества устойчивых перидических траекторий не зависит от  значения величины \((-\ln \lambda)(\ln \mu)^{-1}\).

Цель доклада --- показать, что в окрестности нетрансверсальной гомоклинической точки  могут лежать инвариантные множества диффеоморфизма  \(f\). Каждое из множеств включает в себя подкову Смейла и бесконечое   множество  таких кривых, что  траектории точек, принадлежащих этим кривым, не покидают расширенной окрестности нетрансверсальной гомоклинической точки.

\medskip

\begin{enumerate}
\item[{[1]}]Иванов~Б.\,Ф.,  { \it Устойчивость траекторий, не покидающих окрестность гомоклинической кривой},   Дифференц. уравнения, 15 (1979), 8, 1411-1419.
\item[{[2]}] Гонченко~С.\,В., Тураев~Д.\,В. Шильников~Л.\,П., { \it Динамические явления в многомерных системах с негрубой гомоклинической точкой},  Докл. РАН, 330 (1993), 2, 144-147.
\item[{[3]}] Плисс~ В. \,А. {\it Интегральные множества периодических систем дифференциальных уравнений},  Наука, 1977.
\item[{[4]}] Васильева ~Е.\,В. {\it Диффеоморфизмы плоскости с устойчивыми периодическими точками}, Дифференц. уравнения, 48 (2012), 3, 307-315.
\end{enumerate}
\end{talk}
\end{document}