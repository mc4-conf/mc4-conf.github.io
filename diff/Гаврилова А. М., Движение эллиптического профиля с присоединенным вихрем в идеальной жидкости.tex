\documentclass[12pt]{article}
\usepackage{hyphsubst}
\usepackage[T2A]{fontenc}
\usepackage[english,main=russian]{babel}
\usepackage[utf8]{inputenc}
\usepackage[letterpaper,top=2cm,bottom=2cm,left=2cm,right=2cm,marginparwidth=2cm]{geometry}
\usepackage{float}
\usepackage{mathtools, commath, amssymb, amsthm}
\usepackage{enumitem, tabularx,graphicx,url,xcolor,rotating,multicol,epsfig,colortbl,lipsum}

\setlist{topsep=1pt, itemsep=0em}
\setlength{\parindent}{0pt}
\setlength{\parskip}{6pt}

\usepackage{hyphenat}
\hyphenation{ма-те-ма-ти-ка вос-ста-нав-ли-вать}

\usepackage[math]{anttor}

\newenvironment{talk}[6]{%
\vskip 0pt\nopagebreak%
\vskip 0pt\nopagebreak%
\section*{#1}
\phantomsection
\addcontentsline{toc}{section}{#2. \textit{#1}}
% \addtocontents{toc}{\textit{#1}\par}
\textit{#2}\\\nopagebreak%
#3\\\nopagebreak%
\ifthenelse{\equal{#4}{}}{}{\url{#4}\\\nopagebreak}%
\ifthenelse{\equal{#5}{}}{}{Соавторы: #5\\\nopagebreak}%
\ifthenelse{\equal{#6}{}}{}{Секция: #6\\\nopagebreak}%
}

\definecolor{LovelyBrown}{HTML}{FDFCF5}

\usepackage[pdftex,
breaklinks=true,
bookmarksnumbered=true,
linktocpage=true,
linktoc=all
]{hyperref}

\begin{document}
\pagenumbering{gobble}
\pagestyle{plain}
\pagecolor{LovelyBrown}
\begin{talk}
{Движение эллиптического профиля с присоединенным вихрем в идеальной жидкости}
{Гаврилова Анна Михайловна}
{Уральский математический центр, УдГУ}
{Ann.gavrilova5@mail.ru}
{Артемова Елизавета Марковна}
{Дифференциальные уравнения и динамические системы}

Рассматривается движение эллиптического профиля в идеальной несжимаемой жидкости в предположении, что с профилем связан вихрь интенсивности \(\Gamma\), расположенный на некотором расстоянии от него. Используя подход предложенный Седовым [1] были получены выражения для сил и момента, действующих на профиль со стороны жидкости. Построены уравнения движения эллиптического профиля с присоединенным вихрем.

Показано, что случае постоянной интенсивности вихря \(\Gamma = \mathrm{const}\) в рассматриваемой системе существует два частных случая, в которых система интегрируема. Для каждого случая предложена процедура редукции, показано существование неподвижных точек, соответствующих периодическому движению профиля с вихрем. Приведены бифуркационные диаграммы и указаны характерные фазовые портреты. В общем случае показано, что в системе существует неустойчивый предельный цикл.

\medskip

\begin{enumerate}
\item[{[1]}] Седов Л. И. Плоские задачи гидродинамики и аэродинамики. – Наука. Гл. ред. физ.-мат. лит., 1966.
\end{enumerate}
\end{talk}
\end{document}