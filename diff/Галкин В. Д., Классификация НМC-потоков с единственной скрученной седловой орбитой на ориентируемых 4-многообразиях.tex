\documentclass[12pt]{article}
\usepackage{hyphsubst}
\usepackage[T2A]{fontenc}
\usepackage[english,main=russian]{babel}
\usepackage[utf8]{inputenc}
\usepackage[letterpaper,top=2cm,bottom=2cm,left=2cm,right=2cm,marginparwidth=2cm]{geometry}
\usepackage{float}
\usepackage{mathtools, commath, amssymb, amsthm}
\usepackage{enumitem, tabularx,graphicx,url,xcolor,rotating,multicol,epsfig,colortbl,lipsum}

\setlist{topsep=1pt, itemsep=0em}
\setlength{\parindent}{0pt}
\setlength{\parskip}{6pt}

\usepackage{hyphenat}
\hyphenation{ма-те-ма-ти-ка вос-ста-нав-ли-вать}

\usepackage[math]{anttor}

\newenvironment{talk}[6]{%
\vskip 0pt\nopagebreak%
\vskip 0pt\nopagebreak%
\section*{#1}
\phantomsection
\addcontentsline{toc}{section}{#2. \textit{#1}}
% \addtocontents{toc}{\textit{#1}\par}
\textit{#2}\\\nopagebreak%
#3\\\nopagebreak%
\ifthenelse{\equal{#4}{}}{}{\url{#4}\\\nopagebreak}%
\ifthenelse{\equal{#5}{}}{}{Соавторы: #5\\\nopagebreak}%
\ifthenelse{\equal{#6}{}}{}{Секция: #6\\\nopagebreak}%
}

\definecolor{LovelyBrown}{HTML}{FDFCF5}

\usepackage[pdftex,
breaklinks=true,
bookmarksnumbered=true,
linktocpage=true,
linktoc=all
]{hyperref}

\begin{document}
\pagenumbering{gobble}
\pagestyle{plain}
\pagecolor{LovelyBrown}
\begin{talk}
{Классификация НМC-потоков с единственной скрученной седловой орбитой на ориентируемых 4-многообразиях}
{Галкин Владислав Дмитриевич}
{НИУ ВШЭ, Нижний Новгород}
{vgalkin@hse.ru}
{Починка О.\,В., Шубин Д.\,Д.}
{Дифференциальные уравнения и динамические системы}

Топологической эквивалентности потоков Морса-Смейла без неподвижных точек (НМС-потоков) в предположениях  различной общности посвящен целый ряд статей. В случае малого числа орбит для таких потоков иногда удается получить исчерпывающую классификацию, с перечислением всех типов многообразий, допускающих рассматриваемые потоки, всех классов эквивалентности на допустимом многообразии, а также с решением задачи реализации. Настоящая статья также относится к циклу таких работ. Именно, рассмотрен класс НМС-потоков с единственной седловой орбитой, в предположении, что она скрученная, на замкнутых ориентируемых 4-многообразиях. Доказано, что единственным 4-многообразием, допускающим рассматриваемые потоки является многообразие \(\mathbb S^3\times\mathbb S^1\). Также установлено, что такие потоки разбиваются в точности на восемь классов  эквивалентности и в каждом классе эквивалентности построен стандартный представитель.

\textit{Исследование осуществлено в рамках Программы фундаментальных исследований НИУ ВШЭ.}
\end{talk}
\end{document}