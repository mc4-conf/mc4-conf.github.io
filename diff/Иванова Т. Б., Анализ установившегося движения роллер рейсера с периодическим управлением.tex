\documentclass[12pt]{article}
\usepackage{hyphsubst}
\usepackage[T2A]{fontenc}
\usepackage[english,main=russian]{babel}
\usepackage[utf8]{inputenc}
\usepackage[letterpaper,top=2cm,bottom=2cm,left=2cm,right=2cm,marginparwidth=2cm]{geometry}
\usepackage{float}
\usepackage{mathtools, commath, amssymb, amsthm}
\usepackage{enumitem, tabularx,graphicx,url,xcolor,rotating,multicol,epsfig,colortbl,lipsum}

\setlist{topsep=1pt, itemsep=0em}
\setlength{\parindent}{0pt}
\setlength{\parskip}{6pt}

\usepackage{hyphenat}
\hyphenation{ма-те-ма-ти-ка вос-ста-нав-ли-вать}

\usepackage[math]{anttor}

\newenvironment{talk}[6]{%
\vskip 0pt\nopagebreak%
\vskip 0pt\nopagebreak%
\section*{#1}
\phantomsection
\addcontentsline{toc}{section}{#2. \textit{#1}}
% \addtocontents{toc}{\textit{#1}\par}
\textit{#2}\\\nopagebreak%
#3\\\nopagebreak%
\ifthenelse{\equal{#4}{}}{}{\url{#4}\\\nopagebreak}%
\ifthenelse{\equal{#5}{}}{}{Соавторы: #5\\\nopagebreak}%
\ifthenelse{\equal{#6}{}}{}{Секция: #6\\\nopagebreak}%
}

\definecolor{LovelyBrown}{HTML}{FDFCF5}

\usepackage[pdftex,
breaklinks=true,
bookmarksnumbered=true,
linktocpage=true,
linktoc=all
]{hyperref}

\begin{document}
\pagenumbering{gobble}
\pagestyle{plain}
\pagecolor{LovelyBrown}
\begin{talk}
{Анализ установившегося движения роллер рейсера с периодическим управлением}
{Иванова Татьяна Борисовна}
{Ижевский государственный технический университет имени М.\,Т. Калашникова}
{tbesp@rcd.ru}
{А.\,А. Килин}
{Дифференциальные уравнения и динамические системы}

В этой работе мы рассматриваем задачу об управляемом движении роллер рейсера по плоскости.Это система состоит из двух соединенных между собой (с помощью цилиндрического шарнира) платформ, на  каждой из которой находится жестко закрепленная колесная пара. При этом предполагается, что в точках контакта колес с плоскостью выполняются  условия непроскальзывания (неголономная связь) и действует   сила вязкого трения.  Продвижение роллер рейсера реализуется за счет периодических колебаний платформ относительно друг друга. Мы предполагаем, что угол  между платформами (управляющая функция) является заданной периодической   функцией времени. В работе [1] показано, что  этом случае все траектории  приведенной системы асимптотически стремятся к периодическим решениям.В данной работе исследуются траектории роллер рейсера, соответствующие этим периодическим решениям, в зависимости от параметров управления и массо-геометрических характеристик системы. Определены типы возможных траекторий и проанализирована зависимость средней скорости продвижения вдоль прямой от параметров.

\medskip

\begin{enumerate}
\item[{[1]}] I. A. Bizyaev, A. V. Borisov, I. S. Mamaev,
{\it Exotic dynamics of  nonholonomic roller racer with periodic control},  Regul. Chaotic Dyn., 23(7) (2018), 983--994.
\end{enumerate}
\end{talk}
\end{document}