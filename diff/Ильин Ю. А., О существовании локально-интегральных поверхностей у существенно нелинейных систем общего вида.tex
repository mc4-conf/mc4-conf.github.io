\documentclass[12pt]{article}
\usepackage{hyphsubst}
\usepackage[T2A]{fontenc}
\usepackage[english,main=russian]{babel}
\usepackage[utf8]{inputenc}
\usepackage[letterpaper,top=2cm,bottom=2cm,left=2cm,right=2cm,marginparwidth=2cm]{geometry}
\usepackage{float}
\usepackage{mathtools, commath, amssymb, amsthm}
\usepackage{enumitem, tabularx,graphicx,url,xcolor,rotating,multicol,epsfig,colortbl,lipsum}

\setlist{topsep=1pt, itemsep=0em}
\setlength{\parindent}{0pt}
\setlength{\parskip}{6pt}

\usepackage{hyphenat}
\hyphenation{ма-те-ма-ти-ка вос-ста-нав-ли-вать}

\usepackage[math]{anttor}

\newenvironment{talk}[6]{%
\vskip 0pt\nopagebreak%
\vskip 0pt\nopagebreak%
\section*{#1}
\phantomsection
\addcontentsline{toc}{section}{#2. \textit{#1}}
% \addtocontents{toc}{\textit{#1}\par}
\textit{#2}\\\nopagebreak%
#3\\\nopagebreak%
\ifthenelse{\equal{#4}{}}{}{\url{#4}\\\nopagebreak}%
\ifthenelse{\equal{#5}{}}{}{Соавторы: #5\\\nopagebreak}%
\ifthenelse{\equal{#6}{}}{}{Секция: #6\\\nopagebreak}%
}

\definecolor{LovelyBrown}{HTML}{FDFCF5}

\usepackage[pdftex,
breaklinks=true,
bookmarksnumbered=true,
linktocpage=true,
linktoc=all
]{hyperref}

\begin{document}
\pagenumbering{gobble}
\pagestyle{plain}
\pagecolor{LovelyBrown}
\begin{talk}
{О существовании локально-интегральных поверхностей у существенно нелинейных систем общего вида}
{Ильин Юрий Анатольевич}
{Санкт-Петербургский Государственный университет, математико-механический факультет}
{iljin_y_a@mail.ru}
{}
{Дифференциальные уравнения и динамические системы}

В докладе рассматривается существенно нелинейная система обыкновенных дифференциальных уравнений \(\dot x=X(t,x,y), \dot y=Yt,x,y)\)  в окрестности нулевого решения. Правые части предполагаются непрерывными по всем аргументам и непрерывно дифференцируемыми по \(x\) и \(y\). Термин существенно нелинейная означает, что матрицы Якоби правых  частей обращаются в ноль на нулевом решении. Тем не менее, для таких  систем можно доказывать теоремы о существовании устойчивых и неустойчивых локально-интегральных поверхностей, являющихся аналогами известных теорем Ляпунова и Перрона для систем с невырожденным линейным приближением. Ключевые условия для существенно нелинейных систем формулируются на языке логарифмических норм матриц Якоби, которые как бы заменяют условия на собственные числа или характеристические показатели, используемые для квазилинейных систем. Очень ценно, что условия на логарифмические нормы являются коэффициентно проверяемыми условиями. Интегральные поверхности играют важную роль как в локально-качественной теории динамических систем, так и в задаче об устойчивости (в том числе и условной) нулевого решения, так что результаты, о которых планируется рассказать, могут иметь самое широкое применение.

Как правило, при доказательстве теорем о существовании локально-интегральных поверхностей существенно используется блочно-диагональный вид системы первого приближения. Для квазилинейных систем это не является ограничением и всегда может быть достигнуто приведением или к Жордановой форме (в автономном случае) или преобразованием Перрона (в неавтономном). Для существенно нелинейных систем такое преобразование в принципе невозможно. Поэтому приходиться рассматривать системы, являющиеся как бы возмущением блочно-диагональной системы. Год назад автору удалось несколько отойти от этого ограничения и найти условия существование интегральных поверхностей у систем, не являющихся возмущением блочно-диагональных. Об этом также планируется рассказать в докладе.
\end{talk}
\end{document}