\documentclass[12pt]{article}
\usepackage{hyphsubst}
\usepackage[T2A]{fontenc}
\usepackage[english,main=russian]{babel}
\usepackage[utf8]{inputenc}
\usepackage[letterpaper,top=2cm,bottom=2cm,left=2cm,right=2cm,marginparwidth=2cm]{geometry}
\usepackage{float}
\usepackage{mathtools, commath, amssymb, amsthm}
\usepackage{enumitem, tabularx,graphicx,url,xcolor,rotating,multicol,epsfig,colortbl,lipsum}

\setlist{topsep=1pt, itemsep=0em}
\setlength{\parindent}{0pt}
\setlength{\parskip}{6pt}

\usepackage{hyphenat}
\hyphenation{ма-те-ма-ти-ка вос-ста-нав-ли-вать}

\usepackage[math]{anttor}

\newenvironment{talk}[6]{%
\vskip 0pt\nopagebreak%
\vskip 0pt\nopagebreak%
\section*{#1}
\phantomsection
\addcontentsline{toc}{section}{#2. \textit{#1}}
% \addtocontents{toc}{\textit{#1}\par}
\textit{#2}\\\nopagebreak%
#3\\\nopagebreak%
\ifthenelse{\equal{#4}{}}{}{\url{#4}\\\nopagebreak}%
\ifthenelse{\equal{#5}{}}{}{Соавторы: #5\\\nopagebreak}%
\ifthenelse{\equal{#6}{}}{}{Секция: #6\\\nopagebreak}%
}

\definecolor{LovelyBrown}{HTML}{FDFCF5}

\usepackage[pdftex,
breaklinks=true,
bookmarksnumbered=true,
linktocpage=true,
linktoc=all
]{hyperref}

\begin{document}
\pagenumbering{gobble}
\pagestyle{plain}
\pagecolor{LovelyBrown}
\begin{talk}
{Шильниковский хаос в модели роста раковых клеток}
{Каратецкая Ефросиния Юрьевна}
{Национальный исследовательский университет ``Высшая школа экономики''}
{ekarateczkaya@hse.ru}
{}
{Дифференциальные уравнения и диннамические системы}

Моделирование роста раковых клеток --- одна из важнейших задач в области изучения живых систем. Это важный инструмент в исследовании рака и разработке новых методов лечения, позволяющий понять механизмы взаимодействия инфицированных клеток с окружающими тканями и оценить влияние различных внешних и внутренних факторов на их рост.
В данном докладе будут представлены результаты исследования хаотической динамики в модели де Пиллиса и Радунской, описывающей взаимодействие раковых клеток с двумя типами эффекторных клеток [1,2]:
\begin{equation}
\begin{cases}
\dot x_1 = x_1(1 - x_1) - a_{12}x_1x_2 - a_{13}x_1x_3,\\
\dot x_2 = r_2x_2(1 - x_2) - a_{21}x_1x_2,\\
\dot x_3 = \frac{r_3x_1x_3}{x_1 + k_3} - a_{31}x_1x_3 - d_3x_3
\end{cases}
\label{mainEq}
\end{equation}

Из работ [3,4] известно, что в системе (1) может возникнуть спиральный хаос, связанный с возникновением седло-фокусной петли Шильникова. Показано, что странные аттракторы рождаются в результате реализации сценария Шильникова [5]. Основная часть работы посвящена изучению бифуркаций коразмерности два, которые являются организационными центрами в рассматриваемой системе. В частности, описывается сценарий бифуркации состояния равновесия в случае, когда оно имеет пару нулевых собственных значений (бифуркация Богданова-Такенса), а также ноль и пару чисто мнимых собственных значений (бифуркация Ноль-Хопф). Показано, как эти бифуркации связаны с возникновением аттракторов Шильникова.

\medskip

Данная работа подготовлена в ходе проведения исследования в рамках проекта ``Зеркальные лаборатории НИУ ВШ''.

\begin{enumerate}
\item[{[1]}] De Pillis, L. G.; Radunskaya, A. (2001). A Mathematical Tumor Model with Immune Resistance and Drug Therapy: An Optimal Control Approach. Journal of Theoretical Medicine, 3(2), 79–100.
\item[{[2]}] de Pillis L.G., Radunskaya A. (2003). The dynamics of an optimally controlled tumor model: A case study.  Math. Comp. Modelling 37(11), 1221, 1221-1244.
\item[{[3]}] Itik, M. \& Banks, S. P. (2010 ). Chaos in a threedimensional cancer model, Int. J. Bifurcation and Chaos 20, 71–79.
\item[{[4]}] Duarte, J.; Januário, C.; Rodrigues, C.; Sardanyés, J.(2013). Topological Complexity And Predictability In The Dynamics Of A Tumor Growth Model With Shilnikov's Chaos. International Journal Of Bifurcation And Chaos, 23(7), 1350124.
\item[{[5]}] Shilnikov L.P. (1965). On one case of the existence of a countable set of periodic motions. Dokl. Akad. Nauk SSSR., 169, 3, 558–561
\end{enumerate}
\end{talk}
\end{document}