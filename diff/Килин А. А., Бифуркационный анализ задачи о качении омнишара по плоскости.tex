\documentclass[12pt]{article}
\usepackage{hyphsubst}
\usepackage[T2A]{fontenc}
\usepackage[english,main=russian]{babel}
\usepackage[utf8]{inputenc}
\usepackage[letterpaper,top=2cm,bottom=2cm,left=2cm,right=2cm,marginparwidth=2cm]{geometry}
\usepackage{float}
\usepackage{mathtools, commath, amssymb, amsthm}
\usepackage{enumitem, tabularx,graphicx,url,xcolor,rotating,multicol,epsfig,colortbl,lipsum}

\setlist{topsep=1pt, itemsep=0em}
\setlength{\parindent}{0pt}
\setlength{\parskip}{6pt}

\usepackage{hyphenat}
\hyphenation{ма-те-ма-ти-ка вос-ста-нав-ли-вать}

\usepackage[math]{anttor}

\newenvironment{talk}[6]{%
\vskip 0pt\nopagebreak%
\vskip 0pt\nopagebreak%
\section*{#1}
\phantomsection
\addcontentsline{toc}{section}{#2. \textit{#1}}
% \addtocontents{toc}{\textit{#1}\par}
\textit{#2}\\\nopagebreak%
#3\\\nopagebreak%
\ifthenelse{\equal{#4}{}}{}{\url{#4}\\\nopagebreak}%
\ifthenelse{\equal{#5}{}}{}{Соавторы: #5\\\nopagebreak}%
\ifthenelse{\equal{#6}{}}{}{Секция: #6\\\nopagebreak}%
}

\definecolor{LovelyBrown}{HTML}{FDFCF5}

\usepackage[pdftex,
breaklinks=true,
bookmarksnumbered=true,
linktocpage=true,
linktoc=all
]{hyperref}

\begin{document}
\pagenumbering{gobble}
\pagestyle{plain}
\pagecolor{LovelyBrown}
\begin{talk}
{Бифуркационный анализ задачи о качении омнишара по плоскости}
{Килин Александр Александрович}
{Удмуртский государственный университет}
{kilin@rcd.ru}
{Т.\,Б. Иванова}
{Дифференциальные уравнения и динамические системы}

Рассмотрена задача о тяжелом неуравновешенном шаре  с осесимметричным распределением масс (сферический волчок), который катится по горизонтальной плоскости с частичным проскальзыванием. Полагается, что шар не проскальзывает (катится) в направлении проекции оси симметрии на опорную плоскость. При этом в направлении, перпендикулярном к указанному, шар может скользить относительно плоскости.Рассматриваемая задача описывается в рамках неголоносмной модели, при этом на систему накладывается только одна неголономная связь [1]. В [2] показано, что рассматриваемая система допускает избыточный набор первых интегралов и инвариантную меру, что позволяет свести ее к одной степени свободы.Полученная система зависит от констант четырех первых интегралов и двух массо-геометрических параметров. В работе проводится бифуркационный анализ и классификация различных типов движения рассматриваемой системы. Кроме того, показано, что накладываемая связь вырождается в некоторых точках конфигурационного пространства системы. Это приводит к появлению в системе особенностей, вблизи которых наблюдаются интересные динамические эффекты.

\medskip

\begin{enumerate}
\item[{[1]}] A. V. Borisov, A. A. Kilin, I. S. Mamaev, {\it Dynamics and control of an  omniwheel vehicle}, Regul. Chaotic Dyn., 20(2) (2015), 153--172.
\item[{[2]}] A. A. Kilin, T. B. Ivanova, {\it The Integrable Problem of the Rolling Motion of a Dynamically Symmetric Spherical Top with One Nonholonomic Constraint}, Russian J. Nonlinear Dyn., 19(1) (2023), 3–-17.
\end{enumerate}
\end{talk}
\end{document}