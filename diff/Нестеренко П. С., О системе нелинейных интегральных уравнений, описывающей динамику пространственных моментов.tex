\documentclass[12pt]{article}
\usepackage{hyphsubst}
\usepackage[T2A]{fontenc}
\usepackage[english,main=russian]{babel}
\usepackage[utf8]{inputenc}
\usepackage[letterpaper,top=2cm,bottom=2cm,left=2cm,right=2cm,marginparwidth=2cm]{geometry}
\usepackage{float}
\usepackage{mathtools, commath, amssymb, amsthm}
\usepackage{enumitem, tabularx,graphicx,url,xcolor,rotating,multicol,epsfig,colortbl,lipsum}

\setlist{topsep=1pt, itemsep=0em}
\setlength{\parindent}{0pt}
\setlength{\parskip}{6pt}

\usepackage{hyphenat}
\hyphenation{ма-те-ма-ти-ка вос-ста-нав-ли-вать}

\usepackage[math]{anttor}

\newenvironment{talk}[6]{%
\vskip 0pt\nopagebreak%
\vskip 0pt\nopagebreak%
\section*{#1}
\phantomsection
\addcontentsline{toc}{section}{#2. \textit{#1}}
% \addtocontents{toc}{\textit{#1}\par}
\textit{#2}\\\nopagebreak%
#3\\\nopagebreak%
\ifthenelse{\equal{#4}{}}{}{\url{#4}\\\nopagebreak}%
\ifthenelse{\equal{#5}{}}{}{Соавторы: #5\\\nopagebreak}%
\ifthenelse{\equal{#6}{}}{}{Секция: #6\\\nopagebreak}%
}

\definecolor{LovelyBrown}{HTML}{FDFCF5}

\usepackage[pdftex,
breaklinks=true,
bookmarksnumbered=true,
linktocpage=true,
linktoc=all
]{hyperref}

\begin{document}
\pagenumbering{gobble}
\pagestyle{plain}
\pagecolor{LovelyBrown}
\begin{talk}
{О системе нелинейных интегральных уравнений, описывающей динамику пространственных моментов}
{Нестеренко Полина Сергеевна}
{Университет МГУ-ППИ в Шэньчжэне, факультет ВМК}
{polina_nesterenko2024@mail.ru}
{}
{Дифференциальные уравнения и динамические системы}

В данной работе изучается система \eqref{nesterenko1} нелинейных интегральных уравнений, дополненная условием \eqref{nesterenko2} на бесконечности, возникающая в модели динамики полуляции неподвижных биологических организмов, предложенная У. Дикманом и Р. Лоу.
\begin{equation}\label{nesterenko1}
\begin{cases}
0 = (b-d)N - \overline{s}\int_{B(r_\omega)} C(y)~dy,\\
0 = \overline{b}I_{r_m}(x)N + \overline{b}\int_{B(r_m)}C(x+y)~dy -(d+\overline{s}I_{r_\omega}(x))C(x)-\\ - \overline{s}\int_{B(r_\omega)}T(x,y)~dy.
\end{cases}
\end{equation}
\begin{equation}\label{nesterenko2}
\lim_{\|x\|_{R^n} \to +\infty}C(x) = N^2
\end{equation}
Система \eqref{nesterenko1} описывает состояние равновесия сообщества в случае кусочно-константных ядер разброса и конкуренции.
Система \eqref{nesterenko1} с
помощью замыкания пространственных моментов естественным образом сводится к
нелинейному интегральному уравнению.
Основной целью работы является исследование вышеуказанного нелинейного
интегрального уравнения и ответ на вопрос о существовании его решения. Это
исследование проводится путем построения нелинейного интегрального оператора,
порожденного уравнением, для которого решается поставленная задача, опираясь
на известный результат M.\,A. Красносельского о существовании неподвижной точки
у операторного уравнения. В работе получены условия на биологические параметры, достаточные для существования нетривиального решения данного уравнения.
\end{talk}
\end{document}