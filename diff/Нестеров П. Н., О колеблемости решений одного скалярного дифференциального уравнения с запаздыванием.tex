\documentclass[12pt]{article}
\usepackage{hyphsubst}
\usepackage[T2A]{fontenc}
\usepackage[english,main=russian]{babel}
\usepackage[utf8]{inputenc}
\usepackage[letterpaper,top=2cm,bottom=2cm,left=2cm,right=2cm,marginparwidth=2cm]{geometry}
\usepackage{float}
\usepackage{mathtools, commath, amssymb, amsthm}
\usepackage{enumitem, tabularx,graphicx,url,xcolor,rotating,multicol,epsfig,colortbl,lipsum}

\setlist{topsep=1pt, itemsep=0em}
\setlength{\parindent}{0pt}
\setlength{\parskip}{6pt}

\usepackage{hyphenat}
\hyphenation{ма-те-ма-ти-ка вос-ста-нав-ли-вать}

\usepackage[math]{anttor}

\newenvironment{talk}[6]{%
\vskip 0pt\nopagebreak%
\vskip 0pt\nopagebreak%
\section*{#1}
\phantomsection
\addcontentsline{toc}{section}{#2. \textit{#1}}
% \addtocontents{toc}{\textit{#1}\par}
\textit{#2}\\\nopagebreak%
#3\\\nopagebreak%
\ifthenelse{\equal{#4}{}}{}{\url{#4}\\\nopagebreak}%
\ifthenelse{\equal{#5}{}}{}{Соавторы: #5\\\nopagebreak}%
\ifthenelse{\equal{#6}{}}{}{Секция: #6\\\nopagebreak}%
}

\definecolor{LovelyBrown}{HTML}{FDFCF5}

\usepackage[pdftex,
breaklinks=true,
bookmarksnumbered=true,
linktocpage=true,
linktoc=all
]{hyperref}

\begin{document}
\pagenumbering{gobble}
\pagestyle{plain}
\pagecolor{LovelyBrown}
\begin{talk}
{О колеблемости решений одного скалярного дифференциального уравнения с запаздыванием}
{Нестеров Павел Николаевич}
{ЯрГУ им. П.\,Г. Демидова, РНОМЦ ``Центр интегрируемых систем''}
{p.nesterov@uniyar.ac.ru}
{}
{Дифференциальные уравнения и динамические системы}

В работе строятся асимптотические представления при \(t\to\infty\) для решений следующего скалярного дифференциального  уравнения с переменным запаздыванием:
\[\dot{x}=-a(t)x(t-\tau(t)), \qquad t\ge t_0>0. \eqno(1)
\label{ScalarDDE_Eq1}\]
Здесь функции \(a(t)\) и \(\tau(t)\) предполагаются  действительными и непрерывными на промежутке \([t_0,\infty)\). Основной рассматриваемый в докладе вопрос касается изучения осцилляционных свойств  решений указанного уравнения в так называемом критическом случае, когда  \(a(t)\tau (t)\rightarrow \frac{1}{e}\).  Мы воспользуемся некоторым вариантом метода центральных многообразий, в котором уравнение (1) рассматривается как возмущение автономного уравнения с запаздыванием
\[\dot{x}=-\frac{1}{e}x(t-1).\]
В докладе будет построена двумерная система обыкновенных дифференциальных уравнений, описывающая в главном динамику решений уравнения (1). Асимптотическое интегрирование этой системы позволяет получить ответ на вопрос о колеблемости решений исходного уравнения.
\end{talk}
\end{document}