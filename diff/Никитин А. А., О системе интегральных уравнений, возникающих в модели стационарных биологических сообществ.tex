\documentclass[12pt]{article}
\usepackage{hyphsubst}
\usepackage[T2A]{fontenc}
\usepackage[english,main=russian]{babel}
\usepackage[utf8]{inputenc}
\usepackage[letterpaper,top=2cm,bottom=2cm,left=2cm,right=2cm,marginparwidth=2cm]{geometry}
\usepackage{float}
\usepackage{mathtools, commath, amssymb, amsthm}
\usepackage{enumitem, tabularx,graphicx,url,xcolor,rotating,multicol,epsfig,colortbl,lipsum}

\setlist{topsep=1pt, itemsep=0em}
\setlength{\parindent}{0pt}
\setlength{\parskip}{6pt}

\usepackage{hyphenat}
\hyphenation{ма-те-ма-ти-ка вос-ста-нав-ли-вать}

\usepackage[math]{anttor}

\newenvironment{talk}[6]{%
\vskip 0pt\nopagebreak%
\vskip 0pt\nopagebreak%
\section*{#1}
\phantomsection
\addcontentsline{toc}{section}{#2. \textit{#1}}
% \addtocontents{toc}{\textit{#1}\par}
\textit{#2}\\\nopagebreak%
#3\\\nopagebreak%
\ifthenelse{\equal{#4}{}}{}{\url{#4}\\\nopagebreak}%
\ifthenelse{\equal{#5}{}}{}{Соавторы: #5\\\nopagebreak}%
\ifthenelse{\equal{#6}{}}{}{Секция: #6\\\nopagebreak}%
}

\definecolor{LovelyBrown}{HTML}{FDFCF5}

\usepackage[pdftex,
breaklinks=true,
bookmarksnumbered=true,
linktocpage=true,
linktoc=all
]{hyperref}

\begin{document}
\pagenumbering{gobble}
\pagestyle{plain}
\pagecolor{LovelyBrown}
\begin{talk}
{О системе интегральных уравнений, возникающих в модели стационарных биологических сообществ}
{Никитин Алексей Антонович}
{МГУ им. М.\,В. Ломоносова, факультет ВМК}
{nikitin@cs.msu.ru}
{}
{Дифференциальные уравнения и динамические системы}

В работе рассматриваются основные подходы к изучению стохастического процесса популяционной динамики неподвижных особей с непрерывным временем и пространством, основанного на модели У. Дикмана и Р. Лоу. Динамика пространственного паттерна в этой модели описывается при помощи иерархии уравнений пространственных статистик, которые описывают среднюю плотность и пространственное распределение индивидов, организованных в пары, тройки, четверки, и т. д. В качестве пространственных статистик используются факториальные меры пространственных моментов, которые в случае пары особей пропорциональны функции радиального распределения или функции парной корреляции.

Описывается метод замыкания пространственных моментов и приводятся различные методы исследования получающейся системы интегро-дифференциальных уравнений, соответствующей динамике пространственных моментов этого процесса. Пространственные моменты, полученные при использовании различных методов и замыканий, валидируются при сравнении с пространственными статистиками симуляции стохастического пространственно-временного точечного процесса рождения-разброса-смерти в ограниченной области с периодическими граничными условиями.

\medskip

\begin{enumerate}
\item[{[1]}] Николаев М. В., Никитин А. А., Дикман У. Применение обобщённого принципа неподвижных точек к исследованию системы нелинейных интегральных уравнений, возникающей в модели популяционной динамики // Дифференциальные уравнения. — 2022. — Т. 58, № 9. 1242-1250.
\end{enumerate}
\end{talk}
\end{document}