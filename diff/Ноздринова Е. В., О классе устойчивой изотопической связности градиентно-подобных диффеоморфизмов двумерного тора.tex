\documentclass[12pt]{article}
\usepackage{hyphsubst}
\usepackage[T2A]{fontenc}
\usepackage[english,main=russian]{babel}
\usepackage[utf8]{inputenc}
\usepackage[letterpaper,top=2cm,bottom=2cm,left=2cm,right=2cm,marginparwidth=2cm]{geometry}
\usepackage{float}
\usepackage{mathtools, commath, amssymb, amsthm}
\usepackage{enumitem, tabularx,graphicx,url,xcolor,rotating,multicol,epsfig,colortbl,lipsum}

\setlist{topsep=1pt, itemsep=0em}
\setlength{\parindent}{0pt}
\setlength{\parskip}{6pt}

\usepackage{hyphenat}
\hyphenation{ма-те-ма-ти-ка вос-ста-нав-ли-вать}

\usepackage[math]{anttor}

\newenvironment{talk}[6]{%
\vskip 0pt\nopagebreak%
\vskip 0pt\nopagebreak%
\section*{#1}
\phantomsection
\addcontentsline{toc}{section}{#2. \textit{#1}}
% \addtocontents{toc}{\textit{#1}\par}
\textit{#2}\\\nopagebreak%
#3\\\nopagebreak%
\ifthenelse{\equal{#4}{}}{}{\url{#4}\\\nopagebreak}%
\ifthenelse{\equal{#5}{}}{}{Соавторы: #5\\\nopagebreak}%
\ifthenelse{\equal{#6}{}}{}{Секция: #6\\\nopagebreak}%
}

\definecolor{LovelyBrown}{HTML}{FDFCF5}

\usepackage[pdftex,
breaklinks=true,
bookmarksnumbered=true,
linktocpage=true,
linktoc=all
]{hyperref}

\begin{document}
\pagenumbering{gobble}
\pagestyle{plain}
\pagecolor{LovelyBrown}
\begin{talk}
{О классе устойчивой изотопической связности гра\-ди\-ен\-тно-по\-до\-бных диффеоморфизмов двумерного тора}
{Ноздринова Елена Вячеславовна}
{НИУ Высшая Школа Экономики, Международная лаборатория динамических систем и приложений}
{maati@mail.ru}
{Починка Ольга Витальевна}
{Дифференциальные уравнения и динамические системы}

В докладе речь пойдет о замкнутых связных ориентируемых поверхностях \(M^2\) и сохраняющих ориентацию гомеоморфизмах или диффеоморфизмах, заданных на них. {\it Диффеотопность диффеоморфизмов} \(f_0,f_1:M^2\to M^2\) означает существование некоторой гладкой дуги \(\{f_t:M^2\to M^2,t\in[0,1]\}\), соединяющей их в пространстве диффеоморфизмов. Если диффеотопные  диффеоморфизмы являются {\it структурно устойчивыми} (качественно не меняющими своих свойств при малых шевелениях), то естественно ожидать  существования {\it устойчивой дуги} (качественно не меняющей своих свойств при малых шевелениях) их соединяющей. В этом случае, следуя Ш. Ньюхаусу, Дж. Палису, Ф. Такенсу [1], говорят, что диффеоморфизмы \(f_0,f_1:M^2\to M^2\) {\it устойчиво изотопны} или принадлежат одному и тому же классу {\it устойчивой изотопической связности}.

Простейшими структурно устойчивыми диффеоморфизмами поверхностей являются {\it градиентно-подобные} преобразования, имеющие конечное гиперболическое неблуждающее множество, инвариантные многообразия различных седловых точек которого не пересекаются. Однако, даже градиентно-подобные диффеоморфизмы 2-сферы, которые всегда диффеотопны, в общем случае не являются устойчиво изотопными. Для таких диффеоморфизмов полная классификация, с точностью до устойчивой изотопности, получена Е. Ноздриновой и О. Починкой [2] (см., также обзор [3] по известным на сегодняшний день препятствиям к существованию устойчивых дуг между диффеоморфизмами многообразий). Препятствием к существованию устойчивой дуги между диффеоморфизмами 2-сферы является различие в их периодических данных, что впервые было замечено П. Бланшаром [4].

Хорошо известно, что диффеоморфизмы 2-тора диффеотопны тогда и только тогда, когда индуцированный ими изоморфизм фундаментальной группы  задается одной и той же матрицей  \(A=\begin{pmatrix}
a & b \\
c & d
\end{pmatrix}\in Sl(2,\mathbb{Z})\), то есть \(A\) --- целочисленная квадратная матрица второго порядка с определителем, равным \(1\). Устойчивая связность изотопных тождественному диффеоморфизмов исследована в работе [5], где показано, что диффеоморфизмы, имеющие одинаковые периодические данные могут не соединяться устойчивой дугой из-за разности гомотопических типов кривых, составленных из инвариантных многообразий седловых точек.

В настоящей работе рассмотрен  класс \(G\) градиентно-подобных  диффеоморфизмов 2-тора, индуцирующих изоморфизм фундаментальной группы, определяемый матрицей
\(\begin{pmatrix}
-1 & 0 \\ 0 & -1
\end{pmatrix}\).

Основным результатом, представленным в докладе является доказательство теоремы, что для любых диффеоморфизмов класса \(G\) существует соединяющая их  устойчивая дуга.

\medskip

Исследование осуществлено в рамках Программы фундаментальных исследований НИУ ВШЭ.

\begin{enumerate}
\item[{[1]}]
S. Newhouse, J. Palis, F. Takens, {\it Bifurcations and stability of families of diffeomorphisms}, Publications mathematiques de l' I.H.E.S, 57 (1983), 5--71.
\item[{[2]}] E. Nozdrinova, O. Pochinka, {\it Solution of the 33rd Palis-Pugh problem for gradient-like diffeomorphisms of a two-dimensional sphere}, Discrete and Continuous Dynamical Systems, 41:3 (2021), 1101--1131.
\item[{[3]}] T. Medvedev, E. Nozdrinova, O. Pochinka, {\it Components of Stable Isotopy Connectedness of Morse – Smale Diffeomorphisms}, Regular and Chaotic Dynamics, 27:1 (2022), 77--97.
\item[{[4]}] P. R. Blanchard, {\it Invariants of the NPT isotopy classes of Morse-Smale diffeomorphisms of surfaces}, Duke Mathematical Journal, 47:1 (1980), 33--46.
\item[{[5]}] Д. А. Баранов, Е. В. Ноздринова, О. В. Починка, {\it Сценарий устойчивого перехода от изотопного тождественному диффеоморфизма тора к косому произведению грубых преобразований окружности}, Уфимский математический журнал,  16:1 (2024), 11--23.
\end{enumerate}
\end{talk}
\end{document}