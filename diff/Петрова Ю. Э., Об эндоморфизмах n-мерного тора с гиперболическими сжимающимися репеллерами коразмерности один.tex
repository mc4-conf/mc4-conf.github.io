\documentclass[12pt]{article}
\usepackage{hyphsubst}
\usepackage[T2A]{fontenc}
\usepackage[english,main=russian]{babel}
\usepackage[utf8]{inputenc}
\usepackage[letterpaper,top=2cm,bottom=2cm,left=2cm,right=2cm,marginparwidth=2cm]{geometry}
\usepackage{float}
\usepackage{mathtools, commath, amssymb, amsthm}
\usepackage{enumitem, tabularx,graphicx,url,xcolor,rotating,multicol,epsfig,colortbl,lipsum}

\setlist{topsep=1pt, itemsep=0em}
\setlength{\parindent}{0pt}
\setlength{\parskip}{6pt}

\usepackage{hyphenat}
\hyphenation{ма-те-ма-ти-ка вос-ста-нав-ли-вать}

\usepackage[math]{anttor}

\newenvironment{talk}[6]{%
\vskip 0pt\nopagebreak%
\vskip 0pt\nopagebreak%
\section*{#1}
\phantomsection
\addcontentsline{toc}{section}{#2. \textit{#1}}
% \addtocontents{toc}{\textit{#1}\par}
\textit{#2}\\\nopagebreak%
#3\\\nopagebreak%
\ifthenelse{\equal{#4}{}}{}{\url{#4}\\\nopagebreak}%
\ifthenelse{\equal{#5}{}}{}{Соавторы: #5\\\nopagebreak}%
\ifthenelse{\equal{#6}{}}{}{Секция: #6\\\nopagebreak}%
}

\definecolor{LovelyBrown}{HTML}{FDFCF5}

\usepackage[pdftex,
breaklinks=true,
bookmarksnumbered=true,
linktocpage=true,
linktoc=all
]{hyperref}

\begin{document}
\pagenumbering{gobble}
\pagestyle{plain}
\pagecolor{LovelyBrown}
\begin{talk}
{Об эндоморфизмах n-мерного тора с гиперболическими сжимающимися репеллерами коразмерности один}
{Петрова Юлия Эдуардовна}
{Национальный исследовательский университет ``Высшая школа экономики'',  Нижний Новгород, Россия.}
{yepetrova@hse.ru}
{}
{Дифференциальные уравнения и динамические системы}

В данном докладе представлено обощение результатов, полученных для двумерного тора в работе В.\,З. Гринеса, Е.\,В. Жужомы, Е.\,Д. Куренкова, ``О \(DA\)-эндоморфизмах двумерного тора'', на \(n\)-мерный (\(n \geqslant 3\)) тор.  Рассматриваются непрерывные отображения \(n\)-мерного тора, которые индуцируют гиперболическое действие в фундаментальной группе посредством целочисленной
гиперболической матрицы, у которой одно собственное значение по модулю больше единицы и \(n-1\)  меньше единицы. Предлагается конструкция \(C^{\infty}\)- гладкого эндоморфизма, который задается как суперпозиция эндоморфизма Аносова типа \((1, n-1)\) и некоторого потока. Эта суперпозиция представляет собой \(k\)-листное накрытие и локальный диффеоморфизм, имеющий притягивающий гиперболический сток  и нетривиальный сжимающийся репеллер, который является ламинацией   коразмерности 1.
\end{talk}
\end{document}