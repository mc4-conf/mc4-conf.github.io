\documentclass[12pt]{article}
\usepackage{hyphsubst}
\usepackage[T2A]{fontenc}
\usepackage[english,main=russian]{babel}
\usepackage[utf8]{inputenc}
\usepackage[letterpaper,top=2cm,bottom=2cm,left=2cm,right=2cm,marginparwidth=2cm]{geometry}
\usepackage{float}
\usepackage{mathtools, commath, amssymb, amsthm}
\usepackage{enumitem, tabularx,graphicx,url,xcolor,rotating,multicol,epsfig,colortbl,lipsum}

\setlist{topsep=1pt, itemsep=0em}
\setlength{\parindent}{0pt}
\setlength{\parskip}{6pt}

\usepackage{hyphenat}
\hyphenation{ма-те-ма-ти-ка вос-ста-нав-ли-вать}

\usepackage[math]{anttor}

\newenvironment{talk}[6]{%
\vskip 0pt\nopagebreak%
\vskip 0pt\nopagebreak%
\section*{#1}
\phantomsection
\addcontentsline{toc}{section}{#2. \textit{#1}}
% \addtocontents{toc}{\textit{#1}\par}
\textit{#2}\\\nopagebreak%
#3\\\nopagebreak%
\ifthenelse{\equal{#4}{}}{}{\url{#4}\\\nopagebreak}%
\ifthenelse{\equal{#5}{}}{}{Соавторы: #5\\\nopagebreak}%
\ifthenelse{\equal{#6}{}}{}{Секция: #6\\\nopagebreak}%
}

\definecolor{LovelyBrown}{HTML}{FDFCF5}

\usepackage[pdftex,
breaklinks=true,
bookmarksnumbered=true,
linktocpage=true,
linktoc=all
]{hyperref}

\begin{document}
\pagenumbering{gobble}
\pagestyle{plain}
\pagecolor{LovelyBrown}
\begin{talk}
{Бифуркационный анализ и абсолютная динамика эллипсоида вращения на плоскости}
{Пивоварова Елена Николаевна}
{Уральский математический центр, Удмуртский государственный университет}
{enpiv@rcd.ru}
{А.\,А. Килин}
{Дифференциальные уравнения и динамические системы}

Рассматривается задача о качении эллипсоида вращения по плоскости в предположении, что в точке контакта отсутствует проскальзывание, а проекция угловой скорости эллипсоида на вертикаль равна нулю (модель качения резинового тела). Как известно~[1,2], задача о качении тела вращения произвольной формы в рамках модели резинового тела является интегрируемой и сводится к квадратурам. Более того, дополнительные интегралы системы, описывающей качение тела, представляются в элементарных функциях независимо от формы тела, что значительно упрощает анализ динамики рассматриваемой системы.

Показано существование частных решений эллипсоида, соответствующих его положениям равновесия, качению по окружности, либо качению вдоль прямой. При помощи бифуркационного анализа динамики рассматриваемой системы, была выполнена полная классификация движений в зависимости от параметров эллипсоида (смещения центра масс и соотношений полуосей) и начальных условий (наклона оси симметрии и угловой скорости). Построены все возможные типы бифуркационных диаграмм, и для каждого типа приведены фазовые портреты системы.

Выполнена классификация траекторий движения эллипсоида на основе анализа квадратуры, описывающей динамику его центра масс. Показано, что в общем случае траектории движения центра масс эллипсоида являются квазипериодическими замкнутыми кривыми. Тем не менее, при определенных значениях параметров и начальных условий возможны замкнутые периодические траектории, либо траектории, неограниченно уходящие на бесконечность.

\begin{enumerate}
\item[{[1]}] A. V. Borisov, I. S. Mamaev, {\it Conservation laws, hierarchy of dynamics and explicit integration of nonholonomic systems}, Regular and Chaotic   Dynamics, 13 (2008), 443--490.
\item[{[2]}] A. V. Borisov, I. S. Mamaev, I. A. Bizyaev, {\it The hierarchy of dynamics of a rigid body rolling without slipping and spinning on a plane and a sphere}, Regular and Chaotic Dynamics, 18 (2013), 277--328.
\end{enumerate}
\end{talk}
\end{document}