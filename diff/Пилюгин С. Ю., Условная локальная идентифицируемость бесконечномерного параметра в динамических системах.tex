\documentclass[12pt]{article}
\usepackage{hyphsubst}
\usepackage[T2A]{fontenc}
\usepackage[english,main=russian]{babel}
\usepackage[utf8]{inputenc}
\usepackage[letterpaper,top=2cm,bottom=2cm,left=2cm,right=2cm,marginparwidth=2cm]{geometry}
\usepackage{float}
\usepackage{mathtools, commath, amssymb, amsthm}
\usepackage{enumitem, tabularx,graphicx,url,xcolor,rotating,multicol,epsfig,colortbl,lipsum}

\setlist{topsep=1pt, itemsep=0em}
\setlength{\parindent}{0pt}
\setlength{\parskip}{6pt}

\usepackage{hyphenat}
\hyphenation{ма-те-ма-ти-ка вос-ста-нав-ли-вать}

\usepackage[math]{anttor}

\newenvironment{talk}[6]{%
\vskip 0pt\nopagebreak%
\vskip 0pt\nopagebreak%
\section*{#1}
\phantomsection
\addcontentsline{toc}{section}{#2. \textit{#1}}
% \addtocontents{toc}{\textit{#1}\par}
\textit{#2}\\\nopagebreak%
#3\\\nopagebreak%
\ifthenelse{\equal{#4}{}}{}{\url{#4}\\\nopagebreak}%
\ifthenelse{\equal{#5}{}}{}{Соавторы: #5\\\nopagebreak}%
\ifthenelse{\equal{#6}{}}{}{Секция: #6\\\nopagebreak}%
}

\definecolor{LovelyBrown}{HTML}{FDFCF5}

\usepackage[pdftex,
breaklinks=true,
bookmarksnumbered=true,
linktocpage=true,
linktoc=all
]{hyperref}

\begin{document}
\pagenumbering{gobble}
\pagestyle{plain}
\pagecolor{LovelyBrown}
\begin{talk}
{Условная локальная идентифицируемость бесконечномерного параметра в динамических системах}
{Пилюгин Сергей Юрьевич и Шалгин Владимир Сергеевич}
{Санкт-Петербургский государственный университет}
{sergeipil47@mail.ru}
{}
{Дифференциальные уравнения и динамические системы}

Рассматриваются динамические системы с непрерывным и дискретным временем, зависящие от бесконечномерного параметра. Фиксируются начальное данное траектории и эталонное значение параметра (таким образом, выделяется эталонная траектория). Изучается следующая задача: указать такие семейство допустимых параметров и множество точек наблюдения, что если параметр из этого семейства достаточно близок к эталонному, а соответствующая траектория совпадает с эталонной во всех точках наблюдения, то её параметр совпадает с эталонным.

\medskip

Исследование выполнено за счет гранта Российского научного фонда № 23-21-00025, \url{https://rscf.ru/project/23-21-00025/}.
\end{talk}
\end{document}