\documentclass[12pt]{article}
\usepackage{hyphsubst}
\usepackage[T2A]{fontenc}
\usepackage[english,main=russian]{babel}
\usepackage[utf8]{inputenc}
\usepackage[letterpaper,top=2cm,bottom=2cm,left=2cm,right=2cm,marginparwidth=2cm]{geometry}
\usepackage{float}
\usepackage{mathtools, commath, amssymb, amsthm}
\usepackage{enumitem, tabularx,graphicx,url,xcolor,rotating,multicol,epsfig,colortbl,lipsum}

\setlist{topsep=1pt, itemsep=0em}
\setlength{\parindent}{0pt}
\setlength{\parskip}{6pt}

\usepackage{hyphenat}
\hyphenation{ма-те-ма-ти-ка вос-ста-нав-ли-вать}

\usepackage[math]{anttor}

\newenvironment{talk}[6]{%
\vskip 0pt\nopagebreak%
\vskip 0pt\nopagebreak%
\section*{#1}
\phantomsection
\addcontentsline{toc}{section}{#2. \textit{#1}}
% \addtocontents{toc}{\textit{#1}\par}
\textit{#2}\\\nopagebreak%
#3\\\nopagebreak%
\ifthenelse{\equal{#4}{}}{}{\url{#4}\\\nopagebreak}%
\ifthenelse{\equal{#5}{}}{}{Соавторы: #5\\\nopagebreak}%
\ifthenelse{\equal{#6}{}}{}{Секция: #6\\\nopagebreak}%
}

\definecolor{LovelyBrown}{HTML}{FDFCF5}

\usepackage[pdftex,
breaklinks=true,
bookmarksnumbered=true,
linktocpage=true,
linktoc=all
]{hyperref}

\begin{document}
\pagenumbering{gobble}
\pagestyle{plain}
\pagecolor{LovelyBrown}
\begin{talk}
{О степенной скорости сходимости эргодических средних для групп \(\mathbb{R}^d\) и \(\mathbb{Z}^d\)}
{Подвигин Иван Викторович}
{Институт математики им. С.\,Л. Соболева СО РАН}
{ipodvigin@math.nsc.ru}
{}
{Дифференциальные уравнения и динамические системы}

Скорость сходимости эргодических средних в \(L_2\) норме зависит от двух составляющих. Во-первых, от асимптотики на бесконечности преобразования Фурье индикатора усредняющего множества и, во-вторых, от поведения спектральной меры в окрестности нуля. Известно, что асимптотика преобразования Фурье индикатора существенно зависит от полной кривизны границы усредняющего множества. В докладе рассматриваются два противоположных случая: усреднение вдоль шаров или эллипсоидов~[1] (кривизна границы нигде не зануляется) и усреднение вдоль кубов или параллелепипедов~[2] (кривизна границы нулевая). В обоих случаях приводятся критерии степенной скорости сходимости для всех возможных параметров степеней.

\medskip

\begin{enumerate}
\item[{[1]}] И.В.~Подвигин {\it О степенной скорости сходимости в эргодической теореме Винера}, Алгебра и анализ, 35:6 (2023), 159--168.
\item[{[2]}] А.Г.~Качуровский, И.В.~Подвигин, В.Э.~Тодиков, A.Ж.~Хакимбаев, {\it Спектральный критерий степенной скорости сходимости в эргодической теореме для \(\mathbb{Z}^d\) и \(\mathbb{R}^d\) действий},  Сиб. мат. журн., 65:1 (2024), 92--114.
\end{enumerate}
\end{talk}
\end{document}