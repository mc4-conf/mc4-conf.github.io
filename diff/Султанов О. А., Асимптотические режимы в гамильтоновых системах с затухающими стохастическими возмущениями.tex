\documentclass[12pt]{article}
\usepackage{hyphsubst}
\usepackage[T2A]{fontenc}
\usepackage[english,main=russian]{babel}
\usepackage[utf8]{inputenc}
\usepackage[letterpaper,top=2cm,bottom=2cm,left=2cm,right=2cm,marginparwidth=2cm]{geometry}
\usepackage{float}
\usepackage{mathtools, commath, amssymb, amsthm}
\usepackage{enumitem, tabularx,graphicx,url,xcolor,rotating,multicol,epsfig,colortbl,lipsum}

\setlist{topsep=1pt, itemsep=0em}
\setlength{\parindent}{0pt}
\setlength{\parskip}{6pt}

\usepackage{hyphenat}
\hyphenation{ма-те-ма-ти-ка вос-ста-нав-ли-вать}

\usepackage[math]{anttor}

\newenvironment{talk}[6]{%
\vskip 0pt\nopagebreak%
\vskip 0pt\nopagebreak%
\section*{#1}
\phantomsection
\addcontentsline{toc}{section}{#2. \textit{#1}}
% \addtocontents{toc}{\textit{#1}\par}
\textit{#2}\\\nopagebreak%
#3\\\nopagebreak%
\ifthenelse{\equal{#4}{}}{}{\url{#4}\\\nopagebreak}%
\ifthenelse{\equal{#5}{}}{}{Соавторы: #5\\\nopagebreak}%
\ifthenelse{\equal{#6}{}}{}{Секция: #6\\\nopagebreak}%
}

\definecolor{LovelyBrown}{HTML}{FDFCF5}

\usepackage[pdftex,
breaklinks=true,
bookmarksnumbered=true,
linktocpage=true,
linktoc=all
]{hyperref}

\begin{document}
\pagenumbering{gobble}
\pagestyle{plain}
\pagecolor{LovelyBrown}
\begin{talk}
{Асимптотические режимы в гамильтоновых системах с затухающими стохастическими возмущениями}
{Султанов Оскар Анварович}
{Институт математики с ВЦ УФИЦ РАН}
{oasultanov@gmail.com}
{}
{Дифференциальные уравнения и динамические системы}

Рассматривается класс асимптотически автономных систем дифференциальных уравнений на плоскости с осциллирующими коэффициентами. Предполагается, что предельная система является гамильтоновой с устойчивым равновесием. Обсуждается влияние стохастических возмущений типа белый шума на устойчивость системы при условии, что интенсивность возмущений затухает со временем, а частота удовлетворяет условию резонанса. Показано, что в возмущенной системе имеют место различные долговременные асимптотические режимы, при этом условия стохастической устойчивости равновесия зависят от реализуемого режима и скорости затухания возмущений. В частности, доказана возможность устойчивой фазовой синхронизации в осциллирующих системах за счет затухающих стохастических возмущений. Предлагаемый анализ основан на комбинации метода усреднения и построения стохастических функций Ляпунова.

\medskip

\begin{enumerate}
\item[{[1]}] O. A. Sultanov, {\it Stability of asymptotically Hamiltonian systems with damped oscillatory and stochastic perturbations}, Communications on Pure and Applied Analysis, 23 (2024), 432–462.
\item[{[2]}] O. A. Sultanov, {\it Long-term behaviour of asymptotically autonomous Hamiltonian systems with multiplicative noise}, SIAM Journal on Applied Dynamical Systems, 22 (2023), 1818–1851.
\end{enumerate}
\end{talk}
\end{document}