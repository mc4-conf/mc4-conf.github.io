\documentclass[12pt]{article}
\usepackage{hyphsubst}
\usepackage[T2A]{fontenc}
\usepackage[english,main=russian]{babel}
\usepackage[utf8]{inputenc}
\usepackage[letterpaper,top=2cm,bottom=2cm,left=2cm,right=2cm,marginparwidth=2cm]{geometry}
\usepackage{float}
\usepackage{mathtools, commath, amssymb, amsthm}
\usepackage{enumitem, tabularx,graphicx,url,xcolor,rotating,multicol,epsfig,colortbl,lipsum}

\setlist{topsep=1pt, itemsep=0em}
\setlength{\parindent}{0pt}
\setlength{\parskip}{6pt}

\usepackage{hyphenat}
\hyphenation{ма-те-ма-ти-ка вос-ста-нав-ли-вать}

\usepackage[math]{anttor}

\newenvironment{talk}[6]{%
\vskip 0pt\nopagebreak%
\vskip 0pt\nopagebreak%
\section*{#1}
\phantomsection
\addcontentsline{toc}{section}{#2. \textit{#1}}
% \addtocontents{toc}{\textit{#1}\par}
\textit{#2}\\\nopagebreak%
#3\\\nopagebreak%
\ifthenelse{\equal{#4}{}}{}{\url{#4}\\\nopagebreak}%
\ifthenelse{\equal{#5}{}}{}{Соавторы: #5\\\nopagebreak}%
\ifthenelse{\equal{#6}{}}{}{Секция: #6\\\nopagebreak}%
}

\definecolor{LovelyBrown}{HTML}{FDFCF5}

\usepackage[pdftex,
breaklinks=true,
bookmarksnumbered=true,
linktocpage=true,
linktoc=all
]{hyperref}

\begin{document}
\pagenumbering{gobble}
\pagestyle{plain}
\pagecolor{LovelyBrown}
\begin{talk}
{Необходимые условия для задач управления на бесконечном промежутке со слабо обгоняющим критерием оптимальности}
{Хлопин Дмитрий Валерьевич}
{Институт математики и механики им. Н.\,Н. Красовского}
{khlopin@imm.uran.ru}
{}
{Дифференциальные уравнения и динамические системы}

В задачах управления на бесконечности принцип максимума Л.\,С. Понтрягина, как система необходимых условий оптимальности, не имеет ``общепризнанного'' условия трансверсальности на правом конце. Для  одного из самых общих в таких задачах критериев, слабо обгоняющего критерия, недавно удалось подобрать краевое условие [1], необходимое без каких-либо асимптотических требований на систему. Это условие не может быть усилено как минимум в некоторых линейных задачах, что, впрочем, не мешает ему в  тех же задачах выделять даже континуальный пучок подозрительных на оптимальность экстремалей. Поэтому интересны  и асимптотические предположения, при которых, воспользовавшись условием трансверсальности, можно было бы получить ровно одно решение сопряженной системы. Такие  асимптотические предположения на систему, не слабее работ [1,2], также удалось получить. Основным идеям, позволившим это  сделать, а также разбору нескольких примеров задач управления, и планируется посвятить доклад.

\medskip

\begin{enumerate}
\item[{[1]}]
D. Khlopin,  {\it Necessary conditions in infinite-horizon control problems that need no asymptotic assumptions}. Set-Valued and Variational Analysis, 31:2 (2023), 8.
\item[{[2]}] С. М. Асеев,  В. М. Вельов, {\it Другой взгляд на принцип максимума для задач оптимального управления с бесконечным горизонтом в экономике}, УМН, 74:6 (2019), 3–54
\end{enumerate}
\end{talk}
\end{document}