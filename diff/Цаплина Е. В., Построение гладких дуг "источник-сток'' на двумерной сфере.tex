\documentclass[12pt]{article}
\usepackage{hyphsubst}
\usepackage[T2A]{fontenc}
\usepackage[english,main=russian]{babel}
\usepackage[utf8]{inputenc}
\usepackage[letterpaper,top=2cm,bottom=2cm,left=2cm,right=2cm,marginparwidth=2cm]{geometry}
\usepackage{float}
\usepackage{mathtools, commath, amssymb, amsthm}
\usepackage{enumitem, tabularx,graphicx,url,xcolor,rotating,multicol,epsfig,colortbl,lipsum}

\setlist{topsep=1pt, itemsep=0em}
\setlength{\parindent}{0pt}
\setlength{\parskip}{6pt}

\usepackage{hyphenat}
\hyphenation{ма-те-ма-ти-ка вос-ста-нав-ли-вать}

\usepackage[math]{anttor}

\newenvironment{talk}[6]{%
\vskip 0pt\nopagebreak%
\vskip 0pt\nopagebreak%
\section*{#1}
\phantomsection
\addcontentsline{toc}{section}{#2. \textit{#1}}
% \addtocontents{toc}{\textit{#1}\par}
\textit{#2}\\\nopagebreak%
#3\\\nopagebreak%
\ifthenelse{\equal{#4}{}}{}{\url{#4}\\\nopagebreak}%
\ifthenelse{\equal{#5}{}}{}{Соавторы: #5\\\nopagebreak}%
\ifthenelse{\equal{#6}{}}{}{Секция: #6\\\nopagebreak}%
}

\definecolor{LovelyBrown}{HTML}{FDFCF5}

\usepackage[pdftex,
breaklinks=true,
bookmarksnumbered=true,
linktocpage=true,
linktoc=all
]{hyperref}

\begin{document}
\pagenumbering{gobble}
\pagestyle{plain}
\pagecolor{LovelyBrown}
\begin{talk}
{Построение гладких дуг ``источник-сток'' на двумерной сфере}
{Цаплина Екатерина Вадимовна}
{Аффилиация докладчика}
{ktsaplina11@mail.com}
{Починка О.\,В., Ноздринова Е.\,В.}
{Дифференциальные уравнения и динамические системы}

Пусть \(\Phi:\mathbb S^2\times[0,1]\to\mathbb S^2\) --- гладкое отображение, при каждом фиксированном \(t\in[0,1]\)  являющееся диффеоморфизмом  \(\Phi(x,t)=\phi_t(x)\) сферы \(\mathbb S^2\). Однопараметрическое  семейство диффеоморфизмов  \(\phi_t:\mathbb S^2\to\mathbb S^2,t\in[0,1]\) называется {\it гладкой дугой}, соединяющей диффеоморфизмы \(\phi_0,\,\phi_1\in Diff(\mathbb S^2)\).

Отношение связанности гладкой дугой определяет отношение эквивалентности на множестве  \(Diff(\mathbb S^2)\) диффеоморфизмов сферы и разбивает это множество на два класса эквивалентности, состоящих из сохраняющих и  меняющих  ориентацию диффеоморфизмов согласно работе [1]. При этом в каждом классе существуют диффеоморфизмы ``источник-сток'', неблуждающее множество которых состоит из двух гиперболических точек: источника и стока. Более того, все такие сохраняющие (меняющие) ориентацию диффеоморфизмы попарно топологически, но не гладко,  сопряжены (см., например, [2]). Поэтому, в общем случае дуга, соединяющая два  диффеоморфизма ``источник-сток'', претерпевает бифуркации, в том числе и разрушающие динамику  ``источник-сток''. В силу чего такая дуга может оказаться {\it неустойчивой}, в смысле  чувствительности к малым шевелениям [3].

Основным результатом настоящей работы является конструктивное доказательство теоремы о том, что любые два сохраняющих (меняющих) ориентацию диффеоморфизма ``источник-сток'' двумерной сферы соединяются гладкой дугой, состоящей из диффеоморфизмов ``источник-сток''.

\medskip

Исследование осуществлено в рамках Программы фундаментальных исследований НИУ ВШЭ.

\begin{enumerate}
\item Munkres J., Differentiable isotopies on the \(2 \)-sphere // Michigan Mathematical Journal. – 1960. – Т. 7. – №. 3. – С. 193-197.
\item Grines V. Z., Medvedev T. V., Pochinka O. V., Dynamical systems on 2-and 3-manifolds // Cham : Springer, 2016. – Т. 46.
\item Newhouse S. , Palis J., Takens F., Stable arcs of diffeomorphisms, Bull. Amer. Math. Soc., 82:3 (1976), 499--502.
\end{enumerate}
\end{talk}
\end{document}