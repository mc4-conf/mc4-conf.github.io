\documentclass[12pt]{article}
\usepackage{hyphsubst}
\usepackage[T2A]{fontenc}
\usepackage[english,main=russian]{babel}
\usepackage[utf8]{inputenc}
\usepackage[letterpaper,top=2cm,bottom=2cm,left=2cm,right=2cm,marginparwidth=2cm]{geometry}
\usepackage{float}
\usepackage{mathtools, commath, amssymb, amsthm}
\usepackage{enumitem, tabularx,graphicx,url,xcolor,rotating,multicol,epsfig,colortbl,lipsum}

\setlist{topsep=1pt, itemsep=0em}
\setlength{\parindent}{0pt}
\setlength{\parskip}{6pt}

\usepackage{hyphenat}
\hyphenation{ма-те-ма-ти-ка вос-ста-нав-ли-вать}

\usepackage[math]{anttor}

\newenvironment{talk}[6]{%
\vskip 0pt\nopagebreak%
\vskip 0pt\nopagebreak%
\section*{#1}
\phantomsection
\addcontentsline{toc}{section}{#2. \textit{#1}}
% \addtocontents{toc}{\textit{#1}\par}
\textit{#2}\\\nopagebreak%
#3\\\nopagebreak%
\ifthenelse{\equal{#4}{}}{}{\url{#4}\\\nopagebreak}%
\ifthenelse{\equal{#5}{}}{}{Соавторы: #5\\\nopagebreak}%
\ifthenelse{\equal{#6}{}}{}{Секция: #6\\\nopagebreak}%
}

\definecolor{LovelyBrown}{HTML}{FDFCF5}

\usepackage[pdftex,
breaklinks=true,
bookmarksnumbered=true,
linktocpage=true,
linktoc=all
]{hyperref}

\begin{document}
\pagenumbering{gobble}
\pagestyle{plain}
\pagecolor{LovelyBrown}
\begin{talk}
{О классификации гомеоморфизмов трёхмерных многообразий с псевдоаносовскими аттракторами и репеллерами}
{Чилина Екатерина Евгеньевна}
{Национальный исследовательский университет ``Высшая школа экономики''}
{k.chilina@yandex.ru}
{}
{Дифференциальные уравнения и динамические системы}

Доклад посвящен топологической классификации сохраняющих ориентацию  гомеоморфизмов \(f\) замкнутого ориентируемого топологического 3-многообразия \(M^3\), неблуждающее множество \(NW(f)\) которых состоит из конечного числа компонент связности \(B_0,\dots,B_{m-1}\), удовлетворяющих для любого \(i\in\{0,\dots,m-1\}\) следующим условиям:
\begin{enumerate}
\item \(B_i\) является цилиндрическим вложением  замкнутой  ориентируемой  поверхности  рода большего единицы;
\item  существует натуральное число \(k_i\) такое, что \(f^{k_i}( B_i)= B_i\), \(f^{\tilde k_i}( B_i)\neq B_i\)  для любого натурального \(\tilde k_i< k_i\) и  ограничение отображения \(f^{k_i}|_{ B_i}\) топологически сопряжено сохраняющему ориентацию псевдоаносовскому гомеоморфизму;
\item  \(B_i\) является либо аттрактором, либо репеллером гомеоморфизма \(f^{k_i}\).
\end{enumerate}

\medskip

Исследование осуществлено в рамках Программы фундаментальных исследований НИУ ВШЭ.
\end{talk}
\end{document}