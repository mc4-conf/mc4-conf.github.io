\documentclass[12pt]{article}
\usepackage{hyphsubst}
\usepackage[T2A]{fontenc}
\usepackage[english,main=russian]{babel}
\usepackage[utf8]{inputenc}
\usepackage[letterpaper,top=2cm,bottom=2cm,left=2cm,right=2cm,marginparwidth=2cm]{geometry}
\usepackage{float}
\usepackage{mathtools, commath, amssymb, amsthm}
\usepackage{enumitem, tabularx,graphicx,url,xcolor,rotating,multicol,epsfig,colortbl,lipsum}

\setlist{topsep=1pt, itemsep=0em}
\setlength{\parindent}{0pt}
\setlength{\parskip}{6pt}

\usepackage{hyphenat}
\hyphenation{ма-те-ма-ти-ка вос-ста-нав-ли-вать}

\usepackage[math]{anttor}

\newenvironment{talk}[6]{%
\vskip 0pt\nopagebreak%
\vskip 0pt\nopagebreak%
\section*{#1}
\phantomsection
\addcontentsline{toc}{section}{#2. \textit{#1}}
% \addtocontents{toc}{\textit{#1}\par}
\textit{#2}\\\nopagebreak%
#3\\\nopagebreak%
\ifthenelse{\equal{#4}{}}{}{\url{#4}\\\nopagebreak}%
\ifthenelse{\equal{#5}{}}{}{Соавторы: #5\\\nopagebreak}%
\ifthenelse{\equal{#6}{}}{}{Секция: #6\\\nopagebreak}%
}

\definecolor{LovelyBrown}{HTML}{FDFCF5}

\usepackage[pdftex,
breaklinks=true,
bookmarksnumbered=true,
linktocpage=true,
linktoc=all
]{hyperref}

\begin{document}
\pagenumbering{gobble}
\pagestyle{plain}
\pagecolor{LovelyBrown}
\begin{talk}
{Итерационное решение ОДУ применением аппроксимации Паде}
{Шапеев Василий Павлович}
{Институт теоретической и прикладной механики СО РАН им. С.\,А. Христиановича}
{shapeev.vasily@mail.ru}
{}
{Дифференциальные уравнения и динамические системы}

Аппроксимацией Паде задачи Коши и краевых задач для ОДУ получены приближенные задачи, являющиеся проекциями исходных дифференциальных задач в пространства дробно-рациональных функций.
Искомые приближенные решения отыскиваются в виде аппроксимантов Паде \([L/M]\), которые записываются с неопределенными коэффициентами. Для построения приближенного решения дифференциальной задачи на отрезке решения задачи задаются узлы сетки --- точки коллокации. При этом количество узлов сетки берется больше числа коэффициентов  \([L/M]\). Коллокациями в узлах сетки уравнений приближенной задачи, соответствующей решаемой дифференциальной задаче, получается переопределенная система нелинейных алгебраических уравнений (СНАУ) относительно коэффициентов \([L/M]\). Ее решение после предварительного преобразования полученных уравнений отыскивается в итерационном процессе, в котором численные значения некоторых частей  нелинейных уравнений на текущей итерации берутся с предыдущей итерации так, что на каждой итерации решается переопределенная система линейных алгебраических уравнений (СЛАУ) --- линейная задача наименьших квадратов.

Предложенный алгоритм решения задач для ОДУ запрограммирован на языках системы МАТЕМАТИКА и С. Построены высокоточные решения различных задач для ОДУ. Погрешность полученных решений близка к величине погрешности округления чисел на компьютере.
Показано значительное превосходство по точности предложенного алгоритма над стандартной процедурой NDSOLVE решения задачи Коши для ОДУ в системе МАТЕМАТИКА, а также над методом Рунге-Кутта четвертого порядка аппроксимации.

В численных экспериментах решения конкретных ОДУ показаны зависимость точности решения от степени полиномов \(L\) и \(M\), зависимость обусловленности получающихся СЛАУ
от степени их переопределенности и сложности выражений ОДУ, зависимость скорости сходимости итераций от способа линеаризации СНАУ.

\medskip

Работа выполнена при финансовой поддержке Российского научного фонда грантом по проекту РНФ № 23-21-00499.
\end{talk}
\end{document}