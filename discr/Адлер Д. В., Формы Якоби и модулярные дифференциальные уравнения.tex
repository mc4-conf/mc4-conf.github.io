\documentclass[12pt]{article}
\usepackage{hyphsubst}
\usepackage[T2A]{fontenc}
\usepackage[english,main=russian]{babel}
\usepackage[utf8]{inputenc}
\usepackage[letterpaper,top=2cm,bottom=2cm,left=2cm,right=2cm,marginparwidth=2cm]{geometry}
\usepackage{float}
\usepackage{mathtools, commath, amssymb, amsthm}
\usepackage{enumitem, tabularx,graphicx,url,xcolor,rotating,multicol,epsfig,colortbl,lipsum}

\setlist{topsep=1pt, itemsep=0em}
\setlength{\parindent}{0pt}
\setlength{\parskip}{6pt}

\usepackage{hyphenat}
\hyphenation{ма-те-ма-ти-ка вос-ста-нав-ли-вать}

\usepackage[math]{anttor}

\newenvironment{talk}[6]{%
\vskip 0pt\nopagebreak%
\vskip 0pt\nopagebreak%
\section*{#1}
\phantomsection
\addcontentsline{toc}{section}{#2. \textit{#1}}
% \addtocontents{toc}{\textit{#1}\par}
\textit{#2}\\\nopagebreak%
#3\\\nopagebreak%
\ifthenelse{\equal{#4}{}}{}{\url{#4}\\\nopagebreak}%
\ifthenelse{\equal{#5}{}}{}{Соавторы: #5\\\nopagebreak}%
\ifthenelse{\equal{#6}{}}{}{Секция: #6\\\nopagebreak}%
}

\definecolor{LovelyBrown}{HTML}{FDFCF5}

\usepackage[pdftex,
breaklinks=true,
bookmarksnumbered=true,
linktocpage=true,
linktoc=all
]{hyperref}

\begin{document}
\pagenumbering{gobble}
\pagestyle{plain}
\pagecolor{LovelyBrown}
\begin{talk}
{Формы Якоби и модулярные дифференциальные уравнения}
{Адлер Дмитрий Всеволодович}
{МЦМУ им. Леонарда Эйлера}
{dmitry.v.adler@gmail.com}
{}
{Теория чисел и дискретная математика}

Одним из примеров автоморфных форм, возникающих в различных областях математики, являются формы Якоби. Впервые возникшие в работах И.И. Пятецкого-Шапиро, классические формы Якоби были подробно изучены в книге ``Теория форм Якоби'' М. Эйхлера и Д. Загье.

Для классических модулярных форм можно определить дифференциальный оператор, увеличивающий вес модулярной формы на 2. Как следствие, можно получить дифференциальные уравнения на модулярные формы относительно этого оператора. Примерами таких уравнений являются система дифференциальных уравнений Рамануджана и уравнение Канеко-Загье.

Оказывается, что для форм Якоби также можно определить аналогичный дифференциальный оператор, являющийся своего рода эллиптизацией оператора теплопроводности. В своём докладе я расскажу об этом дифференциальном операторе, о дифференциальных уравнениях на формы Якоби, а также о возникающих приложениях. Данный доклад основан на совместных работах с В.А. Гриценко.
\end{talk}
\end{document}