\documentclass[12pt]{article}
\usepackage{hyphsubst}
\usepackage[T2A]{fontenc}
\usepackage[english,main=russian]{babel}
\usepackage[utf8]{inputenc}
\usepackage[letterpaper,top=2cm,bottom=2cm,left=2cm,right=2cm,marginparwidth=2cm]{geometry}
\usepackage{float}
\usepackage{mathtools, commath, amssymb, amsthm}
\usepackage{enumitem, tabularx,graphicx,url,xcolor,rotating,multicol,epsfig,colortbl,lipsum}

\setlist{topsep=1pt, itemsep=0em}
\setlength{\parindent}{0pt}
\setlength{\parskip}{6pt}

\usepackage{hyphenat}
\hyphenation{ма-те-ма-ти-ка вос-ста-нав-ли-вать}

\usepackage[math]{anttor}

\newenvironment{talk}[6]{%
\vskip 0pt\nopagebreak%
\vskip 0pt\nopagebreak%
\section*{#1}
\phantomsection
\addcontentsline{toc}{section}{#2. \textit{#1}}
% \addtocontents{toc}{\textit{#1}\par}
\textit{#2}\\\nopagebreak%
#3\\\nopagebreak%
\ifthenelse{\equal{#4}{}}{}{\url{#4}\\\nopagebreak}%
\ifthenelse{\equal{#5}{}}{}{Соавторы: #5\\\nopagebreak}%
\ifthenelse{\equal{#6}{}}{}{Секция: #6\\\nopagebreak}%
}

\definecolor{LovelyBrown}{HTML}{FDFCF5}

\usepackage[pdftex,
breaklinks=true,
bookmarksnumbered=true,
linktocpage=true,
linktoc=all
]{hyperref}

\begin{document}
\pagenumbering{gobble}
\pagestyle{plain}
\pagecolor{LovelyBrown}
\begin{talk}
{Пороговая вероятность наличия триангуляции k-уголь\-ни\-ка в случайном графе}
{Вахрушев Степан Викторович}
{Санкт-Петербургский государственный университет, Московский физико-технический институт}
{vakhrushev.sv@phystech.edu}
{Жуковский Максим Евгеньевич}
{Теория чисел и дискретная математика}

В 1991-ом году Бела Боллобаш и Алан Фриз нашли пороговую вероятность для нахождения копии трингуляции треугольника в случайном графе \(G(n, p)\) с точностью до логарифмического множителя. В нашей работе мы находим данную пороговую вероятность триангуляции \(k\)-угольника с точностью до константы. Для этого мы используем недавно разработанную технологию процесса фрагментации и \(q\)-spread леммы. Оказывается, что необходимое условие на соотношение количества вершин и рёбер в подграфе в случае планарной триангуляции эквивалентно тому, что любой простой цикл ограничивает небольшое количество вершин. Более точно, отношение длины цикла к количеству ``внутренних'' вершин не менее чем данное отношение для исходной границы \(k\)-угольника: \(\frac{k}{n-k}\). Данное условие проверяется в работе для конкретной триангуляции, состоящей из вложенных друг в друга циклов.

\medskip

\begin{enumerate}
\item[{[1]}] B. Bollob\'{a}s, A. M. Frieze, {\it Spanning maximal planar subgraphs of random graphs}, Random Structures \& Algorithms, {\bf 2}:2 (1991) 225-231.
\item[{[2]}] A.E. D\'{i}az, Y. Person, {\it Spanning F-cycles in random graphs}, Combinatorics, Probability and Computing, {\bf 32}:5 (2023) 833-850.
\item[{[3]}] J. Kahn, B. Narayanan and J. Park, {\it The threshold for the square of a Hamilton cycle}, Proceedings of the American Mathematical Society {\bf 149} (2020), 3201-3208.
\item[{[4]}] O. Riordan, {\it Spanning subgraphs of random graphs}, Combinatorics, Probability and Computing, {\bf 9}:2 (2000), 125-148.
\end{enumerate}
\end{talk}
\end{document}