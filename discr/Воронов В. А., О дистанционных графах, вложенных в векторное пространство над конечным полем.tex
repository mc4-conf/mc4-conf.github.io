\documentclass[12pt]{article}
\usepackage{hyphsubst}
\usepackage[T2A]{fontenc}
\usepackage[english,main=russian]{babel}
\usepackage[utf8]{inputenc}
\usepackage[letterpaper,top=2cm,bottom=2cm,left=2cm,right=2cm,marginparwidth=2cm]{geometry}
\usepackage{float}
\usepackage{mathtools, commath, amssymb, amsthm}
\usepackage{enumitem, tabularx,graphicx,url,xcolor,rotating,multicol,epsfig,colortbl,lipsum}

\setlist{topsep=1pt, itemsep=0em}
\setlength{\parindent}{0pt}
\setlength{\parskip}{6pt}

\usepackage{hyphenat}
\hyphenation{ма-те-ма-ти-ка вос-ста-нав-ли-вать}

\usepackage[math]{anttor}

\newenvironment{talk}[6]{%
\vskip 0pt\nopagebreak%
\vskip 0pt\nopagebreak%
\section*{#1}
\phantomsection
\addcontentsline{toc}{section}{#2. \textit{#1}}
% \addtocontents{toc}{\textit{#1}\par}
\textit{#2}\\\nopagebreak%
#3\\\nopagebreak%
\ifthenelse{\equal{#4}{}}{}{\url{#4}\\\nopagebreak}%
\ifthenelse{\equal{#5}{}}{}{Соавторы: #5\\\nopagebreak}%
\ifthenelse{\equal{#6}{}}{}{Секция: #6\\\nopagebreak}%
}

\definecolor{LovelyBrown}{HTML}{FDFCF5}

\usepackage[pdftex,
breaklinks=true,
bookmarksnumbered=true,
linktocpage=true,
linktoc=all
]{hyperref}

\begin{document}
\pagenumbering{gobble}
\pagestyle{plain}
\pagecolor{LovelyBrown}
\begin{talk}
{О дистанционных графах, вложенных в векторное пространство над конечным полем}
{Воронов Всеволод Александрович}
{Кавказский математический центр Адыгейского государственного университета, Московский физико-технический институт (национальный исследовательский университет)}
{v-vor@yandex.ru}
{Д.\,Д. Черкашин}
{Теория чисел и дискретная математика}

Рассмотрим граф \(G^{(n)}_{q}\), вершинами которого являются точки \(n\)-мерного векторного пространства \(\mathbb{F}^n_{q}\), где \(q = p^k\), \(p\) простое; ребрами соединены пары точек \(x,y \in \mathbb{F}^n_{q}\), евклидово расстояние между которыми равно единице.  Свойства данного класса графов изучались в работах А. Иосевича и соавторов [1, 2]. В частности, из оценок мощности подмножества \(\mathbb{F}^n_{q}\), в котором реализуются все расстояния, следует, что хроматическое число \(\chi(G^{(n)}_{q})\) стремится к бесконечности при фиксированном \(n \geq 2 \) и \(q \to \infty\). При этом оценка, которую можно получить на основе работы [1], при малых \(q\) является достаточно грубой.

В докладе будут представлены новые оценки хроматических чисел \(G^{(n)}_{q}\), полученные методами спектральной теории графов, а также оценки, основанные на явных построениях.

\medskip

\begin{enumerate}
\item[{[1]}] A. Iosevich, M. Rudnev, {\it Erdos distance problem in vector spaces over finite fields}, Transactions
of the American Mathematical Society, 359 (2007), 6127–6142.
\item[{[2]}] A. Iosevich, H. Parshall, {\it Embedding distance graphs in finite field vector spaces}, J. Korean Math. Soc., 56 (2019), 1515–1528.
\end{enumerate}
\end{talk}
\end{document}