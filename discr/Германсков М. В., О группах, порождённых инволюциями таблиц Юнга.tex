\documentclass[12pt]{article}
\usepackage{hyphsubst}
\usepackage[T2A]{fontenc}
\usepackage[english,main=russian]{babel}
\usepackage[utf8]{inputenc}
\usepackage[letterpaper,top=2cm,bottom=2cm,left=2cm,right=2cm,marginparwidth=2cm]{geometry}
\usepackage{float}
\usepackage{mathtools, commath, amssymb, amsthm}
\usepackage{enumitem, tabularx,graphicx,url,xcolor,rotating,multicol,epsfig,colortbl,lipsum}

\setlist{topsep=1pt, itemsep=0em}
\setlength{\parindent}{0pt}
\setlength{\parskip}{6pt}

\usepackage{hyphenat}
\hyphenation{ма-те-ма-ти-ка вос-ста-нав-ли-вать}

\usepackage[math]{anttor}

\newenvironment{talk}[6]{%
\vskip 0pt\nopagebreak%
\vskip 0pt\nopagebreak%
\section*{#1}
\phantomsection
\addcontentsline{toc}{section}{#2. \textit{#1}}
% \addtocontents{toc}{\textit{#1}\par}
\textit{#2}\\\nopagebreak%
#3\\\nopagebreak%
\ifthenelse{\equal{#4}{}}{}{\url{#4}\\\nopagebreak}%
\ifthenelse{\equal{#5}{}}{}{Соавторы: #5\\\nopagebreak}%
\ifthenelse{\equal{#6}{}}{}{Секция: #6\\\nopagebreak}%
}

\definecolor{LovelyBrown}{HTML}{FDFCF5}

\usepackage[pdftex,
breaklinks=true,
bookmarksnumbered=true,
linktocpage=true,
linktoc=all
]{hyperref}

\begin{document}
\pagenumbering{gobble}
\pagestyle{plain}
\pagecolor{LovelyBrown}
\begin{talk}
{О группах, порождённых инволюциями таблиц Юнга}
{Германсков Михаил Витальевич}
{МЦМУ им. Леонарда Эйлера, СПбГУ}
{mgermanskov@gmail.com}
{}
{Теория чисел и дискретная математика}

Каждой диаграмме Юнга можно сопоставить группу, действующую на множестве таблиц Юнга той же формы. Эта группа порождена инволюциями, которые меняют местами числа  \(i\)  и  \(i+1\)  в таблице, если соответствующие клетки не соседние; в противном случае таблица остаётся неизменной. Такие группы, отвечающие двустрочечным диаграммам, являются либо знакопеременными, либо симметрическими. В докладе я расскажу, как различать эти два случая, глядя на двоичные записи длин двух строк.

\medskip

\begin{enumerate}
\item[{[1]}]А. М. Вершик, Н. В. Цилевич, “Группы, порожденные инволюциями ромбовидных графов, и деформации ортогональной формы Юнга”, Теория представлений, динамические системы, комбинаторные и алгоритмические методы. XXX, Зап. научн. сем. ПОМИ, 481, ПОМИ, СПб., 2019, 29–38
\item[{[2]}] М. В. Германсков, \emph{“Описание групп, порожденных инволюциями двухстрочечных таблиц Юнга”}, Теория представлений, динамические системы, комбинаторные методы. XXXIV, Зап. научн. сем. ПОМИ, 517, ПОМИ, СПб., 2022, 70–81
\item[{[3]}] М. Германсков, \emph{“Классификация групп, порожденных инволюциями двустрочечных таблиц Юнга”}, Теория представлений, динамические системы, комбинаторные методы. XXXV, Зап. научн. сем. ПОМИ, 528, ПОМИ, СПб., 2023, 107–115
\end{enumerate}
\end{talk}
\end{document}