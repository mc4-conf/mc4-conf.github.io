\documentclass[12pt]{article}
\usepackage{hyphsubst}
\usepackage[T2A]{fontenc}
\usepackage[english,main=russian]{babel}
\usepackage[utf8]{inputenc}
\usepackage[letterpaper,top=2cm,bottom=2cm,left=2cm,right=2cm,marginparwidth=2cm]{geometry}
\usepackage{float}
\usepackage{mathtools, commath, amssymb, amsthm}
\usepackage{enumitem, tabularx,graphicx,url,xcolor,rotating,multicol,epsfig,colortbl,lipsum}

\setlist{topsep=1pt, itemsep=0em}
\setlength{\parindent}{0pt}
\setlength{\parskip}{6pt}

\usepackage{hyphenat}
\hyphenation{ма-те-ма-ти-ка вос-ста-нав-ли-вать}

\usepackage[math]{anttor}

\newenvironment{talk}[6]{%
\vskip 0pt\nopagebreak%
\vskip 0pt\nopagebreak%
\section*{#1}
\phantomsection
\addcontentsline{toc}{section}{#2. \textit{#1}}
% \addtocontents{toc}{\textit{#1}\par}
\textit{#2}\\\nopagebreak%
#3\\\nopagebreak%
\ifthenelse{\equal{#4}{}}{}{\url{#4}\\\nopagebreak}%
\ifthenelse{\equal{#5}{}}{}{Соавторы: #5\\\nopagebreak}%
\ifthenelse{\equal{#6}{}}{}{Секция: #6\\\nopagebreak}%
}

\definecolor{LovelyBrown}{HTML}{FDFCF5}

\usepackage[pdftex,
breaklinks=true,
bookmarksnumbered=true,
linktocpage=true,
linktoc=all
]{hyperref}

\begin{document}
\pagenumbering{gobble}
\pagestyle{plain}
\pagecolor{LovelyBrown}
\begin{talk}
{О некоторых вопросах цепных дробей с рациональными неполными частными c левым сдвигом}
{Дмитрий Александрович Долгов}
{ИВМиИТ КФУ}
{Dolgov.kfu@gmail.com}
{}
{Теория чисел и дискретная математика}

Цепные дроби с рациональными неполными частными с левым сдвигом получаются при помощи обобщенного алгоритма Соренсона вычисления НОД двух натуральных чисел. Один из вариантов таких дробей выглядит следующим образом:

\begin{center}
\begin{equation*}
\frac{y_0}{x_0}{k_0}^{e_0} + \cfrac{\delta_0}{\begin{pmatrix}
\cfrac{x_0 y_1}{x_1}{k_1}^{e_1} + \cfrac{\delta_1}{\begin{pmatrix}\ddots+ \cfrac{y_n {k_n}^{e_n} \prod\limits_{\substack{i< n,\\ i\not\equiv n\mod\ 2}} x_i}{x_n \prod\limits_{\substack{j< n,\\ j\equiv n \mod\ 2}} x_j}\end{pmatrix}}\end{pmatrix}},
\end{equation*}
\end{center}
где величина \(\delta_i\) равняется либо 1, либо -1 в зависимости от знака линейной комбинации входных чисел в алгоритме Соренсона, \(x_i\), \(y_i\) -- ненулевые целые числа, а \({k_i}\) -- элементы заранее зафиксированной последовательности \(\mathbb{K}\) (см. [1, 2]).

Многие классические результаты теории цепных дробей в данном случае не будут наблюдаться. Во-первых, разложение в такую дробь может быть неоднозначно. Во-вторых, рассматривая бесконечные цепные дроби, в общем случае ничего нельзя сказать о монотонности последовательности подходящих дробей (аналогично и континуантов таких дробей). Из этого можно заключить, что приближение иррациональных чисел при помощи таких дробей в общем случае может не выполняться. В докладе предполагается обсудить эти вопросы, представить условия при которых для данного класса дробей представленные задачи будут эффективно решаться.

\medskip

\begin{enumerate}
\item[{[1]}] J. Sorenson, {\it Two Fast GCD Algorithms}, J. of Algorithms, vol. 16, no. 1 (1994), 110-144.
\item[{[2]}] D. A. Dolgov, {\it On the continued fraction with rational partial quotients}, Chebyshevskii Sbornik, (2024).
\end{enumerate}
\end{talk}
\end{document}