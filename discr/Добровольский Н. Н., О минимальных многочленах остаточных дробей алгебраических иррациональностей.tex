\documentclass[12pt]{article}
\usepackage{hyphsubst}
\usepackage[T2A]{fontenc}
\usepackage[english,main=russian]{babel}
\usepackage[utf8]{inputenc}
\usepackage[letterpaper,top=2cm,bottom=2cm,left=2cm,right=2cm,marginparwidth=2cm]{geometry}
\usepackage{float}
\usepackage{mathtools, commath, amssymb, amsthm}
\usepackage{enumitem, tabularx,graphicx,url,xcolor,rotating,multicol,epsfig,colortbl,lipsum}

\setlist{topsep=1pt, itemsep=0em}
\setlength{\parindent}{0pt}
\setlength{\parskip}{6pt}

\usepackage{hyphenat}
\hyphenation{ма-те-ма-ти-ка вос-ста-нав-ли-вать}

\usepackage[math]{anttor}

\newenvironment{talk}[6]{%
\vskip 0pt\nopagebreak%
\vskip 0pt\nopagebreak%
\section*{#1}
\phantomsection
\addcontentsline{toc}{section}{#2. \textit{#1}}
% \addtocontents{toc}{\textit{#1}\par}
\textit{#2}\\\nopagebreak%
#3\\\nopagebreak%
\ifthenelse{\equal{#4}{}}{}{\url{#4}\\\nopagebreak}%
\ifthenelse{\equal{#5}{}}{}{Соавторы: #5\\\nopagebreak}%
\ifthenelse{\equal{#6}{}}{}{Секция: #6\\\nopagebreak}%
}

\definecolor{LovelyBrown}{HTML}{FDFCF5}

\usepackage[pdftex,
breaklinks=true,
bookmarksnumbered=true,
linktocpage=true,
linktoc=all
]{hyperref}

\begin{document}
\pagenumbering{gobble}
\pagestyle{plain}
\pagecolor{LovelyBrown}
\begin{talk}
{О минимальных многочленах остаточных дробей алгебраических иррациональностей}
{Добровольский Николай Николаевич}
{Тульский государственный педагогический университет имени Л.\,Н. Толстого}
{nikolai.dobrovolsky@gmail.com}
{}
{Теория чисел и дискретная математика}

В докладе рассматриваются вид и свойства минимальных многочленов остаточных дробей в разложении алгебраических чисел в цепные дроби.

Показано, что для чисто-вещественных алгебраических иррациональностей \(\alpha\) степени \(n\ge 2\), начиная с некоторого номера \(m_0=m_0(\alpha)\), последовательность остаточных дробей \(\alpha_m\) является последовательностью приведённых алгебраических иррациональностей.

Дано определение обобщённого числа Пизо, которое отличается от оп\-ределения чисел Пизо отсутствием требования целочисленности.

Показано, что для произвольной вещественной алгебраической  иррациональности \(\alpha\) степени \(n\ge2\), начиная с некоторого номера \(m_0=m_0(\alpha)\), последовательность остаточных дробей \(\alpha_m\) является последовательностью обобщённых чисел Пизо.

Найдена асимптотическая формула для сопряжённых чисел к остаточным дробям обобщённых чисел Пизо. Из этой формулы вытекает, что сопряжённые к остаточной дроби \(\alpha_m\) концентрируются около дроби \(-\frac{Q_{m-2}}{Q_{m-1}}\) либо в интервале радиуса \(O\left(\frac1{Q_{m-1}^2}\right)\) в случае чисто-вещественной алгебраической иррациональности, либо в круге такого же радиуса в общем случае вещественной алгебраической иррациональности, имеющей комплексные сопряжённые числа.

Установлено, что, начиная с некоторого номера \(m_0=m_0(\alpha)\), справедлива \, рекуррентная \, формула \, для \, неполных \, частных \(q_m\) разложения вещественной алгебраической иррациональности \(\alpha\), выражающая \(q_m\) через значения минимального многочлена \(f_{m-1}(x)\) для остаточной дроби \(\alpha_{m-1}\) и его производной в точке \(q_{m-1}\).

Найдены рекуррентные формулы для нахождения минимальных многочленов остаточных дробей с помощью дробно-линейных преобразований.  Композиция   этих дробно-линейных преобразований является дробно-линейным преобразование, переводящем систему сопряжённых к алгебраической иррациона\-льности \(\alpha\) в систему сопряжённых к остаточной дроби, обладающую ярко выраженным эффектом концентрации около рациональной дроби \(-\frac{Q_{m-2}}{Q_{m-1}}\).

Установлено, что последовательность минимальных многочленов для остаточных дробей  образует последовательность многочленов с равными дискриминантами.

В заключении поставлена проблема о структуре рационального сопряжённого спектра вещественного алгебраического иррационального числа \(\alpha\) и о его предельных точках.
\end{talk}
\end{document}