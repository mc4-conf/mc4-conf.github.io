\documentclass[12pt]{article}
\usepackage{hyphsubst}
\usepackage[T2A]{fontenc}
\usepackage[english,main=russian]{babel}
\usepackage[utf8]{inputenc}
\usepackage[letterpaper,top=2cm,bottom=2cm,left=2cm,right=2cm,marginparwidth=2cm]{geometry}
\usepackage{float}
\usepackage{mathtools, commath, amssymb, amsthm}
\usepackage{enumitem, tabularx,graphicx,url,xcolor,rotating,multicol,epsfig,colortbl,lipsum}

\setlist{topsep=1pt, itemsep=0em}
\setlength{\parindent}{0pt}
\setlength{\parskip}{6pt}

\usepackage{hyphenat}
\hyphenation{ма-те-ма-ти-ка вос-ста-нав-ли-вать}

\usepackage[math]{anttor}

\newenvironment{talk}[6]{%
\vskip 0pt\nopagebreak%
\vskip 0pt\nopagebreak%
\section*{#1}
\phantomsection
\addcontentsline{toc}{section}{#2. \textit{#1}}
% \addtocontents{toc}{\textit{#1}\par}
\textit{#2}\\\nopagebreak%
#3\\\nopagebreak%
\ifthenelse{\equal{#4}{}}{}{\url{#4}\\\nopagebreak}%
\ifthenelse{\equal{#5}{}}{}{Соавторы: #5\\\nopagebreak}%
\ifthenelse{\equal{#6}{}}{}{Секция: #6\\\nopagebreak}%
}

\definecolor{LovelyBrown}{HTML}{FDFCF5}

\usepackage[pdftex,
breaklinks=true,
bookmarksnumbered=true,
linktocpage=true,
linktoc=all
]{hyperref}

\begin{document}
\pagenumbering{gobble}
\pagestyle{plain}
\pagecolor{LovelyBrown}
\begin{talk}
{Cемейства перестановок без s-паросочетаний}
{Иноземцев Эдуард Леонидович}
{Московский физико-технический институт}
{eduard_inozemtsev@bk.ru}
{Колупаев Дмитрий, Купавский Андрей}
{Теория чисел и дискретная математика}

Обозначим через \(S_n\) семейство всех перестановок \([n] \rightarrow [n]\). Перестановки \(\sigma, \pi \in S_n\)  пересекаются, если найдется такой элемент \(i \in [n]\), что \(\sigma(i) = \pi(i)\). Семейство \(\mathcal{F} \subset S_n\) называется пересекающимся, если любые два множества \(A, B \in \mathcal{F}\) пересекаются. Деза и Франкл [1] показали, что пересекающееся семейство перестановок имеет размер не более чем \((n-1)!\). Через \(\nu(\mathcal{F})\) обозначим максимальное количество попарно непересекающихся элементов \(\mathcal{F}\). Мы доказали следующий результат.
Пусть \(\mathcal{F} \subset S_n, \nu(\mathcal{F}) < s\) и \(s \leq \frac{n}{C \log_2(n)},\) тогда \(|\mathcal{F}| \leq (s-1)(n-1)!.\) Основным инструментом доказательства является техника разреженных приближений (spread approximations) из статьи Купавского и  Захарова [2].

\medskip

\begin{enumerate}
\item[{[1]}] M. Deza, P. Frankl, {\it On the maximum number of permutations with given maximal
or minimal distance}, J. Combin. Theory, Ser. A 22 (1977), 352–360.
\item[{[2]}] A. Kupavskii and D. Zakharov, {\it Spread approximations for forbidden
intersections problems}, arXiv:2203.13379v2 (2022).
\end{enumerate}
\end{talk}
\end{document}