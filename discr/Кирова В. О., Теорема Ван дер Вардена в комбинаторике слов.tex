\documentclass[12pt]{article}
\usepackage{hyphsubst}
\usepackage[T2A]{fontenc}
\usepackage[english,main=russian]{babel}
\usepackage[utf8]{inputenc}
\usepackage[letterpaper,top=2cm,bottom=2cm,left=2cm,right=2cm,marginparwidth=2cm]{geometry}
\usepackage{float}
\usepackage{mathtools, commath, amssymb, amsthm}
\usepackage{enumitem, tabularx,graphicx,url,xcolor,rotating,multicol,epsfig,colortbl,lipsum}

\setlist{topsep=1pt, itemsep=0em}
\setlength{\parindent}{0pt}
\setlength{\parskip}{6pt}

\usepackage{hyphenat}
\hyphenation{ма-те-ма-ти-ка вос-ста-нав-ли-вать}

\usepackage[math]{anttor}

\newenvironment{talk}[6]{%
\vskip 0pt\nopagebreak%
\vskip 0pt\nopagebreak%
\section*{#1}
\phantomsection
\addcontentsline{toc}{section}{#2. \textit{#1}}
% \addtocontents{toc}{\textit{#1}\par}
\textit{#2}\\\nopagebreak%
#3\\\nopagebreak%
\ifthenelse{\equal{#4}{}}{}{\url{#4}\\\nopagebreak}%
\ifthenelse{\equal{#5}{}}{}{Соавторы: #5\\\nopagebreak}%
\ifthenelse{\equal{#6}{}}{}{Секция: #6\\\nopagebreak}%
}

\definecolor{LovelyBrown}{HTML}{FDFCF5}

\usepackage[pdftex,
breaklinks=true,
bookmarksnumbered=true,
linktocpage=true,
linktoc=all
]{hyperref}

\begin{document}
\pagenumbering{gobble}
\pagestyle{plain}
\pagecolor{LovelyBrown}
\begin{talk}
{Теорема Ван дер Вардена в комбинаторике слов}
{Кирова Валерия Орлановна}
{МГУ им. М.\,В. Ломоносова}
{kirova_vo@mail.ru}
{}
{Теория чисел и дискретная математика}

Теорема Ван дер Вардена, которую А. Я. Хинчин называл жемчужиной теорией чисел, внесла большой вклад в комбинаторику слов, положив в 2000 г. начало исследованию функции арифметической сложности бесконечных слов, обобщив понятие классической функции комбинаторной сложности. В своей работе, Avgustinovich S. V., Fon-der-Flaass D. G., Frid A. помимо введения понятия функции арифметической сложности, представили теоремы Семереди и Ван дер Вардена в терминах комбинаторики слов. В 1996 г. Leibman A., Bergelson V. представили теорему Ван дер Вардена для полиномиального случая. Эта теорема положила начало исследованию функции полиномиальной сложности --- более обобщенной модификацией комбинаторной и арифметической сложности. Введенные функции прежде всего были рассмотрены на классе слов с наименьшей комбинаторной сложностью --- словах Штурма, для которых получены интересные оценки.

\medskip

\begin{enumerate}
\item[{[1]}] Hedlund G.A. , Morse M. {\it Symbolic dynamics} Amer. J. Math, 1938, 815-866.
\item[{[2]}] Avgustinovich S., Fon-Der-Flaass D., Frid A.
{\it Arithmetical complexity of infnite words }
Proc. Words, Languages and Combinatorics III, 2000. Singapore: World Scientifc,
2003. P. 51-62.
\item[{[3]}] Leibman A., Bergelson  V.  {\it Polynomial extensions of van der Waerden's and Szemeredi's theorems}, Journal of the American Math Society, Vol. 9, 1996, 725-753.
\item[{[4]}] Kirova V.O., Godunov I.V. {\it On the complexity functions of Sturmian words} Chebyshevskii sbornik, 2023, vol. 24, no. 4, pp. 63-44.
\end{enumerate}
\end{talk}
\end{document}