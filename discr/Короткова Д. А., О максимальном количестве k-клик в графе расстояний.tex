\documentclass[12pt]{article}
\usepackage{hyphsubst}
\usepackage[T2A]{fontenc}
\usepackage[english,main=russian]{babel}
\usepackage[utf8]{inputenc}
\usepackage[letterpaper,top=2cm,bottom=2cm,left=2cm,right=2cm,marginparwidth=2cm]{geometry}
\usepackage{float}
\usepackage{mathtools, commath, amssymb, amsthm}
\usepackage{enumitem, tabularx,graphicx,url,xcolor,rotating,multicol,epsfig,colortbl,lipsum}

\setlist{topsep=1pt, itemsep=0em}
\setlength{\parindent}{0pt}
\setlength{\parskip}{6pt}

\usepackage{hyphenat}
\hyphenation{ма-те-ма-ти-ка вос-ста-нав-ли-вать}

\usepackage[math]{anttor}

\newenvironment{talk}[6]{%
\vskip 0pt\nopagebreak%
\vskip 0pt\nopagebreak%
\section*{#1}
\phantomsection
\addcontentsline{toc}{section}{#2. \textit{#1}}
% \addtocontents{toc}{\textit{#1}\par}
\textit{#2}\\\nopagebreak%
#3\\\nopagebreak%
\ifthenelse{\equal{#4}{}}{}{\url{#4}\\\nopagebreak}%
\ifthenelse{\equal{#5}{}}{}{Соавторы: #5\\\nopagebreak}%
\ifthenelse{\equal{#6}{}}{}{Секция: #6\\\nopagebreak}%
}

\definecolor{LovelyBrown}{HTML}{FDFCF5}

\usepackage[pdftex,
breaklinks=true,
bookmarksnumbered=true,
linktocpage=true,
linktoc=all
]{hyperref}

\begin{document}
\pagenumbering{gobble}
\pagestyle{plain}
\pagecolor{LovelyBrown}
\begin{talk}
{О максимальном количестве k-клик в графе расстояний}
{Короткова Дарья Алексеевна}
{Студент МФТИ}
{alekhina.da@phystech.edu}
{Андрей Купавский}
{Теория чисел и дискретная математика}

Рассматривается вопрос об оценке максимального числа k-клик \(c_k(G^{(n)})\), содержащихся в качестве подграфов в графе единичных расстояний (графе диаметров) \(G^{(n)}\) при заданном количестве вершин \(n\) в евклидовом пространстве \(\mathbb{R}^d\). Показывается, что при достаточно больших \(n\) графы, максимизирующие значение \(c_k(G^{(n)})\), удовлетворяют так называемой конструкции Ленца.

\medskip

\begin{enumerate}
\item[{[1]}] Konrad J. Swanepoel, {\it Unit distances and diameters in Euclidean spaces}, Discrete \& Computational Geometry (2009), 1-27.
\item[{[2]}] P. Erdős, {\it On the number of complete subgraphs contained in certain grap}, Mathematical Institute of the Hungarian Academy of Science (1962).
\item[{[3]}] Jürgen Eckhoff, {\it A new Turán-type theorem for cliques in graphs}, Discrete Mathematics (2004), 113-122.
\end{enumerate}
\end{talk}
\end{document}