\documentclass[12pt]{article}
\usepackage{hyphsubst}
\usepackage[T2A]{fontenc}
\usepackage[english,main=russian]{babel}
\usepackage[utf8]{inputenc}
\usepackage[letterpaper,top=2cm,bottom=2cm,left=2cm,right=2cm,marginparwidth=2cm]{geometry}
\usepackage{float}
\usepackage{mathtools, commath, amssymb, amsthm}
\usepackage{enumitem, tabularx,graphicx,url,xcolor,rotating,multicol,epsfig,colortbl,lipsum}

\setlist{topsep=1pt, itemsep=0em}
\setlength{\parindent}{0pt}
\setlength{\parskip}{6pt}

\usepackage{hyphenat}
\hyphenation{ма-те-ма-ти-ка вос-ста-нав-ли-вать}

\usepackage[math]{anttor}

\newenvironment{talk}[6]{%
\vskip 0pt\nopagebreak%
\vskip 0pt\nopagebreak%
\section*{#1}
\phantomsection
\addcontentsline{toc}{section}{#2. \textit{#1}}
% \addtocontents{toc}{\textit{#1}\par}
\textit{#2}\\\nopagebreak%
#3\\\nopagebreak%
\ifthenelse{\equal{#4}{}}{}{\url{#4}\\\nopagebreak}%
\ifthenelse{\equal{#5}{}}{}{Соавторы: #5\\\nopagebreak}%
\ifthenelse{\equal{#6}{}}{}{Секция: #6\\\nopagebreak}%
}

\definecolor{LovelyBrown}{HTML}{FDFCF5}

\usepackage[pdftex,
breaklinks=true,
bookmarksnumbered=true,
linktocpage=true,
linktoc=all
]{hyperref}

\begin{document}
\pagenumbering{gobble}
\pagestyle{plain}
\pagecolor{LovelyBrown}
\begin{talk}
{Обзор результатов о полиадических числах Лиувилля}
{Крупицын Евгений Станиславович}
{Московский педагогический государственный университет}
{krupitsin@gmail.com}
{}
{Теория чисел и дискретная математика}

\medskip
В 1844 г. Ж. Лиувилль [1,2] опубликовал теорему, согласно которой  алгебраическое число не может хорошо приближаться рациональными числами и построил пример трансцендентного числа? допускающего сколь угодно высокий порядок приближения рациональными числами. Фактически работы Лиувилля были началом теории трансцендентных чисел.

Исследования, связанные с числами Лиувилля актуальны по сей день. Можно отметить таких ученых как P. Erd\"{o}s, M. Waldschmidt и многие другие.

Доклад посвящен обзору результатов о полиадических числах Лиувилля.

\medskip

\begin{enumerate}
\item[{[1]}] J. Liouville, {\it Sur des classes tr\`{e}s \'{e}tendues de quantit\'{e}s dont la valeur n'est ni alg\'{e}brique, ni m\^{e}me reductible \`{a} des irrationelles alg\'{e}briques }, Compt. Rend. Acad. Sci. (Paris), 1844, no. 18, pp. 883--885.

\item[{[2]}] J. Liouville, {\it Nouvelle d\'{e}monstration d'un th\'{e}or\`{e}me sur les irrationelles alg\'{e}briques	ins\'{e}r\'{e} dans le Compte rendu de la derni\`{e}re s\'{e}ance}, Compt. Rend. Acad. Sci. (Paris), 1844, no. 18, pp. 910--911.
\end{enumerate}
\end{talk}
\end{document}