\documentclass[12pt]{article}
\usepackage{hyphsubst}
\usepackage[T2A]{fontenc}
\usepackage[english,main=russian]{babel}
\usepackage[utf8]{inputenc}
\usepackage[letterpaper,top=2cm,bottom=2cm,left=2cm,right=2cm,marginparwidth=2cm]{geometry}
\usepackage{float}
\usepackage{mathtools, commath, amssymb, amsthm}
\usepackage{enumitem, tabularx,graphicx,url,xcolor,rotating,multicol,epsfig,colortbl,lipsum}

\setlist{topsep=1pt, itemsep=0em}
\setlength{\parindent}{0pt}
\setlength{\parskip}{6pt}

\usepackage{hyphenat}
\hyphenation{ма-те-ма-ти-ка вос-ста-нав-ли-вать}

\usepackage[math]{anttor}

\newenvironment{talk}[6]{%
\vskip 0pt\nopagebreak%
\vskip 0pt\nopagebreak%
\section*{#1}
\phantomsection
\addcontentsline{toc}{section}{#2. \textit{#1}}
% \addtocontents{toc}{\textit{#1}\par}
\textit{#2}\\\nopagebreak%
#3\\\nopagebreak%
\ifthenelse{\equal{#4}{}}{}{\url{#4}\\\nopagebreak}%
\ifthenelse{\equal{#5}{}}{}{Соавторы: #5\\\nopagebreak}%
\ifthenelse{\equal{#6}{}}{}{Секция: #6\\\nopagebreak}%
}

\definecolor{LovelyBrown}{HTML}{FDFCF5}

\usepackage[pdftex,
breaklinks=true,
bookmarksnumbered=true,
linktocpage=true,
linktoc=all
]{hyperref}

\begin{document}
\pagenumbering{gobble}
\pagestyle{plain}
\pagecolor{LovelyBrown}
\begin{talk}
{Неравенство Эрдёша для примитивных множеств}
{Кучерявый Петр Алексеевич}
{НИУ ВШЭ}
{peter.ktchr@gmail.com}
{}
{Теория чисел и дискретная математика}

Множество натуральных чисел \(A\) называется примитивным, если никакой элемент \(A\) не делит другой элемент \(A\). Например, множество \(A = \{m, m+1, \ldots, 2m - 1\}\) является примитивным.
Обозначим \(\Omega(n)\) количество простых делителей \(n\) с учетом кратности. Для любого \(k\) множество \(\mathbb{P}_k = \{ n : \Omega(n) = k \}\) также является примитивным множеством.
Эрдёш в 1935 году доказал, что \(\sum_{a \in A}\frac{1}{a \log a}\) равномерно ограничено по всем выборам примитивного множества \(A\). Это утверждение допускает такое обобщение.
Положим \(f_z(A) = \sum_{a \in A}\frac{z^{\Omega(a)}}{a (\log a)^z}\), где \(z \in \mathbb{R}_{> 0}\). Тогда \(f_z(A)\) равномерно ограничено по всем выборам примитивного множества \(A\) при фиксированном \(z \in (0, 2)\).
Мы обсудим это утверждение, а также некоторые другие обобщения классических результатов про примитивные множества. В частности будет обсуждаться асимптотика \(f_z(\mathbb{P}_k)\) при растущем \(k\).
\end{talk}
\end{document}