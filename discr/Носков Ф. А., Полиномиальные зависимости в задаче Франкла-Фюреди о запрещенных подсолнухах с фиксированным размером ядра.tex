\documentclass[12pt]{article}
\usepackage{hyphsubst}
\usepackage[T2A]{fontenc}
\usepackage[english,main=russian]{babel}
\usepackage[utf8]{inputenc}
\usepackage[letterpaper,top=2cm,bottom=2cm,left=2cm,right=2cm,marginparwidth=2cm]{geometry}
\usepackage{float}
\usepackage{mathtools, commath, amssymb, amsthm}
\usepackage{enumitem, tabularx,graphicx,url,xcolor,rotating,multicol,epsfig,colortbl,lipsum}

\setlist{topsep=1pt, itemsep=0em}
\setlength{\parindent}{0pt}
\setlength{\parskip}{6pt}

\usepackage{hyphenat}
\hyphenation{ма-те-ма-ти-ка вос-ста-нав-ли-вать}

\usepackage[math]{anttor}

\newenvironment{talk}[6]{%
\vskip 0pt\nopagebreak%
\vskip 0pt\nopagebreak%
\section*{#1}
\phantomsection
\addcontentsline{toc}{section}{#2. \textit{#1}}
% \addtocontents{toc}{\textit{#1}\par}
\textit{#2}\\\nopagebreak%
#3\\\nopagebreak%
\ifthenelse{\equal{#4}{}}{}{\url{#4}\\\nopagebreak}%
\ifthenelse{\equal{#5}{}}{}{Соавторы: #5\\\nopagebreak}%
\ifthenelse{\equal{#6}{}}{}{Секция: #6\\\nopagebreak}%
}

\definecolor{LovelyBrown}{HTML}{FDFCF5}

\usepackage[pdftex,
breaklinks=true,
bookmarksnumbered=true,
linktocpage=true,
linktoc=all
]{hyperref}

\begin{document}
\pagenumbering{gobble}
\pagestyle{plain}
\pagecolor{LovelyBrown}
\begin{talk}
{Полиномиальные зависимости в задаче Франкла-Фюреди о запрещенных подсолнухах с фиксированным размером ядра}
{Носков Федор Андреевич}
{Московский физико-технический институт}
{noskov.fa@phystech.edu}
{Андрей Купавский}
{Теория чисел и дискретная математика}

Назовем систему \(s\) различных множеств \(S_1, \ldots, S_s\) подсолнухом с яром размера \(t - 1\), если существует множество \(C\) размера \(t - 1\) такое, что любые два множества \(S_i, S_j\) пересекаются ровно по \(C\): \(S_i \cap S_j = C\). В 1987 году Франкл и Фюреди рассмотрели следующую задачу: каков максимальный размер подсемейства \(\binom{[n]}{k}\), не содержащего подсолнуха с \(s\) лепестками и ядром размера \(t - 1\)? Они нашли асимптотически точную верхную границу при \(n\) стремящемся к бесконечности и \(k, s, t\) фиксированными, \(k \ge 2t - 1\). В настоящем докладе мы найдем размер максимального семейства при \(n \ge 2^{15} s t^2 k \log k\) и \(k \ge 2t - 1\) с точностью до аддитивной ошибки порядка \(O \left ( \frac{(2 s t)^t \log n}{n^{1/3}} \right ) \binom{n - t}{k - t}\).
\end{talk}
\end{document}