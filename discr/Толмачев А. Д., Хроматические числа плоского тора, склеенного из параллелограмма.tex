\documentclass[12pt]{article}
\usepackage{hyphsubst}
\usepackage[T2A]{fontenc}
\usepackage[english,main=russian]{babel}
\usepackage[utf8]{inputenc}
\usepackage[letterpaper,top=2cm,bottom=2cm,left=2cm,right=2cm,marginparwidth=2cm]{geometry}
\usepackage{float}
\usepackage{mathtools, commath, amssymb, amsthm}
\usepackage{enumitem, tabularx,graphicx,url,xcolor,rotating,multicol,epsfig,colortbl,lipsum}

\setlist{topsep=1pt, itemsep=0em}
\setlength{\parindent}{0pt}
\setlength{\parskip}{6pt}

\usepackage{hyphenat}
\hyphenation{ма-те-ма-ти-ка вос-ста-нав-ли-вать}

\usepackage[math]{anttor}

\newenvironment{talk}[6]{%
\vskip 0pt\nopagebreak%
\vskip 0pt\nopagebreak%
\section*{#1}
\phantomsection
\addcontentsline{toc}{section}{#2. \textit{#1}}
% \addtocontents{toc}{\textit{#1}\par}
\textit{#2}\\\nopagebreak%
#3\\\nopagebreak%
\ifthenelse{\equal{#4}{}}{}{\url{#4}\\\nopagebreak}%
\ifthenelse{\equal{#5}{}}{}{Соавторы: #5\\\nopagebreak}%
\ifthenelse{\equal{#6}{}}{}{Секция: #6\\\nopagebreak}%
}

\definecolor{LovelyBrown}{HTML}{FDFCF5}

\usepackage[pdftex,
breaklinks=true,
bookmarksnumbered=true,
linktocpage=true,
linktoc=all
]{hyperref}

\begin{document}
\pagenumbering{gobble}
\pagestyle{plain}
\pagecolor{LovelyBrown}
\begin{talk}
{Хроматические числа плоского тора, склеенного из параллелограмма}
{Толмачев Александр Дмитриевич}
{Московский физико-технический институт (национальный исследовательский университет), Кавказский математический центр}
{tolmachev.ad@phystech.edu}
{Дмитрий Протасов, Всеволод Воронов}
{Теория чисел и дискретная математика}

Пусть \(\vec{v_1}, \vec{v_2} \in \mathbb{R}^2\) --- произвольные векторы ненулевой длины \(l_1 = |\vec{v_1}|\) и \( l_2 = |\vec{v_2}|\) с углом \(\alpha = \angle(\vec{v_1}, \vec{v_2})\) между ними. Рассмотрим плоский тор \(T^2_{l_1, l_2, \alpha}\) как фактор-пространство по решетке \(\mathbb{R}^2 / L\), где \(L = \left\{ m\vec{v_1} + n\vec{v_2}: m, n \in \mathbb{Z} \right\}\). Другими словами, рассматривается плоский тор, склеенный из параллелограмма, со сторонами \(l_1, l_2\) и углом \(\alpha\) между ними.

Соответствующая метрика \(\rho_{l_1, l_2, \alpha}\) расстояния на таком торе может быть выражена через стандартную евклидову метрику на плоскости следующим образом:
\[\rho_{l_1, l_2, \alpha}\left((x_1,y_1), (x_2, y_2)\right) = \min \left\{ \left| (x_2 - x_1 + m l_1) \cdot \vec{v_1} + (y_2 - y_1 + nl_2) \cdot \vec{v_2} \right | : m, n \in \mathbb{Z} \right\}, \]
что является кратчайшим расстоянием между точками \((x_1, y_1), (x_2, y_2)\) по поверхности тора (здесь и далее считаем, что \(x_1, x_2, \in [0, l_1]\), \(y_1, y_2 \in [0, l_2]\)).

В работе рассматриваются оценки хроматических чисел тора \(T^2_{l_1, l_2, \alpha}\) в зависимости от параметров \(l_1, l_2, \alpha\). Кроме того, получены некоторые оценки хроматических чисел с запрещенным расстоянием для такого тора.

Задача о хроматических числах различных множеств тесно связана с задачей о разбиении множеств на части меньшего диаметра. Данная работа расширяет предыдущие исследования авторов об оценках оптимальных разбиений тора и плоских множеств на части меньшего диаметра (см. [1, 2]). Отметим, что в отличие от работы [2], здесь рассматривается обобщение плоского тора со склейки из квадрата на склейку из произвольного параллелограмма, параметризуемого тремя параметрами. Авторами также получены некоторые оценки разбиений тора \(T^2_{l_1, l_2, \alpha}\) на части меньшего диаметра, которые расширяют исследования из ранних статей [1, 2].

\medskip

\begin{enumerate}
\item[{[1]}] A.D. Tolmachev, D.S. Protasov, V.A. Voronov, {\it Coverings of planar and three-dimensional sets with subsets of smaller diameter}, Discrete Applied Mathematics, 2022, v. 320, p. 270-281.
\item[{[2]}] D.S. Protasov, A.D. Tolmachev, V.A. Voronov, {\it Optimal partitions of the flat torus into parts of smaller diameter}, arxiv preprint. [2024] arXiv:2402.03997
\end{enumerate}
\end{talk}
\end{document}