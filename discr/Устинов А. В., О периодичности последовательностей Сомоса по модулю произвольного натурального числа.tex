\documentclass[12pt]{article}
\usepackage{hyphsubst}
\usepackage[T2A]{fontenc}
\usepackage[english,main=russian]{babel}
\usepackage[utf8]{inputenc}
\usepackage[letterpaper,top=2cm,bottom=2cm,left=2cm,right=2cm,marginparwidth=2cm]{geometry}
\usepackage{float}
\usepackage{mathtools, commath, amssymb, amsthm}
\usepackage{enumitem, tabularx,graphicx,url,xcolor,rotating,multicol,epsfig,colortbl,lipsum}

\setlist{topsep=1pt, itemsep=0em}
\setlength{\parindent}{0pt}
\setlength{\parskip}{6pt}

\usepackage{hyphenat}
\hyphenation{ма-те-ма-ти-ка вос-ста-нав-ли-вать}

\usepackage[math]{anttor}

\newenvironment{talk}[6]{%
\vskip 0pt\nopagebreak%
\vskip 0pt\nopagebreak%
\section*{#1}
\phantomsection
\addcontentsline{toc}{section}{#2. \textit{#1}}
% \addtocontents{toc}{\textit{#1}\par}
\textit{#2}\\\nopagebreak%
#3\\\nopagebreak%
\ifthenelse{\equal{#4}{}}{}{\url{#4}\\\nopagebreak}%
\ifthenelse{\equal{#5}{}}{}{Соавторы: #5\\\nopagebreak}%
\ifthenelse{\equal{#6}{}}{}{Секция: #6\\\nopagebreak}%
}

\definecolor{LovelyBrown}{HTML}{FDFCF5}

\usepackage[pdftex,
breaklinks=true,
bookmarksnumbered=true,
linktocpage=true,
linktoc=all
]{hyperref}

\begin{document}
\pagenumbering{gobble}
\pagestyle{plain}
\pagecolor{LovelyBrown}
\begin{talk}
{О периодичности последовательностей Сомоса по модулю произвольного натурального числа}
{Устинов Алексей Владимирович}
{Национальный исследовательский университет ``Высшая школа экономики'', г. Москва}
{ustinov.alexey@gmail.com}
{}
{Теория чисел и дискретная математика}

Для целого числа \(k\ge 4\) последовательность Сомоса--\(k\) ---  это последовательность, порожденная квадратичным рекуррентным соотношением вида \[s_{n+k}s_n=\sum_{j=1}^{[k/2]}\alpha_js_{n+k-j}s_{n+j},\] где \(\alpha_j\) ---  константы, а \(s_0\), \dots, \(s_{k-1}\)~---  начальные условия. Среди всех последовательностей Сомоса выделяется важный подкласс, обладающих различными нетривиальными свойствами.
Это подкласс \textit{последовательностей конечного ранга}.
Последовательность \(\{s_n\}_{n=-\infty}^\infty\) имеет (конечный) ранг \(r\), если максимальный ранг двух бесконечных матриц
\[\left.\vphantom{\sum}(s_{m+n}s_{m-n})\right|_{m,n=-\infty}^\infty,\qquad \left.\vphantom{\sum}(s_{m+n+1}s_{m-n})\right|_{m,n=-\infty}^\infty\]
равен \(r\). Если \(r=2\), то общий член последовательности Сомоса может быть выражен в терминах эллиптической функции. Общую последовательность конечного ранга можно рассматривать как последовательность, скрывающую за собой более сложную теорему сложения.

Доклад будет в посвящен доказательству периодичности целочисленных последовательностей конечного ранга по модулю произвольного натурального числа. В частности, это означает, что с помощью последовательностей конечного ранга можно строить криптографические протоколы, аналогичные тем, что строятся на эллиптических кривых.

Доклад будет сделан по результатам статьи [1].

Исследование выполнено за счет гранта Российского научного фонда № 22-41-05001,
\href{https://rscf.ru/project/22-41-05001/}{https://rscf.ru/project/22-41-05001/}.

\medskip

\begin{enumerate}
\item[{[1]}] А. В. Устинов, {\it О периодичности последовательностей Сомоса по модулю m}, Матем. заметки, 115:3 (2024),  439--449.
\end{enumerate}
\end{talk}
\end{document}