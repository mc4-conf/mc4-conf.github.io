\documentclass[12pt]{article}
\usepackage{hyphsubst}
\usepackage[T2A]{fontenc}
\usepackage[english,main=russian]{babel}
\usepackage[utf8]{inputenc}
\usepackage[letterpaper,top=2cm,bottom=2cm,left=2cm,right=2cm,marginparwidth=2cm]{geometry}
\usepackage{float}
\usepackage{mathtools, commath, amssymb, amsthm}
\usepackage{enumitem, tabularx,graphicx,url,xcolor,rotating,multicol,epsfig,colortbl,lipsum}

\setlist{topsep=1pt, itemsep=0em}
\setlength{\parindent}{0pt}
\setlength{\parskip}{6pt}

\usepackage{hyphenat}
\hyphenation{ма-те-ма-ти-ка вос-ста-нав-ли-вать}

\usepackage[math]{anttor}

\newenvironment{talk}[6]{%
\vskip 0pt\nopagebreak%
\vskip 0pt\nopagebreak%
\section*{#1}
\phantomsection
\addcontentsline{toc}{section}{#2. \textit{#1}}
% \addtocontents{toc}{\textit{#1}\par}
\textit{#2}\\\nopagebreak%
#3\\\nopagebreak%
\ifthenelse{\equal{#4}{}}{}{\url{#4}\\\nopagebreak}%
\ifthenelse{\equal{#5}{}}{}{Соавторы: #5\\\nopagebreak}%
\ifthenelse{\equal{#6}{}}{}{Секция: #6\\\nopagebreak}%
}

\definecolor{LovelyBrown}{HTML}{FDFCF5}

\usepackage[pdftex,
breaklinks=true,
bookmarksnumbered=true,
linktocpage=true,
linktoc=all
]{hyperref}

\begin{document}
\pagenumbering{gobble}
\pagestyle{plain}
\pagecolor{LovelyBrown}
\begin{talk}
{Накрытия полных графов: симметрии и ассоциированные объекты}
{Циовкина Людмила Юрьевна}
{ИММ УрО РАН, УрФУ}
{l.tsiovkina@gmail.com}
{}
{Теория чисел и дискретная математика}

Доклад посвящен исследованию задачи классификации дистанционно регулярных накрытий полных графов, группа автоморфизмов которых вершинно-транзитивна и имеет не более двух орбит в ее индуцированном действии на множестве дуг накрытия. Из работ автора [2] и [3] следует, что  новые  накрытия с транзитивной на дугах группой автоморфизмов могут быть обнаружены лишь в небольшом числе открытых подслучаев, когда такая группа индуцирует некоторую аффинную группу подстановок ранга 2 на множестве фибр накрытия. В общем случае рассматриваемая задача далека от решения и остается актуальной в контексте проблем поиска новых конструкций  накрытий с различными параметрами и ассоциированных с ними объектов (например, обобщенных матриц Адамара или равноугольных множеств прямых). В докладе мы сосредоточимся на исследовании этой задачи для большого класса т.н. абелевых  (в смысле Годсила и Хензеля [1]) накрытий, новый всплеск интереса к которым вызван их приложениями в дискретной геометрии.  Будет представлен способ ее частичного сведения к рассмотрению   фактор-накрытий, группа автоморфизмов которых  наследует указанные свойства  и индуцирует группу подстановок ранга \(\le 3\) на множестве фибр, а также методы классификации, основанные на классификации таких групп подстановок  и включающие специальную технику ограничения спектра   накрытий в зависимости от вида индуцируемой группы. Особое внимание будет уделено исследованию абелевых накрытий из нескольких потенциально  бесконечных классов, которое, в том числе, мотивировано проблемами существования и построения равноугольных жестких фреймов с заданными параметрами.

\medskip

\begin{enumerate}
\item[{[1]}]  C. D. Godsil, A. D. Hensel, {\it Distance regular covers of the complete graph},  J. Comb. Theory Ser. B., 56 (1992) 205--238.
\item[{[2]}] L. Yu.  Tsiovkina, {\it Arc-transitive groups of automorphisms of antipodal distance-regular graphs of diameter 3 in affine case},  Sib. Elektron. Mat. Izv., 17 (2020) 445--495.
\item[{[3]}]  L. Yu. Tsiovkina, {\it Covers of  complete graphs and related association schemes},  J. Comb. Theory Ser. A., 191  (2022) 105646.
\end{enumerate}
\end{talk}
\end{document}