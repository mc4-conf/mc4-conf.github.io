\documentclass[12pt]{article}
\usepackage{hyphsubst}
\usepackage[T2A]{fontenc}
\usepackage[english,main=russian]{babel}
\usepackage[utf8]{inputenc}
\usepackage[letterpaper,top=2cm,bottom=2cm,left=2cm,right=2cm,marginparwidth=2cm]{geometry}
\usepackage{float}
\usepackage{mathtools, commath, amssymb, amsthm}
\usepackage{enumitem, tabularx,graphicx,url,xcolor,rotating,multicol,epsfig,colortbl,lipsum}

\setlist{topsep=1pt, itemsep=0em}
\setlength{\parindent}{0pt}
\setlength{\parskip}{6pt}

\usepackage{hyphenat}
\hyphenation{ма-те-ма-ти-ка вос-ста-нав-ли-вать}

\usepackage[math]{anttor}

\newenvironment{talk}[6]{%
\vskip 0pt\nopagebreak%
\vskip 0pt\nopagebreak%
\section*{#1}
\phantomsection
\addcontentsline{toc}{section}{#2. \textit{#1}}
% \addtocontents{toc}{\textit{#1}\par}
\textit{#2}\\\nopagebreak%
#3\\\nopagebreak%
\ifthenelse{\equal{#4}{}}{}{\url{#4}\\\nopagebreak}%
\ifthenelse{\equal{#5}{}}{}{Соавторы: #5\\\nopagebreak}%
\ifthenelse{\equal{#6}{}}{}{Секция: #6\\\nopagebreak}%
}

\definecolor{LovelyBrown}{HTML}{FDFCF5}

\usepackage[pdftex,
breaklinks=true,
bookmarksnumbered=true,
linktocpage=true,
linktoc=all
]{hyperref}

\begin{document}
\pagenumbering{gobble}
\pagestyle{plain}
\pagecolor{LovelyBrown}
\begin{talk}
{Бесконечная алгебраическая независимость рядов с периодическими коэффициентами}
{Чирский Владимир Григорьевич}
{МГУ им. М.\,В. Ломоносова}
{vgchirskii@yandex.ru}
{}
{Теория чисел и дискретная математика}

Рассмотрим ряды вида
\[\sum_{n=0}^{\infty}a^{i}_n n!\;z^n,\eqno{(1)}\]
коэффициенты которых -- целые числа с условиями \(a^{i}_{n+T}=a^{i}_{n},i=1,\ldots,m\). Такие ряды, если они отличны от многочленов, расходятся в поле \(\mathbb{C}\).

Эти ряды сходятся в полях \(p\)--адических чисел, что позволяет рассматривать бесконечномерные векторы, координаты которых представляют собой суммы рассматриваемых рядов в полях \(p\)-адических чисел. Это позволяет ввести понятия бесконечной и глобальной линейной или алгебраической независимости. В работах [1, 2] установлены теоремы, подобные теоремам А.\,Б. Шидловского для \(E\)-функций (ряды вида \(\sum_{n=0}^{\infty}\frac{a_n}{n!}z^n\) ).

Суммы этих рядов (1) в кольце полиадических чисел, т.е. в  прямом произведении колец целых \(p\)--адических чисел обозначим \(\alpha_{i}^{(p)},i=1,\ldots,m \).
Утверждается, что если векторы \(A_{i}=(a^{i}_{0},\ldots,a^{i}_{T-1}), i=1,\ldots,m\) линейно независимы, то полиадические числа \(\alpha_{i}^{(p)},i=1,\ldots,m \) бесконечно алгебраически независимы.

\medskip

\begin{enumerate}
\item[{[1]}] Chirskii V.G. Product Formula, {\it Global relations and Polyadic Integers} //Russ. J. Math. Phys. 2019. V.26. No.3 P.286-305.
\item[{[2]}] Bertrand D.,Chirskii V.G., Yebbou J. {\it Effective estimates for global relations on Euler-type series} //Ann.Fac.Sci.Toulouse.2004.V.13,No.2.P.241-260.
\end{enumerate}
\end{talk}
\end{document}