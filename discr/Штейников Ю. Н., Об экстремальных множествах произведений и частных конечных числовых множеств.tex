\documentclass[12pt]{article}
\usepackage{hyphsubst}
\usepackage[T2A]{fontenc}
\usepackage[english,main=russian]{babel}
\usepackage[utf8]{inputenc}
\usepackage[letterpaper,top=2cm,bottom=2cm,left=2cm,right=2cm,marginparwidth=2cm]{geometry}
\usepackage{float}
\usepackage{mathtools, commath, amssymb, amsthm}
\usepackage{enumitem, tabularx,graphicx,url,xcolor,rotating,multicol,epsfig,colortbl,lipsum}

\setlist{topsep=1pt, itemsep=0em}
\setlength{\parindent}{0pt}
\setlength{\parskip}{6pt}

\usepackage{hyphenat}
\hyphenation{ма-те-ма-ти-ка вос-ста-нав-ли-вать}

\usepackage[math]{anttor}

\newenvironment{talk}[6]{%
\vskip 0pt\nopagebreak%
\vskip 0pt\nopagebreak%
\section*{#1}
\phantomsection
\addcontentsline{toc}{section}{#2. \textit{#1}}
% \addtocontents{toc}{\textit{#1}\par}
\textit{#2}\\\nopagebreak%
#3\\\nopagebreak%
\ifthenelse{\equal{#4}{}}{}{\url{#4}\\\nopagebreak}%
\ifthenelse{\equal{#5}{}}{}{Соавторы: #5\\\nopagebreak}%
\ifthenelse{\equal{#6}{}}{}{Секция: #6\\\nopagebreak}%
}

\definecolor{LovelyBrown}{HTML}{FDFCF5}

\usepackage[pdftex,
breaklinks=true,
bookmarksnumbered=true,
linktocpage=true,
linktoc=all
]{hyperref}

\begin{document}
\pagenumbering{gobble}
\pagestyle{plain}
\pagecolor{LovelyBrown}
\begin{talk}
{Об экстремальных множествах произведений и частных конечных числовых множеств}
{Штейников Юрий Николаевич}
{НИЦ Курчатовский иснтитут, НИИСИ}
{yuriisht@yandex.ru}
{}
{Теория чисел и дискретная математика}

Произведением множеств \(A\) и \(B\)  называется множество \(AB\), которое  по определению есть
\(AB= \{ab : a \in A, b \in B \}.\) Обозначим через \([N]\) интервал натуральных чисел \(\{1,2, \ldots ,N \}\)  и пусть  \(M_N = |[N][N]|\). Одна из давних классических задач П. Эрдеша о таблице умножения  ставит вопрос  об отыскании правильного порядка роста величины \(M_N\) при \(N \rightarrow \infty\). П. Эрдешем и впоследствии К.Фордом были установлены точные порядок роста этой величины. Некоторые направления дальнейших исследований, тесно связанной с этой задачей, были предложены  Х. Силлеруело с соавторами. Он формулируется следующим образом.  Каков максимальный размер множества \(A \subset [N]\), что \(|AA| \sim |A|^2/2\). К. Форд доказал существование множества с такими свойствами, размер которого достаточно близок к оптимальной границе. В своем докладе я расскажу об уточнении этого результата, а также о ряде задач, возникающих в теории произведений и частных целочисленных множеств.

\medskip

\begin{enumerate}
\item[{[1]}] К. Форд, “Экстремальные свойства произведений множеств”, Труды МИАН, 303 (2018),  239–245
\item[{[2]}] Ю. Н. Штейников, “Множества с экстремальным свойством произведения и его вариации”, Матем. заметки, 114:6 (2023), 922–930.
\item[{[3]}] Х. Силлеруело, Д. С. Рамана, О. Рамаре, “Частные и произведения подмножеств нулевой плотности множества натуральных чисел”, Труды МИАН, 296 (2017),  58–71 .
\item[{[4]}] Ю. Н. Штейников, “Частные плотных подмножеств целых чисел и короткие расстояния элементов произведения”, Матем. заметки, 111:1 (2022), 117–124
\end{enumerate}
\end{talk}
\end{document}