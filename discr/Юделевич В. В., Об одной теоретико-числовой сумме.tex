\documentclass[12pt]{article}
\usepackage{hyphsubst}
\usepackage[T2A]{fontenc}
\usepackage[english,main=russian]{babel}
\usepackage[utf8]{inputenc}
\usepackage[letterpaper,top=2cm,bottom=2cm,left=2cm,right=2cm,marginparwidth=2cm]{geometry}
\usepackage{float}
\usepackage{mathtools, commath, amssymb, amsthm}
\usepackage{enumitem, tabularx,graphicx,url,xcolor,rotating,multicol,epsfig,colortbl,lipsum}

\setlist{topsep=1pt, itemsep=0em}
\setlength{\parindent}{0pt}
\setlength{\parskip}{6pt}

\usepackage{hyphenat}
\hyphenation{ма-те-ма-ти-ка вос-ста-нав-ли-вать}

\usepackage[math]{anttor}

\newenvironment{talk}[6]{%
\vskip 0pt\nopagebreak%
\vskip 0pt\nopagebreak%
\section*{#1}
\phantomsection
\addcontentsline{toc}{section}{#2. \textit{#1}}
% \addtocontents{toc}{\textit{#1}\par}
\textit{#2}\\\nopagebreak%
#3\\\nopagebreak%
\ifthenelse{\equal{#4}{}}{}{\url{#4}\\\nopagebreak}%
\ifthenelse{\equal{#5}{}}{}{Соавторы: #5\\\nopagebreak}%
\ifthenelse{\equal{#6}{}}{}{Секция: #6\\\nopagebreak}%
}

\definecolor{LovelyBrown}{HTML}{FDFCF5}

\usepackage[pdftex,
breaklinks=true,
bookmarksnumbered=true,
linktocpage=true,
linktoc=all
]{hyperref}

\begin{document}
\pagenumbering{gobble}
\pagestyle{plain}
\pagecolor{LovelyBrown}
\begin{talk}
{Об одной теоретико-числовой сумме}
{Юделевич Виталий Викторович}
{МГУ им. М.\,В.~Ломоносова}
{Vitaliiyudelevich@mail.ru}
{}
{Теория чисел и дискретная математика}

В докладе пойдёт речь о доказательстве асимптотической формулы
\[ \mathop{{\sum}'}_{n\leqslant x}\dfrac{r(n)}{r(n+1)} = {x}{(\ln x)^{-3/4}}(c+o(1)),\ \ (x \to +\infty),\]
где \(r(n)\) обозначает число представлений \(n\) суммою двух квадратов:
\[r(n) = \#\left\{(a, b)\in \mathbb{Z}^2: n = a^2 + b^2\right\}\!,\]
\(c\)~--- это некоторая положительная постоянная, а штрих означает суммирование по тем \(n\), для которых \(r(n+1)\neq 0\).
\end{talk}
\end{document}