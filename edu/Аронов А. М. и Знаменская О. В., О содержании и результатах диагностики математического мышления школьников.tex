\documentclass[12pt]{article}
\usepackage{hyphsubst}
\usepackage[T2A]{fontenc}
\usepackage[english,main=russian]{babel}
\usepackage[utf8]{inputenc}
\usepackage[letterpaper,top=2cm,bottom=2cm,left=2cm,right=2cm,marginparwidth=2cm]{geometry}
\usepackage{float}
\usepackage{mathtools, commath, amssymb, amsthm}
\usepackage{enumitem, tabularx,graphicx,url,xcolor,rotating,multicol,epsfig,colortbl,lipsum}

\setlist{topsep=1pt, itemsep=0em}
\setlength{\parindent}{0pt}
\setlength{\parskip}{6pt}

\usepackage{hyphenat}
\hyphenation{ма-те-ма-ти-ка вос-ста-нав-ли-вать}

\usepackage[math]{anttor}

\newenvironment{talk}[6]{%
\vskip 0pt\nopagebreak%
\vskip 0pt\nopagebreak%
\section*{#1}
\phantomsection
\addcontentsline{toc}{section}{#2. \textit{#1}}
% \addtocontents{toc}{\textit{#1}\par}
\textit{#2}\\\nopagebreak%
#3\\\nopagebreak%
\ifthenelse{\equal{#4}{}}{}{\url{#4}\\\nopagebreak}%
\ifthenelse{\equal{#5}{}}{}{Соавторы: #5\\\nopagebreak}%
\ifthenelse{\equal{#6}{}}{}{Секция: #6\\\nopagebreak}%
}

\definecolor{LovelyBrown}{HTML}{FDFCF5}

\usepackage[pdftex,
breaklinks=true,
bookmarksnumbered=true,
linktocpage=true,
linktoc=all
]{hyperref}

\begin{document}
\pagenumbering{gobble}
\pagestyle{plain}
\pagecolor{LovelyBrown}
\begin{talk}
{О содержании и результатах диагностики математического мышления школьников на переходе из начальных классов в основную школу}
{Аронов Александр Моисеевич и Знаменская Оксана Витальевна}
{Московский городской педагогический университет, Сибирский федеральный университет}
{aronovam@mgpu.ru; ovznamenskaya@sfu-kras.ru}
{}
{Математическое образование и просвещение}

Рассматривается подход к математическому мышлению как системе средств организации математической деятельности. В основе этого подхода лежит трехуровневая модель ``Дельта'', позволяющая диагносцировать понимание и рефлексию при решении школьниками математических задач. Будут представлены результаты диагностики школьников 4–6  классов и дан анализ этих результатов с точки зрения выявления мыслительных средств, необходимых для освоения математических способов решения. Указаны риски на переходе из начальных классов в основную школу, связанные с дефицитами мышления учеников в понятиях начальной школы. Полученные данные диагностики позволяют проектировать дифференцированное обучение, применять методы индивидуализации и персонализации.
\end{talk}
\end{document}