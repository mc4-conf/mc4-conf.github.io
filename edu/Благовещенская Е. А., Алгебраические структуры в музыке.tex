\documentclass[12pt]{article}
\usepackage{hyphsubst}
\usepackage[T2A]{fontenc}
\usepackage[english,main=russian]{babel}
\usepackage[utf8]{inputenc}
\usepackage[letterpaper,top=2cm,bottom=2cm,left=2cm,right=2cm,marginparwidth=2cm]{geometry}
\usepackage{float}
\usepackage{mathtools, commath, amssymb, amsthm}
\usepackage{enumitem, tabularx,graphicx,url,xcolor,rotating,multicol,epsfig,colortbl,lipsum}

\setlist{topsep=1pt, itemsep=0em}
\setlength{\parindent}{0pt}
\setlength{\parskip}{6pt}

\usepackage{hyphenat}
\hyphenation{ма-те-ма-ти-ка вос-ста-нав-ли-вать}

\usepackage[math]{anttor}

\newenvironment{talk}[6]{%
\vskip 0pt\nopagebreak%
\vskip 0pt\nopagebreak%
\section*{#1}
\phantomsection
\addcontentsline{toc}{section}{#2. \textit{#1}}
% \addtocontents{toc}{\textit{#1}\par}
\textit{#2}\\\nopagebreak%
#3\\\nopagebreak%
\ifthenelse{\equal{#4}{}}{}{\url{#4}\\\nopagebreak}%
\ifthenelse{\equal{#5}{}}{}{Соавторы: #5\\\nopagebreak}%
\ifthenelse{\equal{#6}{}}{}{Секция: #6\\\nopagebreak}%
}

\definecolor{LovelyBrown}{HTML}{FDFCF5}

\usepackage[pdftex,
breaklinks=true,
bookmarksnumbered=true,
linktocpage=true,
linktoc=all
]{hyperref}

\begin{document}
\pagenumbering{gobble}
\pagestyle{plain}
\pagecolor{LovelyBrown}
\begin{talk}
{Алгебраические структуры в музыке}
{Благовещенская Екатерина Анатольевна}
{Петербургский государственный университет путей сообщения Императора Александра~I}
{kblag2002@yahoo.com}
{Александр Костроминов, Евгений Спиридонов}
{Математическое образование и просвещение}

Найдены различные способы формализации музыкальных построений, которые  дают возможность анализировать  существующие в них внутренние закономерности и формулировать их  в терминах алгебраических структур. Среди многочисленных связей музыки и математики существуют формальные связи, близкие к фундаментальной алгебре и обусловленные  достаточно жесткой структурой музыкальных построений.
Как известно, в алгебре с помощью операций и соотношений между элементами, подчиняющихся определенным законам, задаются связи на элементах множеств, которыми определяются различные алгебраические структуры. Поскольку природа самих элементов не важна, при осуществлении данного подхода допускается его распространение на множества элементов различной природы, в том числе, на звуки. Мы ограничимся звуками дискретной системы, создаваемыми при игре на фортепиано. При этом делается теоретическое предположение об идеальной ситуации, в которой  акустический интервал октава равномерно делится на 12 полутонов. Музыкальный фрагмент (произведение) может рассматриваться как элемент множества  с двумя алгебраическими операциями: сложение (коммутативная операция, заключающаяся в одновременном звучании нот, создающим  аккорд) и умножение (некоммутативная операция, заключающаяся в последовательном звучании отдельно взятых нот или аккордов). Образующими данной структуры являются сами ноты. Коммутативность первой операции и ассоциативность обеих позволяют рассматривать данную структуру как  квази-кольцевую структуру. Добавление временной характеристики (длительности звучания), превращает данную структуру в предалгебу, в которой определено умножение на элементы некоторого числового множества. Таким образом, имеет место скрытое математическое моделирование при создании музыкальных произведений, благозвучность которых не учитывается в данной модели. Для удовлетворения этого требования можно использовать добавление некоторых ограничений. Однако, подчеркнем, что формализация языка музыки не включает такую важную составляющую как талант композитора, обладание которым недоступно искусственному интеллекту.
\end{talk}
\end{document}