\documentclass[12pt]{article}
\usepackage{hyphsubst}
\usepackage[T2A]{fontenc}
\usepackage[english,main=russian]{babel}
\usepackage[utf8]{inputenc}
\usepackage[letterpaper,top=2cm,bottom=2cm,left=2cm,right=2cm,marginparwidth=2cm]{geometry}
\usepackage{float}
\usepackage{mathtools, commath, amssymb, amsthm}
\usepackage{enumitem, tabularx,graphicx,url,xcolor,rotating,multicol,epsfig,colortbl,lipsum}

\setlist{topsep=1pt, itemsep=0em}
\setlength{\parindent}{0pt}
\setlength{\parskip}{6pt}

\usepackage{hyphenat}
\hyphenation{ма-те-ма-ти-ка вос-ста-нав-ли-вать}

\usepackage[math]{anttor}

\newenvironment{talk}[6]{%
\vskip 0pt\nopagebreak%
\vskip 0pt\nopagebreak%
\section*{#1}
\phantomsection
\addcontentsline{toc}{section}{#2. \textit{#1}}
% \addtocontents{toc}{\textit{#1}\par}
\textit{#2}\\\nopagebreak%
#3\\\nopagebreak%
\ifthenelse{\equal{#4}{}}{}{\url{#4}\\\nopagebreak}%
\ifthenelse{\equal{#5}{}}{}{Соавторы: #5\\\nopagebreak}%
\ifthenelse{\equal{#6}{}}{}{Секция: #6\\\nopagebreak}%
}

\definecolor{LovelyBrown}{HTML}{FDFCF5}

\usepackage[pdftex,
breaklinks=true,
bookmarksnumbered=true,
linktocpage=true,
linktoc=all
]{hyperref}

\begin{document}
\pagenumbering{gobble}
\pagestyle{plain}
\pagecolor{LovelyBrown}
\begin{talk}
{О понятии математической грамотности}
{Боровских Алексей Владиславович}
{МГУ имени М.В.Ломоносова}
{aleksey.borovskikh@math.msu.ru}
{}
{Математическое образование и просвещение}

Математической грамотностью называется интеллектуальная способность, состоящая во владении математическими знаковыми средствами и проявляющаяся в решении задач с использованием этих средств. Она обеспечивает выстраивание отношения между задачей, сформулированной на общеупотребительном или профессиональном языке, и задачей математической. В состав математической грамотности входят:
\begin{itemize}
\item анализ задачи и выделение необходимых данных;
\item схематизация основных отношений между этими данными;
\item поиск таких же математических отношений и перенос на них данных задачи;
\item формулировка, с помощью схемы, математической постановки задачи;
\item решение математической задачи в рамках той или иной системы операций со знаковыми средствами;
\item переход, при необходимости, от одной знаковой системы к другой (например, от алгебраической к графической и обратно);
\item формулировка ответа для математической задачи;
\item интерпретация ответа и, при необходимости, промежуточных результатов, на схеме;
\item формулировка ответа на исходную задачу в терминах этой задачи и оценка соответствия ответа смыслу задачи.
\end{itemize}

Математическая грамотность характеризуется:
\begin{itemize}
\item набором освоенных знаковых средств и способов их использования;
\item классом решаемых задач.
\end{itemize}

\medskip

\begin{enumerate}
\item[{[1]}] Боровских А.В. {\it О понятии математической грамотности}, Педагогика, 86 (2022), 3, 33–45.
\end{enumerate}
\end{talk}
\end{document}