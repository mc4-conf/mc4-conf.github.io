\documentclass[12pt]{article}
\usepackage{hyphsubst}
\usepackage[T2A]{fontenc}
\usepackage[english,main=russian]{babel}
\usepackage[utf8]{inputenc}
\usepackage[letterpaper,top=2cm,bottom=2cm,left=2cm,right=2cm,marginparwidth=2cm]{geometry}
\usepackage{float}
\usepackage{mathtools, commath, amssymb, amsthm}
\usepackage{enumitem, tabularx,graphicx,url,xcolor,rotating,multicol,epsfig,colortbl,lipsum}

\setlist{topsep=1pt, itemsep=0em}
\setlength{\parindent}{0pt}
\setlength{\parskip}{6pt}

\usepackage{hyphenat}
\hyphenation{ма-те-ма-ти-ка вос-ста-нав-ли-вать}

\usepackage[math]{anttor}

\newenvironment{talk}[6]{%
\vskip 0pt\nopagebreak%
\vskip 0pt\nopagebreak%
\section*{#1}
\phantomsection
\addcontentsline{toc}{section}{#2. \textit{#1}}
% \addtocontents{toc}{\textit{#1}\par}
\textit{#2}\\\nopagebreak%
#3\\\nopagebreak%
\ifthenelse{\equal{#4}{}}{}{\url{#4}\\\nopagebreak}%
\ifthenelse{\equal{#5}{}}{}{Соавторы: #5\\\nopagebreak}%
\ifthenelse{\equal{#6}{}}{}{Секция: #6\\\nopagebreak}%
}

\definecolor{LovelyBrown}{HTML}{FDFCF5}

\usepackage[pdftex,
breaklinks=true,
bookmarksnumbered=true,
linktocpage=true,
linktoc=all
]{hyperref}

\begin{document}
\pagenumbering{gobble}
\pagestyle{plain}
\pagecolor{LovelyBrown}
\begin{talk}
{Изменение целей математического образования в контексте развития цифровых технологий}
{Вдовин Евгений Петрович}
{Тюменский государственный университет}
{e.p.vdovin@utmn.ru}
{}
{Математическое образование и просвещение}

В предлагаемом докладе мы различим на принципиальном уровне несколько тактов действий человека в ситуации достижения какой-либо цели (примеры таких различений приведены в [1] и [2]). На основании сформированного различения мы покажем, как выглядят текущие цели массового математического образования, сформулируем тезис о том, что в сложившейся технологической ситуации все эти цели сейчас закрывают цифровые технологии. После этого будут сформулированы те цели, которые более соответствуют сложившейся технологической ситуации. Завершим доклад примерами из практики автора и его команды по перестройке математического образования в соответствии с новыми целями.

\medskip

\begin{enumerate}
\item[{[1]}] Werner Blum, Rita Borromeo Ferri, {\it Mathematical Modelling: Can It Be Taught and Learnt?}, Journal of Mathematical Modelling and Application, 1 (2009), No. 1, 45--58.
\item[{[2]}] А.В. Боровских, {\it О понятии математической грамотности}, Бедагогика, 86 (2022), 3б 33--45.
\end{enumerate}
\end{talk}
\end{document}