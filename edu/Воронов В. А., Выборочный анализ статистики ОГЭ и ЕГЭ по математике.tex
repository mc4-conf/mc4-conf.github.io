\documentclass[12pt]{article}
\usepackage{hyphsubst}
\usepackage[T2A]{fontenc}
\usepackage[english,main=russian]{babel}
\usepackage[utf8]{inputenc}
\usepackage[letterpaper,top=2cm,bottom=2cm,left=2cm,right=2cm,marginparwidth=2cm]{geometry}
\usepackage{float}
\usepackage{mathtools, commath, amssymb, amsthm}
\usepackage{enumitem, tabularx,graphicx,url,xcolor,rotating,multicol,epsfig,colortbl,lipsum}

\setlist{topsep=1pt, itemsep=0em}
\setlength{\parindent}{0pt}
\setlength{\parskip}{6pt}

\usepackage{hyphenat}
\hyphenation{ма-те-ма-ти-ка вос-ста-нав-ли-вать}

\usepackage[math]{anttor}

\newenvironment{talk}[6]{%
\vskip 0pt\nopagebreak%
\vskip 0pt\nopagebreak%
\section*{#1}
\phantomsection
\addcontentsline{toc}{section}{#2. \textit{#1}}
% \addtocontents{toc}{\textit{#1}\par}
\textit{#2}\\\nopagebreak%
#3\\\nopagebreak%
\ifthenelse{\equal{#4}{}}{}{\url{#4}\\\nopagebreak}%
\ifthenelse{\equal{#5}{}}{}{Соавторы: #5\\\nopagebreak}%
\ifthenelse{\equal{#6}{}}{}{Секция: #6\\\nopagebreak}%
}

\definecolor{LovelyBrown}{HTML}{FDFCF5}

\usepackage[pdftex,
breaklinks=true,
bookmarksnumbered=true,
linktocpage=true,
linktoc=all
]{hyperref}

\begin{document}
\pagenumbering{gobble}
\pagestyle{plain}
\pagecolor{LovelyBrown}
\begin{talk}
{Выборочный анализ статистики ОГЭ и ЕГЭ по математике}
{Воронов Всеволод Александрович}
{Кавказский математический центр Адыгейского государственного университета, Московский физико-технический институт (национальный исследовательский университет)}
{v-vor@yandex.ru}
{}
{Математическое образование и просвещение}

Статистические данные, которые обрабатывают Федеральный институт педагогических измерений и Рособрнадзор по результатам государственной итоговой аттестации в 9 и 11 классах школы, к сожалению, лишь частично являются открытыми. Далеко не все регионы публикуют статистические отчеты, и не всегда статистический отчет по единой форме доступен для страны в целом. Деперсонифицированная база результатов ОГЭ/ЕГЭ доступна лишь на информационных ресурсах отдельных регионов. Анализ статистики затрудняется, кроме того, запретом на публикацию контрольно-измерительных материалов, не имеющим срока давности. Тем не менее на основе неполных данных можно сделать ряд любопытных наблюдений.

\begin{enumerate}
\item Для Сибири и Дальнего Востока типичны сравнительно низкие результаты по профильной математике в сравнении с регионами Европейской части. Это не всегда может быть объяснено различным уровнем социально-экономического развития регионов.
\item Из тех регионов, для которых статистика доступна, наилучшие показатели по профильной математике имеет Татарстан.
\item В 2021-м году средний балл ЕГЭ по профильной математике в Москве был ниже, чем в среднем по России.
\item В нескольких регионах на графике распределения баллов ОГЭ по математике наблюдается резкое падение числа участников при переходе от 19 первичных баллов к 20 (граница первой части ОГЭ).
\item В статистических отчетах редко указывают процент двоек, полученных в основной период (без учета пересдач).
\end{enumerate}
\end{talk}
\end{document}