\documentclass[12pt]{article}
\usepackage{hyphsubst}
\usepackage[T2A]{fontenc}
\usepackage[english,main=russian]{babel}
\usepackage[utf8]{inputenc}
\usepackage[letterpaper,top=2cm,bottom=2cm,left=2cm,right=2cm,marginparwidth=2cm]{geometry}
\usepackage{float}
\usepackage{mathtools, commath, amssymb, amsthm}
\usepackage{enumitem, tabularx,graphicx,url,xcolor,rotating,multicol,epsfig,colortbl,lipsum}

\setlist{topsep=1pt, itemsep=0em}
\setlength{\parindent}{0pt}
\setlength{\parskip}{6pt}

\usepackage{hyphenat}
\hyphenation{ма-те-ма-ти-ка вос-ста-нав-ли-вать}

\usepackage[math]{anttor}

\newenvironment{talk}[6]{%
\vskip 0pt\nopagebreak%
\vskip 0pt\nopagebreak%
\section*{#1}
\phantomsection
\addcontentsline{toc}{section}{#2. \textit{#1}}
% \addtocontents{toc}{\textit{#1}\par}
\textit{#2}\\\nopagebreak%
#3\\\nopagebreak%
\ifthenelse{\equal{#4}{}}{}{\url{#4}\\\nopagebreak}%
\ifthenelse{\equal{#5}{}}{}{Соавторы: #5\\\nopagebreak}%
\ifthenelse{\equal{#6}{}}{}{Секция: #6\\\nopagebreak}%
}

\definecolor{LovelyBrown}{HTML}{FDFCF5}

\usepackage[pdftex,
breaklinks=true,
bookmarksnumbered=true,
linktocpage=true,
linktoc=all
]{hyperref}

\begin{document}
\pagenumbering{gobble}
\pagestyle{plain}
\pagecolor{LovelyBrown}
\begin{talk}
{Учебный курс ``Культура математических рассуждений''
для студентов-программистов в техническом вузе}
{Герасимов Александр Сергеевич}
{Санкт-Петербургский политехнический университет Петра Великого}
{alexander.s.gerasimov@ya.ru}
{}
{Математическое образование и просвещение}

Представляется учебный курс ``Культура математических рассуждений'',
введенный автором доклада для студентов, обучающихся по направлению
подготовки ``Фундаментальная информатика и информационные технологи''
в Санкт-Петербургском политехническом университете.
Целью этого курса является систематическое освоение базовых приемов
математических рассуждений, что нужно по меньшей мере для доказательства
корректности алгоримов, которые изучаются в последующем курсе
``Алгоритмы и анализ сложности'', также читаемом автором доклада.
На курс ``Культура математических рассуждений'' отведено 30 академических часов
практических занятий, включающих в себя элементы лекции.
Большая часть этого курса посвящена изучению исчисления натуральных выводов
в стиле С. Яськовского, построению формальных доказательств (или выводов)
в этом исчислении и построению неформальных (или содержательных) доказательств,
близких по структуре к формальным.
Также в данном курсе систематизируются базовые понятия и факты теории множеств;
изучаются метод возвратной индукции (применяемый, в частности,
для доказательства корректности рекурсивных алгоритмов)
и метод инвариантов циклов для доказательства корректности алгоритмов,содержащих циклы.
\end{talk}
\end{document}