\documentclass[12pt]{article}
\usepackage{hyphsubst}
\usepackage[T2A]{fontenc}
\usepackage[english,main=russian]{babel}
\usepackage[utf8]{inputenc}
\usepackage[letterpaper,top=2cm,bottom=2cm,left=2cm,right=2cm,marginparwidth=2cm]{geometry}
\usepackage{float}
\usepackage{mathtools, commath, amssymb, amsthm}
\usepackage{enumitem, tabularx,graphicx,url,xcolor,rotating,multicol,epsfig,colortbl,lipsum}

\setlist{topsep=1pt, itemsep=0em}
\setlength{\parindent}{0pt}
\setlength{\parskip}{6pt}

\usepackage{hyphenat}
\hyphenation{ма-те-ма-ти-ка вос-ста-нав-ли-вать}

\usepackage[math]{anttor}

\newenvironment{talk}[6]{%
\vskip 0pt\nopagebreak%
\vskip 0pt\nopagebreak%
\section*{#1}
\phantomsection
\addcontentsline{toc}{section}{#2. \textit{#1}}
% \addtocontents{toc}{\textit{#1}\par}
\textit{#2}\\\nopagebreak%
#3\\\nopagebreak%
\ifthenelse{\equal{#4}{}}{}{\url{#4}\\\nopagebreak}%
\ifthenelse{\equal{#5}{}}{}{Соавторы: #5\\\nopagebreak}%
\ifthenelse{\equal{#6}{}}{}{Секция: #6\\\nopagebreak}%
}

\definecolor{LovelyBrown}{HTML}{FDFCF5}

\usepackage[pdftex,
breaklinks=true,
bookmarksnumbered=true,
linktocpage=true,
linktoc=all
]{hyperref}

\begin{document}
\pagenumbering{gobble}
\pagestyle{plain}
\pagecolor{LovelyBrown}
\begin{talk}
{Использование платформы GeoLin для проведения ранжирующего теста с целью определения уровня освоения дисциплин математического цикла в высшей школе}
{Ершов Александр Романович}
{Университет ИТМО, НОЦ Математики}
{alex2002andr@mail.ru}
{}
{Математическое образование и просвещения}


В работе представлена информация о том, как в Университете ИТМО проводится тест на определение уровня знаний математики у студентов первого курса. Представлено описание возможностей платформы GeoLin, структура теста. Представлена аналитика результатов теста 2023 года. Показано решение таких задач как составление портрета целевой аудитории, алгоритм подбора задач. Приведены пример заданий для теста и данные, полученные в ходе тестирования, которые используются для дальнейшей аналитики.

\medskip

\begin{enumerate}
\item[{[1]}] Математика. Адаптационный курс : учеб. пособие / ЗЕНШ при СФУ; сост. : А.М. Кытманов, Е.К. Лейнартас, С.Г. Мысливец. — Красноярск ИПК СФУ, 2009. — 196~с.
\item[{[2]}] Математика и математическое образование: проблемы, технологии, перспективы. Материалы 42-го Международного научного семинара преподавателей математики и информатики университетов и педагогических вузов. Смоленск, 2023. С. 170-173.
\end{enumerate}
\end{talk}
\end{document}