\documentclass[12pt]{article}
\usepackage{hyphsubst}
\usepackage[T2A]{fontenc}
\usepackage[english,main=russian]{babel}
\usepackage[utf8]{inputenc}
\usepackage[letterpaper,top=2cm,bottom=2cm,left=2cm,right=2cm,marginparwidth=2cm]{geometry}
\usepackage{float}
\usepackage{mathtools, commath, amssymb, amsthm}
\usepackage{enumitem, tabularx,graphicx,url,xcolor,rotating,multicol,epsfig,colortbl,lipsum}

\setlist{topsep=1pt, itemsep=0em}
\setlength{\parindent}{0pt}
\setlength{\parskip}{6pt}

\usepackage{hyphenat}
\hyphenation{ма-те-ма-ти-ка вос-ста-нав-ли-вать}

\usepackage[math]{anttor}

\newenvironment{talk}[6]{%
\vskip 0pt\nopagebreak%
\vskip 0pt\nopagebreak%
\section*{#1}
\phantomsection
\addcontentsline{toc}{section}{#2. \textit{#1}}
% \addtocontents{toc}{\textit{#1}\par}
\textit{#2}\\\nopagebreak%
#3\\\nopagebreak%
\ifthenelse{\equal{#4}{}}{}{\url{#4}\\\nopagebreak}%
\ifthenelse{\equal{#5}{}}{}{Соавторы: #5\\\nopagebreak}%
\ifthenelse{\equal{#6}{}}{}{Секция: #6\\\nopagebreak}%
}

\definecolor{LovelyBrown}{HTML}{FDFCF5}

\usepackage[pdftex,
breaklinks=true,
bookmarksnumbered=true,
linktocpage=true,
linktoc=all
]{hyperref}

\begin{document}
\pagenumbering{gobble}
\pagestyle{plain}
\pagecolor{LovelyBrown}
\begin{talk}
{Большая математическая мастерская: итоги потока ``Образование''}
{Карпенко Анастасия Валерьевна}
{Новосибирский государственный университет, НГУ}
{anastasia.v.karpenko@gmail.com}
{Абдыкеров Жанат Сергеевич}
{Математическое образование и просвещение}

Большая математическая мастерская (БММ) --- научно-образовательное мероприятие, в рамках которого команды школьников, студентов и педагогов при сопровождении кураторов в интенсивном формате работают над решением реальных задач, имеющих математическую составляющую.

БММ реализуется с 2020 года. В 2024 году площадками для проведения Мастерской выступили: Математический центр в Академгородке, Омский филиал Института математики им. С.Л. Соболева СО РАН, Региональный научно-образовательный математический центр Томского государственного университета, Региональный научно-образова-тельный математический центр Адыгейского государственного университета — ``Кавказский математический центр'' и Школа компьютернаых наук Тюменского государственного университета.

С момента создания БММ приросла не только количественно, но и качественно: в 2021 появился школьный поток, а с 2022 года Мастерская набирает проекты в области образования. Реализация в параллели потоков для школьников и педагогов позволяет педагогам за время Мастерской планировать и проводить полноценные эксперименты, а также обобщать полученный опыт.

Так, например, команда проекта <<Исследовательские блуждания по стереометрической задаче>> в рамках первого модуля проработала концепт применения метода работы с задачной ситуацией, дважды аппробировали его на школьниках, а затем обобщила результаты в виде заметки об использовании метода при решении стереометрических задач.

Другим примером результата проекта является полноценная потенциально коммерциализуемая игра ``Геоформы'', геймифицирующая изучение школьного курса геометрии 7-9 классов. Игра разработана помандой проекта ``Использование игровых инструментов в преподавании математики'', аппробирована на БММ и представлена на Августовской конференции в г. Томске в 2023 году.

В рамках доклада будет представлен концепт реализации потока ``Образование'' на Большой математической мастерской, а также результаты, которых удалось добиться командам проектов.
\end{talk}
\end{document}