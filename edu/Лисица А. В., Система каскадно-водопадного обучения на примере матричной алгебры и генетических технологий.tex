\documentclass[12pt]{article}
\usepackage{hyphsubst}
\usepackage[T2A]{fontenc}
\usepackage[english,main=russian]{babel}
\usepackage[utf8]{inputenc}
\usepackage[letterpaper,top=2cm,bottom=2cm,left=2cm,right=2cm,marginparwidth=2cm]{geometry}
\usepackage{float}
\usepackage{mathtools, commath, amssymb, amsthm}
\usepackage{enumitem, tabularx,graphicx,url,xcolor,rotating,multicol,epsfig,colortbl,lipsum}

\setlist{topsep=1pt, itemsep=0em}
\setlength{\parindent}{0pt}
\setlength{\parskip}{6pt}

\usepackage{hyphenat}
\hyphenation{ма-те-ма-ти-ка вос-ста-нав-ли-вать}

\usepackage[math]{anttor}

\newenvironment{talk}[6]{%
\vskip 0pt\nopagebreak%
\vskip 0pt\nopagebreak%
\section*{#1}
\phantomsection
\addcontentsline{toc}{section}{#2. \textit{#1}}
% \addtocontents{toc}{\textit{#1}\par}
\textit{#2}\\\nopagebreak%
#3\\\nopagebreak%
\ifthenelse{\equal{#4}{}}{}{\url{#4}\\\nopagebreak}%
\ifthenelse{\equal{#5}{}}{}{Соавторы: #5\\\nopagebreak}%
\ifthenelse{\equal{#6}{}}{}{Секция: #6\\\nopagebreak}%
}

\definecolor{LovelyBrown}{HTML}{FDFCF5}

\usepackage[pdftex,
breaklinks=true,
bookmarksnumbered=true,
linktocpage=true,
linktoc=all
]{hyperref}

\begin{document}
\pagenumbering{gobble}
\pagestyle{plain}
\pagecolor{LovelyBrown}
\begin{talk}
{Система каскадно-водопадного обучения на примере матричной алгебры и генетических технологий}
{Лисица Андрей Валерьевич}
{ФГАОУ ВО ``Тюменский государственный университет''}
{lisitsa052@gmail.com}
{Андреюк Денис Сергеевич, Российская ассоциация содействия науке (РАСН); Козлова Анна Сергеена, ФГБНУ ``Научно-исследовательский институт биомедицинской химии имени В.\,Н. Ореховича'' (ИБМХ)}
{Математическое образование и просвещение}

Рассматриваются результаты применения системы ``Таблекс'' (разработчик --- ООО ``КуБ'') на базе Центра научно-практического образования ИБМХ с 2021 по 2024 г. Система [1] предоставляет возможность организации научных кружков для школьников и студентов. Проведены следующие курсы: генетические технологии (сборка геномов), анализ широкомасштабных протеомных и метаболомных данных, 3D моделирование белков, с применением облачных технологий Яндекс-Клауд. Математическое мышление формируется с использованием стандартных модулей Питона — а именно гистограмм, диаграмм Венна, методов распознавания образов, анализ главных компонент, дисперсионный анализ.

Особенностью методологии преподавания в системе ``Таблекс'' является парное программирование, где есть роли пилота, штурмана и инструктора. Пилоты и штурманы, достигая уровня инструктора, вовлекаются в систему монетизации и школьник может заработать до 500 рублей в день. Результатом такого подхода являются команды, которые обновляют кадровый состав при реализации федеральных программ, связанных с генетическими технологиями и применением искусственного интеллекта. Общая логика построения кадрового резерва предложена в рамках системы ``Кванториум''.

\medskip

\begin{enumerate}
\item[{[1]}] http://oookub.ru/tablex-main.html ``Таблекс - каскадно-водопадная система обучения'' св. о рег. программы для ЭВМ №2022685715 от 27.12.2022.
\end{enumerate}
\end{talk}
\end{document}