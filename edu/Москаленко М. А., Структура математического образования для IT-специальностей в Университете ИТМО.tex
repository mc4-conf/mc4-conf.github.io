\documentclass[12pt]{article}
\usepackage{hyphsubst}
\usepackage[T2A]{fontenc}
\usepackage[english,main=russian]{babel}
\usepackage[utf8]{inputenc}
\usepackage[letterpaper,top=2cm,bottom=2cm,left=2cm,right=2cm,marginparwidth=2cm]{geometry}
\usepackage{float}
\usepackage{mathtools, commath, amssymb, amsthm}
\usepackage{enumitem, tabularx,graphicx,url,xcolor,rotating,multicol,epsfig,colortbl,lipsum}

\setlist{topsep=1pt, itemsep=0em}
\setlength{\parindent}{0pt}
\setlength{\parskip}{6pt}

\usepackage{hyphenat}
\hyphenation{ма-те-ма-ти-ка вос-ста-нав-ли-вать}

\usepackage[math]{anttor}

\newenvironment{talk}[6]{%
\vskip 0pt\nopagebreak%
\vskip 0pt\nopagebreak%
\section*{#1}
\phantomsection
\addcontentsline{toc}{section}{#2. \textit{#1}}
% \addtocontents{toc}{\textit{#1}\par}
\textit{#2}\\\nopagebreak%
#3\\\nopagebreak%
\ifthenelse{\equal{#4}{}}{}{\url{#4}\\\nopagebreak}%
\ifthenelse{\equal{#5}{}}{}{Соавторы: #5\\\nopagebreak}%
\ifthenelse{\equal{#6}{}}{}{Секция: #6\\\nopagebreak}%
}

\definecolor{LovelyBrown}{HTML}{FDFCF5}

\usepackage[pdftex,
breaklinks=true,
bookmarksnumbered=true,
linktocpage=true,
linktoc=all
]{hyperref}

\begin{document}
\pagenumbering{gobble}
\pagestyle{plain}
\pagecolor{LovelyBrown}
\begin{talk}
{Структура математического образования для IT-спе\-ци\-аль\-нос\-тей в Университете ИТМО}
{Москаленко Мария Александровна}
{НОЦ Математики Университета ИТМО}
{moskalenko.mary@gmail.com}
{Табиева Арина Вадимовна, Трифанов Александр Игоревич}
{Математическое образование и просвещение}

В докладе будет представлена структура математического образования для IT-спе\-ци\-аль\-но\-стей Университета ИТМО. В современных реалиях выпускник, чтобы быть востребованным специалистом на рынке труда, должен обладать набором уникальных компетенций, что возможно только при персонифицированном образовательном треке.
Персонификация образования возможна только при наличии понятных критериев, отвечающих современным требованиям стандартов математических дисциплин. Критерии позволяют сформировать единую образовательную среду, которая предоставляет одинаковые возможности каждому студенту и соответствует следующим потребностям обучающихся: получение качественного образования, защита от перегрузок, сохранение психического и физического здоровья. В рамках  образовательной системы, созданной благодаря критериям, возникает преемственность образовательных программ на разных ступенях образования, что вкупе с инструментами обратной связи между студентами и педагогами позволяет выйти на новый уровень качества математического образования.
\end{talk}
\end{document}