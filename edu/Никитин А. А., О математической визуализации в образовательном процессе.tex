\documentclass[12pt]{article}
\usepackage{hyphsubst}
\usepackage[T2A]{fontenc}
\usepackage[english,main=russian]{babel}
\usepackage[utf8]{inputenc}
\usepackage[letterpaper,top=2cm,bottom=2cm,left=2cm,right=2cm,marginparwidth=2cm]{geometry}
\usepackage{float}
\usepackage{mathtools, commath, amssymb, amsthm}
\usepackage{enumitem, tabularx,graphicx,url,xcolor,rotating,multicol,epsfig,colortbl,lipsum}

\setlist{topsep=1pt, itemsep=0em}
\setlength{\parindent}{0pt}
\setlength{\parskip}{6pt}

\usepackage{hyphenat}
\hyphenation{ма-те-ма-ти-ка вос-ста-нав-ли-вать}

\usepackage[math]{anttor}

\newenvironment{talk}[6]{%
\vskip 0pt\nopagebreak%
\vskip 0pt\nopagebreak%
\section*{#1}
\phantomsection
\addcontentsline{toc}{section}{#2. \textit{#1}}
% \addtocontents{toc}{\textit{#1}\par}
\textit{#2}\\\nopagebreak%
#3\\\nopagebreak%
\ifthenelse{\equal{#4}{}}{}{\url{#4}\\\nopagebreak}%
\ifthenelse{\equal{#5}{}}{}{Соавторы: #5\\\nopagebreak}%
\ifthenelse{\equal{#6}{}}{}{Секция: #6\\\nopagebreak}%
}

\definecolor{LovelyBrown}{HTML}{FDFCF5}

\usepackage[pdftex,
breaklinks=true,
bookmarksnumbered=true,
linktocpage=true,
linktoc=all
]{hyperref}

\begin{document}
\pagenumbering{gobble}
\pagestyle{plain}
\pagecolor{LovelyBrown}
\begin{talk}
{О математической визуализации в образовательном процессе}
{Никитин Алексей Антонович}
{МГУ им. М.\,В. Ломоносова, факультет ВМК}
{nikitin@cs.msu.ru}
{}
{Математическое образование и просвещение}

Настоящая работа посвящена вопросу использования современных информационных технологий в аудиторном образовательном процессе. В ней подчёркивается необходимость объединения символьной и визуальной математики, описываются проблемы, связанные с этим вопросом, делается обзор существующих систем и определяются требования, которым должна удовлетворять современная система визуализации. В работе обсуждаются существующие наработки, созданные командой авторов. Описывается работа библиотеки визуализаций visualmath.ru. Этот ресурс содержит объёмный архив текстовых и визуальных модулей,  на основе которых преподаватели смогут создавать свои лекции-презентации, снабжённые большим количеством визуальных материалов. Другой важнейшей частью доклада является описание работы быстрых и мощных графических JavaScript-библиотек: Skeleton и Grafar. Первая из этих библиотек предназначена для отображения двумерных графиков и способна обрабатывать очень большие массивы элементов за исключительно короткое время, а вторая позволяет визуализировать красивейшие трёхмерные объекты,  прорабатывать их освещённость, прозрачность и т.п. В заключении приводится ряд примеров использования вышеописанных библиотек. Демонстрируются уже созданные визуализации для курсов математического анализа и аналитической геометрии.

\medskip

\begin{enumerate}
\item[{[1]}] Karpov A. D., Klepov V. Y., Nikitin A. A. On mathematical visualization in education // Communications in Computer and Information Science. — 2020. — Vol. 1140, no. 1. --- P. 11–27.
\end{enumerate}
\end{talk}
\end{document}