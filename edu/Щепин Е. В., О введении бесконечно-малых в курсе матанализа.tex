\documentclass[12pt]{article}
\usepackage{hyphsubst}
\usepackage[T2A]{fontenc}
\usepackage[english,main=russian]{babel}
\usepackage[utf8]{inputenc}
\usepackage[letterpaper,top=2cm,bottom=2cm,left=2cm,right=2cm,marginparwidth=2cm]{geometry}
\usepackage{float}
\usepackage{mathtools, commath, amssymb, amsthm}
\usepackage{enumitem, tabularx,graphicx,url,xcolor,rotating,multicol,epsfig,colortbl,lipsum}

\setlist{topsep=1pt, itemsep=0em}
\setlength{\parindent}{0pt}
\setlength{\parskip}{6pt}

\usepackage{hyphenat}
\hyphenation{ма-те-ма-ти-ка вос-ста-нав-ли-вать}

\usepackage[math]{anttor}

\newenvironment{talk}[6]{%
\vskip 0pt\nopagebreak%
\vskip 0pt\nopagebreak%
\section*{#1}
\phantomsection
\addcontentsline{toc}{section}{#2. \textit{#1}}
% \addtocontents{toc}{\textit{#1}\par}
\textit{#2}\\\nopagebreak%
#3\\\nopagebreak%
\ifthenelse{\equal{#4}{}}{}{\url{#4}\\\nopagebreak}%
\ifthenelse{\equal{#5}{}}{}{Соавторы: #5\\\nopagebreak}%
\ifthenelse{\equal{#6}{}}{}{Секция: #6\\\nopagebreak}%
}

\definecolor{LovelyBrown}{HTML}{FDFCF5}

\usepackage[pdftex,
breaklinks=true,
bookmarksnumbered=true,
linktocpage=true,
linktoc=all
]{hyperref}

\begin{document}
\pagenumbering{gobble}
\pagestyle{plain}
\pagecolor{LovelyBrown}
\begin{talk}
{О введении бесконечно-малых в курсе матанализа}
{Щепин Евгений Витальевич}
{Математический институт им. В.А. Стеклова РАН}
{scepin@mi-ras.ru}
{}
{Математическое образование и просвещение}

В докладе изложен подход к
введению и активному использованию актуально бес\-ко\-неч\-но-ма\-лых величин в начальном курсе математического анализа, который достаточно строг и
хорошо адаптирован к применениям в физике и геометрии.
Для описания дифференциалов функций одной переменной используются числа, известные под именем \emph{дуальных},
которые автор предпочитает называть \emph{числами двойной точности}, мотивируя это название компьютерной аналогией.
Числа двойной точности представляют собой минимальное неархимедово расширение действительных чисел. А именно,
к полю действительных чисел добавляется один "идеальный" бесконечно-малый  элемент, обозначаемый \(\sqrt0\), который положителен
но имеет нулевой квадрат.
В результате возникает линейно упорядоченное кольцо чисел вида \(a+b\sqrt0\) с интуитивно понятными операциями
сложения и умножения. Для определения значений трансцендентных функций на числах двойной точности достаточно постулировать,
что все известные для них  нестрогие неравенства для действительных чисел остаются справедливыми для чисел двойной точности.

Связь с теорией пределов обеспечивается следующей моделью построения чисел двойной точности. Действительные числа
интерпретируются как постоянные последовательности. \(\sqrt0\) интерпретируется как монотонная стремящаяся к нулю последовательность
положительных чисел \(o_n\). А числа двойной точности интерпретируются как сходящиеся последовательности \(x_n\), такие что
сходятся последовательности отношений \(\frac {x_n-\lim x_n}{o_n}\). Число двойной точности \(a+b\sqrt0\)  представляется как
совокупность всех последовательностей с описанными условиями сходимости, для которых \(\lim x_n=a\),  \(\lim \frac {x_n-a}{o_n}=b\).
\end{talk}
\end{document}