\documentclass[12pt]{article}
\usepackage{hyphsubst}
\usepackage[T2A]{fontenc}
\usepackage[english,main=russian]{babel}
\usepackage[utf8]{inputenc}
\usepackage[letterpaper,top=2cm,bottom=2cm,left=2cm,right=2cm,marginparwidth=2cm]{geometry}
\usepackage{float}
\usepackage{mathtools, commath, amssymb, amsthm}
\usepackage{enumitem, tabularx,graphicx,url,xcolor,rotating,multicol,epsfig,colortbl,lipsum}

\setlist{topsep=1pt, itemsep=0em}
\setlength{\parindent}{0pt}
\setlength{\parskip}{6pt}

\usepackage{hyphenat}
\hyphenation{ма-те-ма-ти-ка вос-ста-нав-ли-вать}

\usepackage[math]{anttor}

\newenvironment{talk}[6]{%
\vskip 0pt\nopagebreak%
\vskip 0pt\nopagebreak%
\section*{#1}
\phantomsection
\addcontentsline{toc}{section}{#2. \textit{#1}}
% \addtocontents{toc}{\textit{#1}\par}
\textit{#2}\\\nopagebreak%
#3\\\nopagebreak%
\ifthenelse{\equal{#4}{}}{}{\url{#4}\\\nopagebreak}%
\ifthenelse{\equal{#5}{}}{}{Соавторы: #5\\\nopagebreak}%
\ifthenelse{\equal{#6}{}}{}{Секция: #6\\\nopagebreak}%
}

\definecolor{LovelyBrown}{HTML}{FDFCF5}

\usepackage[pdftex,
breaklinks=true,
bookmarksnumbered=true,
linktocpage=true,
linktoc=all
]{hyperref}

\begin{document}
\pagenumbering{gobble}
\pagestyle{plain}
\pagecolor{LovelyBrown}
\begin{talk}
{Аналитическое представление \(\sigma\)-функции Вейерштрасса и её вычисление}
{Алексеев Максим Максимович}
{ФИЦ ИЦ РАН}
{alienkseev@gmail.com}
{Безродных Сергей Игоревич}
{Уравнения в частных производных, математическая физика и спектральная теория}

\textit{Функции Вейерштрасса} возникают естественным образом в контексте \textit{эллиптических функций} --- двоякопериодечиских мероморфных функций, описывающих решения дифференциальных уравнений в механике, реализующих некоторые классические конформные отображения и формирующих фундамент трансцендентых методов в алгебраической геометрии. Одной из функций Вейерштрасса является \(\sigma\)\textit{-функция} --- целая функция одного комплексного переменного, которая может быть использована для вычисления любой эллиптической функции с согласованной периодической структурой.

В данном докладе будут представлены результаты исследования \textit{системы уравнений в частных производных} для \(\sigma\)-функции Вейерштрасса [1] и эквивалентных ей счетных системы ОДУ, а также будут описаны новые \textit{рекуррентные и нерекуррентные представления для коэффициентов} ряда Тейлора [2]. Кроме того, будет затронута тематика вычисления эллиптических функций и функций Вейерштрасса с известной относительной точностью на основании расчётов значений \(\sigma\){-функции}.

\medskip

\begin{enumerate}
\item[{[1]}] Weierstrass K. \textit{Zur Theorie der elliptischen Functionen}, Sitzungsberichte der Akademie der Wissenschaften zu Berlin, 1882-1883.
\item[{[2]}] {Alekseev M., Bezrodnykh S.} \textit{System of Partial Differential Equations and Analytical Representations of the Weierstrass Sigma Function}, Mathematical Notes, 114 (2024), 1094–1102.
\end{enumerate}
\end{talk}
\end{document}