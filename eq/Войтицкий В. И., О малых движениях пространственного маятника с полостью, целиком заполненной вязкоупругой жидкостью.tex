\documentclass[12pt]{article}
\usepackage{hyphsubst}
\usepackage[T2A]{fontenc}
\usepackage[english,main=russian]{babel}
\usepackage[utf8]{inputenc}
\usepackage[letterpaper,top=2cm,bottom=2cm,left=2cm,right=2cm,marginparwidth=2cm]{geometry}
\usepackage{float}
\usepackage{mathtools, commath, amssymb, amsthm}
\usepackage{enumitem, tabularx,graphicx,url,xcolor,rotating,multicol,epsfig,colortbl,lipsum}

\setlist{topsep=1pt, itemsep=0em}
\setlength{\parindent}{0pt}
\setlength{\parskip}{6pt}

\usepackage{hyphenat}
\hyphenation{ма-те-ма-ти-ка вос-ста-нав-ли-вать}

\usepackage[math]{anttor}

\newenvironment{talk}[6]{%
\vskip 0pt\nopagebreak%
\vskip 0pt\nopagebreak%
\section*{#1}
\phantomsection
\addcontentsline{toc}{section}{#2. \textit{#1}}
% \addtocontents{toc}{\textit{#1}\par}
\textit{#2}\\\nopagebreak%
#3\\\nopagebreak%
\ifthenelse{\equal{#4}{}}{}{\url{#4}\\\nopagebreak}%
\ifthenelse{\equal{#5}{}}{}{Соавторы: #5\\\nopagebreak}%
\ifthenelse{\equal{#6}{}}{}{Секция: #6\\\nopagebreak}%
}

\definecolor{LovelyBrown}{HTML}{FDFCF5}

\usepackage[pdftex,
breaklinks=true,
bookmarksnumbered=true,
linktocpage=true,
linktoc=all
]{hyperref}

\begin{document}
\pagenumbering{gobble}
\pagestyle{plain}
\pagecolor{LovelyBrown}
\begin{talk}
{О малых движениях пространственного маятника с полостью, целиком заполненной вязкоупругой жидкостью}
{Войтицкий Виктор Иванович}
{Математический институт имени С.\,М.~Никольского, РУДН}
{voytitskiy_vi@rudn.ru}
{Цветков Денис Олегович (КФУ имени В.\,И. Вернадского)}
{Уравнения в частных производных, математическая физика и спектральная теория}


Рассматривается линеаризованная проблема малых пространственных движений маятника с полостью целиком заполненной вязкоупругой жидкостью обобщенной модели Олдройта. С применением теории операторов, действующих в гильбертовом пространстве, задача сводится к дифференциально-операторному уравнению первого порядка с главным макси-мальным аккретивным оператором. Отсюда при естественных условиях на начальные данные и правую часть доказывается теорема о существовании и единственности сильного решения. Соответствующая спектральная задача имеет дискретный спектр, располага-ющийся в правой комплексной полуплоскости симметрично относительно вещественной оси. С помощью теории операторов, действующих в пространстве с индефинитной метрикой доказано, что  невещественный спектр содержит не более конечного числа собственных значений, при этом предельными точками кроме бесконечности являются также некоторые положительные числа (являющиеся нулями характеристической функ-ции). Эта модель обобщает случай классической вязкой жидкости в полости, где спектральная задача дискретный спектр с единственной предельной точкой на бесконечности. Для частей корневых элементов установлено, что они образуют (после проектирования) базисы Рисса с конечным дефектом в основных гильбертовых пространствах.

\medskip

\begin{enumerate}
\item[{[1]}] Батыр Э.И., Копачевский Н.Д., {\it Малые движения и нормальные колебания системы сочленённых гиростатов}// СМФН. -- 2013, том 49. -- C.~5--88.
\item[{[2]}] Войтицкий В.И., Копачевский Н.Д., {\it О колебаниях сочлененных маятников с полостями, заполненными однородными жидкостями}// СМФН. -- 2019, том 65,
выпуск 3. -- C.~434--512.
\end{enumerate}
\end{talk}
\end{document}