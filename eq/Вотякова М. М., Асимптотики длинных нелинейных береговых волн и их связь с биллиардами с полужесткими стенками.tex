\documentclass[12pt]{article}
\usepackage{hyphsubst}
\usepackage[T2A]{fontenc}
\usepackage[english,main=russian]{babel}
\usepackage[utf8]{inputenc}
\usepackage[letterpaper,top=2cm,bottom=2cm,left=2cm,right=2cm,marginparwidth=2cm]{geometry}
\usepackage{float}
\usepackage{mathtools, commath, amssymb, amsthm}
\usepackage{enumitem, tabularx,graphicx,url,xcolor,rotating,multicol,epsfig,colortbl,lipsum}

\setlist{topsep=1pt, itemsep=0em}
\setlength{\parindent}{0pt}
\setlength{\parskip}{6pt}

\usepackage{hyphenat}
\hyphenation{ма-те-ма-ти-ка вос-ста-нав-ли-вать}

\usepackage[math]{anttor}

\newenvironment{talk}[6]{%
\vskip 0pt\nopagebreak%
\vskip 0pt\nopagebreak%
\section*{#1}
\phantomsection
\addcontentsline{toc}{section}{#2. \textit{#1}}
% \addtocontents{toc}{\textit{#1}\par}
\textit{#2}\\\nopagebreak%
#3\\\nopagebreak%
\ifthenelse{\equal{#4}{}}{}{\url{#4}\\\nopagebreak}%
\ifthenelse{\equal{#5}{}}{}{Соавторы: #5\\\nopagebreak}%
\ifthenelse{\equal{#6}{}}{}{Секция: #6\\\nopagebreak}%
}

\definecolor{LovelyBrown}{HTML}{FDFCF5}

\usepackage[pdftex,
breaklinks=true,
bookmarksnumbered=true,
linktocpage=true,
linktoc=all
]{hyperref}

\begin{document}
\pagenumbering{gobble}
\pagestyle{plain}
\pagecolor{LovelyBrown}
\begin{talk}
{Асимптотики длинных нелинейных береговых волн и их связь с биллиардами с полужесткими стенками}
{Вотякова Мария Михайловна}
{Москва, МФТИ, ИПМех РАН}
{votiakova.mm@phystech.edu}
{С.\,Ю. ~Доброхотов, Д.\,С.~ Миненков}
{Уравнения в частных производных, математическая физика и спектральная теория}

Под береговыми волнами мы понимаем  периодические или близкие к периодическим по времени гравитационные волны на воде
в бассейне глубины \(D(x)\), \(x=(x_1,x_2)\), локализованные в окрестности береговой линии \(\Gamma^0=\{D(x)=0\}\).
В двух конкретных примерах мы строим отвечающие береговым волнам асимптотические решения системы нелинейных уравнений мелкой воды
в виде параметрически заданных функций, определяемых через асимптотики линеаризованной системы (см. [1]), которые в свою очередь связаны с асимптотическими собственными функциями оператора \(\hat L = - \nabla g D(x) \nabla\). Область определения оператора --- гладкие функции \(\xi(x)\) в области \(\Omega = \{x : D(x) > 0\}\) с конечной энергией: \(|\xi|_{x\in\Gamma^0} < \infty\).
Также обсуждается связь построенных асимптотик с классическими (почти интегрируемыми) ``биллиардами с полужесткими стенками''.

\medskip

Работа выполнена при финансовой поддержке гранта РНФ 24-11-00213.

\begin{enumerate}
\item[{[1]}] S.,Y.~Dobrokhotov, D.S.~Minenkov, V.E.~Nazaikinskii~//  Russ. J. Math. Phys.~--- 2022.~--- vol.~29,~--- p.~28--36.
\end{enumerate}
\end{talk}
\end{document}