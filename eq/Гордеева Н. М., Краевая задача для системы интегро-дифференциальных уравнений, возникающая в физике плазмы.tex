\documentclass[12pt]{article}
\usepackage{hyphsubst}
\usepackage[T2A]{fontenc}
\usepackage[english,main=russian]{babel}
\usepackage[utf8]{inputenc}
\usepackage[letterpaper,top=2cm,bottom=2cm,left=2cm,right=2cm,marginparwidth=2cm]{geometry}
\usepackage{float}
\usepackage{mathtools, commath, amssymb, amsthm}
\usepackage{enumitem, tabularx,graphicx,url,xcolor,rotating,multicol,epsfig,colortbl,lipsum}

\setlist{topsep=1pt, itemsep=0em}
\setlength{\parindent}{0pt}
\setlength{\parskip}{6pt}

\usepackage{hyphenat}
\hyphenation{ма-те-ма-ти-ка вос-ста-нав-ли-вать}

\usepackage[math]{anttor}

\newenvironment{talk}[6]{%
\vskip 0pt\nopagebreak%
\vskip 0pt\nopagebreak%
\section*{#1}
\phantomsection
\addcontentsline{toc}{section}{#2. \textit{#1}}
% \addtocontents{toc}{\textit{#1}\par}
\textit{#2}\\\nopagebreak%
#3\\\nopagebreak%
\ifthenelse{\equal{#4}{}}{}{\url{#4}\\\nopagebreak}%
\ifthenelse{\equal{#5}{}}{}{Соавторы: #5\\\nopagebreak}%
\ifthenelse{\equal{#6}{}}{}{Секция: #6\\\nopagebreak}%
}

\definecolor{LovelyBrown}{HTML}{FDFCF5}

\usepackage[pdftex,
breaklinks=true,
bookmarksnumbered=true,
linktocpage=true,
linktoc=all
]{hyperref}

\begin{document}
\pagenumbering{gobble}
\pagestyle{plain}
\pagecolor{LovelyBrown}
\begin{talk}
{Краевая задача для системы интегро--дифференциальных уравнений, возникающая в физике плазмы}
{Гордеева Надежда Михайловна}
{Федеральный исследовательский центр ``Информатика и управление'' Российской академии наук (ФИЦ ИУ РАН), Москва}
{nmgordeeva@bmstu.ru}
{}
{Уравнения в частных производных, математическая физика и спектральная теория}

Рассматривается основанная на системе уравнений Больцмана--Максвелла модель воздействия электрического поля на слой плазмы. В качестве невозмущенной плотности распределения заряженных частиц принимается функция Ферми--Дирака или Максвелла. Для описания состояния плазмы в предположении малой амплитуды внешнего электрического поля рассматривается краевая задача для системы двух интегро-дифференциальных уравнений. Искомыми величинами в них являются: возмущение функции распределения электронов и возмущение напряженности электрического поля. Система зависит от двух комплексных параметров, характеризующих свойства плазмы и внешнее поле, а в качестве ядра интегрального оператора принята функция, родственная распределению Ферми--Дирака или Максвелла.

В работе построено аналитическое представление общего решения указанной системы интегро-дифференциальных уравнений в виде интеграла с явно выписанным ядром. Такой вид решения найден с помощью новых, являющихся развитием работ И.\,М.~Гельфанда и Г.\,Е.~Шилова, результатов в теории преобразования Фурье обобщенных функций. Для плотности интегральных представлений решения возникает сингулярное интегральное уравнение с ядром Коши на вещественной прямой. Решение этого интегрального уравнения получено с использованием метода Ф.\,Д.~Гахова и Н.\,И.~Мусхелишвили и теории задачи Римана линейного сопряжения. Представлены результаты численной реализации построенного решения и исследована его зависимость от параметров задачи.
\end{talk}
\end{document}