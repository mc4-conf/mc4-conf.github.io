\documentclass[12pt]{article}
\usepackage{hyphsubst}
\usepackage[T2A]{fontenc}
\usepackage[english,main=russian]{babel}
\usepackage[utf8]{inputenc}
\usepackage[letterpaper,top=2cm,bottom=2cm,left=2cm,right=2cm,marginparwidth=2cm]{geometry}
\usepackage{float}
\usepackage{mathtools, commath, amssymb, amsthm}
\usepackage{enumitem, tabularx,graphicx,url,xcolor,rotating,multicol,epsfig,colortbl,lipsum}

\setlist{topsep=1pt, itemsep=0em}
\setlength{\parindent}{0pt}
\setlength{\parskip}{6pt}

\usepackage{hyphenat}
\hyphenation{ма-те-ма-ти-ка вос-ста-нав-ли-вать}

\usepackage[math]{anttor}

\newenvironment{talk}[6]{%
\vskip 0pt\nopagebreak%
\vskip 0pt\nopagebreak%
\section*{#1}
\phantomsection
\addcontentsline{toc}{section}{#2. \textit{#1}}
% \addtocontents{toc}{\textit{#1}\par}
\textit{#2}\\\nopagebreak%
#3\\\nopagebreak%
\ifthenelse{\equal{#4}{}}{}{\url{#4}\\\nopagebreak}%
\ifthenelse{\equal{#5}{}}{}{Соавторы: #5\\\nopagebreak}%
\ifthenelse{\equal{#6}{}}{}{Секция: #6\\\nopagebreak}%
}

\definecolor{LovelyBrown}{HTML}{FDFCF5}

\usepackage[pdftex,
breaklinks=true,
bookmarksnumbered=true,
linktocpage=true,
linktoc=all
]{hyperref}

\begin{document}
\pagenumbering{gobble}
\pagestyle{plain}
\pagecolor{LovelyBrown}
\begin{talk}
{Тождество ошибки для некоторых задач с препятствием}
{Даровская Ксения Александровна}
{Первый МГМУ имени И.\,М. Сеченова, Российский университет дружбы народов}
{k.darovsk@gmail.com}
{}
{Уравнения в частных производных, математическая физика и спектральная теория}

Рассмотрим задачу минимизации функционала, порожденного линейным дифференциальным оператором, на некотором выпуклом замкнутом множестве. Подобные постановки возникают во~многих прикладных областях, в~частности, в~механике --- при~изучении поведения упругих балок и пластин над жестким препятствием.

Если нас интересуют приближенные решениях таких задач, то для измерения их ``качества'' (т.\,е.  близости к точному решению) хорошо себя зарекомендовали функциональные апостериорные оценки, поскольку они не накладывают условий на способ построения аппроксимаций. Особую роль в получении апостериорных оценок играет так называемое ``тождество ошибки'' (ТО), описывающее разрыв между точным решением задачи и произвольной функцией из соответствующего энергетического класса.

В~рамках доклада предполагается обсудить новый подход к получению ТО для задачи с~линейным дифференциальным оператором и~гладким тензором и~сформулировать соответствующий общий результат. В качестве иллюстрации будут представлены удобные формы тождества ошибки для гармонической и~бигармонической задач с~толстым препятствием.

\medskip

Настоящее исследование выполнено при финансовой поддержке~Российского научного фонда, проект~№~24-11-00073.
\end{talk}
\end{document}