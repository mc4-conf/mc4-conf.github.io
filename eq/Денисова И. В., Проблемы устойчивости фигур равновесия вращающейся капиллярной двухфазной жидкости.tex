\documentclass[12pt]{article}
\usepackage{hyphsubst}
\usepackage[T2A]{fontenc}
\usepackage[english,main=russian]{babel}
\usepackage[utf8]{inputenc}
\usepackage[letterpaper,top=2cm,bottom=2cm,left=2cm,right=2cm,marginparwidth=2cm]{geometry}
\usepackage{float}
\usepackage{mathtools, commath, amssymb, amsthm}
\usepackage{enumitem, tabularx,graphicx,url,xcolor,rotating,multicol,epsfig,colortbl,lipsum}

\setlist{topsep=1pt, itemsep=0em}
\setlength{\parindent}{0pt}
\setlength{\parskip}{6pt}

\usepackage{hyphenat}
\hyphenation{ма-те-ма-ти-ка вос-ста-нав-ли-вать}

\usepackage[math]{anttor}

\newenvironment{talk}[6]{%
\vskip 0pt\nopagebreak%
\vskip 0pt\nopagebreak%
\section*{#1}
\phantomsection
\addcontentsline{toc}{section}{#2. \textit{#1}}
% \addtocontents{toc}{\textit{#1}\par}
\textit{#2}\\\nopagebreak%
#3\\\nopagebreak%
\ifthenelse{\equal{#4}{}}{}{\url{#4}\\\nopagebreak}%
\ifthenelse{\equal{#5}{}}{}{Соавторы: #5\\\nopagebreak}%
\ifthenelse{\equal{#6}{}}{}{Секция: #6\\\nopagebreak}%
}

\definecolor{LovelyBrown}{HTML}{FDFCF5}

\usepackage[pdftex,
breaklinks=true,
bookmarksnumbered=true,
linktocpage=true,
linktoc=all
]{hyperref}

\begin{document}
\pagenumbering{gobble}
\pagestyle{plain}
\pagecolor{LovelyBrown}
\begin{talk}
{Проблемы устойчивости фигур равновесия вращающейся капиллярной двухфазной жидкости}
{Денисова Ирина Владимировна}
{Институт проблем машиноведения   РАН, Санкт-Петербург}
{denisovairinavlad@gmail.com}
{}
{Уравнения в частных производных, математическая физика и спектральная теория}

Обсуждается вопрос устойчивости вращения вязкой двухфазной  капли, состоящей из  сжимаемой и несжимаемой жидкостей. Предполагается, что угловая скорость мала, а форма капли близка к двухслойной фигуре равновесия, при этом внутренней является несжимаемая жидкость. Она ограничена замкнутой неизвестной поверхностью, не пересекающейся с внешней границей. Сжимаемая жидкость баротропна. На границах действуют силы поверхностного натяжения.   Решение стационарной задачи с  неизвестными границами для уравнений Навье--Стокса, соответствующих медленному ж\"{е}сткому вращению двухфазной  капли, даёт
существование  фигур равновесия  \( \cal F^+ \), \(\cal F \), близких к вложенным шарам \( B_{R_0^\pm}\)
соответствующего радиуса (\(R_0^+< R_0^-\)); \(| {\cal F}^+|=|B_{R_0^+}|\), \(\cal F^+ \subset{\cal F}\).
Доказательство проводится с помощью теоремы о неявной функции [1].
Глобальная разрешимость задачи без вращения была  доказана в [2]. Там была получена устойчивость состояния покоя капли с начальной границей раздела жидкостей близкой к шару.
Исследование проводится в случае отсутствия силы тяжести, т. е. наше двухфазное тело можно рассматривать, например, как  планету с газовой атмосферой.

\medskip

\begin{enumerate}
\item[{[1]}]  И. В. Денисова,   Алгебра и  анализ,   \textbf{36}(3) (2024), 62–80.
\item[{[2]}]  В. А. Солонников, Алгебра и анализ, \textbf{32}(1) (2020), 121-186.
\end{enumerate}
Работа выполнена по теме государственного задания Министерства науки и  высшего образования РФ для ИПМаш РАН  № 124040800009-8.
\end{talk}
\end{document}