\documentclass[12pt]{article}
\usepackage{hyphsubst}
\usepackage[T2A]{fontenc}
\usepackage[english,main=russian]{babel}
\usepackage[utf8]{inputenc}
\usepackage[letterpaper,top=2cm,bottom=2cm,left=2cm,right=2cm,marginparwidth=2cm]{geometry}
\usepackage{float}
\usepackage{mathtools, commath, amssymb, amsthm}
\usepackage{enumitem, tabularx,graphicx,url,xcolor,rotating,multicol,epsfig,colortbl,lipsum}

\setlist{topsep=1pt, itemsep=0em}
\setlength{\parindent}{0pt}
\setlength{\parskip}{6pt}

\usepackage{hyphenat}
\hyphenation{ма-те-ма-ти-ка вос-ста-нав-ли-вать}

\usepackage[math]{anttor}

\newenvironment{talk}[6]{%
\vskip 0pt\nopagebreak%
\vskip 0pt\nopagebreak%
\section*{#1}
\phantomsection
\addcontentsline{toc}{section}{#2. \textit{#1}}
% \addtocontents{toc}{\textit{#1}\par}
\textit{#2}\\\nopagebreak%
#3\\\nopagebreak%
\ifthenelse{\equal{#4}{}}{}{\url{#4}\\\nopagebreak}%
\ifthenelse{\equal{#5}{}}{}{Соавторы: #5\\\nopagebreak}%
\ifthenelse{\equal{#6}{}}{}{Секция: #6\\\nopagebreak}%
}

\definecolor{LovelyBrown}{HTML}{FDFCF5}

\usepackage[pdftex,
breaklinks=true,
bookmarksnumbered=true,
linktocpage=true,
linktoc=all
]{hyperref}

\begin{document}
\pagenumbering{gobble}
\pagestyle{plain}
\pagecolor{LovelyBrown}
\begin{talk}
{Усреднение гиперболических уравнений: операторные оценки при учёте корректоров}
{Дородный Марк Александрович}
{Санкт-Петербургский государственный университет}
{mdorodni@yandex.ru}
{Суслина Т.\,А.}
{Уравнения в частных производных, математическая физика и спектральная теория}

В \(L_2(\mathbb{R}^d;\mathbb{C}^n)\) рассматривается самосопряжённый сильно эллиптический дифференциальный оператор \(A_\varepsilon\) второго порядка. Предполагается, что коэффициенты оператора \(A_\varepsilon\) периодичны и зависят от \(\mathbf{x}/\varepsilon\), где \(\varepsilon > 0\). Мы изучаем поведение операторов \(\cos(\tau A_{\varepsilon}^{1/2})\) и \(A_{\varepsilon}^{-1/2} \sin(\tau A_{\varepsilon}^{1/2})\) при малом \(\varepsilon\) и \(\tau \in \mathbb{R}\). Результаты применяются к усреднению решений задачи Коши для гиперболического уравнения \(\partial^2_\tau \mathbf{u}_\varepsilon(\mathbf{x},\tau)= - (A_\varepsilon \mathbf{u}_\varepsilon)(\mathbf{x},\tau)\) с начальными данными из специального класса. При фиксированном \(\tau\) получена аппроксимация решения \(\mathbf{u}_\varepsilon(\cdot,\tau)\) по норме в \(L_2(\mathbb{R}^d;\mathbb{C}^n)\) с погрешностью \(O(\varepsilon^2)\), а также аппроксимация решения по норме в \(H^1(\mathbb{R}^d;\mathbb{C}^n)\) с погрешностью \(O(\varepsilon)\). В этих аппроксимациях учитываются корректоры. Отслежена зависимость погрешностей от параметра \(\tau\).

\medskip

Исследование выполнено за счёт гранта Российского научного фонда №~22-11-00092, \texttt{https://rscf.ru/project/22-11-00092/}.
\end{talk}
\end{document}