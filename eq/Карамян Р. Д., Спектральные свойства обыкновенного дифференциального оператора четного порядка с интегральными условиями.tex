\documentclass[12pt]{article}
\usepackage{hyphsubst}
\usepackage[T2A]{fontenc}
\usepackage[english,main=russian]{babel}
\usepackage[utf8]{inputenc}
\usepackage[letterpaper,top=2cm,bottom=2cm,left=2cm,right=2cm,marginparwidth=2cm]{geometry}
\usepackage{float}
\usepackage{mathtools, commath, amssymb, amsthm}
\usepackage{enumitem, tabularx,graphicx,url,xcolor,rotating,multicol,epsfig,colortbl,lipsum}

\setlist{topsep=1pt, itemsep=0em}
\setlength{\parindent}{0pt}
\setlength{\parskip}{6pt}

\usepackage{hyphenat}
\hyphenation{ма-те-ма-ти-ка вос-ста-нав-ли-вать}

\usepackage[math]{anttor}

\newenvironment{talk}[6]{%
\vskip 0pt\nopagebreak%
\vskip 0pt\nopagebreak%
\section*{#1}
\phantomsection
\addcontentsline{toc}{section}{#2. \textit{#1}}
% \addtocontents{toc}{\textit{#1}\par}
\textit{#2}\\\nopagebreak%
#3\\\nopagebreak%
\ifthenelse{\equal{#4}{}}{}{\url{#4}\\\nopagebreak}%
\ifthenelse{\equal{#5}{}}{}{Соавторы: #5\\\nopagebreak}%
\ifthenelse{\equal{#6}{}}{}{Секция: #6\\\nopagebreak}%
}

\definecolor{LovelyBrown}{HTML}{FDFCF5}

\usepackage[pdftex,
breaklinks=true,
bookmarksnumbered=true,
linktocpage=true,
linktoc=all
]{hyperref}

\begin{document}
\pagenumbering{gobble}
\pagestyle{plain}
\pagecolor{LovelyBrown}
\begin{talk}
{Спектральные свойства обыкновенного дифференциального оператора четного порядка с интегральными условиями}
{Карамян Рубен Дженсикович}
{Российский университет дружбы народов имени Патриса Лумумбы}
{rkaramyan@yandex.ru}
{}
{Уравнения в частных производных, математическая физика и спектральная теория}

Рассматривается обыкновенный дифференциальный оператор четного порядка с нелокальными краевыми условиями и спектральным параметром. Граничные условия задаются интегралами Римана, которые содержат как неизвестную функцию, так и производные от неизвестной функции. В пространстве Соболева вводится эквивалентная норма, зависящая от спектрального параметра \(\lambda\).
Получены результаты о фредгольмовой разрешимости и спектре задачи, а также априорная оценка решений задачи при достаточно больших значениях параметра \(\lambda\) в терминах эквивалентных норм.

\medskip

\begin{enumerate}
\item[{[1]}] Karamyan~R.\,D., Skubachevskii~A.\,L.,
\textit{Spectral Properties of the Fourth Order Differential Operator with Integral Conditions}, Lobachevskii J.  Math., 2024, Vol. 45, No. 4, pp. 1561--1577.
\end{enumerate}
\end{talk}
\end{document}