\documentclass[12pt]{article}
\usepackage{hyphsubst}
\usepackage[T2A]{fontenc}
\usepackage[english,main=russian]{babel}
\usepackage[utf8]{inputenc}
\usepackage[letterpaper,top=2cm,bottom=2cm,left=2cm,right=2cm,marginparwidth=2cm]{geometry}
\usepackage{float}
\usepackage{mathtools, commath, amssymb, amsthm}
\usepackage{enumitem, tabularx,graphicx,url,xcolor,rotating,multicol,epsfig,colortbl,lipsum}

\setlist{topsep=1pt, itemsep=0em}
\setlength{\parindent}{0pt}
\setlength{\parskip}{6pt}

\usepackage{hyphenat}
\hyphenation{ма-те-ма-ти-ка вос-ста-нав-ли-вать}

\usepackage[math]{anttor}

\newenvironment{talk}[6]{%
\vskip 0pt\nopagebreak%
\vskip 0pt\nopagebreak%
\section*{#1}
\phantomsection
\addcontentsline{toc}{section}{#2. \textit{#1}}
% \addtocontents{toc}{\textit{#1}\par}
\textit{#2}\\\nopagebreak%
#3\\\nopagebreak%
\ifthenelse{\equal{#4}{}}{}{\url{#4}\\\nopagebreak}%
\ifthenelse{\equal{#5}{}}{}{Соавторы: #5\\\nopagebreak}%
\ifthenelse{\equal{#6}{}}{}{Секция: #6\\\nopagebreak}%
}

\definecolor{LovelyBrown}{HTML}{FDFCF5}

\usepackage[pdftex,
breaklinks=true,
bookmarksnumbered=true,
linktocpage=true,
linktoc=all
]{hyperref}

\begin{document}
\pagenumbering{gobble}
\pagestyle{plain}
\pagecolor{LovelyBrown}
\begin{talk}
{Исследование граничных интегральных уравнений Фредгольма второго рода, условно эквивалентных трехмерной задаче дифракции акустических волн. Осреднение слабо сингулярных интегральных операторов и численное решение трехмерных задач Неймана для уравнения Гельмгольца}
{Каширин Алексей Алексеевич}
{Вычислительный центр Дальневосточного отделения Российской академии наук}
{elomer@mail.ru}
{Смагин Сергей Иванович, Погорелов Сергей Анатольевич}
{Уравнения в частных производных, математическая физика и спектральная теория}

Рассматривается задача дифракции стационарных акустических волн на трехмерном однородном включении. Аналитическое решение этой задачи может быть найдено лишь в исключительных случаях, поэтому чаще всего она решается численно. Эффективные алгоритмы численного решения задачи дифракции могут быть созданы на основе условно эквивалентных ей граничных интегральных уравнений с одной неизвестной функцией. Различные уравнения такого вида получены в работе [1].

Используя теорию Фредгольма, мы исследуем два слабо сингулярных интегральных уравнения второго рода, к каждому из которых может быть сведена задача дифракции, на их собственных частотах. В этих случаях уравнения, в отличие от исходной задачи, некорректно разрешимы. Поскольку для областей общей формы собственные частоты неизвестны, это может привести к недостоверным результатам при численном решении данных уравнений. Установлено, что одно из них на собственных частотах может не иметь решения, а другое разрешимо неединственным образом. При этом существует единственное решение второго уравнения, которое позволяет найти решение задачи дифракции. На всех частотах оно может быть найдено приближенно методом интерполяции [2].

Для численного решения указанных уравнений необходимо построить дискретные аналоги поверхностных потенциалов и их нормальных производных. Для этого может быть использован метод осреднения интегральных операторов со слабыми особенностями в ядрах. В результате интегральные уравнения аппроксимируются системами линейных алгебраических уравнений с легко вычисляемыми коэффициентами, которые затем решаются численно обобщенным методом минимальных невязок (GMRES). После этого приближенное решение исходной задачи находится в любой точке пространства.

Ранее такой подход использовался для приближенного решения трехмерных задач Дирихле для уравнений Лапласа и Гельмгольца потенциалами простого слоя [3], [4]. Мы получили формулы для нормальной производной потенциала простого слоя и применили их для численного решения трехмерных задач Неймана для уравнения Гельмгольца. Результаты численных экспериментов демонстрируют возможности данного подхода.

\medskip

\begin{enumerate}
\item[{[1]}] R. E. Kleinman, P. A. Martin, {\it On single integral equations for the transmission problem of acoustics}, SIAM Journal on Applied Mathematics, 48 (1988), 307–325.
\item[{[2]}] А. А. Каширин, С. И. Смагин, {\it О численном решении скалярных задач дифракции в интегральных постановках на спектрах интегральных операторов}, Докл. РАН. Матем., информ., проц. упр., 494 (2020),  38–42.
\item[{[3]}] А. А. Каширин, С. И. Смагин, {\it О численном решении задач Дирихле для уравнения Гельмгольца методом потенциалов}, ЖВМиМФ, 52 (2012), 1492–1505.
\item[{[4]}] С. И. Смагин, {\it О численном решении интегрального уравнения I рода со слабой особенностью в ядре на замкнутой поверхности}, Докл. РАН. Матем., информ., проц. упр., 505 (2022),  14–18.
\end{enumerate}
\end{talk}
\end{document}