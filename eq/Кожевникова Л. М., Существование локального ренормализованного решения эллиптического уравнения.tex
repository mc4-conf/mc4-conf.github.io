\documentclass[12pt]{article}
\usepackage{hyphsubst}
\usepackage[T2A]{fontenc}
\usepackage[english,main=russian]{babel}
\usepackage[utf8]{inputenc}
\usepackage[letterpaper,top=2cm,bottom=2cm,left=2cm,right=2cm,marginparwidth=2cm]{geometry}
\usepackage{float}
\usepackage{mathtools, commath, amssymb, amsthm}
\usepackage{enumitem, tabularx,graphicx,url,xcolor,rotating,multicol,epsfig,colortbl,lipsum}

\setlist{topsep=1pt, itemsep=0em}
\setlength{\parindent}{0pt}
\setlength{\parskip}{6pt}

\usepackage{hyphenat}
\hyphenation{ма-те-ма-ти-ка вос-ста-нав-ли-вать}

\usepackage[math]{anttor}

\newenvironment{talk}[6]{%
\vskip 0pt\nopagebreak%
\vskip 0pt\nopagebreak%
\section*{#1}
\phantomsection
\addcontentsline{toc}{section}{#2. \textit{#1}}
% \addtocontents{toc}{\textit{#1}\par}
\textit{#2}\\\nopagebreak%
#3\\\nopagebreak%
\ifthenelse{\equal{#4}{}}{}{\url{#4}\\\nopagebreak}%
\ifthenelse{\equal{#5}{}}{}{Соавторы: #5\\\nopagebreak}%
\ifthenelse{\equal{#6}{}}{}{Секция: #6\\\nopagebreak}%
}

\definecolor{LovelyBrown}{HTML}{FDFCF5}

\usepackage[pdftex,
breaklinks=true,
bookmarksnumbered=true,
linktocpage=true,
linktoc=all
]{hyperref}

\begin{document}
\pagenumbering{gobble}
\pagestyle{plain}
\pagecolor{LovelyBrown}
\begin{talk}
{Существование локального ренормализованного решения эллиптического уравнения с переменными   показателями  в пространстве \(\mathbb{R}^n\)}
{Кожевникова Лариса Михайловна}
{Стерлитамакский филиал Уфимского университета науки и технологий}
{kosul@mail.ru}
{}
{Уравнения в частных производных, математическая физика и спектральная теория}

Концепция ренормализованных решений является важным шагом в изучении общих
вырождающихся эллиптических уравнений с данными в виде меры. Первоначальное определение
приведено в работе [1] и распространено М.\,Ф. Бидо-Верон [2] в локальную
и очень полезную форму  для уравнения с \(p\)-лапласианом, поглощением и мерой Радона \(\mu\):
\[-\Delta_pu+|u|^{p_0-2}u=\mu,\quad p\in(1,n),\quad p<p_0.\eqno(1)\]
В частности, М.\,Ф. Бидо-Верон доказала существование в пространстве \(\mathbb{R}^n\) локального энтропийного решения уравнения (1)  c правой частью из пространства \(L_{1,{\rm loc}}(\mathbb{R}^n)\). В монографии [3] Л. Верон обобщил  понятия локального ренормализованного решения для уравнения со степенными нелинейностями вида
\[-{\rm div}\,{\rm a}({\rm x},\nabla
u)+b({\rm x},u,\nabla u)=\mu.\eqno(2)\]

В докладе понятие  локального ренормализованного решения адаптируется на  уравнение вида (2) c переменными показателями роста.
В локальном пространстве Соболева с переменным показателем доказано существование локального ренормализованного решения уравнения (2)  c \(\mu\in L_{1,{\rm loc}}(\mathbb{R}^n)\) в пространстве \(\mathbb{R}^n\).

Ранее автором в работе [4] для уравнения вида (2) с \(p({\rm x})\)-ростом и общей мерой Радона \(\mu\) с конечной полной вариацией доказано существование ренормализованного решения задачи Дирихле в ограниченной области \(\Omega\).

\medskip

\begin{enumerate}
\item[{[1]}] G. Dal Maso, F. Murat, L. Orsina, A. Prignet, {\it Renormalized solutions of elliptic
equations with general measure data}, Ann. Scuola Norm. Sup. Pisa Cl. Sci. (4),
28:4 (1999), 741--808.
\item[{[2]}]
M. F. Bidaut-V\(\acute{e}\)ron, {\it Removable singularities and existence for a quasilinear equation
with absorption or source term and measure data}, Adv. Nonlinear Stud., 3:1
(2003), 25--63.
\item[{[3]}] L. V\(\acute{e}\)ron,  {\it Local and global aspects of quasilinear degenerate elliptic equations. Quasilinear
elliptic singular problems}, World Sci. Publ., Hackensack,  2017.
\item[{[4]}] Л. М. Кожевникова,  {\it Ренормализованные решения эллиптических уравнений с переменными показателями и данными в виде общей меры}, Матем. сб., 211:12 (2020), 83--122.
\end{enumerate}
\end{talk}
\end{document}