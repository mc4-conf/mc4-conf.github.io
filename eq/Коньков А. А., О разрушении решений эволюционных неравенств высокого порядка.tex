\documentclass[12pt]{article}
\usepackage{hyphsubst}
\usepackage[T2A]{fontenc}
\usepackage[english,main=russian]{babel}
\usepackage[utf8]{inputenc}
\usepackage[letterpaper,top=2cm,bottom=2cm,left=2cm,right=2cm,marginparwidth=2cm]{geometry}
\usepackage{float}
\usepackage{mathtools, commath, amssymb, amsthm}
\usepackage{enumitem, tabularx,graphicx,url,xcolor,rotating,multicol,epsfig,colortbl,lipsum}

\setlist{topsep=1pt, itemsep=0em}
\setlength{\parindent}{0pt}
\setlength{\parskip}{6pt}

\usepackage{hyphenat}
\hyphenation{ма-те-ма-ти-ка вос-ста-нав-ли-вать}

\usepackage[math]{anttor}

\newenvironment{talk}[6]{%
\vskip 0pt\nopagebreak%
\vskip 0pt\nopagebreak%
\section*{#1}
\phantomsection
\addcontentsline{toc}{section}{#2. \textit{#1}}
% \addtocontents{toc}{\textit{#1}\par}
\textit{#2}\\\nopagebreak%
#3\\\nopagebreak%
\ifthenelse{\equal{#4}{}}{}{\url{#4}\\\nopagebreak}%
\ifthenelse{\equal{#5}{}}{}{Соавторы: #5\\\nopagebreak}%
\ifthenelse{\equal{#6}{}}{}{Секция: #6\\\nopagebreak}%
}

\definecolor{LovelyBrown}{HTML}{FDFCF5}

\usepackage[pdftex,
breaklinks=true,
bookmarksnumbered=true,
linktocpage=true,
linktoc=all
]{hyperref}

\begin{document}
\pagenumbering{gobble}
\pagestyle{plain}
\pagecolor{LovelyBrown}
\begin{talk}
{О разрушении решений эволюционных неравенств высокого порядка}
{Коньков Андрей Александрович}
{Московский государственный университет имени~М.\,В.~Ломоносова, Московский центр фундаментальной и прикладной математики}
{konkov@mech.math.msu.su}
{А.\,Е.~Шишков}
{Уравнения в частных производных, математическая физика и спектральная теория}

Рассматривается задача Коши
\[
\partial_t^k u
-
\sum_{|\alpha| = m}
\partial^\alpha
a_\alpha (x, t, u)
\ge
f (|u|)
\quad
\mbox{в }
{\mathbb R}_+^{n+1} = {\mathbb R}^n \times (0, \infty),
\eqno (1)
\]
\[
u (x, 0) = u_0 (x),
\partial_t u (x, 0) = u_1 (x),
\ldots,
\partial_t^{k-1} u (x, 0) = u_{k-1} (x) \ge 0,
\eqno (2)
\]
где
\(k, m, n \ge 1\)
и
\(a_\alpha\) --- каратеодориевы функции такие, что
\[
|a_\alpha (x, t, \zeta)| \le A |\zeta|^p,
\quad
|\alpha| = m,
\quad
A, p = const > 0,
\]
для почти всех
\((x, t) \in {\mathbb R}_+^{n+1}\)
и для всех
\(\zeta \in {\mathbb R}\).

Для слабых решений~(1), (2) найдены точные условия blow-up.
В случае степенной нелинейности \(f (\zeta) = \zeta^\lambda\) наши условия совпадают с условиями, раннее полученными в работах~[1--4] и, в частности, с известными условиями Фуджиты-Хаякавы и Като.
Нам также удалось обобщить результаты работы [5] на случай неравенств высокого порядка вида~(1). При этом в отличие от~[5] мы не накладываем никаких условий эллиптичности на коэффициенты \(a_\alpha\) дифференциального оператора. Нам также не требуются никакие условия роста на решение задачи Коши на начальные значения.
Все формулировки и доказательства доступны в архиве Корнельского университета~[6].

\medskip

\begin{enumerate}
\item[{[1]}]
Yu.\,V.Egorov, V.\,A.~Galaktionov, V.\,A.~Kondratiev, S.\,I.~Pohozaev,
{\it On the necessary conditions of global existence of solutions to a quasilinear inequality in the half-space},
C. R. Acad. Sci. Paris, S\'er. 1: Math. 330 (2000), 93--98.
\item[{[2]}]
H.~Fujita,
{\it On the blowing up of solutions of the Cauchy problem for \(u_t = \Delta u + u^{1 + \alpha}\)},
J. Fac. Sci. Univ. Tokyo Sect. IA Math. 13 (1966), 109--124.
\item[{[3]}]
K.~Hayakawa,
{\it On nonexistence of global solutions of some semilinear parabolic diffe\-ren\-tial equations},
Proc. Japan Acad. 49 (1973), 503--505.
\item[{[4]}]
T.~Kato,
{\it Blow-up of solutions of some nonlinear hyperbolic equations},
Commun. Pure Appl. Math. 33 (1980), 501--505.
\item[{[5]}]
R.~Laister, M.~Sier\.{z}\c{e}ga,
{\it A blow-up dichotomy for semilinear fractional heat equations},
Math. Ann. 381 (2021), 75--90.
\item[{[6]}]
A.\,A.~Kon'kov, A.\,E.~Shishkov,
{\it On blow-up conditions for nonlinear higher order evo\-lu\-tion inequalities},
arXiv:2309.00574 [math.AP], DOI: 10.48550/arXiv.2309.00574
\end{enumerate}
\end{talk}
\end{document}