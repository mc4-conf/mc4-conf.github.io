\documentclass[12pt]{article}
\usepackage{hyphsubst}
\usepackage[T2A]{fontenc}
\usepackage[english,main=russian]{babel}
\usepackage[utf8]{inputenc}
\usepackage[letterpaper,top=2cm,bottom=2cm,left=2cm,right=2cm,marginparwidth=2cm]{geometry}
\usepackage{float}
\usepackage{mathtools, commath, amssymb, amsthm}
\usepackage{enumitem, tabularx,graphicx,url,xcolor,rotating,multicol,epsfig,colortbl,lipsum}

\setlist{topsep=1pt, itemsep=0em}
\setlength{\parindent}{0pt}
\setlength{\parskip}{6pt}

\usepackage{hyphenat}
\hyphenation{ма-те-ма-ти-ка вос-ста-нав-ли-вать}

\usepackage[math]{anttor}

\newenvironment{talk}[6]{%
\vskip 0pt\nopagebreak%
\vskip 0pt\nopagebreak%
\section*{#1}
\phantomsection
\addcontentsline{toc}{section}{#2. \textit{#1}}
% \addtocontents{toc}{\textit{#1}\par}
\textit{#2}\\\nopagebreak%
#3\\\nopagebreak%
\ifthenelse{\equal{#4}{}}{}{\url{#4}\\\nopagebreak}%
\ifthenelse{\equal{#5}{}}{}{Соавторы: #5\\\nopagebreak}%
\ifthenelse{\equal{#6}{}}{}{Секция: #6\\\nopagebreak}%
}

\definecolor{LovelyBrown}{HTML}{FDFCF5}

\usepackage[pdftex,
breaklinks=true,
bookmarksnumbered=true,
linktocpage=true,
linktoc=all
]{hyperref}

\begin{document}
\pagenumbering{gobble}
\pagestyle{plain}
\pagecolor{LovelyBrown}
\begin{talk}
{Усреднение уравнений равновесия волокнистого композита методом двухмасштабной сходимости}
{Леонова Эвелина Ивановна}
{Новосибирский государственный университет;
Институт гидродинамики им. М.\,А. Лаврентьева СО РАН}
{e.leonova1@g.nsu.ru}
{}
{Уравнения в частных производных, математическая физика и спектральная теория}

Доклад посвящен исследованию статической задачи антиплоского сдвига волокнистого композита, армированного тонкими нитями. Исходная постановка содержит два малых параметра \(\delta\) и \(\varepsilon\), которые отвечают за толщину нити и расстояние между соседними нитями соответственно. Изучено асимптотическое поведение решений при стремлении параметров к нулю --- сначала параметра \(\delta\), затем параметра \(\varepsilon\). Оба предельных перехода математически строго обоснованы.  Предельный переход при \(\delta \rightarrow 0+\) основан на асимптотическом методе, предложенном в [1]. Предельный переход при \(\varepsilon \rightarrow 0+\) представляет собой процедуру гомогенизации, основанную на применении метода двухмасштабной сходимости Г. Аллера и соавторов [2]. Проведены численные расчеты, которые показывают хорошую сходимость с теоретическими результатами.

\medskip

\begin{enumerate}
\item[{[1]}] A. I. Furtsev, E. M. Rudoy, {\it Variational approach to modeling soft and stiff interfaces in the Kirchhoff-Love theory of plates,} International Journal of Solids and Structures, 2020, Vol. 202, P. 562–574.
\item[{[2]}] G. Allaire, A. Damlamian, U. Hornung, {\it Two-scale convergence on periodic surfaces and applications,} Proceedings of the International Conference on Mathematical Modelling of Flow through Porous Media, 1996, P. 15–25.
\end{enumerate}
\end{talk}
\end{document}