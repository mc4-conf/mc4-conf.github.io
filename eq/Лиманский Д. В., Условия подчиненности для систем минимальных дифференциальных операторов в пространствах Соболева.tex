\documentclass[12pt]{article}
\usepackage{hyphsubst}
\usepackage[T2A]{fontenc}
\usepackage[english,main=russian]{babel}
\usepackage[utf8]{inputenc}
\usepackage[letterpaper,top=2cm,bottom=2cm,left=2cm,right=2cm,marginparwidth=2cm]{geometry}
\usepackage{float}
\usepackage{mathtools, commath, amssymb, amsthm}
\usepackage{enumitem, tabularx,graphicx,url,xcolor,rotating,multicol,epsfig,colortbl,lipsum}

\setlist{topsep=1pt, itemsep=0em}
\setlength{\parindent}{0pt}
\setlength{\parskip}{6pt}

\usepackage{hyphenat}
\hyphenation{ма-те-ма-ти-ка вос-ста-нав-ли-вать}

\usepackage[math]{anttor}

\newenvironment{talk}[6]{%
\vskip 0pt\nopagebreak%
\vskip 0pt\nopagebreak%
\section*{#1}
\phantomsection
\addcontentsline{toc}{section}{#2. \textit{#1}}
% \addtocontents{toc}{\textit{#1}\par}
\textit{#2}\\\nopagebreak%
#3\\\nopagebreak%
\ifthenelse{\equal{#4}{}}{}{\url{#4}\\\nopagebreak}%
\ifthenelse{\equal{#5}{}}{}{Соавторы: #5\\\nopagebreak}%
\ifthenelse{\equal{#6}{}}{}{Секция: #6\\\nopagebreak}%
}

\definecolor{LovelyBrown}{HTML}{FDFCF5}

\usepackage[pdftex,
breaklinks=true,
bookmarksnumbered=true,
linktocpage=true,
linktoc=all
]{hyperref}

\begin{document}
\pagenumbering{gobble}
\pagestyle{plain}
\pagecolor{LovelyBrown}
\begin{talk}
{Условия подчиненности для систем минимальных дифференциальных операторов в пространствах Соболева}
{Лиманский Дмитрий Владимирович}
{Донецкий государственный университет, г. Донецк}
{d.limanskiy.dongu@mail.ru}
{}
{Уравнения в частных производных, математическая физика и спектральная теория}

В работе приводится обзор результатов об априорных оценках для систем минимальных дифференциальных операторов в шкале пространств \(L^p(\Omega)\), где \(p\in[1,\infty]\). Приведены результаты о характеризации эллиптических и \(l\)-квазиэллиптических систем при помощи априорных оценок в изотропных и анизотропных пространствах Соболева \(W_{p,0}^l(\mathbb R^n)\),\ \(p\in[1,\infty]\). При заданном наборе \(l=(l_1,\dots,l_n)\in\mathbb N^n\) доказаны критерии существования \(l\)-квазиэллиптических и слабо коэрцитивных систем, а также указаны широкие классы слабо коэрцитивных в \(W_{p,0}^l(\mathbb R^n)\),\ \(p\in[1,\infty]\), неэллиптических и неквазиэллиптических систем. Кроме того, описаны линейные пространства операторов, подчиненных в \(L^\infty(\mathbb R^n)\)-норме тензорному произведению двух эллиптических дифференциальных полиномов.

\medskip

\begin{enumerate}
\item[{[1]}] Д. В. Лиманский, М. М. Маламуд, {\it Об условиях подчиненности для систем минимальных дифференциальных операторов}, Современная математика. Фундаментальные направления, 70:1 (2024), 121-149.
\end{enumerate}
\end{talk}
\end{document}