\documentclass[12pt]{article}
\usepackage{hyphsubst}
\usepackage[T2A]{fontenc}
\usepackage[english,main=russian]{babel}
\usepackage[utf8]{inputenc}
\usepackage[letterpaper,top=2cm,bottom=2cm,left=2cm,right=2cm,marginparwidth=2cm]{geometry}
\usepackage{float}
\usepackage{mathtools, commath, amssymb, amsthm}
\usepackage{enumitem, tabularx,graphicx,url,xcolor,rotating,multicol,epsfig,colortbl,lipsum}

\setlist{topsep=1pt, itemsep=0em}
\setlength{\parindent}{0pt}
\setlength{\parskip}{6pt}

\usepackage{hyphenat}
\hyphenation{ма-те-ма-ти-ка вос-ста-нав-ли-вать}

\usepackage[math]{anttor}

\newenvironment{talk}[6]{%
\vskip 0pt\nopagebreak%
\vskip 0pt\nopagebreak%
\section*{#1}
\phantomsection
\addcontentsline{toc}{section}{#2. \textit{#1}}
% \addtocontents{toc}{\textit{#1}\par}
\textit{#2}\\\nopagebreak%
#3\\\nopagebreak%
\ifthenelse{\equal{#4}{}}{}{\url{#4}\\\nopagebreak}%
\ifthenelse{\equal{#5}{}}{}{Соавторы: #5\\\nopagebreak}%
\ifthenelse{\equal{#6}{}}{}{Секция: #6\\\nopagebreak}%
}

\definecolor{LovelyBrown}{HTML}{FDFCF5}

\usepackage[pdftex,
breaklinks=true,
bookmarksnumbered=true,
linktocpage=true,
linktoc=all
]{hyperref}

\begin{document}
\pagenumbering{gobble}
\pagestyle{plain}
\pagecolor{LovelyBrown}
\begin{talk}
{Асимптотики типа шепчущей галереи в трехмерной области, диффеоморфной полноторию}
{Миненков Дмитрий Сергеевич}
{ИПМех им. А.\,Ю.~Ишлинского РАН, МГУ им. М.\,В.~Ломоносова}
{minenkov.ds@gmail.com}
{}
{Уравнения в частных производных, математическая физика и спектральная теория}

Рассматривается задача на собственные функции для оператора Лапласа внутри трехмерной области вращения \(\Omega\), диффеоморфной полноторию, с условиями Дирихле на границе. Построена серия асимптотических собственных чисел и функций (квазимод) \(\{E_k,u_k\}_{k=k_0}^\infty\) типа шепчущей галереи (см. [1]):

\(\|\Delta u_k - E_k u_k \|_{L^2(\Omega)} = O(k^{2/3}), \quad
\|u_k\|_{L^2(\Omega)}\sim 1, \quad
E_k \asymp k^2,\quad
E_{k+1} - E_k \asymp k, \quad
k \to\infty.\)

Именно, исследуются коротковолновые асимптотики, локализованные у границы или у части границы \(\partial \Omega\). Исходная задача сводится к решению одномерных уравнений с помощью адиабатического приближения, применяемого в виде операторного разделения переменных (см. [2]).

\medskip

Результаты получены совместно с С.\,А.~Сергеевым в МГУ им. М.\,В.~Ломоносова в рамках гранта РНФ 22-71-10106.

\begin{enumerate}
\item[{[1]}] D. S. Minenkov, S. A. Sergeev, {\it Asymptotics of the Whispering Gallery-Type in the Eigenproblem for the Laplacian in a Domain of Revolution Diffeomorphic To a Solid Torus}, Russ. J. Math. Phys. 30 4 (2023), 599-620.
\item[{[2]}] С. Ю. Доброхотов, {\it Методы Маслова в линеаризованной теории гравитационных волн на поверхности жидкости}, Докл. АН СССР 269 1 (1983),  76–80.
\end{enumerate}
\end{talk}
\end{document}