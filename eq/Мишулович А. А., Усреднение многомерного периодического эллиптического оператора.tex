\documentclass[12pt]{article}
\usepackage{hyphsubst}
\usepackage[T2A]{fontenc}
\usepackage[english,main=russian]{babel}
\usepackage[utf8]{inputenc}
\usepackage[letterpaper,top=2cm,bottom=2cm,left=2cm,right=2cm,marginparwidth=2cm]{geometry}
\usepackage{float}
\usepackage{mathtools, commath, amssymb, amsthm}
\usepackage{enumitem, tabularx,graphicx,url,xcolor,rotating,multicol,epsfig,colortbl,lipsum}

\setlist{topsep=1pt, itemsep=0em}
\setlength{\parindent}{0pt}
\setlength{\parskip}{6pt}

\usepackage{hyphenat}
\hyphenation{ма-те-ма-ти-ка вос-ста-нав-ли-вать}

\usepackage[math]{anttor}

\newenvironment{talk}[6]{%
\vskip 0pt\nopagebreak%
\vskip 0pt\nopagebreak%
\section*{#1}
\phantomsection
\addcontentsline{toc}{section}{#2. \textit{#1}}
% \addtocontents{toc}{\textit{#1}\par}
\textit{#2}\\\nopagebreak%
#3\\\nopagebreak%
\ifthenelse{\equal{#4}{}}{}{\url{#4}\\\nopagebreak}%
\ifthenelse{\equal{#5}{}}{}{Соавторы: #5\\\nopagebreak}%
\ifthenelse{\equal{#6}{}}{}{Секция: #6\\\nopagebreak}%
}

\definecolor{LovelyBrown}{HTML}{FDFCF5}

\usepackage[pdftex,
breaklinks=true,
bookmarksnumbered=true,
linktocpage=true,
linktoc=all
]{hyperref}

\begin{document}
\pagenumbering{gobble}
\pagestyle{plain}
\pagecolor{LovelyBrown}
\begin{talk}
{Усреднение многомерного периодического эллиптического оператора на краю спектральной лакуны: операторные оценки в энергетической норме}
{Мишулович Арсений Александрович}
{Санкт-Петербургский государственный университет}
{st062829@student.spbu.ru}
{}
{Уравнения в частных производных, математическая физика и спектральная теория}

В пространстве \(L_{2}(\mathbb{R}^d)\) рассматривается эллиптический самосопряженный дифференциальный оператор второго порядка \(\mathcal{A}_{\varepsilon}\) с периодическими быстро осциллирующими коэффициентами: \(\mathcal{A}_{\varepsilon} = -\operatorname{div} g(\boldsymbol{x}/\varepsilon)\nabla + \varepsilon^{-2} p(\boldsymbol{x}/\varepsilon) \).
Известно, что спектр оператора \( \mathcal{A}_{\varepsilon} \) имеет зонную структуру: он является объединением замкнутых отрезков (спектральных зон).
Зоны могут перекрываться.
Между зонами могут открываться лакуны.
Согласно гипотезе Бете-Зоммерфельда, в многомерном случае число лакун конечно.
Получена аппроксимация резольвенты в регулярной точке оператора \(\mathcal{A}_{\varepsilon}\), близкой к краю внутренней спектральной лакуны, по ``энергетической'' норме (т.е. по операторной норме из пространства \(L_{2}(\mathbb{R}^d)\) в класс Соболева \(H^{1}(\mathbb{R}^d)\)).

\medskip

Исследование поддержано Министерством науки и высшего образования
Российской Федерации (соглашение № 075–15–2022–287 от 06.04.2022).
\end{talk}
\end{document}