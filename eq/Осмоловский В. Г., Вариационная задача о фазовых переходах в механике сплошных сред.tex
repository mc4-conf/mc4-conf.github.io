\documentclass[12pt]{article}
\usepackage{hyphsubst}
\usepackage[T2A]{fontenc}
\usepackage[english,main=russian]{babel}
\usepackage[utf8]{inputenc}
\usepackage[letterpaper,top=2cm,bottom=2cm,left=2cm,right=2cm,marginparwidth=2cm]{geometry}
\usepackage{float}
\usepackage{mathtools, commath, amssymb, amsthm}
\usepackage{enumitem, tabularx,graphicx,url,xcolor,rotating,multicol,epsfig,colortbl,lipsum}

\setlist{topsep=1pt, itemsep=0em}
\setlength{\parindent}{0pt}
\setlength{\parskip}{6pt}

\usepackage{hyphenat}
\hyphenation{ма-те-ма-ти-ка вос-ста-нав-ли-вать}

\usepackage[math]{anttor}

\newenvironment{talk}[6]{%
\vskip 0pt\nopagebreak%
\vskip 0pt\nopagebreak%
\section*{#1}
\phantomsection
\addcontentsline{toc}{section}{#2. \textit{#1}}
% \addtocontents{toc}{\textit{#1}\par}
\textit{#2}\\\nopagebreak%
#3\\\nopagebreak%
\ifthenelse{\equal{#4}{}}{}{\url{#4}\\\nopagebreak}%
\ifthenelse{\equal{#5}{}}{}{Соавторы: #5\\\nopagebreak}%
\ifthenelse{\equal{#6}{}}{}{Секция: #6\\\nopagebreak}%
}

\definecolor{LovelyBrown}{HTML}{FDFCF5}

\usepackage[pdftex,
breaklinks=true,
bookmarksnumbered=true,
linktocpage=true,
linktoc=all
]{hyperref}

\begin{document}
\pagenumbering{gobble}
\pagestyle{plain}
\pagecolor{LovelyBrown}
\begin{talk}
{Вариационная задача о фазовых переходах в механике сплошных сред}
{Осмоловский Виктор Георгиевич}
{Санкт-Петербургский государственный университет}
{victor.osmolovskii@gmail.com}
{}
{Уравнения в частных производных, математическая физика и спектральная теория}

В докладе обсуждается свойства решений вариационной задачи для класса невыпуклых функционалов, возникающих при описании процесса фазовых переходов в механике сплошных сред. Особое внимание уделяется зависимости состояний равновесия от параметров задачи, например,
температуры и области, занимаемой двухфазовой средой. Показано, какие из явно получаемых результатов для одномерного случая, хотя бы на качественном уровне переносятся на многомерный. Приводится пример  регуляризации функционала энергии, основанный на учёте поверхностногй энергии границы раздела фаз, гарантирующей существование решений в любых областях при всех значениях температуры.

\medskip

\begin{enumerate}
\item[{[1]}] V.G. Osmolovskii, {\it Boundary Value Problems with Free Surfaces in the Theory of Phase Transitions}, Differential Equations, Vol.53, N13 (2017), 1734-1763.
\item[{[2]}] В.Г. Осмоловский, {\it Математические вопросы теории фазовых переходов в механике сплошных сред}, Алгебра и анализ, т.29, N5, (2012), 111–178.
\item[{[3]}] В.Г. Осмоловский, {\it Метод расщепления в вариационной задаче теории фазовых переходов в механике двухфазовых сплошных сред}, Проблемы математического анализа, вып.126, (2024), 17–28.
\end{enumerate}
\end{talk}
\end{document}