\documentclass[12pt]{article}
\usepackage{hyphsubst}
\usepackage[T2A]{fontenc}
\usepackage[english,main=russian]{babel}
\usepackage[utf8]{inputenc}
\usepackage[letterpaper,top=2cm,bottom=2cm,left=2cm,right=2cm,marginparwidth=2cm]{geometry}
\usepackage{float}
\usepackage{mathtools, commath, amssymb, amsthm}
\usepackage{enumitem, tabularx,graphicx,url,xcolor,rotating,multicol,epsfig,colortbl,lipsum}

\setlist{topsep=1pt, itemsep=0em}
\setlength{\parindent}{0pt}
\setlength{\parskip}{6pt}

\usepackage{hyphenat}
\hyphenation{ма-те-ма-ти-ка вос-ста-нав-ли-вать}

\usepackage[math]{anttor}

\newenvironment{talk}[6]{%
\vskip 0pt\nopagebreak%
\vskip 0pt\nopagebreak%
\section*{#1}
\phantomsection
\addcontentsline{toc}{section}{#2. \textit{#1}}
% \addtocontents{toc}{\textit{#1}\par}
\textit{#2}\\\nopagebreak%
#3\\\nopagebreak%
\ifthenelse{\equal{#4}{}}{}{\url{#4}\\\nopagebreak}%
\ifthenelse{\equal{#5}{}}{}{Соавторы: #5\\\nopagebreak}%
\ifthenelse{\equal{#6}{}}{}{Секция: #6\\\nopagebreak}%
}

\definecolor{LovelyBrown}{HTML}{FDFCF5}

\usepackage[pdftex,
breaklinks=true,
bookmarksnumbered=true,
linktocpage=true,
linktoc=all
]{hyperref}

\begin{document}
\pagenumbering{gobble}
\pagestyle{plain}
\pagecolor{LovelyBrown}
\begin{talk}
{Об автомодельных решениях задачи Стефана с бесконечным числом фазовых переходов}
{Панов Евгений Юрьевич}
{Новгородский государственный университет имени Ярослава Мудрого}
{}
{}
{Уравнения в частных производных, математическая физика и спектральная теория}

В области \(t,x>0\) рассматривается задача Стефана для уравнения теплопроводности
с фазовыми переходами при температурах
\(u_i>u_0\), \(i\in\mathbb{N}\). Считаем, что \(u_{i+1}>u_i\) \(\forall i\in\mathbb{N}\) и что \(\lim\limits_{i\to\infty}u_i=u_*\le+\infty\). Здесь \(i\)-ая фаза соответствует температурному интервалу \((u_i,u_{i+1})\) и характеризуется коэффициентами диффузии \(a_i>0\) и теплопроводности \(k_i>0\), \(i\in\{0\}\cup \mathbb{N}\). На неизвестных линиях \(x=x_i(t)\) фазовых переходов, где \(u=u_i\), задается условие Стефана
\(d_ix_i'(t)+k_iu_x(t,x_i(t)+)-k_{i-1}u_x(x,x_i(t)-)=0\),  \(i\in\mathbb{N}\),
\(d_i\ge 0\). Ставятся также начальное и краевое условия \(u(0,x)\equiv u_0\), \(u(t,0)\equiv u_*\). Изучаются убывающие по \(\xi=x/\sqrt{t}\) автомодельные решения \(u=u(\xi)\). Оказалось, что условия Стефана на линиях фазовых переходов \(\xi=\xi_i\) сводятся к равенствам \(\frac{\partial}{\partial\xi_i} E(\bar\xi)=0\), где
\(\displaystyle
E(\bar\xi)=-\sum_{i=0}^\infty  k_i(u_{i+1}-u_i)\ln (F(\xi_i/a_i)-F(\xi_{i+1}/a_i))+\sum_{i=1}^\infty d_i\xi_i^2/4\),
\(\displaystyle F(z)=\frac{1}{\sqrt{\pi}}\int_0^z e^{-s^2/4}ds\)
и \(\xi_0=+\infty\). Этот функционал задан на выпуклом конусе
\(\{ \ \bar\xi=(\xi_i)_{i\in\mathbb{N}}\in l_\infty \ | \ \xi_i>\xi_{i+1}>0 \ \forall i\in\mathbb{N} \ \}\)
и является коэрцитивным и строго выпуклым функционалом. Если \(E(\bar\xi)\not\equiv+\infty\), то существует единственная точка \(\bar\xi^0\) глобального минимума функционала \(E\). Координаты \(\xi^0_i\) этой точки определяют решение нашей задачи. Показано, что в случае \(E(\bar\xi)\equiv+\infty\) решение может отсутствовать.
\end{talk}
\end{document}