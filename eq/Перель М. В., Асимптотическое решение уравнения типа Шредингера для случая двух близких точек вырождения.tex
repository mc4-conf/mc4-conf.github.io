\documentclass[12pt]{article}
\usepackage{hyphsubst}
\usepackage[T2A]{fontenc}
\usepackage[english,main=russian]{babel}
\usepackage[utf8]{inputenc}
\usepackage[letterpaper,top=2cm,bottom=2cm,left=2cm,right=2cm,marginparwidth=2cm]{geometry}
\usepackage{float}
\usepackage{mathtools, commath, amssymb, amsthm}
\usepackage{enumitem, tabularx,graphicx,url,xcolor,rotating,multicol,epsfig,colortbl,lipsum}

\setlist{topsep=1pt, itemsep=0em}
\setlength{\parindent}{0pt}
\setlength{\parskip}{6pt}

\usepackage{hyphenat}
\hyphenation{ма-те-ма-ти-ка вос-ста-нав-ли-вать}

\usepackage[math]{anttor}

\newenvironment{talk}[6]{%
\vskip 0pt\nopagebreak%
\vskip 0pt\nopagebreak%
\section*{#1}
\phantomsection
\addcontentsline{toc}{section}{#2. \textit{#1}}
% \addtocontents{toc}{\textit{#1}\par}
\textit{#2}\\\nopagebreak%
#3\\\nopagebreak%
\ifthenelse{\equal{#4}{}}{}{\url{#4}\\\nopagebreak}%
\ifthenelse{\equal{#5}{}}{}{Соавторы: #5\\\nopagebreak}%
\ifthenelse{\equal{#6}{}}{}{Секция: #6\\\nopagebreak}%
}

\definecolor{LovelyBrown}{HTML}{FDFCF5}

\usepackage[pdftex,
breaklinks=true,
bookmarksnumbered=true,
linktocpage=true,
linktoc=all
]{hyperref}

\begin{document}
\pagenumbering{gobble}
\pagestyle{plain}
\pagecolor{LovelyBrown}
\begin{talk}
{Асимптотическое решение уравнения типа Шредингера для случая двух близких точек вырождения}
{Перель Мария Владимировна}
{Кафедра высшей математики и математической физики, СПбГУ}
{m.perel@spbu.ru}
{}
{Уравнения в частных производных, математическая физика и спектральная теория}

Асимптотические разложения решения
дифференциального уравнения в гильбертовом пространстве
\(
\mathbf{K}(x) \boldsymbol{\Psi} = - i \varepsilon {\boldsymbol \Gamma}  \frac{\partial\boldsymbol{\Psi}  }{\partial x}, \;\varepsilon \ll 1,
\)
где \(\mathbf{K}(x),\) \({\boldsymbol \Gamma}\)-- самосопряженные операторы, строятся на основе собственных  значений  \( \beta_n(x)\) и собственных элементов \(\boldsymbol{\varphi_n}(x)\) линейного операторного пучка
\(
\mathbf{K}(x) \boldsymbol{\varphi_n}(x) = \beta_n(x) {\boldsymbol \Gamma}\boldsymbol{\varphi_n}(x).
\)
Если невырожденное \(\beta_n(x)\)   отделено от остального спектра постоянной, не зависящей от \(\varepsilon\),  \(\beta=\beta_n(x)\) и \(\boldsymbol{\varphi}=\boldsymbol{\varphi_n}(x)\) -- гладкие функции, то соответствующее разложение называем адиабатическим. Оно, как правило, неприменимо  вблизи локальных точек вырождения, то есть таких  \(x_*\), в которых  \(\beta=\beta_n(x)\) пересекается с другим собственным значением \(\beta=\beta_j(x)\).
Если  	\(\mathbf{K}(x)=	\mathbf{K}_0(x) + \sqrt{\varepsilon}\mathbf{B}\), причем  собственные значения  \(\beta_n^0(x)\) и \(\beta_j^0(x)\)  невозмущенной спектральной задачи, пересекаются в некоторой точке \(x_*\),  \(\beta_n^0(x)-\beta_j^0(x) \approx Q(x-x_*)\), собственное пространство, соответствующее собственному значению  \(\beta_n^0(x_*)=\beta_j^0(x_*)\) двумерно,  весь остальной спектр отделен от  \(\beta_n^0(x)\) и \(\beta_j^0(x)\), то на интервале изменения \(x\), содержащем \(x_*\), найдено асимптотическое решение возмущенной задачи. В случае, когда на краях этого интервала
адиабатические разложения применимы, выведена явно матрица перехода, связывающая главные члены адиабатических разложений с разных сторон от точки \(x_*\).
Рассмотрены задачи,  к которым применен полученный  результат [1]. Они описываются системой обыкновенных дифференциальных уравнений или уравнениями в частных производных, причем \(\mathbf{K}(x)\) является дифференциальным по любым переменным, кроме \(x\).
Приводится матрица перехода для случая, когда двукратно вырожденное собственное значение имеет геометрическую кратность один.

\medskip

\begin{enumerate}
\item[{[1]}] Ignat \,V.\;Fialkovsky, Maria\,V.\;Perel, {\it Mode transformation for a Schrödinger type equation: Avoided and unavoidable level crossings}, {J Math Phys}, 61 (2020), 043506.
\end{enumerate}
\end{talk}
\end{document}