\documentclass[12pt]{article}
\usepackage{hyphsubst}
\usepackage[T2A]{fontenc}
\usepackage[english,main=russian]{babel}
\usepackage[utf8]{inputenc}
\usepackage[letterpaper,top=2cm,bottom=2cm,left=2cm,right=2cm,marginparwidth=2cm]{geometry}
\usepackage{float}
\usepackage{mathtools, commath, amssymb, amsthm}
\usepackage{enumitem, tabularx,graphicx,url,xcolor,rotating,multicol,epsfig,colortbl,lipsum}

\setlist{topsep=1pt, itemsep=0em}
\setlength{\parindent}{0pt}
\setlength{\parskip}{6pt}

\usepackage{hyphenat}
\hyphenation{ма-те-ма-ти-ка вос-ста-нав-ли-вать}

\usepackage[math]{anttor}

\newenvironment{talk}[6]{%
\vskip 0pt\nopagebreak%
\vskip 0pt\nopagebreak%
\section*{#1}
\phantomsection
\addcontentsline{toc}{section}{#2. \textit{#1}}
% \addtocontents{toc}{\textit{#1}\par}
\textit{#2}\\\nopagebreak%
#3\\\nopagebreak%
\ifthenelse{\equal{#4}{}}{}{\url{#4}\\\nopagebreak}%
\ifthenelse{\equal{#5}{}}{}{Соавторы: #5\\\nopagebreak}%
\ifthenelse{\equal{#6}{}}{}{Секция: #6\\\nopagebreak}%
}

\definecolor{LovelyBrown}{HTML}{FDFCF5}

\usepackage[pdftex,
breaklinks=true,
bookmarksnumbered=true,
linktocpage=true,
linktoc=all
]{hyperref}

\begin{document}
\pagenumbering{gobble}
\pagestyle{plain}
\pagecolor{LovelyBrown}
\begin{talk}
{Теорема единственности типа Кружкова для системы законов сохранения, описывающей химическое заводнение}
{Растегаев Никита Владимирович}
{Санкт-Петербургское отделение Математического института им. В.\,А.Стеклова РАН}
{rastmusician@gmail.com}
{Матвеенко~С.\,Г.}
{Уравнения в частных производных, математическая физика и спектральная теория}

Рассматривается система из двух гиперболических законов сохранения
\begin{equation*}
\begin{cases}
s_t + f(s, c)_x = 0, \\
(cs + a(c))_t + (cf(s,c))_x = 0,
\end{cases}
\end{equation*}
обычно описывающая заводнение нефтяного пласта раствором химического агента. Эта система не является ни истинно нелинейной, ни строго гиперболической, что ограничивает применение к ней общих результатов, относящихся к строго гиперболическим истинно нелинейным системам.
Решения некоторых начально-граничных задач (например, задачи Римана [1] или задачи закачки оторочки химического агента [2], [3]) для этой системы были исследованы ранее. В работах [2], [3] решения строятся методом характеристик с использованием перехода к лагранжевым координатам, в которых уравнения разделяются. При этом вопрос единственности построенных решений не исследован.

Мы используем предложенную замену координат для доказательства теоремы единственности типа Кружкова для решения начально-краевой задачи при определенных ограничениях на начальные данные и класс допустимых слабых решений. При определении допустимости разрывов используется локальный вариант критерия малого параметра (исчезающая вязкость).

\medskip

\begin{enumerate}
\item[{[1]}] Johansen~T. and Winther~R., {\it The solution of the Riemann problem for a hyperbolic system of conservation laws modeling polymer flooding}, SIAM journal on mathematical analysis, 1988, 19(3), 541--566.
\item[{[2]}] Pires~A.~P., Bedrikovetsky~P.~G. and Shapiro~A.~A., {\it A splitting technique for analytical modelling of two-phase multicomponent flow in porous media}, Journal of Petroleum Science and Engineering, 2006, 51(1-2), 54--67.
\item[{[3]}] Ribeiro~P.~M. and Pires~A.~P., {\it The displacement of oil by polymer slugs considering adsorption effects}, SPE Annual Technical Conference and Exhibition, 2008, September, SPE-115272.
\end{enumerate}
\end{talk}
\end{document}