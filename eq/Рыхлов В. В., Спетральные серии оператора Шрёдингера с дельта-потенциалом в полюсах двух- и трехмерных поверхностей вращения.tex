\documentclass[12pt]{article}
\usepackage{hyphsubst}
\usepackage[T2A]{fontenc}
\usepackage[english,main=russian]{babel}
\usepackage[utf8]{inputenc}
\usepackage[letterpaper,top=2cm,bottom=2cm,left=2cm,right=2cm,marginparwidth=2cm]{geometry}
\usepackage{float}
\usepackage{mathtools, commath, amssymb, amsthm}
\usepackage{enumitem, tabularx,graphicx,url,xcolor,rotating,multicol,epsfig,colortbl,lipsum}

\setlist{topsep=1pt, itemsep=0em}
\setlength{\parindent}{0pt}
\setlength{\parskip}{6pt}

\usepackage{hyphenat}
\hyphenation{ма-те-ма-ти-ка вос-ста-нав-ли-вать}

\usepackage[math]{anttor}

\newenvironment{talk}[6]{%
\vskip 0pt\nopagebreak%
\vskip 0pt\nopagebreak%
\section*{#1}
\phantomsection
\addcontentsline{toc}{section}{#2. \textit{#1}}
% \addtocontents{toc}{\textit{#1}\par}
\textit{#2}\\\nopagebreak%
#3\\\nopagebreak%
\ifthenelse{\equal{#4}{}}{}{\url{#4}\\\nopagebreak}%
\ifthenelse{\equal{#5}{}}{}{Соавторы: #5\\\nopagebreak}%
\ifthenelse{\equal{#6}{}}{}{Секция: #6\\\nopagebreak}%
}

\definecolor{LovelyBrown}{HTML}{FDFCF5}

\usepackage[pdftex,
breaklinks=true,
bookmarksnumbered=true,
linktocpage=true,
linktoc=all
]{hyperref}

\begin{document}
\pagenumbering{gobble}
\pagestyle{plain}
\pagecolor{LovelyBrown}
\begin{talk}
{Спетральные серии оператора Шрёдингера с дельта-по\-тен\-ци\-ал\-ом в полюсах двух- и трехмерных поверхностей вращения}
{Рыхлов Владислав Владимирович}
{Московский Государственный Университет им. М.\,В.~Ломоносова}
{vladderq@gmail.com}
{А.\,И.~Шафаревич}
{Уравнения в частных производных, математическая физика и спектральная теория}

Рассматривается спектральная задача для оператора Шрёдингера
\[H\psi = E\psi + o(h),\quad
H = -\frac{h^2}{2}\Delta + \delta_{x_1}(x) + \delta_{x_2}(x),
\quad x\in M,\]
где \(h\to+0\), \(x_j\) --- полюса \(M\), \(M\) --- двумерная \(M^2\subset\mathbb R^3\) или трехмерная \(M^3\subset\mathbb R^4\) поверхность вращения:
\[M^2 = \{ (v(s)\cos{\varphi}, v(s)\sin{\varphi}, w(s))\,\big|\, s\in [s_1,s_2], \varphi\in\mathbb S^1\}\quad\text{и}\]
\[M^3 = \{ (v(s)\cos{\theta}\cos{\varphi}, v(s)\cos{\theta}\sin{\varphi}, v(s)\sin{\theta}, w(s))\,\big|\, s\in [s_1,s_2], \varphi\in\mathbb S^1, \theta\in [-\frac{\pi}{2},\frac{\pi}{2}]\},\]
\(s\in [s_1,s_2]\) --- натуральный параметр кривой \((v(s),w(s))\subset\mathbb R^2\), и значения \(s_j\) соответствуют точкам \(x_j\).

Оператор~\(H\) определен на функциях из \(L^2(M)\) как самосопряженное расширение оператора~\(\Delta\), действующего на функциях \(\psi_0 \in W_2^2(M)\), обращающихся в нуль в полюсах~\(x_j\).
В области определения оператора~\(H\) лежат функции, имеющие особенности в точках~\(x_j\), а именно, для \(\psi\in D(H)\) имеется асимптотическое равенство~[1] при \(x\to x_j\)
\begin{equation}\label{asymp_expans}
\psi(x) = -\frac{a_j}{2\pi} \ln{d(x,x_j)} + b_j + o(1) \text{ на } M^2,\quad
\psi(x) = \frac{a_j}{4\pi} \frac{1}{d(x,x_j)} + b_j + o(1)\text{ на } M^3,
\end{equation}
где \(d(x,x_j)\) --- геодезическое расстояние на~\(M\) между точками~\(x\) и~\(x_j\), а коэффициенты~\(a_j\) при сингулярной части и \(b_j\) при регулярной связаны следующим образом:
\begin{equation}\label{boundary_cond}
i (\mathbf{I} + U) a + \frac{2}{h^2}(\mathbf{I}-U)b=0,\quad a = \begin{pmatrix} a_1 \\ a_2\end{pmatrix},\quad b = \begin{pmatrix} b_1 \\ b_2\end{pmatrix},
\end{equation}
где \(\mathbf{I}\) --- единичная матрица, \(U\) --- некоторый унитарный оператор.

В работе получены явные выражения для условий квантования Бора--Зоммерфельда--Маслова, позволяющие изучить поведение асимптотического спектра. Асимптотические собственные функции имеют представление в терминах функций Бесселя и Неймана.

\medskip

\begin{enumerate}
\item J.~Br\"uning, V.~A.~Geyler, Scattering on compact manifolds with infinitely thin horns, J. Math. Phys., 44:2 (2003), 371--405
\item В.~П.~Маслов, М.~В.~Федорюк, Квазиклассическое приближение для уравнений квантовой механики, Изд. Наука, Mосква, 1976
\item W.~Wasow, Asymptotic Expansions for Ordinary Differential Equations, Изд. Interscience Publ. John Wiley \& Sons, Inc., New York--London--Sydney, 1965
\end{enumerate}
\end{talk}
\end{document}