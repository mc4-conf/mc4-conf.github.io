\documentclass[12pt]{article}
\usepackage{hyphsubst}
\usepackage[T2A]{fontenc}
\usepackage[english,main=russian]{babel}
\usepackage[utf8]{inputenc}
\usepackage[letterpaper,top=2cm,bottom=2cm,left=2cm,right=2cm,marginparwidth=2cm]{geometry}
\usepackage{float}
\usepackage{mathtools, commath, amssymb, amsthm}
\usepackage{enumitem, tabularx,graphicx,url,xcolor,rotating,multicol,epsfig,colortbl,lipsum}

\setlist{topsep=1pt, itemsep=0em}
\setlength{\parindent}{0pt}
\setlength{\parskip}{6pt}

\usepackage{hyphenat}
\hyphenation{ма-те-ма-ти-ка вос-ста-нав-ли-вать}

\usepackage[math]{anttor}

\newenvironment{talk}[6]{%
\vskip 0pt\nopagebreak%
\vskip 0pt\nopagebreak%
\section*{#1}
\phantomsection
\addcontentsline{toc}{section}{#2. \textit{#1}}
% \addtocontents{toc}{\textit{#1}\par}
\textit{#2}\\\nopagebreak%
#3\\\nopagebreak%
\ifthenelse{\equal{#4}{}}{}{\url{#4}\\\nopagebreak}%
\ifthenelse{\equal{#5}{}}{}{Соавторы: #5\\\nopagebreak}%
\ifthenelse{\equal{#6}{}}{}{Секция: #6\\\nopagebreak}%
}

\definecolor{LovelyBrown}{HTML}{FDFCF5}

\usepackage[pdftex,
breaklinks=true,
bookmarksnumbered=true,
linktocpage=true,
linktoc=all
]{hyperref}

\begin{document}
\pagenumbering{gobble}
\pagestyle{plain}
\pagecolor{LovelyBrown}
\begin{talk}
{Многоточечные формулы в обратных задачах и их численная реализация}
{Сивкин Владимир Николаевич}
{МГУ им. Ломоносова}
{name@gmail.com}
{Новиков Р.\,Г.}
{Уравнения в частных производных, математическая физика и спектральная теория}


В ряде обратных задач рассеяния возникают асимптотические разложения, старшие члены которых содержат основную информацию о неизвестной функции. Чтобы повысить скорость сходимости, к таким разложениям можно применять метот многоточечных формул, предложенный в [1]. При этом, сами многоточечные формулы неустойчивы к шуму, и требуют дополнительных модификаций. В работе [2], в частности, представлена численная регуляризации многоточечных формул. При этом, реализованы различные формулы для нахождения
преобразования Фурье потенциала рассеяния уравнения Шредингера по данным амплитуды рассеяния при нескольких высоких энергиях. Данные формулы, в частности, улучшают классическую медленно сходящуюся формулу Борна-Фаддеева.

\medskip

\begin{enumerate}
\item[{[1]}] R.G. Novikov, {\it Multipoint formulas for scattered far field in multidimensions},  Inverse Problems 36.9 (2020): 095001.
\item[{[2]}] R.G. Novikov, V.N. Sivkin, G.V. Sabinin, {\it Multipoint formulas in inverse problems and their
numerical implementation}, Inverse Problems, 2023, 39 (12), pp.125016.
\end{enumerate}
\end{talk}
\end{document}