\documentclass[12pt]{article}
\usepackage{hyphsubst}
\usepackage[T2A]{fontenc}
\usepackage[english,main=russian]{babel}
\usepackage[utf8]{inputenc}
\usepackage[letterpaper,top=2cm,bottom=2cm,left=2cm,right=2cm,marginparwidth=2cm]{geometry}
\usepackage{float}
\usepackage{mathtools, commath, amssymb, amsthm}
\usepackage{enumitem, tabularx,graphicx,url,xcolor,rotating,multicol,epsfig,colortbl,lipsum}

\setlist{topsep=1pt, itemsep=0em}
\setlength{\parindent}{0pt}
\setlength{\parskip}{6pt}

\usepackage{hyphenat}
\hyphenation{ма-те-ма-ти-ка вос-ста-нав-ли-вать}

\usepackage[math]{anttor}

\newenvironment{talk}[6]{%
\vskip 0pt\nopagebreak%
\vskip 0pt\nopagebreak%
\section*{#1}
\phantomsection
\addcontentsline{toc}{section}{#2. \textit{#1}}
% \addtocontents{toc}{\textit{#1}\par}
\textit{#2}\\\nopagebreak%
#3\\\nopagebreak%
\ifthenelse{\equal{#4}{}}{}{\url{#4}\\\nopagebreak}%
\ifthenelse{\equal{#5}{}}{}{Соавторы: #5\\\nopagebreak}%
\ifthenelse{\equal{#6}{}}{}{Секция: #6\\\nopagebreak}%
}

\definecolor{LovelyBrown}{HTML}{FDFCF5}

\usepackage[pdftex,
breaklinks=true,
bookmarksnumbered=true,
linktocpage=true,
linktoc=all
]{hyperref}

\begin{document}
\pagenumbering{gobble}
\pagestyle{plain}
\pagecolor{LovelyBrown}
\begin{talk}
{Асимптотика решений начально-краевой задачи для одномерного уравнения Клейна-Гордона и моделирование распространения акустических возмущений в атмосфере}
{Смирнова Екатерина Сергеевна}
{Институт проблем механики им. А.\,Ю. Ишлинского РАН; Балтийский Федеральный Университет им. Канта}
{smirnova.ekaterina.serg@gmail.com}
{}
{Уравнения в частных производных, математическая физика и спектральная теория}

В работе рассматривается начально-краевая задача для одномерного уравнения Клейна-Гордона \(h^2\frac{\partial^2 U}{\partial \tau^2}-h^2 c^2(y)\frac{\partial^2 U}{\partial y^2}+a(y) U=0\) с переменными коэффициентами на полуосях \(y \geq 0\), \(\tau \geq 0\), из физической задачи о моделировании волновых возмущений, распространяющихся в атмосферном газе.  Построено асимптотическое при \(0<h<<1\) решение этой задачи. Показано, что оно состоит из двух частей [1, 2]: погранслойной быстроубывающей при отдалении от точки \(y=0\) и бегущей осциллирующей, представляемой в виде канонического оператора Маслова.

\medskip

Работа выполнена по теме государственного задания №124012500437-9.

\begin{enumerate}
\item[{[1]}] S. Dobrokhotov, E. Smirnova, {\it  Asymptotics of the Solution of the Initial Boundary Value Problem for the One-Dimensional Klein–Gordon Equation with Variable Coefficients,} Russian Journal of Mathematical Physics, 31:2 (2024), 187-198.
\item[{[2]}] Е.С. Смирнова, {\it  Асимптотика решения одной начально-краевой задачи для одномерного уравнения Клейна–Гордона на полуоси,}  Математические заметки, 114:4 (2023), 602-614.
\end{enumerate}
\end{talk}
\end{document}