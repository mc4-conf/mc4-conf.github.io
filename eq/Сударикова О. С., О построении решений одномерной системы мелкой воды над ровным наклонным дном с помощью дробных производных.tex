\documentclass[12pt]{article}
\usepackage{hyphsubst}
\usepackage[T2A]{fontenc}
\usepackage[english,main=russian]{babel}
\usepackage[utf8]{inputenc}
\usepackage[letterpaper,top=2cm,bottom=2cm,left=2cm,right=2cm,marginparwidth=2cm]{geometry}
\usepackage{float}
\usepackage{mathtools, commath, amssymb, amsthm}
\usepackage{enumitem, tabularx,graphicx,url,xcolor,rotating,multicol,epsfig,colortbl,lipsum}

\setlist{topsep=1pt, itemsep=0em}
\setlength{\parindent}{0pt}
\setlength{\parskip}{6pt}

\usepackage{hyphenat}
\hyphenation{ма-те-ма-ти-ка вос-ста-нав-ли-вать}

\usepackage[math]{anttor}

\newenvironment{talk}[6]{%
\vskip 0pt\nopagebreak%
\vskip 0pt\nopagebreak%
\section*{#1}
\phantomsection
\addcontentsline{toc}{section}{#2. \textit{#1}}
% \addtocontents{toc}{\textit{#1}\par}
\textit{#2}\\\nopagebreak%
#3\\\nopagebreak%
\ifthenelse{\equal{#4}{}}{}{\url{#4}\\\nopagebreak}%
\ifthenelse{\equal{#5}{}}{}{Соавторы: #5\\\nopagebreak}%
\ifthenelse{\equal{#6}{}}{}{Секция: #6\\\nopagebreak}%
}

\definecolor{LovelyBrown}{HTML}{FDFCF5}

\usepackage[pdftex,
breaklinks=true,
bookmarksnumbered=true,
linktocpage=true,
linktoc=all
]{hyperref}

\begin{document}
\pagenumbering{gobble}
\pagestyle{plain}
\pagecolor{LovelyBrown}
\begin{talk}
{О построении решений одномерной системы мелкой воды над ровным наклонным дном с помощью дробных производных}
{Сударикова Ольга Сергеевна}
{Московский физико-технический институт (национальный исследовательский университет), Институт проблем механики им. А.\,Ю. Ишлинского Российской академии наук}
{sudarikova.os@phystech.edu}
{Дмитрий Сергеевич Миненков}
{Уравнения в частных производных, математическая физика и спектральная теория}

Рассматривается одномерная нелинейная система уравнений мелкой воды, описывающая набег необрушающихся длинных волн на пологий берег. С помощью специальной замены система асимптотически сводится к линеаризованной системе мелкой воды (эквивалентной волновому уравнению с переменной скоростью), заданной в фиксированной области с вырождением функции дна на границе. Асимптотики нелинейной системы таким образом строятся с помощью решений линеаризованной системы [1]. Для линеаризованной системы известны асимптотики в виде стоячих или бегущих волн, а в специальных случаях и семейства точных решений (для ровного наклонного или параболического дна). С помощью дифференцирования по времени можно строить новые точные решения линеаризованной системы. В данной работе исследуются точные решения, полученные с помощью дробного дифференцирования по времени решений в виде бегущих волн из [2]. Также обсуждается связь некоторых определений дробного дифференцирования и вопрос удобной реализации финальных формул с помощью программных пакетов.

\medskip

Работа выполнена по теме государственного задания ИПМех РАН (№ госрегистрации 124012500442-3).

\begin{enumerate}
\item[{[1]}] Dobrokhotov S.Yu, Minenkov D.S., Nazaikinskii V.E., {\it Asymptotic Solutions of the Cauchy Problem for the Nonlinear Shallow Water Equations in a Basin with a Gently Sloping Beach}, Russian J. of Math. Phys., 2022, T.29 № 1. C.28-36
\item[{[2]}] 	Доброхотов С. Ю., Тироцци Б., {\it Локализованные решения одномерной нелинейной системы уравнений мелкой воды со скоростью \(c=\sqrt x\)},  УМН, 2010, Т. 65. В. 1 (391). С. 185-186.
\end{enumerate}
\end{talk}
\end{document}