\documentclass[12pt]{article}
\usepackage{hyphsubst}
\usepackage[T2A]{fontenc}
\usepackage[english,main=russian]{babel}
\usepackage[utf8]{inputenc}
\usepackage[letterpaper,top=2cm,bottom=2cm,left=2cm,right=2cm,marginparwidth=2cm]{geometry}
\usepackage{float}
\usepackage{mathtools, commath, amssymb, amsthm}
\usepackage{enumitem, tabularx,graphicx,url,xcolor,rotating,multicol,epsfig,colortbl,lipsum}

\setlist{topsep=1pt, itemsep=0em}
\setlength{\parindent}{0pt}
\setlength{\parskip}{6pt}

\usepackage{hyphenat}
\hyphenation{ма-те-ма-ти-ка вос-ста-нав-ли-вать}

\usepackage[math]{anttor}

\newenvironment{talk}[6]{%
\vskip 0pt\nopagebreak%
\vskip 0pt\nopagebreak%
\section*{#1}
\phantomsection
\addcontentsline{toc}{section}{#2. \textit{#1}}
% \addtocontents{toc}{\textit{#1}\par}
\textit{#2}\\\nopagebreak%
#3\\\nopagebreak%
\ifthenelse{\equal{#4}{}}{}{\url{#4}\\\nopagebreak}%
\ifthenelse{\equal{#5}{}}{}{Соавторы: #5\\\nopagebreak}%
\ifthenelse{\equal{#6}{}}{}{Секция: #6\\\nopagebreak}%
}

\definecolor{LovelyBrown}{HTML}{FDFCF5}

\usepackage[pdftex,
breaklinks=true,
bookmarksnumbered=true,
linktocpage=true,
linktoc=all
]{hyperref}

\begin{document}
\pagenumbering{gobble}
\pagestyle{plain}
\pagecolor{LovelyBrown}
\begin{talk}
{Усточивость МГД течений полимерной жидкости в цилиндрическом канале (обобщение модели Виноградова-Покровского)}
{Ткачев Дмитрий Леонидович}
{Институт математики им. С.\,Л. Соболева}
{tkachev@math.nsc.ru}
{Бибердорф Элина Арнольдовна}
{Уравнения в частных производных, математическая физика и спектральная теория}

Изучается устойчивость состояния покоя для течений несжимаемой вязкоупругой полимерной жидкости в бесконечном цилиндрическом канале в классе осесимметрических возмущений. В качестве математической модели используется структурно-феноменологическая модель Виноградова--Покровского [1, 2].

Сформулированы два уравнения для радиальной компоненты скорости, в основном определяющие спектр задачи в случае абсолютной проводимости \(b_{m} = 0\) и в общем случае \(b_{m} \neq 0\). Проведенные вычислительные эксперименты показывают, что с ростом частоты возмущений вдоль оси канала у  спектрального уравнения (в случае \(b_{m} = 0\)) появляются собственные значения с положительными вещественными частями, однако по величине они малы.

В целом исследования показывают, что введение в модель внешнего магнитного поля позволяет ослабить или даже погасить линейную неустойчивость по Ляпунову состояния покоя в отличие от базовой модели [3].

\medskip

Работа первого автора выполнена при поддержке Математического Центра в Академгородке, соглашение с Министерством науки и высшего образования Российской Федерации № 075-15-2022-281.

\begin{enumerate}
\item[{[1]}] Pokrovskii V.\,N., \emph{The mesoscopic theory of polymer dynamics}, Springer Ser. Chem. Phys., 95, Springer, Dordrecht (2010).
\item[{[2]}] Altukhov Yu.\,A., Gusev A.\,S., Pishnograi G.\,V., \emph{Introduction into mesoscopic theory of flowing polymeric systems}, Alt. GPA, Barnaul (2012).
\item[{[3]}] Tkachev D.\,L. and Biberdorf E.\,A., \emph{Spectrum of a problem about the flow of a polymeric viscoelastic fluid in a cylindrical channel (Vinogradov-Pokrovski model)}, Siberian Electro\-nic Mathematical Reports, \textbf{20}(2), 1269--1289 (2023).
\end{enumerate}
\end{talk}
\end{document}