\documentclass[12pt]{article}
\usepackage{hyphsubst}
\usepackage[T2A]{fontenc}
\usepackage[english,main=russian]{babel}
\usepackage[utf8]{inputenc}
\usepackage[letterpaper,top=2cm,bottom=2cm,left=2cm,right=2cm,marginparwidth=2cm]{geometry}
\usepackage{float}
\usepackage{mathtools, commath, amssymb, amsthm}
\usepackage{enumitem, tabularx,graphicx,url,xcolor,rotating,multicol,epsfig,colortbl,lipsum}

\setlist{topsep=1pt, itemsep=0em}
\setlength{\parindent}{0pt}
\setlength{\parskip}{6pt}

\usepackage{hyphenat}
\hyphenation{ма-те-ма-ти-ка вос-ста-нав-ли-вать}

\usepackage[math]{anttor}

\newenvironment{talk}[6]{%
\vskip 0pt\nopagebreak%
\vskip 0pt\nopagebreak%
\section*{#1}
\phantomsection
\addcontentsline{toc}{section}{#2. \textit{#1}}
% \addtocontents{toc}{\textit{#1}\par}
\textit{#2}\\\nopagebreak%
#3\\\nopagebreak%
\ifthenelse{\equal{#4}{}}{}{\url{#4}\\\nopagebreak}%
\ifthenelse{\equal{#5}{}}{}{Соавторы: #5\\\nopagebreak}%
\ifthenelse{\equal{#6}{}}{}{Секция: #6\\\nopagebreak}%
}

\definecolor{LovelyBrown}{HTML}{FDFCF5}

\usepackage[pdftex,
breaklinks=true,
bookmarksnumbered=true,
linktocpage=true,
linktoc=all
]{hyperref}

\begin{document}
\pagenumbering{gobble}
\pagestyle{plain}
\pagecolor{LovelyBrown}
\begin{talk}
{Cлед резольвенты оператора Лапласа на метрическом графе}
{Толченников Антон Александрович}
{ИПМех РАН, Механико-математический факультет МГУ}
{}
{}
{Уравнения в частных производных, математическая физика и спектральная теория}

Доклад будет посвящен оператору Лапласа на метрическом конечном графе \(G=(V,E)\) с длинами ребер \(\{ l_j \}_{j =1}^{|E|}\),
который на каждом ребре задан выражением \(-\frac{d^2}{dx^2}\), а в вершинах графа заданы граничные условия, обеспечивающие самосопряженность оператора. Для работы с такими операторами есть эффективный метод ---
формализм Крейна, позволяющий удобно записывать резольвенту такого оператора и вычислять его след (или след подходящей степени резольвенты в случае пространств большей размерности, например, декорированных графов).
Каждый такой оператор однозначно определяется лагранжевой плоскостью в \(\mathbb{C}^{|2E|}\oplus \mathbb{C}^{2|E|}\): \(\Lambda \leftrightarrow \Delta^{\Lambda}\). Цель --- написать  коэффициенты
разложения \(\mathrm{tr} (\Delta^\Lambda -z)^{-1}\) при больших \(z\) (за исключением сектора, содержащего положительную вещественную полуось).
Первые два слагаемых в разложении особенно просты:
\( \mathrm{tr} (\Delta^\Lambda + w^2)^{-1} =
\frac{\sum_{i=1}^{|E|} l_i}{2w} + \frac{|E| - \dim \Lambda\cap \Lambda_X}{2w^2} +
O(w^{-3}),
\)
где \(\Lambda_X = \mathbb{C}^{2|E|} \oplus 0\). Этот подход, связанный с резольвентной формулой Крейна, можно применять для более сложных пространств -- декорированных графов.

\medskip

Работа поддержана грантом РНФ 22-11-00272.
\end{talk}
\end{document}