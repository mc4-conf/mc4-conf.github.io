\documentclass[12pt]{article}
\usepackage{hyphsubst}
\usepackage[T2A]{fontenc}
\usepackage[english,main=russian]{babel}
\usepackage[utf8]{inputenc}
\usepackage[letterpaper,top=2cm,bottom=2cm,left=2cm,right=2cm,marginparwidth=2cm]{geometry}
\usepackage{float}
\usepackage{mathtools, commath, amssymb, amsthm}
\usepackage{enumitem, tabularx,graphicx,url,xcolor,rotating,multicol,epsfig,colortbl,lipsum}

\setlist{topsep=1pt, itemsep=0em}
\setlength{\parindent}{0pt}
\setlength{\parskip}{6pt}

\usepackage{hyphenat}
\hyphenation{ма-те-ма-ти-ка вос-ста-нав-ли-вать}

\usepackage[math]{anttor}

\newenvironment{talk}[6]{%
\vskip 0pt\nopagebreak%
\vskip 0pt\nopagebreak%
\section*{#1}
\phantomsection
\addcontentsline{toc}{section}{#2. \textit{#1}}
% \addtocontents{toc}{\textit{#1}\par}
\textit{#2}\\\nopagebreak%
#3\\\nopagebreak%
\ifthenelse{\equal{#4}{}}{}{\url{#4}\\\nopagebreak}%
\ifthenelse{\equal{#5}{}}{}{Соавторы: #5\\\nopagebreak}%
\ifthenelse{\equal{#6}{}}{}{Секция: #6\\\nopagebreak}%
}

\definecolor{LovelyBrown}{HTML}{FDFCF5}

\usepackage[pdftex,
breaklinks=true,
bookmarksnumbered=true,
linktocpage=true,
linktoc=all
]{hyperref}

\begin{document}
\pagenumbering{gobble}
\pagestyle{plain}
\pagecolor{LovelyBrown}
\begin{talk}
{Интегрируемые двумерные геодезические потоки в магнитном поле}
{Агапов Сергей Вадимович}
{Институт математики им. С.\,Л. Соболева СО РАН, Новосибирск}
{agapov.sergey.v@gmail.com}
{}
{Геометрия} %

В докладе речь пойдет об интегрируемых геодезических потоках на двумерных поверхностях в ненулевом магнитном поле.
Известно, что требование интегрируемости таких потоков одновременно на всех (или хотя бы на нескольких различных) уровнях энергии
является весьма ограничительным (см., например, [1], [2]).
С другой стороны, хорошо известны примеры метрик и магнитных полей, допускающих дополнительный интеграл лишь на фиксированном уровне энергии.
Так, например, в [3], [4] доказано, что на двумерном торе существуют семейства аналитических римановых метрик и магнитных полей
с дополнительным квадратичным по импульсам первым интегралом.
Различные примеры локальных рациональных по импульсам первых интегралов магнитных геодезических потоков в явном виде построены в [5].

\medskip

Доклад основан на совместных работах с М. Бялым, А.\,А. Валюженичем, А.\,Е. Мироновым, А.\,И. Поташниковым, В.\,В. Шубиным.

\begin{enumerate}
\item[{[1]}] И.А. Тайманов, {\it О первых интегралах геодезических потоков на двумерном торе}, Труды МИАН, 295 (2016), 241–260.
\item[{[2]}] S. Agapov, A. Valyuzhenich, {\it Polynomial integrals of magnetic geodesic flows on the 2-torus on several energy levels},
Disc. Cont. Dyn. Syst. - A., 39:11 (2019), 6565–6583.
\item[{[3]}] B. Dorizzi, B. Grammaticos, A. Ramani, P. Winternitz, {\it Integrable Hamiltonian systems with velocity-dependent potentials},
J. Math. Phys., 26:12 (1985), 3070–3079.
\item[{[4]}] S.V. Agapov, M. Bialy, A.E. Mironov, {\it Integrable magnetic geodesic flows on 2-torus: new examples via quasi-linear system of PDEs},
Comm. Math. Phys., 351:3 (2017), 993–1007.
\item[{[5]}] S. Agapov, A. Potashnikov, V. Shubin, {\it Integrable magnetic geodesic flows on 2-surfaces}, Nonlinearity, 36:4 (2023), 2128–2147.
\end{enumerate}
\end{talk}
\end{document}