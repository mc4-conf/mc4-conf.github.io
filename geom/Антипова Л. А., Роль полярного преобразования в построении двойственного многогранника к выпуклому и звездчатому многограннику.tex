\documentclass[12pt]{article}
\usepackage{hyphsubst}
\usepackage[T2A]{fontenc}
\usepackage[english,main=russian]{babel}
\usepackage[utf8]{inputenc}
\usepackage[letterpaper,top=2cm,bottom=2cm,left=2cm,right=2cm,marginparwidth=2cm]{geometry}
\usepackage{float}
\usepackage{mathtools, commath, amssymb, amsthm}
\usepackage{enumitem, tabularx,graphicx,url,xcolor,rotating,multicol,epsfig,colortbl,lipsum}

\setlist{topsep=1pt, itemsep=0em}
\setlength{\parindent}{0pt}
\setlength{\parskip}{6pt}

\usepackage{hyphenat}
\hyphenation{ма-те-ма-ти-ка вос-ста-нав-ли-вать}

\usepackage[math]{anttor}

\newenvironment{talk}[6]{%
\vskip 0pt\nopagebreak%
\vskip 0pt\nopagebreak%
\section*{#1}
\phantomsection
\addcontentsline{toc}{section}{#2. \textit{#1}}
% \addtocontents{toc}{\textit{#1}\par}
\textit{#2}\\\nopagebreak%
#3\\\nopagebreak%
\ifthenelse{\equal{#4}{}}{}{\url{#4}\\\nopagebreak}%
\ifthenelse{\equal{#5}{}}{}{Соавторы: #5\\\nopagebreak}%
\ifthenelse{\equal{#6}{}}{}{Секция: #6\\\nopagebreak}%
}

\definecolor{LovelyBrown}{HTML}{FDFCF5}

\usepackage[pdftex,
breaklinks=true,
bookmarksnumbered=true,
linktocpage=true,
linktoc=all
]{hyperref}

\begin{document}
\pagenumbering{gobble}
\pagestyle{plain}
\pagecolor{LovelyBrown}
\begin{talk}
{Роль полярного преобразования в построении двойственного многогранника к выпуклому и звездчатому многограннику}
{Антипова Любовь Александровна}
{РГПУ им. А.\,И. Герцена}
{pridoroga31@ya.ru}
{Вернер Алексей Леонидович}
{Геометрия} %

Каждому многограннику в трехмерном пространстве можно поставить в соответствие двойственный ему абстрактный многогранник, то есть такой, вершинам, ребрам и граням которого отвечают соответственно грани, ребра и вершины исходного, и при этом инцидентным парам элементов одного многогранника соответствуют инцидентные пары второго. Возникает вопрос –-- существует ли универсальный способ построения двойственного многогранника?

В евклидовом трехмерном пространстве двойственный многогранник к выпуклому многограннику можно построить с помощью полярного преобразования относительно некоторой сферы. Для выпуклого многогранника, обладающего центром симметрии, можно использовать сферу с центром в этой точке. В случае существования описанной сферы, вписанной сферы или средневписанной сферы за центр поляритета можно брать центр соответствующей сферы. Заметим, что, меняя расположение центра сферы относительно данного многогранника, меняется результирующая форма двойственного многогранника. Также известно, что выбор центра сферы, относительно которой осуществляется построение полярного многогранника, определяет его с точностью до подобия. Таким образом, применение полярного преобразования является универсальным способом построения двойственных многогранников к выпуклым.

В докладе будет обобщено понятие полярного образа многогранника с выпуклого случая на множество звездчатых многогранников и предъявлен универсальный способ построения двойственных многогранников к звездчатым. Рассматривая класс однородных многогранников в соответствии со статьей [1], будет получен класс двойственных многогранников и доказаны их свойства.

\medskip

\begin{enumerate}
\item[{[1]}] Антипова Л. А. Конфигурации полярно-двойственных многогранников. Каталановы звезды / Антипова Л. А. // Современные проблемы математики и математического образования: Герценовские чтения, 76 : сборник научных статей Международной научной конференции, Санкт-Петербург, 18-20 апреля 2023 года / Российский государственный педагогический университет им. А. И. Герцена. — Санкт-Петербург, 2023. — С. 326-332. — URL: \url{https://elibrary.ru/item.asp?id=54907979}.
\end{enumerate}
\end{talk}
\end{document}