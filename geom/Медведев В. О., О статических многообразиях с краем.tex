\documentclass[12pt]{article}
\usepackage{hyphsubst}
\usepackage[T2A]{fontenc}
\usepackage[english,main=russian]{babel}
\usepackage[utf8]{inputenc}
\usepackage[letterpaper,top=2cm,bottom=2cm,left=2cm,right=2cm,marginparwidth=2cm]{geometry}
\usepackage{float}
\usepackage{mathtools, commath, amssymb, amsthm}
\usepackage{enumitem, tabularx,graphicx,url,xcolor,rotating,multicol,epsfig,colortbl,lipsum}

\setlist{topsep=1pt, itemsep=0em}
\setlength{\parindent}{0pt}
\setlength{\parskip}{6pt}

\usepackage{hyphenat}
\hyphenation{ма-те-ма-ти-ка вос-ста-нав-ли-вать}

\usepackage[math]{anttor}

\newenvironment{talk}[6]{%
\vskip 0pt\nopagebreak%
\vskip 0pt\nopagebreak%
\section*{#1}
\phantomsection
\addcontentsline{toc}{section}{#2. \textit{#1}}
% \addtocontents{toc}{\textit{#1}\par}
\textit{#2}\\\nopagebreak%
#3\\\nopagebreak%
\ifthenelse{\equal{#4}{}}{}{\url{#4}\\\nopagebreak}%
\ifthenelse{\equal{#5}{}}{}{Соавторы: #5\\\nopagebreak}%
\ifthenelse{\equal{#6}{}}{}{Секция: #6\\\nopagebreak}%
}

\definecolor{LovelyBrown}{HTML}{FDFCF5}

\usepackage[pdftex,
breaklinks=true,
bookmarksnumbered=true,
linktocpage=true,
linktoc=all
]{hyperref}

\begin{document}
\pagenumbering{gobble}
\pagestyle{plain}
\pagecolor{LovelyBrown}
\begin{talk}
{О статических многообразиях с краем}
{Медведев Владимир Олегович}
{Национальный исследовательский университет ``Высшая школа экономики''}
{}
{}
{Геометрия} %

Статические многообразия с краем были введены в геометрию совсем недавно и сразу же вызвали живой интерес в связи с их приложениями к различным вопросам геометрической теории относительности. В римановой геометрии данные многообразия возникают естественным образом при изучении вопросов деформации скалярной кривизны многообразий с краем. Родственным понятием является понятие статической тройки, чья важность была осознана в классических работах Кобаяши, Ляфонтэна и Бургиньона. В докладе будут обсуждаться свойства статических многообразий с краем и их приложения.

\medskip

\begin{enumerate}
\item[{[1]}] L. Ambrozio. On static three-manifolds with positive scalar curvature. Journal of Differen\-tial Geometry, 107(1):1–45, 2017.
\item[{[2]}] T. Cruz and F. Vitorio. Prescribing the curvature of Riemannian manifolds with boundary. Calculus of Variations and Partial Differential Equations, 58(4):124, 2019.
\item[{[3]}] T. Cruz and I. Nunes. On static manifolds satisfying an overdetermined Robin type condition on the boundary. Proceedings of the American Mathematical Society, 151(11): 4971–4982, 2023.
\item[{[4]}] V.Medvedev. On static manifolds with boundary. In progress.
\end{enumerate}
\end{talk}
\end{document}