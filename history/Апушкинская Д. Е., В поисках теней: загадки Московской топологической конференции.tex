\documentclass[12pt]{article}
\usepackage{hyphsubst}
\usepackage[T2A]{fontenc}
\usepackage[english,main=russian]{babel}
\usepackage[utf8]{inputenc}
\usepackage[letterpaper,top=2cm,bottom=2cm,left=2cm,right=2cm,marginparwidth=2cm]{geometry}
\usepackage{float}
\usepackage{mathtools, commath, amssymb, amsthm}
\usepackage{enumitem, tabularx,graphicx,url,xcolor,rotating,multicol,epsfig,colortbl,lipsum}

\setlist{topsep=1pt, itemsep=0em}
\setlength{\parindent}{0pt}
\setlength{\parskip}{6pt}

\usepackage{hyphenat}
\hyphenation{ма-те-ма-ти-ка вос-ста-нав-ли-вать}

\usepackage[math]{anttor}

\newenvironment{talk}[6]{%
\vskip 0pt\nopagebreak%
\vskip 0pt\nopagebreak%
\section*{#1}
\phantomsection
\addcontentsline{toc}{section}{#2. \textit{#1}}
% \addtocontents{toc}{\textit{#1}\par}
\textit{#2}\\\nopagebreak%
#3\\\nopagebreak%
\ifthenelse{\equal{#4}{}}{}{\url{#4}\\\nopagebreak}%
\ifthenelse{\equal{#5}{}}{}{Соавторы: #5\\\nopagebreak}%
\ifthenelse{\equal{#6}{}}{}{Секция: #6\\\nopagebreak}%
}

\definecolor{LovelyBrown}{HTML}{FDFCF5}

\usepackage[pdftex,
breaklinks=true,
bookmarksnumbered=true,
linktocpage=true,
linktoc=all
]{hyperref}

\begin{document}
\pagenumbering{gobble}
\pagestyle{plain}
\pagecolor{LovelyBrown}
\begin{talk}
{В поисках теней: загадки Московской топологической конференции}
{Апушкинская Дарья Евгеньевна и Назаров Александр Ильич}
{РУДН; ПОМИ РАН и СПбГУ}
{al.il.nazarov@gmail.com, apushkinskaya@gmail.com}
{Г.\,И. Синкевич}
{История математики}

Первая международная топологическая конференция проходила в Москве с 4 по 10 сентября 1935 года. Это была вторая (после конференции по дифференциальной геометрии 1934 г.) специализированная конференция в истории международного математического сообщества, собравшая выдающийся состав участников из 10 стран Европы и Америки. Представленные на ней результаты оказали колоссальное влияние на развитие топологии.

Работа конференции была широко освещена как в официальных публикациях, так и в многочисленных воспоминаниях участников. Тем не менее фактическая информация (список докладчиков, количество докладов и т. д.) почти во всех источниках была дана неполно или неточно, а зачастую и противоречиво, что довольно загадочно для события, произошедшего сравнительно недавно.

Основываясь на доступных источниках, мы попытались представить полную и непротиворечивую картину событий. В частности, мы приводим полный список докладов и докладчиков, а также даем полное описание фотографии участников конференции.

Доклад основан на статье [1].

\medskip

\begin{enumerate}
\item[{[1]}] D.E. Apushkinskaya, A.I. Nazarov, G.I. Sinkevich, {\it In Search of Shadows: The First Topological Conference, Moscow 1935}, The Mathematical Intelligencer, 41:4 (2019), 37-42.
\end{enumerate}
\end{talk}
\end{document}