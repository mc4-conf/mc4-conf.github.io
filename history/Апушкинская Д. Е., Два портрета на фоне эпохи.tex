\documentclass[12pt]{article}
\usepackage{hyphsubst}
\usepackage[T2A]{fontenc}
\usepackage[english,main=russian]{babel}
\usepackage[utf8]{inputenc}
\usepackage[letterpaper,top=2cm,bottom=2cm,left=2cm,right=2cm,marginparwidth=2cm]{geometry}
\usepackage{float}
\usepackage{mathtools, commath, amssymb, amsthm}
\usepackage{enumitem, tabularx,graphicx,url,xcolor,rotating,multicol,epsfig,colortbl,lipsum}

\setlist{topsep=1pt, itemsep=0em}
\setlength{\parindent}{0pt}
\setlength{\parskip}{6pt}

\usepackage{hyphenat}
\hyphenation{ма-те-ма-ти-ка вос-ста-нав-ли-вать}

\usepackage[math]{anttor}

\newenvironment{talk}[6]{%
\vskip 0pt\nopagebreak%
\vskip 0pt\nopagebreak%
\section*{#1}
\phantomsection
\addcontentsline{toc}{section}{#2. \textit{#1}}
% \addtocontents{toc}{\textit{#1}\par}
\textit{#2}\\\nopagebreak%
#3\\\nopagebreak%
\ifthenelse{\equal{#4}{}}{}{\url{#4}\\\nopagebreak}%
\ifthenelse{\equal{#5}{}}{}{Соавторы: #5\\\nopagebreak}%
\ifthenelse{\equal{#6}{}}{}{Секция: #6\\\nopagebreak}%
}

\definecolor{LovelyBrown}{HTML}{FDFCF5}

\usepackage[pdftex,
breaklinks=true,
bookmarksnumbered=true,
linktocpage=true,
linktoc=all
]{hyperref}

\begin{document}
\pagenumbering{gobble}
\pagestyle{plain}
\pagecolor{LovelyBrown}
\begin{talk}
{Два портрета на фоне эпохи}
{Апушкинская Дарья Евгеньевна и Назаров Александр Ильич}
{РУДН; ПОМИ РАН и СПбГУ}
{apushkinskaya@gmail.com, al.il.nazarov@gmail.com}
{}
{История математики} %

Начало XX века в России. Эпоха расцвета промышленности, науки, культуры, и в то же время - эпоха нарастания общественных противоречий, вылившихся в кровавую кашу гражданской войны. Эпоха неустойчивости и ``экспоненциального разбегания'' человеческих судеб. Мы рассмотрим этот период на примере двух ``мировых линий'' с близкими начальными условиями.

Владимир Иванович Смирнов и Яков Давидович Тамаркин - ровесники, друзья, выпускники Петербургского университета. Выдающиеся математики, ставшие и выдающимися организаторами науки и создателями научных школ – один в Советском Союзе, другой в Соединенных Штатах Америки. Драматических, а порой и трагических поворотов в их биографиях хватило бы на пару голливудских боевиков. Синóпсис такого боевика мы и хотим представить слушателям.
\end{talk}
\end{document}