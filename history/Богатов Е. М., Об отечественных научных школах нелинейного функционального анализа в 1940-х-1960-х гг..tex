\documentclass[12pt]{article}
\usepackage{hyphsubst}
\usepackage[T2A]{fontenc}
\usepackage[english,main=russian]{babel}
\usepackage[utf8]{inputenc}
\usepackage[letterpaper,top=2cm,bottom=2cm,left=2cm,right=2cm,marginparwidth=2cm]{geometry}
\usepackage{float}
\usepackage{mathtools, commath, amssymb, amsthm}
\usepackage{enumitem, tabularx,graphicx,url,xcolor,rotating,multicol,epsfig,colortbl,lipsum}

\setlist{topsep=1pt, itemsep=0em}
\setlength{\parindent}{0pt}
\setlength{\parskip}{6pt}

\usepackage{hyphenat}
\hyphenation{ма-те-ма-ти-ка вос-ста-нав-ли-вать}

\usepackage[math]{anttor}

\newenvironment{talk}[6]{%
\vskip 0pt\nopagebreak%
\vskip 0pt\nopagebreak%
\section*{#1}
\phantomsection
\addcontentsline{toc}{section}{#2. \textit{#1}}
% \addtocontents{toc}{\textit{#1}\par}
\textit{#2}\\\nopagebreak%
#3\\\nopagebreak%
\ifthenelse{\equal{#4}{}}{}{\url{#4}\\\nopagebreak}%
\ifthenelse{\equal{#5}{}}{}{Соавторы: #5\\\nopagebreak}%
\ifthenelse{\equal{#6}{}}{}{Секция: #6\\\nopagebreak}%
}

\definecolor{LovelyBrown}{HTML}{FDFCF5}

\usepackage[pdftex,
breaklinks=true,
bookmarksnumbered=true,
linktocpage=true,
linktoc=all
]{hyperref}

\begin{document}
\pagenumbering{gobble}
\pagestyle{plain}
\pagecolor{LovelyBrown}
\begin{talk}
{Об отечественных научных школах нелинейного функционального анализа в 1940-х-1960-х гг.}
{Богатов Егор Михайлович}
{ГФ НИТУ МИСИС; СТИ НИТУ МИСИС}
{embogatov@inbox.ru}
{}
{История математики} %

Рассмотрение истории математики через призму научных школ даёт дополнительное представление о  математике и её истории [1]. Математические мыслительные средства, выработанные в рамках какой-либо научной школы, являются продуктом коллективной деятельности, что многократно увеличивает скорость их приведения в систему, генерацию продуктивных методов их использования и распространение в пределах национальных и международных научных сообществ.

Основным результатом работы является:
\begin{enumerate}
\item введение в историко-научный оборот в области  математики XX в. нового материала - истории отечественных школ нелинейного функционального анализа (НФА);
\item характеризация научных школ НФА, функционирующих в СССР в 1940-х-1960-х гг. с выделением времени и места их основания, руководителей и основных представителей, продолжительности функционирования и конкретизацией области исследований;
\item определение вклада отечественных школ НФА в развитие следующих его разделов - вариационного исчисления в целом, теории ветвления и бифуркаций, теории положительных операторов, топологических методов нелинейного анализа, вариационных и приближённых  методов решения нелинейных операторных уравнений.
\end{enumerate}

\medskip

\begin{enumerate}
\item[{[1]}] Demidov S. L'histoire des mathématiques en Russie et l'URSS en tant qu'histoire des écoles // ИМИ. 2-я серия. Спец. выпуск. М., 1997. С.9-21.
\end{enumerate}
\end{talk}
\end{document}