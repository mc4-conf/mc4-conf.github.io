\documentclass[12pt]{article}
\usepackage{hyphsubst}
\usepackage[T2A]{fontenc}
\usepackage[english,main=russian]{babel}
\usepackage[utf8]{inputenc}
\usepackage[letterpaper,top=2cm,bottom=2cm,left=2cm,right=2cm,marginparwidth=2cm]{geometry}
\usepackage{float}
\usepackage{mathtools, commath, amssymb, amsthm}
\usepackage{enumitem, tabularx,graphicx,url,xcolor,rotating,multicol,epsfig,colortbl,lipsum}

\setlist{topsep=1pt, itemsep=0em}
\setlength{\parindent}{0pt}
\setlength{\parskip}{6pt}

\usepackage{hyphenat}
\hyphenation{ма-те-ма-ти-ка вос-ста-нав-ли-вать}

\usepackage[math]{anttor}

\newenvironment{talk}[6]{%
\vskip 0pt\nopagebreak%
\vskip 0pt\nopagebreak%
\section*{#1}
\phantomsection
\addcontentsline{toc}{section}{#2. \textit{#1}}
% \addtocontents{toc}{\textit{#1}\par}
\textit{#2}\\\nopagebreak%
#3\\\nopagebreak%
\ifthenelse{\equal{#4}{}}{}{\url{#4}\\\nopagebreak}%
\ifthenelse{\equal{#5}{}}{}{Соавторы: #5\\\nopagebreak}%
\ifthenelse{\equal{#6}{}}{}{Секция: #6\\\nopagebreak}%
}

\definecolor{LovelyBrown}{HTML}{FDFCF5}

\usepackage[pdftex,
breaklinks=true,
bookmarksnumbered=true,
linktocpage=true,
linktoc=all
]{hyperref}

\begin{document}
\pagenumbering{gobble}
\pagestyle{plain}
\pagecolor{LovelyBrown}
\begin{talk}
{Из истории высшего математического образования в Оренбургском крае}
{Зубова Инна Каримовна}
{Оренбургский государственный университет}
{zubova-inna@yandtx.ru}
{Игнатушина Инесса Васильевна}
{История математики} %

Говоря о предыстории высшего образования в Оренбургском крае, нельзя не вспомнить о тех средних учебных заведениях, в которых уже начиная с 30-х гг. XIX в.
работали выпускники лучших университетов страны, приезжавшие сюда по назначению и, как правило, сразу начинавшие играть значительную роль в общественной и
культурной жизни города. Эти преподаватели сыграли большую роль в формировании предпосылок для создания в городе высших учебных заведений.
Авторы останавливаются на деятельности некоторых преподавателей математики, которые не только добросовестно выполняли служебные обязанности, но и все свои знания,
энтузиазм и творческие возможности отдавали делу просвещения, организуя научные кружки, читая публичные лекции, совершенствуясь в методике преподавания своих
предметов. Достойными преемниками этих преподавателей являются специалисты советского времени, которых можно назвать создателями современных вузов города.
Оренбургский государственный университет, возникший на базе политехнического института --– сегодня крупнейший вуз области. Богатую историю имеет и
Оренбургский педагогический университет, который в этом году отмечает свое 105-летие. В докладе представлен краткий обзор истории этих высших учебных заведений Оренбурга
и деятельности преподавателей математики, на протяжении многих лет в них трудившихся.
\end{talk}
\end{document}