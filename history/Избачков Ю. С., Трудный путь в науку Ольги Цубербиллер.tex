\documentclass[12pt]{article}
\usepackage{hyphsubst}
\usepackage[T2A]{fontenc}
\usepackage[english,main=russian]{babel}
\usepackage[utf8]{inputenc}
\usepackage[letterpaper,top=2cm,bottom=2cm,left=2cm,right=2cm,marginparwidth=2cm]{geometry}
\usepackage{float}
\usepackage{mathtools, commath, amssymb, amsthm}
\usepackage{enumitem, tabularx,graphicx,url,xcolor,rotating,multicol,epsfig,colortbl,lipsum}

\setlist{topsep=1pt, itemsep=0em}
\setlength{\parindent}{0pt}
\setlength{\parskip}{6pt}

\usepackage{hyphenat}
\hyphenation{ма-те-ма-ти-ка вос-ста-нав-ли-вать}

\usepackage[math]{anttor}

\newenvironment{talk}[6]{%
\vskip 0pt\nopagebreak%
\vskip 0pt\nopagebreak%
\section*{#1}
\phantomsection
\addcontentsline{toc}{section}{#2. \textit{#1}}
% \addtocontents{toc}{\textit{#1}\par}
\textit{#2}\\\nopagebreak%
#3\\\nopagebreak%
\ifthenelse{\equal{#4}{}}{}{\url{#4}\\\nopagebreak}%
\ifthenelse{\equal{#5}{}}{}{Соавторы: #5\\\nopagebreak}%
\ifthenelse{\equal{#6}{}}{}{Секция: #6\\\nopagebreak}%
}

\definecolor{LovelyBrown}{HTML}{FDFCF5}

\usepackage[pdftex,
breaklinks=true,
bookmarksnumbered=true,
linktocpage=true,
linktoc=all
]{hyperref}

\begin{document}
\pagenumbering{gobble}
\pagestyle{plain}
\pagecolor{LovelyBrown}
\begin{talk}
{Трудный путь в науку Ольги Цубербиллер}
{Избачков Юрий Сергеевич}
{Российский научно-исследовательский институт культурного и природного наследия имени Д.\,С. Лихачёва}
{strax5nature@gmail.com}
{Рыбак Кирилл Евгеньевич, доктор культурологии}
{История математики} %

Ольга Николаевна Губонина (в замужестве Цубербиллер) (1885-1975) --- известный математик и педагог, автор задачника ``Задачи и упражнения по аналитической геометрии'', выдержавшего внушительное количество переизданий, в том числе на иностранных языках, в течение нескольких десятилетий была заведующей кафедрой Института тонких химических технологий (Московского государственного университета тонких химических технологий имени М.В.Ломоносова). Кроме того, Ольга Николаевна известна своими дружескими связями с представителями творческой интеллигенции Серебряного века, театральными деятелями, художниками и даже адептами советских эзотерических кружков.
Становление и развитие ученого обуславливаются причинами, которые побудили его выбрать научное поприще. У Цубербиллер путь в науку был гораздо более сложным, нежели его описывали советские биографы.
Ольга Николаевна в советское время по понятным причинам была вынуждена скрывать свое родство с купцами-миллионщиками, дворянское происхождение и связь с деятельностью партии эсеров. В 1906-1911 годах она была вовлечена в работу Междупартийного Красного Креста (структура помощи политическим арестантам и ссыльным).
Ее дедушка Петр Ионович Губонин из крепостных через благотворительность дослужится до чина тайного советника. Ни коим образом не умаляя математические способности Ольги Николаевны, отметим, что учреждённая Петром Ионовичем в 1870-е гг. стипендия очевидно способствовала ее обучению на математическом отделении Московских Высших Женских Курсах (один из ее преподавателей был получателем стипендии еще до ее рождения).
В 1908 году родственники выдали Ольгу Николаевну замуж за товарища прокурора Московской Судебной Палаты Владимира Владимировича Цубербиллера (1866-1910). При этом сын Владимира Владимировича от первого брака --- Владимир был младше Ольги Николаевны всего на семь лет. В 1911 году после смерти Владимира Владимировича его сын Владимир и вдова Ольга Николаевна были арестованы в рамках уголовного дела о распространении нелегальной литературы и участии в запрещенном преступном сообществе. Владимир Цубербиллер взял вину на себя, и очевидно не без усилий бывших коллег мужа, Ольга Николаевна была представлена случайным свидетелем развернутой пасынком революционной работы. По результатам рассмотрения дела суд приговорил Владимира Цубербиллера к тюремному заключению сроком на один год, а его соучастников Эмиля Нордштрема и Николая Витка (в будущем известного советского ученого-урбаниста) на 8 месяцев. Полагаем, такое серьезное потрясение заставило Ольгу Николаевну пересмотреть свое участие в революционной деятельности и сконцентрироваться на научной работе, к которой у нее были очевидные способности.
Принимала участие в работе Московского математического общества.
\end{talk}
\end{document}