\documentclass[12pt]{article}
\usepackage{hyphsubst}
\usepackage[T2A]{fontenc}
\usepackage[english,main=russian]{babel}
\usepackage[utf8]{inputenc}
\usepackage[letterpaper,top=2cm,bottom=2cm,left=2cm,right=2cm,marginparwidth=2cm]{geometry}
\usepackage{float}
\usepackage{mathtools, commath, amssymb, amsthm}
\usepackage{enumitem, tabularx,graphicx,url,xcolor,rotating,multicol,epsfig,colortbl,lipsum}

\setlist{topsep=1pt, itemsep=0em}
\setlength{\parindent}{0pt}
\setlength{\parskip}{6pt}

\usepackage{hyphenat}
\hyphenation{ма-те-ма-ти-ка вос-ста-нав-ли-вать}

\usepackage[math]{anttor}

\newenvironment{talk}[6]{%
\vskip 0pt\nopagebreak%
\vskip 0pt\nopagebreak%
\section*{#1}
\phantomsection
\addcontentsline{toc}{section}{#2. \textit{#1}}
% \addtocontents{toc}{\textit{#1}\par}
\textit{#2}\\\nopagebreak%
#3\\\nopagebreak%
\ifthenelse{\equal{#4}{}}{}{\url{#4}\\\nopagebreak}%
\ifthenelse{\equal{#5}{}}{}{Соавторы: #5\\\nopagebreak}%
\ifthenelse{\equal{#6}{}}{}{Секция: #6\\\nopagebreak}%
}

\definecolor{LovelyBrown}{HTML}{FDFCF5}

\usepackage[pdftex,
breaklinks=true,
bookmarksnumbered=true,
linktocpage=true,
linktoc=all
]{hyperref}

\begin{document}
\pagenumbering{gobble}
\pagestyle{plain}
\pagecolor{LovelyBrown}
\begin{talk}
{Д.\,Д. Мордухай-Болтовской и его связь с Петербургским университетом}
{Налбандян Юлия Сергеевна}
{Южный федеральный университет, Институт математики, механики и компьютерных наук имени И.\,И. Воровича}
{ysnalbandyan@sfedu.ru}
{}
{История математики} %

Дмитрий Дмитриевич Мордухай-Болтовской, по праву считающийся одним из основателей ростовской математической школы, учился в Санкт-Петербургском университете в 1894--1898 гг. По рекомендации К.А. Поссе и А.А. Маркова по окончании университета он был направлен в Варшаву, в Варшавский политехнический институт, ``штатным преподавателем с функциями ассистента'' при Г.Ф. Вороном. Впоследствии Мордухай-Болтовской сдавал в Санкт-Петербурге магистерские экзамены и в 1906 г. защитил магистерскую диссертацию.  Став профессором Варшавского Императорского университета, переехав вместе с университетом в Ростов-на-Дону, он всегда тепло вспоминал своих учителей, которые, по его образному выражению, ``жили под солнцем Чебышёва''. К этой школе на правах внука он причислял и себя.

В докладе предполагается остановиться на студенческих годах выдающегося математика, проанализировать трудности, с которыми он столкнулся в первые годы своей работы (например, упор на практические занятия в политехническом институте), а также рассмотреть его взаимоотношения с такими учёными как А.\,А. Марков, К.\,А. Поссе, Г.\,Ф. Вороной, И.\,Л. Пташицкий. Используются материалы из опубликованных статей, архивные документы и письма Д.\,Д. Мордухай-Болтовского, адресованные сыну.
\end{talk}
\end{document}