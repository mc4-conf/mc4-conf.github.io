\documentclass[12pt]{article}
\usepackage{hyphsubst}
\usepackage[T2A]{fontenc}
\usepackage[english,main=russian]{babel}
\usepackage[utf8]{inputenc}
\usepackage[letterpaper,top=2cm,bottom=2cm,left=2cm,right=2cm,marginparwidth=2cm]{geometry}
\usepackage{float}
\usepackage{mathtools, commath, amssymb, amsthm}
\usepackage{enumitem, tabularx,graphicx,url,xcolor,rotating,multicol,epsfig,colortbl,lipsum}

\setlist{topsep=1pt, itemsep=0em}
\setlength{\parindent}{0pt}
\setlength{\parskip}{6pt}

\usepackage{hyphenat}
\hyphenation{ма-те-ма-ти-ка вос-ста-нав-ли-вать}

\usepackage[math]{anttor}

\newenvironment{talk}[6]{%
\vskip 0pt\nopagebreak%
\vskip 0pt\nopagebreak%
\section*{#1}
\phantomsection
\addcontentsline{toc}{section}{#2. \textit{#1}}
% \addtocontents{toc}{\textit{#1}\par}
\textit{#2}\\\nopagebreak%
#3\\\nopagebreak%
\ifthenelse{\equal{#4}{}}{}{\url{#4}\\\nopagebreak}%
\ifthenelse{\equal{#5}{}}{}{Соавторы: #5\\\nopagebreak}%
\ifthenelse{\equal{#6}{}}{}{Секция: #6\\\nopagebreak}%
}

\definecolor{LovelyBrown}{HTML}{FDFCF5}

\usepackage[pdftex,
breaklinks=true,
bookmarksnumbered=true,
linktocpage=true,
linktoc=all
]{hyperref}

\begin{document}
\pagenumbering{gobble}
\pagestyle{plain}
\pagecolor{LovelyBrown}
\begin{talk}
{Российской науке -- 300 лет. Петербургская школа прикладной
математики и кораблестроения}
{Павилайнен Галина Вольдемаровна и Поляхова Елена Николаевна}
{Санкт-Петербургский государственный университет}
{g.pavilaynen@spbu.ru}
{}
{История математики} %

В докладе делается обзор развития прикладной математики и кораблестроения в юбилейный год Российской Академии наук, Академического университета и Академической гимназии, когда необходимо вспомнить и оценить заново достижения и открытия многих поколений учёных. В предисловии ко второму изданию книги ``Очерки истории отечественного кораблестроения'', опубликованной под редакцией Г.\,В. Павилайнен в издательстве ``ВВМ'' в 2023 году, академик Н.\,Ф. Морозов пишет: ``Россия – морская держава, и достижения отечественного кораблестроения, великие имена Л. Эйлера, А.Н. Крылова, И.Г. Бубнова, С.О. Макарова, В.В. Новожилова наши отечественные приоритеты, научные прорывы в различных областях гидродинамики, теории корабля, механике разрушения, создании новых конструкционных материалов, обязательно должны становиться предметом изучения историков науки и публиковаться в научных и научно-популярных журналах, в монографиях и учебниках по истории''.
Огромно влияние Академии наук с самого её основания в научном кораблестроении. Завершение создания русского регулярного военного флота и учреждение в России Академии наук совпали по времени. Для этого были как объективные, так и субъективные предпосылки. Среди субъективных --– реформаторская роль Петра I и появление на научном небосклоне звезды Леонарда Эйлера, научные труды которого до сих пор не исследованы до конца и каждый раз изумляют исследователей гениальными научными предвидениями.
Основателем отечественной Академии наук (была открыта в 1724 г.) явился Пётр I, который уже с 1717 г. состоял членом Парижской Академии наук. Опыт стран Западной Европы подсказывал, что отсутствие масштабного мореплавания вело к отставанию России в развитии астрономии и механики. Россия не имела не только флота, но и даже выходов к Балтийскому и Чёрному морям. К концу правления Петра I страна превратилась в крупную морскую державу. Один из секретов такого превращения состоит в научном подходе к созданию флота. Крупнейший вклад в кораблестроение внёс Л. Эйлер (1707—1783). Академик А.\,Н. Крылов писал: ``В нашей Академии наук зародилась теория корабля в виде двухтомного сочинения Л. Эйлера''. Трактат о ``Морской науке'', написанный по заказу Академии наук, вышел в Петербурге на латинском языке в 1749 году. Много выдающихся флотоводцев снискали славу в России и обеспечили ее морское превосходство во все времена.
\end{talk}
\end{document}