\documentclass[12pt]{article}
\usepackage{hyphsubst}
\usepackage[T2A]{fontenc}
\usepackage[english,main=russian]{babel}
\usepackage[utf8]{inputenc}
\usepackage[letterpaper,top=2cm,bottom=2cm,left=2cm,right=2cm,marginparwidth=2cm]{geometry}
\usepackage{float}
\usepackage{mathtools, commath, amssymb, amsthm}
\usepackage{enumitem, tabularx,graphicx,url,xcolor,rotating,multicol,epsfig,colortbl,lipsum}

\setlist{topsep=1pt, itemsep=0em}
\setlength{\parindent}{0pt}
\setlength{\parskip}{6pt}

\usepackage{hyphenat}
\hyphenation{ма-те-ма-ти-ка вос-ста-нав-ли-вать}

\usepackage[math]{anttor}

\newenvironment{talk}[6]{%
\vskip 0pt\nopagebreak%
\vskip 0pt\nopagebreak%
\section*{#1}
\phantomsection
\addcontentsline{toc}{section}{#2. \textit{#1}}
% \addtocontents{toc}{\textit{#1}\par}
\textit{#2}\\\nopagebreak%
#3\\\nopagebreak%
\ifthenelse{\equal{#4}{}}{}{\url{#4}\\\nopagebreak}%
\ifthenelse{\equal{#5}{}}{}{Соавторы: #5\\\nopagebreak}%
\ifthenelse{\equal{#6}{}}{}{Секция: #6\\\nopagebreak}%
}

\definecolor{LovelyBrown}{HTML}{FDFCF5}

\usepackage[pdftex,
breaklinks=true,
bookmarksnumbered=true,
linktocpage=true,
linktoc=all
]{hyperref}

\begin{document}
\pagenumbering{gobble}
\pagestyle{plain}
\pagecolor{LovelyBrown}
\begin{talk}
{Семён Ефимович Белозёров: к 120-летию со дня рождения}
{Пырков Вячеслав Евгеньевич}
{Южный федеральный университет}
{vepyrkov@sfedu.ru}
{}
{История математики} %

В докладе будут освещены уточненные и ранее неизвестные сведения о творческом пути и научном наследии С.Е. Белозёрова --- историка математики, ректора Ростовского университета (1938--1954). Эти сведения касаются его деятельности по созданию  кафедры Истории физико-математических наук и по руководству этой кафедрой, постановки курса ``История математики'', а также создания историко-математической школы в Ростовском университете.

\medskip

\begin{enumerate}
\item[{[1]}] С.Е. Белозёров, {\it Математика в Ростовском университете}, Исто\-ри\-ко-ма\-те\-ма\-ти\-че\-ские исследования.  Вып.VI (1953), 247-352.
\item[{[2]}] С.Е. Белозёров, {\it Первые шаги в исследовательской работе по истории наук}, Ученые записки Ростовского-н/Д государственного университета им. В.М. Молотова.  Т.XXIV. Труды кафедры истории наук. Вып.1 (1955), 213-214.
\end{enumerate}
\end{talk}
\end{document}