\documentclass[12pt]{article}
\usepackage{hyphsubst}
\usepackage[T2A]{fontenc}
\usepackage[english,main=russian]{babel}
\usepackage[utf8]{inputenc}
\usepackage[letterpaper,top=2cm,bottom=2cm,left=2cm,right=2cm,marginparwidth=2cm]{geometry}
\usepackage{float}
\usepackage{mathtools, commath, amssymb, amsthm}
\usepackage{enumitem, tabularx,graphicx,url,xcolor,rotating,multicol,epsfig,colortbl,lipsum}

\setlist{topsep=1pt, itemsep=0em}
\setlength{\parindent}{0pt}
\setlength{\parskip}{6pt}

\usepackage{hyphenat}
\hyphenation{ма-те-ма-ти-ка вос-ста-нав-ли-вать}

\usepackage[math]{anttor}

\newenvironment{talk}[6]{%
\vskip 0pt\nopagebreak%
\vskip 0pt\nopagebreak%
\section*{#1}
\phantomsection
\addcontentsline{toc}{section}{#2. \textit{#1}}
% \addtocontents{toc}{\textit{#1}\par}
\textit{#2}\\\nopagebreak%
#3\\\nopagebreak%
\ifthenelse{\equal{#4}{}}{}{\url{#4}\\\nopagebreak}%
\ifthenelse{\equal{#5}{}}{}{Соавторы: #5\\\nopagebreak}%
\ifthenelse{\equal{#6}{}}{}{Секция: #6\\\nopagebreak}%
}

\definecolor{LovelyBrown}{HTML}{FDFCF5}

\usepackage[pdftex,
breaklinks=true,
bookmarksnumbered=true,
linktocpage=true,
linktoc=all
]{hyperref}

\begin{document}
\pagenumbering{gobble}
\pagestyle{plain}
\pagecolor{LovelyBrown}
\begin{talk}
{Обсуждение аналогов теоремы Дезарга}
{Селиверстов Александр Владиславович}
{Институт проблем передачи информации им. А.\,А. Харкевича РАН}
{slvstv@iitp.ru}
{Алексей А. Бойков (РТУ МИРЭА)}
{История математики} %

Целью работы служит иллюстрация изменений в доказательствах теорем с развитием многомерной геометрии. Теорема Дезарга (Girard Desargues, 1591--1661) о перспективных треугольниках переносится на случай перспективных тетраэдров. Эту теорему о тетраэдрах впервые доказал Понселе (Jean-Victor Poncelet, 1788--1867). Теорема может быть доказана как в трёхмерном проективном пространстве, так и вовлекая многомерное проективное пространство. Ранние публикации упоминают только первое доказательство, а второе было найдено позже. По свидетельству Нины Васильевны Наумович, выход в пятимерное пространство использовал в 1913 г. Дмитрий Дмитриевич Мордухай-Болтовской (1876--1952). С другой стороны, в 1899 г. Гильберт (David Hilbert, 1862--1943) показал, что нельзя вывести теорему Дезарга из аксиом проективной плоскости. Астроном Моултон (Forest Ray Moulton, 1872--1952) упростил доказательство в 1902 г. Поэтому доказательство теоремы Дезарга о перспективных треугольниках на плоскости требует выхода в трёхмерное пространство. Начиная с работ Кэли (Arthur Cayley, 1821--1895) и Шлефли (Ludwig Schläfli, 1814--1895) в середине XIX века, многомерная геометрия быстро развивалась. Геометрический смысл алгебраических уравнений от многих переменных был осознан к 1844 г., прежде чем многомерная геометрия стала общепризнанной. Но даже в начале XX века доказательство теоремы о перспективных тетраэдрах, использующее выход в многомерное пространство, не было привлекательным из-за возможности провести доказательство в трёхмерном пространстве. Напротив, в середине XX века доказательство, вовлекающее многомерное пространство, стало восприниматься как естественное обобщение доказательства теоремы Дезарга. Многомерные пространства стали обычными. При этом снизился интерес к основаниям геометрии. Но расширение доступных методов позволяет поддерживать единство математики, чтобы видеть красоту взаимосвязей между разделами.
\end{talk}
\end{document}