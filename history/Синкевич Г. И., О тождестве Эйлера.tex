\documentclass[12pt]{article}
\usepackage{hyphsubst}
\usepackage[T2A]{fontenc}
\usepackage[english,main=russian]{babel}
\usepackage[utf8]{inputenc}
\usepackage[letterpaper,top=2cm,bottom=2cm,left=2cm,right=2cm,marginparwidth=2cm]{geometry}
\usepackage{float}
\usepackage{mathtools, commath, amssymb, amsthm}
\usepackage{enumitem, tabularx,graphicx,url,xcolor,rotating,multicol,epsfig,colortbl,lipsum}

\setlist{topsep=1pt, itemsep=0em}
\setlength{\parindent}{0pt}
\setlength{\parskip}{6pt}

\usepackage{hyphenat}
\hyphenation{ма-те-ма-ти-ка вос-ста-нав-ли-вать}

\usepackage[math]{anttor}

\newenvironment{talk}[6]{%
\vskip 0pt\nopagebreak%
\vskip 0pt\nopagebreak%
\section*{#1}
\phantomsection
\addcontentsline{toc}{section}{#2. \textit{#1}}
% \addtocontents{toc}{\textit{#1}\par}
\textit{#2}\\\nopagebreak%
#3\\\nopagebreak%
\ifthenelse{\equal{#4}{}}{}{\url{#4}\\\nopagebreak}%
\ifthenelse{\equal{#5}{}}{}{Соавторы: #5\\\nopagebreak}%
\ifthenelse{\equal{#6}{}}{}{Секция: #6\\\nopagebreak}%
}

\definecolor{LovelyBrown}{HTML}{FDFCF5}

\usepackage[pdftex,
breaklinks=true,
bookmarksnumbered=true,
linktocpage=true,
linktoc=all
]{hyperref}

\begin{document}
\pagenumbering{gobble}
\pagestyle{plain}
\pagecolor{LovelyBrown}
\begin{talk}
{О тождестве Эйлера}
{Синкевич Галина Ивановна}
{Санкт-Петербургский государственный архитектурно-строительный университет}
{galina.sinkevich@gmail.com}
{}
{История математики} % [6] название секции

Об истории формул и тождества Эйлера много писали, но в ней остается немало противоречий и лакун. Еще когда Эйлер был
ребенком, равенство $\ln(\cos x+\sqrt{-1}\sin x)=x\sqrt{-1}$ было получено в словесной форме при расчете поверхности геоида Р. Коутсом
(R. Cotes) с помощью метода логарифмических пропорций. Некоторые комментаторы полагают, что у Коутса содержалась ошибка. Так
ли это?

В 1743--1748 гг. Эйлер определил показательную функцию через ряды синуса и косинуса и получил уравнение
$\cos\phi+\sqrt{-1}\sin\phi=e^{\sqrt{-1}\phi}$, а также выразил тригонометрические функции через экспоненту.
Но у Эйлера нет знаменитого тождества $e^{i\pi}=-1$,
или $ e^{i\pi}+1=0$. Нам удалось выяснить, когда и у кого оно появилось впервые. Мы покажем, как постепенно менялось мнение об
этой формуле как о незначительном частном случае и до признания ее красивейшей формулой математики.
\medskip
\begin{enumerate}
\item[{[1]}] Синкевич Г.И. История самой красивой формулы математики. Тождество Эйлера // История науки и техники, 2023, {\bf 3}.
-- с. 3--25.
\end{enumerate}
\end{talk}
\end{document}