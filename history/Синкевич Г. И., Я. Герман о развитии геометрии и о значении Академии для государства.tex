\documentclass[12pt]{article}
\usepackage{hyphsubst}
\usepackage[T2A]{fontenc}
\usepackage[english,main=russian]{babel}
\usepackage[utf8]{inputenc}
\usepackage[letterpaper,top=2cm,bottom=2cm,left=2cm,right=2cm,marginparwidth=2cm]{geometry}
\usepackage{float}
\usepackage{mathtools, commath, amssymb, amsthm}
\usepackage{enumitem, tabularx,graphicx,url,xcolor,rotating,multicol,epsfig,colortbl,lipsum}

\setlist{topsep=1pt, itemsep=0em}
\setlength{\parindent}{0pt}
\setlength{\parskip}{6pt}

\usepackage{hyphenat}
\hyphenation{ма-те-ма-ти-ка вос-ста-нав-ли-вать}

\usepackage[math]{anttor}

\newenvironment{talk}[6]{%
\vskip 0pt\nopagebreak%
\vskip 0pt\nopagebreak%
\section*{#1}
\phantomsection
\addcontentsline{toc}{section}{#2. \textit{#1}}
% \addtocontents{toc}{\textit{#1}\par}
\textit{#2}\\\nopagebreak%
#3\\\nopagebreak%
\ifthenelse{\equal{#4}{}}{}{\url{#4}\\\nopagebreak}%
\ifthenelse{\equal{#5}{}}{}{Соавторы: #5\\\nopagebreak}%
\ifthenelse{\equal{#6}{}}{}{Секция: #6\\\nopagebreak}%
}

\definecolor{LovelyBrown}{HTML}{FDFCF5}

\usepackage[pdftex,
breaklinks=true,
bookmarksnumbered=true,
linktocpage=true,
linktoc=all
]{hyperref}

\begin{document}
\pagenumbering{gobble}
\pagestyle{plain}
\pagecolor{LovelyBrown}
\begin{talk}
{Якоб Герман о развитии геометрии и о значении Академии для государства. Речь на открытом заседании Петербургской Академии наук, 1726 г.}
{Синкевич Галина Ивановна}
{Санкт-Петербургский государственный архитектурно-строительный университет}
{galina.sinkevich@gmail.com}
{}
{История математики} %

Якоб Герман (1678–1733), швейцарский математик и механик, любимый ученик Якоба Бернулли, родственник Леонарда Эйлера, был первым профессором высшей математики Петербургской академии наук. Ему, известному математику, самому старшему из академиков, светскому и разносторонне образованному человеку, надлежало выступать на открытых заседаниях в присутствии императрицы. Он прибыл в Петербург в 1725 г. и выступал на первом (1725) и втором (1726) открытых заседаниях Академии. После смерти Петра I перспективы существования Академии стали неясны. Речь Германа можно смело назвать программной, или, как мы сейчас бы сказали, пленарной, так как она, по словам автора, раскрывает не только ``начало, развитие и торжество математики'', но и дает современную к тому моменту картину состояния математики, определяет задачи для молодых академиков по развитию математики и механики и обучению студентов. Главной же  целью Германа было внушить сильным мира сего мысль о необходимости сохранения Академии для государства. Речь 1726 г. ``О происхождении и развитии геометрии'' была опубликована на латыни в 1728 г. и без малого 300 лет оставалась непереведенной.  Русский перевод  будет звучать впервые.

\medskip

\begin{enumerate}
\item[{[1]}]  J. Hermann. De ortu et progressu Geometriae. Sermones in secundo solenni Academiae Scientiarum imperialis conventu die 1 augusti anni MDCCXXVI publice recitati. Petropoli, 1728. - 125 p.
\end{enumerate}
\end{talk}
\end{document}