\documentclass[12pt]{article}
\usepackage{hyphsubst}
\usepackage[T2A]{fontenc}
\usepackage[english,main=russian]{babel}
\usepackage[utf8]{inputenc}
\usepackage[letterpaper,top=2cm,bottom=2cm,left=2cm,right=2cm,marginparwidth=2cm]{geometry}
\usepackage{float}
\usepackage{mathtools, commath, amssymb, amsthm}
\usepackage{enumitem, tabularx,graphicx,url,xcolor,rotating,multicol,epsfig,colortbl,lipsum}

\setlist{topsep=1pt, itemsep=0em}
\setlength{\parindent}{0pt}
\setlength{\parskip}{6pt}

\usepackage{hyphenat}
\hyphenation{ма-те-ма-ти-ка вос-ста-нав-ли-вать}

\usepackage[math]{anttor}

\newenvironment{talk}[6]{%
\vskip 0pt\nopagebreak%
\vskip 0pt\nopagebreak%
\section*{#1}
\phantomsection
\addcontentsline{toc}{section}{#2. \textit{#1}}
% \addtocontents{toc}{\textit{#1}\par}
\textit{#2}\\\nopagebreak%
#3\\\nopagebreak%
\ifthenelse{\equal{#4}{}}{}{\url{#4}\\\nopagebreak}%
\ifthenelse{\equal{#5}{}}{}{Соавторы: #5\\\nopagebreak}%
\ifthenelse{\equal{#6}{}}{}{Секция: #6\\\nopagebreak}%
}

\definecolor{LovelyBrown}{HTML}{FDFCF5}

\usepackage[pdftex,
breaklinks=true,
bookmarksnumbered=true,
linktocpage=true,
linktoc=all
]{hyperref}

\begin{document}
\pagenumbering{gobble}
\pagestyle{plain}
\pagecolor{LovelyBrown}
\begin{talk}
{Элементы теории поля в работах академика О.\,И. Сомова}
{Юлина Анна Олеговна}
{Санкт-Петербургский государственный архитектурно-строительный университет}
{parfenova19761976@mail.ru}
{}
{История математики} %

Петербургский математик и механик О.\,И. Сомов первый в России использовал аппарат векторного исчисления в  курсе теоретической механики.  Ему принадлежит
введение математического понятия градиента, годографа, векторного произведения, линии и поверхности уровня, потенциала и их геометрического и векторного смысла. Все вопросы механики Сомов рассматривает в тесной взаимосвязи с математикой. В его фундаментальных работах блестяще показано как математический анализ помогает раскрывать законы движения и действия сил природы с одной стороны, а с другой как механика помогает развитию аналитических и геометрических методов исследования.
Однако же работы академика Сомова незаслуженно забыты. Постараемся восполнить этот пробел в данном докладе.

\medskip

\begin{enumerate}
\item[{[1]}] Сомов О.И. Рациональная механика. Кинематика. С-Петербург. Типография Императорской Академии Наук. 1872г.- 491 с.
\item[{[2]}] Сомов О.И. О решении одного вопроса механики, предложенного Абелем // Записки Императорской Академии наук. Санкт-Петербург: Типография Императорской Академии Наук. Т.9, кн.1. т. 1866г. Раздельная пагинация.
491 с.
\item[{[3]}] А. О. Юлина.  Векторное исчисление в механике Сомова. // История науки и техники. 2023. №3. с. 26-33.
\end{enumerate}
\end{talk}
\end{document}