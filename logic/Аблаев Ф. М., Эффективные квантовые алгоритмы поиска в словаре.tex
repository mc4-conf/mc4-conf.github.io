\documentclass[12pt]{article}
\usepackage{hyphsubst}
\usepackage[T2A]{fontenc}
\usepackage[english,main=russian]{babel}
\usepackage[utf8]{inputenc}
\usepackage[letterpaper,top=2cm,bottom=2cm,left=2cm,right=2cm,marginparwidth=2cm]{geometry}
\usepackage{float}
\usepackage{mathtools, commath, amssymb, amsthm}
\usepackage{enumitem, tabularx,graphicx,url,xcolor,rotating,multicol,epsfig,colortbl,lipsum}

\setlist{topsep=1pt, itemsep=0em}
\setlength{\parindent}{0pt}
\setlength{\parskip}{6pt}

\usepackage{hyphenat}
\hyphenation{ма-те-ма-ти-ка вос-ста-нав-ли-вать}

\usepackage[math]{anttor}

\newenvironment{talk}[6]{%
\vskip 0pt\nopagebreak%
\vskip 0pt\nopagebreak%
\section*{#1}
\phantomsection
\addcontentsline{toc}{section}{#2. \textit{#1}}
% \addtocontents{toc}{\textit{#1}\par}
\textit{#2}\\\nopagebreak%
#3\\\nopagebreak%
\ifthenelse{\equal{#4}{}}{}{\url{#4}\\\nopagebreak}%
\ifthenelse{\equal{#5}{}}{}{Соавторы: #5\\\nopagebreak}%
\ifthenelse{\equal{#6}{}}{}{Секция: #6\\\nopagebreak}%
}

\definecolor{LovelyBrown}{HTML}{FDFCF5}

\usepackage[pdftex,
breaklinks=true,
bookmarksnumbered=true,
linktocpage=true,
linktoc=all
]{hyperref}

\begin{document}
\pagenumbering{gobble}
\pagestyle{plain}
\pagecolor{LovelyBrown}
\begin{talk}
{Эффективные квантовые алгоритмы поиска в словаре}
{Аблаев Фарид Мансурович}
{Казанский федеральный университет}
{farid.ablayev@gmail.com}
{Марат Аблаев, Наиля Салехова}
{Математическая логика и теоретическая информатика}

В  последние десятилетия  развиты различные подходы к решению проблемы поиска в словаре.  Различные варианты классических  алгоритмов требуют  линейного  (от объема словаря) числа запросов [1].

Квантовые алгоритмы квадратично ускоряют процесс запросов. Квантовые алгоритмы используют метод квантового амплитудного усиления, лежащий в основе известного алгоритма Гровера, который квадратично ускоряет процесс поиска.  Информация, достаточная для знакомством с этими алгоритмами, приводится, например, в [2].

Мы  описываем  наш  классически-квантовый  алгоритм поиска [2] и представляем новый подход к поиску элемента в словаре, основанный на квантовом хешировании (quantum hashing) и квантовой технике отпечатков  (quantum fingerprinting). Мы предлагаем ``чистый'' квантовый алгоритм, который, по сути, является дальнейшим  ``шагом'' по сравнению с нашим недавним гибридным классически-квантовым алгоритмом поиска в словаре [2].

\medskip

\begin{enumerate}
\item[{[1]}]Knuth, D.E.; Morris, J.H., Jr.; Pratt, V.R. {\it Fast pattern matching in strings}, SIAM J. Comput. 1977, 6, 323–350.
\item[{[2]}] F. Ablayev, M. Ablayev, N. Salekhova {\it Hybrid Classical–Quantum Text Search Based on Hashing},  Mathematics 2024, 12(12), 1858; https://doi.org/10.3390/math12121858.
\end{enumerate}
\end{talk}
\end{document}