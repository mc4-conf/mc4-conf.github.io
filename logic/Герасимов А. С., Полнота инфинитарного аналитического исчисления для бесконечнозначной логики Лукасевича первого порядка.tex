\documentclass[12pt]{article}
\usepackage{hyphsubst}
\usepackage[T2A]{fontenc}
\usepackage[english,main=russian]{babel}
\usepackage[utf8]{inputenc}
\usepackage[letterpaper,top=2cm,bottom=2cm,left=2cm,right=2cm,marginparwidth=2cm]{geometry}
\usepackage{float}
\usepackage{mathtools, commath, amssymb, amsthm}
\usepackage{enumitem, tabularx,graphicx,url,xcolor,rotating,multicol,epsfig,colortbl,lipsum}

\setlist{topsep=1pt, itemsep=0em}
\setlength{\parindent}{0pt}
\setlength{\parskip}{6pt}

\usepackage{hyphenat}
\hyphenation{ма-те-ма-ти-ка вос-ста-нав-ли-вать}

\usepackage[math]{anttor}

\newenvironment{talk}[6]{%
\vskip 0pt\nopagebreak%
\vskip 0pt\nopagebreak%
\section*{#1}
\phantomsection
\addcontentsline{toc}{section}{#2. \textit{#1}}
% \addtocontents{toc}{\textit{#1}\par}
\textit{#2}\\\nopagebreak%
#3\\\nopagebreak%
\ifthenelse{\equal{#4}{}}{}{\url{#4}\\\nopagebreak}%
\ifthenelse{\equal{#5}{}}{}{Соавторы: #5\\\nopagebreak}%
\ifthenelse{\equal{#6}{}}{}{Секция: #6\\\nopagebreak}%
}

\definecolor{LovelyBrown}{HTML}{FDFCF5}

\usepackage[pdftex,
breaklinks=true,
bookmarksnumbered=true,
linktocpage=true,
linktoc=all
]{hyperref}

\begin{document}
\pagenumbering{gobble}
\pagestyle{plain}
\pagecolor{LovelyBrown}
\begin{talk}
{Полнота инфинитарного аналитического исчисления для бесконечнозначной логики Лукасевича первого порядка}
{Герасимов Александр Сергеевич}
{Санкт-Петербургский политехнический университет Петра Великого}
{alexander.s.gerasimov@ya.ru}
{}
{Математическая логика и теоретическая информатика} %

Бесконечнозначная логика Лукасевича первого порядка \(\textnormal{\L}\forall\)
является одной из важнейших математических нечетких логик и
служит для формализации приближенных рассуждений.
Множество всех общезначимых \(\textnormal{\L}\forall\)-предложений
(достаточно богатой сигнатуры) неперечислимо.
В [1] М.~Баац и Дж.~Меткалф предложили для логики \(\textnormal{\L}\forall\)
инфинитарное аналитическое гиперсеквенциальное исчисление,
но привели существенно неверное доказательство его полноты,
что нам подтвердил Дж.~Меткалф.
В [3] мы доказали полноту этого исчисления для предваренных
\(\textnormal{\L}\forall\)-предложений.
Теперь, опираясь также на вспомогательное исчисление из [2] и
на результаты из [4] по сравнению исчислений для \(\textnormal{\L}\forall\),
мы устанавливаем полноту указанного инфинитарного исчисления
(для произвольных \(\textnormal{\L}\forall\)-предложений).

\medskip

\begin{enumerate}
\item[{[1]}]
M.~Baaz, G.~Metcalfe,
{\it Herbrand's theorem, skolemization and proof systems
for first-order {\L}ukasiewicz logic},
Journal of Logic and Computation, Vol.~20, No.~1 (2010), 35--54.
\item[{[2]}]
А.~С.~Герасимов,
{\it Бесконечнозначная логика Лукасевича первого порядка:
гиперсеквенциальные исчисления без структурных правил
и поиск вывода предваренных предложений},
Математические труды, т.~20, N~2 (2017), 3--34.
\item[{[3]}]
A.~S.~Gerasimov,
{\it Repetition-free and infinitary analytic calculi for
first-order rational Pavelka logic},
Siberian Electronic Mathematical Reports, Vol.~17 (2020), 1869--1899.
\item[{[4]}]
A.~S.~Gerasimov,
{\it Comparing calculi for first-order infinite-valued {\L}ukasiewicz logic and
first-order rational Pavelka logic},
Logic and Logical Philosophy, Vol.~32, No.~2 (2022), 269--318.
\end{enumerate}
\end{talk}
\end{document}