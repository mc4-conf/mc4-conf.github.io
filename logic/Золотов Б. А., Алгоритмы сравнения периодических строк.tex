\documentclass[12pt]{article}
\usepackage{hyphsubst}
\usepackage[T2A]{fontenc}
\usepackage[english,main=russian]{babel}
\usepackage[utf8]{inputenc}
\usepackage[letterpaper,top=2cm,bottom=2cm,left=2cm,right=2cm,marginparwidth=2cm]{geometry}
\usepackage{float}
\usepackage{mathtools, commath, amssymb, amsthm}
\usepackage{enumitem, tabularx,graphicx,url,xcolor,rotating,multicol,epsfig,colortbl,lipsum}

\setlist{topsep=1pt, itemsep=0em}
\setlength{\parindent}{0pt}
\setlength{\parskip}{6pt}

\usepackage{hyphenat}
\hyphenation{ма-те-ма-ти-ка вос-ста-нав-ли-вать}

\usepackage[math]{anttor}

\newenvironment{talk}[6]{%
\vskip 0pt\nopagebreak%
\vskip 0pt\nopagebreak%
\section*{#1}
\phantomsection
\addcontentsline{toc}{section}{#2. \textit{#1}}
% \addtocontents{toc}{\textit{#1}\par}
\textit{#2}\\\nopagebreak%
#3\\\nopagebreak%
\ifthenelse{\equal{#4}{}}{}{\url{#4}\\\nopagebreak}%
\ifthenelse{\equal{#5}{}}{}{Соавторы: #5\\\nopagebreak}%
\ifthenelse{\equal{#6}{}}{}{Секция: #6\\\nopagebreak}%
}

\definecolor{LovelyBrown}{HTML}{FDFCF5}

\usepackage[pdftex,
breaklinks=true,
bookmarksnumbered=true,
linktocpage=true,
linktoc=all
]{hyperref}

\begin{document}
\pagenumbering{gobble}
\pagestyle{plain}
\pagecolor{LovelyBrown}
\begin{talk}
{Алгоритмы сравнения периодических строк}
{Золотов Борис Алексеевич}
{Санкт-Петербургский государственный университет, МЦМУ им. Леонарда Эйлера}
{boris.a.zolotov@yandex.com}
{Гаевой Никита Сергеевич,
Тискин Александр Владимирович}
{Математическая логика и теоретическая информатика} %

Задача поиска наибольшей общей подпоследовательности {\it (LCS)} формулируется
следующим образом: по паре входных строк \(a\), \(b\) определить длину
наибольшей строки, которая являлась бы подпоследовательностью обеих этих строк.
Это важнейшая алгоритмическая задача, имеющая множество применений.
А.\,В.~Тискиным была предложена алгебраическая интерпретация для этой задачи
и её аналогов (задачи о редакционном расстоянии, о взвешенном выравнивании
и других). Основные объект этой интерпретации --- моноид Гекке
и представление его элементов в виде липких кос.
Алгебраическая интерпретация неожиданно эффективна для решения задачи LCS
в случае, когда одна или обе входные строки имеют период (представляются как
конкатенация нескольких копий одной строки).
Случай, когда только одна из двух входных строк периодична, был решен ранее
при помощи обобщения указанной интерпретации на аффинный моноид Гекке.
Центральный результат настоящего доклада --- алгоритмически эффективное сведение
умножения в аффинном моноиде Гекке к умножению в стандартном моноиде Гекке.
С его помощью нами разработан эффективный алгоритм для более общего случая,
когда обе строки периодичны.
Предлагаемый алгоритм был реализован Н.\,С.~Гаевым,
полученная программа способна находить длину наибольшей общей
подпоследовательности для пар периодических строк, размеры которых
не позволили бы обработать их известными ранее алгоритмами.

\medskip

\begin{enumerate}
\item[{[1]}]
D.~E. Knuth,
{\it The Art of Computer Programming: Sorting and Searching, Volume~3.}
Addison Wesley, 1998.
\item[{[2]}]
Joel~Brewster Lewis,
{\it Affine symmetric group,}
WikiJournal of Science, 4(1):3, 2021.
\item[{[3]}]
A.~Tiskin,
{\it Semi-local string comparison: Algorithmic techniques and applications,}
\linebreak
Mathematics in Computer Science, 1(4):571--603, 2008.
\item[{[4]}]
A.~Tiskin,
{\it Periodic String Comparison,}
In Proceedings of CPM, volume 5577 of
Lecture Notes in Computer Science, pages 193--206, 2009.
\item[{[5]}]
A.~Tiskin,
{\it Fast Distance Multiplication of Unit-Monge Matrices,}
Algorithmica, 71:859--888, 2015.
\end{enumerate}
\end{talk}
\end{document}