\documentclass[12pt]{article}
\usepackage{hyphsubst}
\usepackage[T2A]{fontenc}
\usepackage[english,main=russian]{babel}
\usepackage[utf8]{inputenc}
\usepackage[letterpaper,top=2cm,bottom=2cm,left=2cm,right=2cm,marginparwidth=2cm]{geometry}
\usepackage{float}
\usepackage{mathtools, commath, amssymb, amsthm}
\usepackage{enumitem, tabularx,graphicx,url,xcolor,rotating,multicol,epsfig,colortbl,lipsum}

\setlist{topsep=1pt, itemsep=0em}
\setlength{\parindent}{0pt}
\setlength{\parskip}{6pt}

\usepackage{hyphenat}
\hyphenation{ма-те-ма-ти-ка вос-ста-нав-ли-вать}

\usepackage[math]{anttor}

\newenvironment{talk}[6]{%
\vskip 0pt\nopagebreak%
\vskip 0pt\nopagebreak%
\section*{#1}
\phantomsection
\addcontentsline{toc}{section}{#2. \textit{#1}}
% \addtocontents{toc}{\textit{#1}\par}
\textit{#2}\\\nopagebreak%
#3\\\nopagebreak%
\ifthenelse{\equal{#4}{}}{}{\url{#4}\\\nopagebreak}%
\ifthenelse{\equal{#5}{}}{}{Соавторы: #5\\\nopagebreak}%
\ifthenelse{\equal{#6}{}}{}{Секция: #6\\\nopagebreak}%
}

\definecolor{LovelyBrown}{HTML}{FDFCF5}

\usepackage[pdftex,
breaklinks=true,
bookmarksnumbered=true,
linktocpage=true,
linktoc=all
]{hyperref}

\begin{document}
\pagenumbering{gobble}
\pagestyle{plain}
\pagecolor{LovelyBrown}
\begin{talk}
{Аксиоматизация арифметики Бюхи}
{Ковалёв Константин Андреевич}
{МФТИ}
{kovalev.ka@phystech.edu}
{}
{Математическая логика и теоретическая информатика} %

Арифметикой Бюхи (с основанием \(p \ge 2\)) называется теория натуральных чисел в языке с нулем, функцией последователя, сложением и специальной функцией, обозначаемой \(V_p\), которая число \(n\) отображает в наибольшую степень \(p\), делящую \(n\). Хорошо известно, что данная теория является разрешимой, однако неизвестно никакой явной аксиоматизации этой теории (есть некоторые отрицательные результаты, см. [1]). Цель данной работы состоит в отыскании такой аксиоматизации. Наша аксиоматизация состоит из некоторого конечного числа простых аксиом (похожих на арифметику Робинсона) и схемы аксиом вида \(\exists x \: \phi(x) \to \exists x \le n_\phi \: \phi(x)\), где \(n_\phi\) подбирается так, чтобы данная формула была истинна в стандартной модели. Выбор такого числа \(n_\phi\) опирается на важное свойство арифметики Бюхи, а именно, что любое определимое в рассматриваемом языке множество \(A\) является \(p\)-автоматным (т. е., множество \(p\)-ичных записей элементов \(A\) распознаваемо некоторым конечным автоматом).

\medskip

\begin{enumerate}
\item[{[1]}] Alexander Zapryagaev, {\it Some properties of B\"uchi Arithmetics}, arXiv:2310.16019 [math.LO], 2023.
\end{enumerate}
\end{talk}
\end{document}