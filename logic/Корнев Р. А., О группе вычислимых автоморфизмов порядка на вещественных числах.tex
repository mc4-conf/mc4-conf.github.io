\documentclass[12pt]{article}
\usepackage{hyphsubst}
\usepackage[T2A]{fontenc}
\usepackage[english,main=russian]{babel}
\usepackage[utf8]{inputenc}
\usepackage[letterpaper,top=2cm,bottom=2cm,left=2cm,right=2cm,marginparwidth=2cm]{geometry}
\usepackage{float}
\usepackage{mathtools, commath, amssymb, amsthm}
\usepackage{enumitem, tabularx,graphicx,url,xcolor,rotating,multicol,epsfig,colortbl,lipsum}

\setlist{topsep=1pt, itemsep=0em}
\setlength{\parindent}{0pt}
\setlength{\parskip}{6pt}

\usepackage{hyphenat}
\hyphenation{ма-те-ма-ти-ка вос-ста-нав-ли-вать}

\usepackage[math]{anttor}

\newenvironment{talk}[6]{%
\vskip 0pt\nopagebreak%
\vskip 0pt\nopagebreak%
\section*{#1}
\phantomsection
\addcontentsline{toc}{section}{#2. \textit{#1}}
% \addtocontents{toc}{\textit{#1}\par}
\textit{#2}\\\nopagebreak%
#3\\\nopagebreak%
\ifthenelse{\equal{#4}{}}{}{\url{#4}\\\nopagebreak}%
\ifthenelse{\equal{#5}{}}{}{Соавторы: #5\\\nopagebreak}%
\ifthenelse{\equal{#6}{}}{}{Секция: #6\\\nopagebreak}%
}

\definecolor{LovelyBrown}{HTML}{FDFCF5}

\usepackage[pdftex,
breaklinks=true,
bookmarksnumbered=true,
linktocpage=true,
linktoc=all
]{hyperref}

\begin{document}
\pagenumbering{gobble}
\pagestyle{plain}
\pagecolor{LovelyBrown}
\begin{talk}
{О группе вычислимых автоморфизмов порядка на вещественных числах}
{Корнев Руслан Александрович}
{Новосибирский государственный университет, МЦМУ в Академгородке}
{kornevrus@gmail.com}
{}
{Математическая логика и теоретическая информатика} %

Изучаются алгебраические свойства группы \(\mathrm{Aut}_c(\mathbb{R})\) вычислимых автоморфизмов линейного порядка \((\mathbb{R},\leqslant)\) в сравнении с классической группой всех автоморфизмов этого порядка, а также группой \(\mathrm{Aut}_c(\mathbb{Q})\) вычислимых автоморфизмов порядка на рациональных числах. Обсуждаются привычные для этой тематики вопросы в применении к \(\mathrm{Aut}_c(\mathbb{R})\): показывается, что эта группа не является делимой, содержит в точности три нетривиальные нормальные подгруппы, а также содержит элемент, не сопряжённый со своим квадратом (т.о., критерий Холланда не выполняется в \(\mathrm{Aut}_c(\mathbb{R})\)).

Показывается, что \(\mathrm{Aut}_c(\mathbb{R})\) содержит бамп, не сопряжённый со своим квадратом. Кроме того, любое вычислимо перечислимое вещественное число \(z\) реализуется в качестве верхней границы некоторого бампа из \(\mathrm{Aut}_c(\mathbb{R})\). Эти свойства отличают \(\mathrm{Aut}_c(\mathbb{R})\) от \(\mathrm{Aut}_c(\mathbb{Q})\).

Также обсуждаются определимые в языке теории групп свойства \(\mathrm{Aut}_c(\mathbb{R})\).
\end{talk}
\end{document}