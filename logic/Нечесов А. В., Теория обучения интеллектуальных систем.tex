\documentclass[12pt]{article}
\usepackage{hyphsubst}
\usepackage[T2A]{fontenc}
\usepackage[english,main=russian]{babel}
\usepackage[utf8]{inputenc}
\usepackage[letterpaper,top=2cm,bottom=2cm,left=2cm,right=2cm,marginparwidth=2cm]{geometry}
\usepackage{float}
\usepackage{mathtools, commath, amssymb, amsthm}
\usepackage{enumitem, tabularx,graphicx,url,xcolor,rotating,multicol,epsfig,colortbl,lipsum}

\setlist{topsep=1pt, itemsep=0em}
\setlength{\parindent}{0pt}
\setlength{\parskip}{6pt}

\usepackage{hyphenat}
\hyphenation{ма-те-ма-ти-ка вос-ста-нав-ли-вать}

\usepackage[math]{anttor}

\newenvironment{talk}[6]{%
\vskip 0pt\nopagebreak%
\vskip 0pt\nopagebreak%
\section*{#1}
\phantomsection
\addcontentsline{toc}{section}{#2. \textit{#1}}
% \addtocontents{toc}{\textit{#1}\par}
\textit{#2}\\\nopagebreak%
#3\\\nopagebreak%
\ifthenelse{\equal{#4}{}}{}{\url{#4}\\\nopagebreak}%
\ifthenelse{\equal{#5}{}}{}{Соавторы: #5\\\nopagebreak}%
\ifthenelse{\equal{#6}{}}{}{Секция: #6\\\nopagebreak}%
}

\definecolor{LovelyBrown}{HTML}{FDFCF5}

\usepackage[pdftex,
breaklinks=true,
bookmarksnumbered=true,
linktocpage=true,
linktoc=all
]{hyperref}

\begin{document}
\pagenumbering{gobble}
\pagestyle{plain}
\pagecolor{LovelyBrown}
\begin{talk}
{Теория обучения интеллектуальных систем}
{Нечесов Андрей Витальевич}
{Математический центр в Академгородке: ИМ СО РАН и НГУ}
{nechesov@math.nsc.ru}
{Сергей Гончаров}
{Математическая логика и теоретическая информатика}

В докладе будет рассмотрена теория обучения интеллектуальных систем. Данная тематика взяла свое начало с двух довольно популярных логических направлений: концепции семантического программирования и концепции задачного подхода. В рамка этих двух концепций мы идем от понятия задачи, ее решения и эффективности решения в сторону построения систем искусственного интеллекта, в основе которых лежит логическое мышление. Это удается достичь за счет формализации понятия знания и иерархии знаний. Что позволяет строить динамические деревья знаний и, в конечном счете, эффективно выдавать финальный результат. Данная теория обучения может стать хорошей альтернативой нейронным сетям или усиливать и контролировать их в выдаче правильных ответов.

\medskip

\begin{enumerate}
\item[{[1]}] Goncharov, S.; Nechesov, A. AI-Driven Digital Twins for Smart Cities. Eng. Proc. 2023, 58, 94. \url{https://doi.org/10.3390/ecsa-10-16223}
\item[{[2]}] Vityaev, E.E.; Goncharov, S.S.; Gumirov, V.S.; Mantsivoda, A.V.; Nechesov, A.V.; Sviridenko, D.I. Task approach: on the way to trusting artificial intelligence. WORLD CONGRESS SYSTEMS THEORY, ALGEBRAIC BIOLOGY, ARTIFICIAL INTELLIGENCE: Mathematical Foundations and Applications SELECTED WORKS. 2023. pp. 179-243, \url{https://doi.org/10.18699/sblai2023-41}
\item[{[3]}] Goncharov, S.; Nechesov, A. Semantic programming for AI and Robotics,  SIBIRCON, 2022, pp. 810-815, \url{https://doi.org/10.1109/SIBIRCON56155.2022.10017077}
\end{enumerate}
\end{talk}
\end{document}