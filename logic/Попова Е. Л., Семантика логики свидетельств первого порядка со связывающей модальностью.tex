\documentclass[12pt]{article}
\usepackage{hyphsubst}
\usepackage[T2A]{fontenc}
\usepackage[english,main=russian]{babel}
\usepackage[utf8]{inputenc}
\usepackage[letterpaper,top=2cm,bottom=2cm,left=2cm,right=2cm,marginparwidth=2cm]{geometry}
\usepackage{float}
\usepackage{mathtools, commath, amssymb, amsthm}
\usepackage{enumitem, tabularx,graphicx,url,xcolor,rotating,multicol,epsfig,colortbl,lipsum}

\setlist{topsep=1pt, itemsep=0em}
\setlength{\parindent}{0pt}
\setlength{\parskip}{6pt}

\usepackage{hyphenat}
\hyphenation{ма-те-ма-ти-ка вос-ста-нав-ли-вать}

\usepackage[math]{anttor}

\newenvironment{talk}[6]{%
\vskip 0pt\nopagebreak%
\vskip 0pt\nopagebreak%
\section*{#1}
\phantomsection
\addcontentsline{toc}{section}{#2. \textit{#1}}
% \addtocontents{toc}{\textit{#1}\par}
\textit{#2}\\\nopagebreak%
#3\\\nopagebreak%
\ifthenelse{\equal{#4}{}}{}{\url{#4}\\\nopagebreak}%
\ifthenelse{\equal{#5}{}}{}{Соавторы: #5\\\nopagebreak}%
\ifthenelse{\equal{#6}{}}{}{Секция: #6\\\nopagebreak}%
}

\definecolor{LovelyBrown}{HTML}{FDFCF5}

\usepackage[pdftex,
breaklinks=true,
bookmarksnumbered=true,
linktocpage=true,
linktoc=all
]{hyperref}

\begin{document}
\pagenumbering{gobble}
\pagestyle{plain}
\pagecolor{LovelyBrown}
\begin{talk}
{Семантика логики свидетельств первого порядка со связывающей модальностью}
{Попова Елена Леонидовна}
{НИУ ВШЭ}
{elpop.logics@gmail.com}
{Яворская Татьяна Леонидовна}
{Математическая логика и теоретическая информатика} %


Работа посвящена семантике логики свидетельств первого порядка. Пропозициональные логики свидетельств были введены С. Артемовым в [1]. Они сформулированы в расширении пропозиционального языка формулами вида t:F, где t --- свидетельский терм, F --- формула. Подразумеваемая семантика таких атомов ``t является свидетельством F’’. Для таких логик изучена семантика в стиле Крипке и арифметическая семантика, а также их связь с модальной логикой [2].


Логика свидетельств первого порядка была введена в работе [3]. Свидетельские формулы t:F в этой логике доопределены таким образом, чтобы сделать возможным различие между локальными и глобальными параметрами. А именно, рассматриваются формулы вида \(t:_X\) F, где X – список параметров (т. е. свободных переменных), которые являются глобальными, т. е., открытыми для подстановки.

В работе [3] описана логика доказательств первого порядка FOLP и арифметическая семантика для нее. Семантика в стиле возможных миров Крипке была описана в [4], там же доказаны полнота и корректность.

В нашей статье рассматривается логика доказательств первого порядка \(FOLP^\Box\) в языке, расширенном модальностью Box, которая также допускает связывание параметров. Для этой логики мы описываем модели в стиле Фиттинга, доказываем полноту и корректность относительно этих моделей. В отличие от подхода, выбранного М.Фиттингом, мы описываем модели с терминах означивания свободных переменных, не используя расширение языка с помощью дополнительных констант. Это позволяет придать семантической значение формулам, содержащим свободные переменные. Главными результатами являются полнота и корректность логики \(FOLP^\Box\) в описанной семантике.

\medskip

\begin{enumerate}
\item[{[1]}] S.N. Artemov. Operational modal logic. Technical Report 95-29, Mathematical Sciences Institute, Cornell University, 1995.
\item[{[2]}] S.N. Artemov, M. Fitting, Justification Logic. Reasoning with reasons, Cambridge University Press, 2019.
\item[{[3]}] S.N. Artemov, T.L. Yavorskaya. On first order logic of proofs. Moscow Mathematical Journal, 1:475–490, 2001.
\item[{[4]}] M. Fitting. Possible world semantics for first-order logic of proofs. Annals of Pure and Applied Logic, 165(1):225–240, 2014.
\end{enumerate}
\end{talk}
\end{document}