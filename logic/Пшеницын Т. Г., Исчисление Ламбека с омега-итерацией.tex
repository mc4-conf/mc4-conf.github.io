\documentclass[12pt]{article}
\usepackage{hyphsubst}
\usepackage[T2A]{fontenc}
\usepackage[english,main=russian]{babel}
\usepackage[utf8]{inputenc}
\usepackage[letterpaper,top=2cm,bottom=2cm,left=2cm,right=2cm,marginparwidth=2cm]{geometry}
\usepackage{float}
\usepackage{mathtools, commath, amssymb, amsthm}
\usepackage{enumitem, tabularx,graphicx,url,xcolor,rotating,multicol,epsfig,colortbl,lipsum}

\setlist{topsep=1pt, itemsep=0em}
\setlength{\parindent}{0pt}
\setlength{\parskip}{6pt}

\usepackage{hyphenat}
\hyphenation{ма-те-ма-ти-ка вос-ста-нав-ли-вать}

\usepackage[math]{anttor}

\newenvironment{talk}[6]{%
\vskip 0pt\nopagebreak%
\vskip 0pt\nopagebreak%
\section*{#1}
\phantomsection
\addcontentsline{toc}{section}{#2. \textit{#1}}
% \addtocontents{toc}{\textit{#1}\par}
\textit{#2}\\\nopagebreak%
#3\\\nopagebreak%
\ifthenelse{\equal{#4}{}}{}{\url{#4}\\\nopagebreak}%
\ifthenelse{\equal{#5}{}}{}{Соавторы: #5\\\nopagebreak}%
\ifthenelse{\equal{#6}{}}{}{Секция: #6\\\nopagebreak}%
}

\definecolor{LovelyBrown}{HTML}{FDFCF5}

\usepackage[pdftex,
breaklinks=true,
bookmarksnumbered=true,
linktocpage=true,
linktoc=all
]{hyperref}

\begin{document}
\pagenumbering{gobble}
\pagestyle{plain}
\pagecolor{LovelyBrown}
\begin{talk}
{Исчисление Ламбека с омега-итерацией}
{Пшеницын Тихон Григорьевич}
{Математический институт им. В.\,А. Стеклова Российской академии наук}
{tpshenitsyn@mi-ras.ru}
{}
{Математическая логика и теоретическая информатика} %

Исчисление Ламбека --- субструктурная логика, аксиоматизирующая инэквациональную теорию полугрупп с делениями; она находит применения в лингвистике для моделирования синтаксиса естественных языков. В литературе изучаются расширения исчисления Ламбека различными операциями. Одной из таких операций является итерация Клини, которую можно рассматривать как ``конечную итерацию'': если \(R\) --- это отношение, то \(R^\ast\) --- его рефлексивное транзитивное замыкание, состоящее из \(n\)-кратных композиций \(R\) с собой для \(n \in \mathbb{N}\). Исчисление Ламбека с итерацией Клини и с решеточными операциями называется логикой действий. В статье [1] вводится инфинитарная логика действий и доказывается, что задача выводимости в этом исчислении \(\Pi^0_1\)-полна.

Нами вводится и исследуется расширение исчисления Ламбека с помощью бесконечной итерации, или \(\omega\)-итерации. Данная операция мотивирована теорией формальных языков с бесконечными словами. В рассматриваемой логике секвенции имеют вид \(\Pi \vdash B\), где \(\Pi\) --- либо конечная, либо бесконечная последовательность формул. Вводится правило сечения и доказывается его допустимость. Доказывается полнота исчисления Ламбека с \(\omega\)-итерацией относительно реляционных моделей. Наконец, устанавливается нижняя оценка на сложность задачи выводимости для конечных секвенций в этом исчислении: показывается, что данная задача \(\Pi^1_2\)-трудна. Таким образом, исчисление с бесконечной итерацией оказывается существенно сложнее исчисления с конечной итерацией.

\medskip

\begin{enumerate}
\item[{[1]}] Palka, E. (2007). An infinitary sequent system for the equational theory of *-continuous action lattices. Fundamenta Informaticae 78(2), 295–309.
\end{enumerate}
\end{talk}
\end{document}