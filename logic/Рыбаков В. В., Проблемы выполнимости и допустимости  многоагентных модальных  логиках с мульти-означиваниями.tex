\documentclass[12pt]{article}
\usepackage{hyphsubst}
\usepackage[T2A]{fontenc}
\usepackage[english,main=russian]{babel}
\usepackage[utf8]{inputenc}
\usepackage[letterpaper,top=2cm,bottom=2cm,left=2cm,right=2cm,marginparwidth=2cm]{geometry}
\usepackage{float}
\usepackage{mathtools, commath, amssymb, amsthm}
\usepackage{enumitem, tabularx,graphicx,url,xcolor,rotating,multicol,epsfig,colortbl,lipsum}

\setlist{topsep=1pt, itemsep=0em}
\setlength{\parindent}{0pt}
\setlength{\parskip}{6pt}

\usepackage{hyphenat}
\hyphenation{ма-те-ма-ти-ка вос-ста-нав-ли-вать}

\usepackage[math]{anttor}

\newenvironment{talk}[6]{%
\vskip 0pt\nopagebreak%
\vskip 0pt\nopagebreak%
\section*{#1}
\phantomsection
\addcontentsline{toc}{section}{#2. \textit{#1}}
% \addtocontents{toc}{\textit{#1}\par}
\textit{#2}\\\nopagebreak%
#3\\\nopagebreak%
\ifthenelse{\equal{#4}{}}{}{\url{#4}\\\nopagebreak}%
\ifthenelse{\equal{#5}{}}{}{Соавторы: #5\\\nopagebreak}%
\ifthenelse{\equal{#6}{}}{}{Секция: #6\\\nopagebreak}%
}

\definecolor{LovelyBrown}{HTML}{FDFCF5}

\usepackage[pdftex,
breaklinks=true,
bookmarksnumbered=true,
linktocpage=true,
linktoc=all
]{hyperref}

\begin{document}
\pagenumbering{gobble}
\pagestyle{plain}
\pagecolor{LovelyBrown}
\begin{talk}
{Проблемы выполнимости и допустимости  многоагентных модальных  логиках с мульти-означиваниями}
{Рыбаков Владимир Владимирович}
{Институт Математики и фундаментальной информатики, Красноярск, Сибирский Федеральный Университет}
{Vladimir_Rybakov@mail.ru}
{}
{Математическая логика и теоретическая информатика}

Мы изучаем реляционные Крипке-Хинтикка подобные модели, описывающие исследование информации на корректность и совместимость суждений агентов (о накопленной информации). Базой служат реляционные модели с мультиагентными отношениями достижимости и мульти-означиваниями истинностных значений информации. Соответственно, выбранный логический язык -- это язык мульти-модальной логики модифицированной в указанном направлении. Предложенные правила вычисления истинности скомпонованных формул отличаются от стандартных в отношении того, что модальности могут переключать мульти-означивания. Концепция общепринятого знания, при таком подходе, претерпевает значительные изменения и отличается от трактовки сорта итерации  S5-модальностей,
принятой широка в начале исследования концепции общезначимости.

В техническом отношении, мы общепринято начинаем с проблемы разрешимости проблемы выполнимости
и разрешимости самих предложенных логик. Найдено положительное решение указанных проблем и для введенных логик найдены алгоритмы разрешимости. Более трудная и интригующая часть состоит в попытках расширить такие результаты на алгоритмы распознающие допустимость правил вывода. Здесь нам удается расширить найденную ранее технику, включающую концепцию проективных подстановок, и найденную возможность абсорбирования внешних информационных состояний. Это позволяет в указанных случаях решить проблему допустимости.

\medskip

Благодарности: Эта работа выполнена при поддержке
фонда РНС (Проект -- 23-21-00213).
\end{talk}
\end{document}