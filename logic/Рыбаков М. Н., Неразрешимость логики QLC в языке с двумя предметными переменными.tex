\documentclass[12pt]{article}
\usepackage{hyphsubst}
\usepackage[T2A]{fontenc}
\usepackage[english,main=russian]{babel}
\usepackage[utf8]{inputenc}
\usepackage[letterpaper,top=2cm,bottom=2cm,left=2cm,right=2cm,marginparwidth=2cm]{geometry}
\usepackage{float}
\usepackage{mathtools, commath, amssymb, amsthm}
\usepackage{enumitem, tabularx,graphicx,url,xcolor,rotating,multicol,epsfig,colortbl,lipsum}

\setlist{topsep=1pt, itemsep=0em}
\setlength{\parindent}{0pt}
\setlength{\parskip}{6pt}

\usepackage{hyphenat}
\hyphenation{ма-те-ма-ти-ка вос-ста-нав-ли-вать}

\usepackage[math]{anttor}

\newenvironment{talk}[6]{%
\vskip 0pt\nopagebreak%
\vskip 0pt\nopagebreak%
\section*{#1}
\phantomsection
\addcontentsline{toc}{section}{#2. \textit{#1}}
% \addtocontents{toc}{\textit{#1}\par}
\textit{#2}\\\nopagebreak%
#3\\\nopagebreak%
\ifthenelse{\equal{#4}{}}{}{\url{#4}\\\nopagebreak}%
\ifthenelse{\equal{#5}{}}{}{Соавторы: #5\\\nopagebreak}%
\ifthenelse{\equal{#6}{}}{}{Секция: #6\\\nopagebreak}%
}

\definecolor{LovelyBrown}{HTML}{FDFCF5}

\usepackage[pdftex,
breaklinks=true,
bookmarksnumbered=true,
linktocpage=true,
linktoc=all
]{hyperref}

\begin{document}
\pagenumbering{gobble}
\pagestyle{plain}
\pagecolor{LovelyBrown}
\begin{talk}
{Неразрешимость логики QLC в языке с двумя предметными переменными} % [1] название доклада
{Рыбаков Михаил Николаевич}
{ВШМ МФТИ, НИУ ВШЭ, ТвГУ}% [3] аффилиация
{m_rybakov@mail.ru} % [4] адрес электронной почты (НЕОБЯЗАТЕЛЬНО)
{}
{Математическая логика и теоретическая информатика} % [6] название секции

{\bf Известные факты.} Классическая логика предикатов $\mathbf{QCl}$ неразрешима в языке с тремя предметными переменными и одной бинарной предикатной буквой, при этом е\"{е} фрагмент с двумя предметными переменными разрешим. 
Любая логика, лежащая между интуиционистской предикатной логикой $\mathbf{QInt}$ и предикатной логикой слабого закона исключённого третьего $\mathbf{QKC}$, неразрешима в языке с двумя предметными переменными и одной унарной предикатной буквой, при этом используемые в доказательствах методы требуют наличия у логики шкал Крипке с бесконечными антицепями, и неприменимы к предикатной логике $\mathbf{QLC}$, определяемой классом линейных шкал Крипке. 
Логика $\mathbf{QS4.3}$~--- модальная предикатная логика класса линейных шкал~--- неразрешима при наличии двух предметных переменных и двух унарных предикатных букв в языке, 
но методы доказательства тоже не переносятся на $\mathbf{QLC}$, 
оставляя вопрос о разрешимости её фрагмента от двух переменных 
``интригующей открытой проблемой'' [1], которую мы и рассматриваем здесь.

{\bf Результаты.} Пусть $\mathbf{QL\mathbb{N}}$~--- логика шкалы Крипке $\langle \mathbb{N},\leqslant\rangle$, а $\mathbf{QL\mathbb{N}.cd}$~--- расширение этой логики формулой $\bf{cd}$, требующей постоянства предметных областей. Тогда
\begin{itemize}
\item
если $\mathbf{QLC}\subseteq L\subseteq \mathbf{QL\mathbb{N}.cd}$, то $L$ является $\Sigma^0_1$-трудной в языке с двумя предметными переменными;
\item
если $\mathbf{QL\mathbb{N}}\subseteq L\subseteq \mathbf{QL\mathbb{N}.cd}$, то $L$ является $\Sigma^0_1$-трудной и $\Pi^0_1$-трудной в языке с двумя предметными переменными.
\end{itemize}

{\bf Предмет обсуждения.} Если время позволит, то предполагается обсудить не столько сами результаты, сколько методы их получения и возможные обобщения.

{\bf Финансирование.} Работа выполнена при финансовой поддержке программы ``Научный фонд НИУ ВШЭ'', проект \mbox{23-00-022}.

\medskip

\begin{enumerate}
\item[{[1]}] X. Caicedo, G. Metcalfe, R. Rodr\'{i}guez, O. Tuyt, {\it One-variable fragments of intermediate logics over linear frames}, Information and Computation, 287, 2022.
\end{enumerate}
\end{talk}
\end{document}