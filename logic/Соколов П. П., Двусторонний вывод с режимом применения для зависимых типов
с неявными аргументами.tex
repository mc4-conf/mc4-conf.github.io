\documentclass[12pt]{article}
\usepackage{hyphsubst}
\usepackage[T2A]{fontenc}
\usepackage[english,main=russian]{babel}
\usepackage[utf8]{inputenc}
\usepackage[letterpaper,top=2cm,bottom=2cm,left=2cm,right=2cm,marginparwidth=2cm]{geometry}
\usepackage{float}
\usepackage{mathtools, commath, amssymb, amsthm}
\usepackage{enumitem, tabularx,graphicx,url,xcolor,rotating,multicol,epsfig,colortbl,lipsum}

\setlist{topsep=1pt, itemsep=0em}
\setlength{\parindent}{0pt}
\setlength{\parskip}{6pt}

\usepackage{hyphenat}
\hyphenation{ма-те-ма-ти-ка вос-ста-нав-ли-вать}

\usepackage[math]{anttor}

\newenvironment{talk}[6]{%
\vskip 0pt\nopagebreak%
\vskip 0pt\nopagebreak%
\section*{#1}
\phantomsection
\addcontentsline{toc}{section}{#2. \textit{#1}}
% \addtocontents{toc}{\textit{#1}\par}
\textit{#2}\\\nopagebreak%
#3\\\nopagebreak%
\ifthenelse{\equal{#4}{}}{}{\url{#4}\\\nopagebreak}%
\ifthenelse{\equal{#5}{}}{}{Соавторы: #5\\\nopagebreak}%
\ifthenelse{\equal{#6}{}}{}{Секция: #6\\\nopagebreak}%
}

\definecolor{LovelyBrown}{HTML}{FDFCF5}

\usepackage[pdftex,
breaklinks=true,
bookmarksnumbered=true,
linktocpage=true,
linktoc=all
]{hyperref}

\begin{document}
\pagenumbering{gobble}
\pagestyle{plain}
\pagecolor{LovelyBrown}
\begin{talk}
{Двусторонний вывод с режимом применения для зависимых типов
с неявными аргументами}
{Соколов Павел Павлович}
{МФТИ}
{sokolov.p64@gmail.com}
{}
{Математическая логика и теоретическая информатика}

Системы типов с зависимыми типами, изначально применявшиеся в инструментах
интерактивного доказательства теорем в качестве универсального логического
основания, постепенно начинают вводиться в существующие языки программирования
общего назначения или даже становятся основой для новых языков программирования
с целью более тонкого контроля за поведением программы, уменьшения вероятности
ошибки программиста и повышения его удобства.

Прежде всего, использование той или иной системы типов в языке выражается в
компиляторе в виде алгоритма, разрешающего задачу типизации: имеет ли данное
выражение \(t\) данный тип \(T\)? И, вообще говоря, в системах с зависимыми типами
решение этой задачи нетривиально и в классическом варианте требует от
программиста большого числа аннотаций типами, что мешает эргономике и загрязняет
код. В данном докладе мы рассмотрим недавно появившуюся технику двустороннего
вывода типов с режимом применения и применим её в построении системы с
зависимыми типами с неявными аргументами, критическим образом сокращающую
необходимое от программиста число аннотаций типов.

\medskip

\begin{enumerate}
\item[{[1]}] Richard A. Eisenberg,
{\it Dependent Types in Haskell: Theory and Practice}
[Doctoral dissertation, University of Pennsylvania]. arXiv, 2017.
\item[{[2]}] Ningning Xie, Bruno C. d. S. Oliveira,
{\it Let Arguments Go First}, Programming Lan\-guages and Systems.
ESOP 2018. Lecture Notes in Computer Science. 10801 (2018), 272-299.
\end{enumerate}
\end{talk}
\end{document}