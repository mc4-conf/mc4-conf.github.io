\documentclass[12pt]{article}
\usepackage{hyphsubst}
\usepackage[T2A]{fontenc}
\usepackage[english,main=russian]{babel}
\usepackage[utf8]{inputenc}
\usepackage[letterpaper,top=2cm,bottom=2cm,left=2cm,right=2cm,marginparwidth=2cm]{geometry}
\usepackage{float}
\usepackage{mathtools, commath, amssymb, amsthm}
\usepackage{enumitem, tabularx,graphicx,url,xcolor,rotating,multicol,epsfig,colortbl,lipsum}

\setlist{topsep=1pt, itemsep=0em}
\setlength{\parindent}{0pt}
\setlength{\parskip}{6pt}

\usepackage{hyphenat}
\hyphenation{ма-те-ма-ти-ка вос-ста-нав-ли-вать}

\usepackage[math]{anttor}

\newenvironment{talk}[6]{%
\vskip 0pt\nopagebreak%
\vskip 0pt\nopagebreak%
\section*{#1}
\phantomsection
\addcontentsline{toc}{section}{#2. \textit{#1}}
% \addtocontents{toc}{\textit{#1}\par}
\textit{#2}\\\nopagebreak%
#3\\\nopagebreak%
\ifthenelse{\equal{#4}{}}{}{\url{#4}\\\nopagebreak}%
\ifthenelse{\equal{#5}{}}{}{Соавторы: #5\\\nopagebreak}%
\ifthenelse{\equal{#6}{}}{}{Секция: #6\\\nopagebreak}%
}

\definecolor{LovelyBrown}{HTML}{FDFCF5}

\usepackage[pdftex,
breaklinks=true,
bookmarksnumbered=true,
linktocpage=true,
linktoc=all
]{hyperref}

\begin{document}
\pagenumbering{gobble}
\pagestyle{plain}
\pagecolor{LovelyBrown}
\begin{talk}
{О нижних сложностных оценках в кванторной вероятностной логике}
{Сперанский Станислав Олегович}
{Математический институт им. В.\,А. Стеклова Российской академии наук}
{katze.tail@gmail.com}
{}
{Математическая логика и теоретическая информатика}

Доклад посвящён двум естественным обогащениям популярной бескванторной <<полиномиальной>> вероятностной логики из [1]. Одно из них, обозначаемое через \(\mathsf{QPL}^{\mathrm{e}}\), получается посредством добавления кванторов по произвольным событиям, а в другом, обозначаемом через \(\underline{\mathsf{QPL}}^{\mathrm{e}}\), используются кванторы по пропозициональным формулам (т.е.\ фактически по событиям, выразимым посредством таких формул). Предыдущие доказательства результатов о нижних сложностных оценках для \(\mathsf{QPL}^{\mathrm{e}}\) и \(\underline{\mathsf{QPL}}^{\mathrm{e}}\) сильно зависели от наличия умножения и, стало быть, от полиноминальности на бескванторном уровне; см.\ [2] и [3]. В настоящем докладе будет показано, как можно получить те же самые нижние оценки для фрагментов \(\mathsf{QPL}^{\mathrm{e}}\) и \(\underline{\mathsf{QPL}}^{\mathrm{e}}\), в которых допускаются лишь линейные комбинации весьма специального вида. Также мы обсудим, что происходит при добавлении кванторов по вещественным числам к \(\mathsf{QPL}^{\mathrm{e}}\) и \(\underline{\mathsf{QPL}}^{\mathrm{e}}\).

\medskip

\begin{enumerate}

\item[{[1]}] R.\ Fagin, J.\,Y.\ Halpern, N.\ Megiddo.
A logic for reasoning about probabilities.
\emph{Information and Computation} 87(1--2), 78--128, 1990.
\item[{[2}] S.\,O.\ Speranski.
Complexity for probability logic with quantifiers over propositions.
\emph{Journal of Logic and Computation} 23(5), 1035--1055, 2013.
\item[{[3]}] S.\,O.\ Speranski.
Quantifying over events in probability logic: An introduction.
\emph{Mathematical Structures in Computer Science} 27(8), 1581--1600, 2017.
\end{enumerate}
\end{talk}
\end{document}