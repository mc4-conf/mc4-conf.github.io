\documentclass[12pt]{article}
\usepackage{hyphsubst}
\usepackage[T2A]{fontenc}
\usepackage[english,main=russian]{babel}
\usepackage[utf8]{inputenc}
\usepackage[letterpaper,top=2cm,bottom=2cm,left=2cm,right=2cm,marginparwidth=2cm]{geometry}
\usepackage{float}
\usepackage{mathtools, commath, amssymb, amsthm}
\usepackage{enumitem, tabularx,graphicx,url,xcolor,rotating,multicol,epsfig,colortbl,lipsum}

\setlist{topsep=1pt, itemsep=0em}
\setlength{\parindent}{0pt}
\setlength{\parskip}{6pt}

\usepackage{hyphenat}
\hyphenation{ма-те-ма-ти-ка вос-ста-нав-ли-вать}

\usepackage[math]{anttor}

\newenvironment{talk}[6]{%
\vskip 0pt\nopagebreak%
\vskip 0pt\nopagebreak%
\section*{#1}
\phantomsection
\addcontentsline{toc}{section}{#2. \textit{#1}}
% \addtocontents{toc}{\textit{#1}\par}
\textit{#2}\\\nopagebreak%
#3\\\nopagebreak%
\ifthenelse{\equal{#4}{}}{}{\url{#4}\\\nopagebreak}%
\ifthenelse{\equal{#5}{}}{}{Соавторы: #5\\\nopagebreak}%
\ifthenelse{\equal{#6}{}}{}{Секция: #6\\\nopagebreak}%
}

\definecolor{LovelyBrown}{HTML}{FDFCF5}

\usepackage[pdftex,
breaklinks=true,
bookmarksnumbered=true,
linktocpage=true,
linktoc=all
]{hyperref}

\begin{document}
\pagenumbering{gobble}
\pagestyle{plain}
\pagecolor{LovelyBrown}
\begin{talk}
{Структуры на сигнатурах структур}
{Стукачев Алексей Ильич}
{Новосибирский государственный университет}
{aistu@math.nsc.ru}
{}
{Математическая логика и теоретическая информатика}

В математической лингвистике слова естественного языка используются как символы (или знаки) для обозначения сущностей, свойств сущностей, свойств свойств сущностей и т. д. Совокупность этих слов, символов или знаков образует лексикон или сигнатуру. Однако это не просто множество, на нем существует определенная структура.  Например, каждое слово имеет грамматическую категорию (иногда две и более),  обладать монотонностью разной направленности  и т. д.

В докладе рассматриваются проблемы сложности структур такого типа в терминах \(\Sigma\)-определимости и  \(\Sigma\)-сводимости [1], и обсуждаются некоторые новые результаты из [2, 3, 4].

\medskip

\begin{enumerate}
\item[{[1]}]
A.I. Stukachev,  {\it Effective model theory: an approach via Sigma-definability}, Lecture Notes in Logic, 41 (2013), 164-197.
\item[{[2]}]
A.S. Burnistov, A.I. Stukachev,  {\it Generalized computable models and Montague semantics}, Studies in Computational Intelligence, 1081 (2023), 107-124.
\item[{[3]}]
A.S. Burnistov, A.I. Stukachev, {\it Computable functionals of finite types in Montague semantics}, Lecture Notes in Computer Science (to appear).
\item[{[4]}]
A.I. Stukachev, U.D. Penzina, {\it
Skolem functions and generalized quantifiers for negative polarity items semantics}, Lecture Notes in Networks and Systems (to appear).
\end{enumerate}
\end{talk}
\end{document}