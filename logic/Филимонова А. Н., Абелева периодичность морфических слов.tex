\documentclass[12pt]{article}
\usepackage{hyphsubst}
\usepackage[T2A]{fontenc}
\usepackage[english,main=russian]{babel}
\usepackage[utf8]{inputenc}
\usepackage[letterpaper,top=2cm,bottom=2cm,left=2cm,right=2cm,marginparwidth=2cm]{geometry}
\usepackage{float}
\usepackage{mathtools, commath, amssymb, amsthm}
\usepackage{enumitem, tabularx,graphicx,url,xcolor,rotating,multicol,epsfig,colortbl,lipsum}

\setlist{topsep=1pt, itemsep=0em}
\setlength{\parindent}{0pt}
\setlength{\parskip}{6pt}

\usepackage{hyphenat}
\hyphenation{ма-те-ма-ти-ка вос-ста-нав-ли-вать}

\usepackage[math]{anttor}

\newenvironment{talk}[6]{%
\vskip 0pt\nopagebreak%
\vskip 0pt\nopagebreak%
\section*{#1}
\phantomsection
\addcontentsline{toc}{section}{#2. \textit{#1}}
% \addtocontents{toc}{\textit{#1}\par}
\textit{#2}\\\nopagebreak%
#3\\\nopagebreak%
\ifthenelse{\equal{#4}{}}{}{\url{#4}\\\nopagebreak}%
\ifthenelse{\equal{#5}{}}{}{Соавторы: #5\\\nopagebreak}%
\ifthenelse{\equal{#6}{}}{}{Секция: #6\\\nopagebreak}%
}

\definecolor{LovelyBrown}{HTML}{FDFCF5}

\usepackage[pdftex,
breaklinks=true,
bookmarksnumbered=true,
linktocpage=true,
linktoc=all
]{hyperref}

\begin{document}
\pagenumbering{gobble}
\pagestyle{plain}
\pagecolor{LovelyBrown}
\begin{talk}
{Абелева периодичность морфических слов}
{Филимонова Арина Николаевна}
{СПбГУ}
{arina4filimonova@gmail.com}
{С.\,А. Пузынина}
{Математическая логика и теоретическая информатика}

Цель настоящего исследования --- дать характеризацию морфизмов над конечным алфавитом, порождающих абелево периодические слова. Два слова \(u\) и \(v\) называются \textit{абелево эквивалентными}, если в слове \(u\) можно переставить местами буквы так, чтобы получилось слово \(v\). Слово \(w\) называется \textit{абелево периодическим}, если  \(w\) представляет из себя конкатенацию абелево эквивалентных слов. Говоря об абелево периодичских бесконечных словах, мы будем допускать наличие предпериода, то есть условие абелевой периодичности должно начинаться с некоторого индекса.

\textit{Морфизмом} называется эндоморфизм на множестве слов, сохраняющий конкатенацию. Если морфизм \(f\) нестирающий, то есть образ никакого слова не пуст, и образ некоторой буквы \(a\) начинается с \(a\) и имеет длину больше двух, то говорят, что морфизм порождает бесконечное слово \(w=\lim_{n\to\infty}f^n(a)\).
Построение бесконечных слов с помощью морфизмов является одной из основных конструкций бесконечных слов в комбинаторике слов. Одним из ее преимуществ является возможность порождать слова быстро.

Всякому морфизму \(f\) над алфавитом \(\Sigma = \{a_1, a_2,\dots a_n\}\) ставится в соответствие матрица \(M(f)\) размера \(n\times n\), коэффициенты которой таковы: \(m_{ij}\) равно количеству вхождений буквы \(a_i\) в слово \(f(a_j)\). Обозачим собственные числа такой матрицы за \(\theta_1,\; \theta_2\; \dots, \theta_n\) (числа пронумерованы в порядке убывания их модуля). В терминах такой матрицы можно, в частности, получить некоторые необходимые условия абелевой периодичности.

Получены следующие результаты. Во-первых, задача решена для бинарных равноблочных (образы букв имеют равную длину) морфизмов:

\textbf{Теорема 1.}
{\it Пусть \(f\) равноблочный морфизм над бинарным алфавитом. Тогда морфическое слово \(f^{\omega}(a)\) абелево периодично в следующих случаях и только в них:
\begin{enumerate}
\item Морфизму \(f\) соответствует матрица с \(\theta_2 = 0\), то есть матрица вида
\(\begin{pmatrix}
A & A\\
B & B
\end{pmatrix}\);
\item Морфизм \(f\) имеет вид \(f(a) = a(ba)^k\), \(f(b) = b(ab)^k\) для некоторого \(k\).
\end{enumerate}}

Для неравноблочных бинарных морфизмов получен ряд необходимых и ряд достаточных условий абелевой периодичности. Для произвольных бинарных морфизмов остаётся неразобранным только случай \(\theta_2 = 0\), то есть не получена характеризация абелевой периодичности морфизмов, которым отвечают матрицы вида \(\begin{pmatrix}
A & \alpha A\\
B & \alpha B
\end{pmatrix}\).

Результаты получены совместно с С.\,А. Пузыниной.
\end{talk}
\end{document}