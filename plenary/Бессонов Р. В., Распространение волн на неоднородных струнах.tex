\documentclass[12pt]{article}
\usepackage{hyphsubst}
\usepackage[T2A]{fontenc}
\usepackage[english,main=russian]{babel}
\usepackage[utf8]{inputenc}
\usepackage[letterpaper,top=2cm,bottom=2cm,left=2cm,right=2cm,marginparwidth=2cm]{geometry}
\usepackage{float}
\usepackage{mathtools, commath, amssymb, amsthm}
\usepackage{enumitem, tabularx,graphicx,url,xcolor,rotating,multicol,epsfig,colortbl,lipsum}

\setlist{topsep=1pt, itemsep=0em}
\setlength{\parindent}{0pt}
\setlength{\parskip}{6pt}

\usepackage{hyphenat}
\hyphenation{ма-те-ма-ти-ка вос-ста-нав-ли-вать}

\usepackage[math]{anttor}

\newenvironment{talk}[6]{%
\vskip 0pt\nopagebreak%
\vskip 0pt\nopagebreak%
\section*{#1}
\phantomsection
\addcontentsline{toc}{section}{#2. \textit{#1}}
% \addtocontents{toc}{\textit{#1}\par}
\textit{#2}\\\nopagebreak%
#3\\\nopagebreak%
\ifthenelse{\equal{#4}{}}{}{\url{#4}\\\nopagebreak}%
\ifthenelse{\equal{#5}{}}{}{Соавторы: #5\\\nopagebreak}%
\ifthenelse{\equal{#6}{}}{}{Секция: #6\\\nopagebreak}%
}

\definecolor{LovelyBrown}{HTML}{FDFCF5}

\usepackage[pdftex,
breaklinks=true,
bookmarksnumbered=true,
linktocpage=true,
linktoc=all
]{hyperref}

\begin{document}
\pagenumbering{gobble}
\pagestyle{plain}
\pagecolor{LovelyBrown}
\begin{talk}
{Распространение волн на неоднородных струнах}
{Бессонов Роман Викторович}
{Санкт-Петербургский государственный университет Санкт-Петербургское отделение математического института им. В.\,А. Стеклова РАН}
{r.bessonov@gmail.com}
{С.\,А. Денисов}
{Пленарный доклад} %

Рассматривается распространение волн по неоднородной полубесконечной струне общего вида. В терминах динамики волн описывается условие Крейна-Винера конечности логарифмического интеграла спектральной функции струны:
\[\int_{0}^{\infty}\frac{\log v_{\rm ac}(\lambda)}{\sqrt{\lambda}(1+\lambda)}\,d\lambda > -\infty.\]
Указанное условие играет ключевую роль в спектральной теории стационарных гауссовских процессов. Оказывается, что оно равносильно наличию ``асимптотически бегущих'' волн, распространяющихся по струне.

Помимо динамического описания струн Крейна-Винера, приводится их полное описание в терминах функций плотности. В частности, струны, составленные из участков двух разных материалов принадлежат классу Крейна-Винера тогда и только тогда, когда общая длина одного из материалов конечна.

Задача о распространении волн на неоднородной струне допускает интерпретацию в теории рассеяния. Доказывается, что условие Крейна-Винера равносильно существованию и полноте модифицированных волновых операторов для рассматриваемой струны на фоне однородной струны.

Доклад содержит результаты цикла работ автора и С.\,А. Денисова (University of Wisconsin--Madyson).
\end{talk}
\end{document}