\documentclass[12pt]{article}
\usepackage{hyphsubst}
\usepackage[T2A]{fontenc}
\usepackage[english,main=russian]{babel}
\usepackage[utf8]{inputenc}
\usepackage[letterpaper,top=2cm,bottom=2cm,left=2cm,right=2cm,marginparwidth=2cm]{geometry}
\usepackage{float}
\usepackage{mathtools, commath, amssymb, amsthm}
\usepackage{enumitem, tabularx,graphicx,url,xcolor,rotating,multicol,epsfig,colortbl,lipsum}

\setlist{topsep=1pt, itemsep=0em}
\setlength{\parindent}{0pt}
\setlength{\parskip}{6pt}

\usepackage{hyphenat}
\hyphenation{ма-те-ма-ти-ка вос-ста-нав-ли-вать}

\usepackage[math]{anttor}

\newenvironment{talk}[6]{%
\vskip 0pt\nopagebreak%
\vskip 0pt\nopagebreak%
\section*{#1}
\phantomsection
\addcontentsline{toc}{section}{#2. \textit{#1}}
% \addtocontents{toc}{\textit{#1}\par}
\textit{#2}\\\nopagebreak%
#3\\\nopagebreak%
\ifthenelse{\equal{#4}{}}{}{\url{#4}\\\nopagebreak}%
\ifthenelse{\equal{#5}{}}{}{Соавторы: #5\\\nopagebreak}%
\ifthenelse{\equal{#6}{}}{}{Секция: #6\\\nopagebreak}%
}

\definecolor{LovelyBrown}{HTML}{FDFCF5}

\usepackage[pdftex,
breaklinks=true,
bookmarksnumbered=true,
linktocpage=true,
linktoc=all
]{hyperref}

\begin{document}
\pagenumbering{gobble}
\pagestyle{plain}
\pagecolor{LovelyBrown}
\begin{talk}
{Весовые структуры и \(t\)-структуры}
{Бондарко Михаил Владимирович}
{Санкт-Петербургский Государственный Университет}
{mbond77@mail.ru}
{}
{Пленарный доклад}

Часто важные (и функториальные) инварианты определяются или вычисляются при помощи неканонических конструкций.

Производные функторы вычисляются при помощи проективных и инъективных резольвент (модулей, объектов и комплексов); (ко)гомологии многообразий --- при помощи разбиений на симплексы; (ко)гомологии спектров --- при помощи клеточных фильтрации. Смешанные структуры Ходжа для когомологий непроективного (или негладкого) комплексного многообразия определяются при помощи хороших компактификаций (соотв., гладких гиперпокрытий). При этом, две разных резольвенты можно соединить морфизмом; для компактификаций получается только выбрать третью, ``мажорирующую'' первые две.

Иногда получается ``вложить исходные объекты'' в триангулированную категорию \(\underline{C}\) и связать искомый ``инвариант'' со срезками, соответствующими весовым структурам. Хоть весовые срезки и не каноничны, они позволяют строить канонические инварианты. В частности, при некоторых условиях (выполненных, например, для производных категорий регулярных собственных схем) весовые структуры позволяют строить \(t\)-структуры, срезки по которым (всегда) каноничны.
\end{talk}
\end{document}