\documentclass[12pt]{article}
\usepackage{hyphsubst}
\usepackage[T2A]{fontenc}
\usepackage[english,main=russian]{babel}
\usepackage[utf8]{inputenc}
\usepackage[letterpaper,top=2cm,bottom=2cm,left=2cm,right=2cm,marginparwidth=2cm]{geometry}
\usepackage{float}
\usepackage{mathtools, commath, amssymb, amsthm}
\usepackage{enumitem, tabularx,graphicx,url,xcolor,rotating,multicol,epsfig,colortbl,lipsum}

\setlist{topsep=1pt, itemsep=0em}
\setlength{\parindent}{0pt}
\setlength{\parskip}{6pt}

\usepackage{hyphenat}
\hyphenation{ма-те-ма-ти-ка вос-ста-нав-ли-вать}

\usepackage[math]{anttor}

\newenvironment{talk}[6]{%
\vskip 0pt\nopagebreak%
\vskip 0pt\nopagebreak%
\section*{#1}
\phantomsection
\addcontentsline{toc}{section}{#2. \textit{#1}}
% \addtocontents{toc}{\textit{#1}\par}
\textit{#2}\\\nopagebreak%
#3\\\nopagebreak%
\ifthenelse{\equal{#4}{}}{}{\url{#4}\\\nopagebreak}%
\ifthenelse{\equal{#5}{}}{}{Соавторы: #5\\\nopagebreak}%
\ifthenelse{\equal{#6}{}}{}{Секция: #6\\\nopagebreak}%
}

\definecolor{LovelyBrown}{HTML}{FDFCF5}

\usepackage[pdftex,
breaklinks=true,
bookmarksnumbered=true,
linktocpage=true,
linktoc=all
]{hyperref}

\begin{document}
\pagenumbering{gobble}
\pagestyle{plain}
\pagecolor{LovelyBrown}
\begin{talk}
{Порядки элементов конечных почти простых групп}
{Гречкосеева Мария Александровна}
{Институт математики им. С.\,Л. Соболева СО РАН}
{grechkoseeva@gmail.com}
{}
{Пленарный доклад}

Конечная группа \(G\) называется почти простой, если она удовлетворяет условию \(S\leq G\leq \operatorname{Aut} S\) для некоторой конечной неабелевой простой группы  \(S\); при этом  \(S\) является цоколем группы \(G\). Многие вопросы теории конечных групп сводятся не к простым, а к почти простым группам, поскольку важна информация не только о том, каковы композиционные факторы данной конечной группы, но и каково действие группы  на этих композиционных факторах.

Доклад посящен задаче вычисления множеств порядков элементов конечных почти простых групп. Эта задача легко решается для групп со знакопеременным цоколем и
давно решена для групп со спорадическим цоколем, поэтому речь пойдет о группах с цоколем лиева типа и, как легко понять, можно ограничиться группами вида \(\langle S, \alpha\rangle\), где~\(\alpha\in \operatorname{Aut} S\).

Будет дан обзор известных результатов, в том числе,  будет рассказано, как были найдены множества порядков элементов самих простых групп лиева типа и как случай внешнего автоморфизма \(\alpha\) был сведен к случаю, когда \(\alpha\) --- диагонально-графовый автоморфизм. Также будут представлены недавние результаты о диагонально-графовых автоморфизмах.
\end{talk}
\end{document}