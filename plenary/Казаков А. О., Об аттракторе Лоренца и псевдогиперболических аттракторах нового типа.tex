\documentclass[12pt]{article}
\usepackage{hyphsubst}
\usepackage[T2A]{fontenc}
\usepackage[english,main=russian]{babel}
\usepackage[utf8]{inputenc}
\usepackage[letterpaper,top=2cm,bottom=2cm,left=2cm,right=2cm,marginparwidth=2cm]{geometry}
\usepackage{float}
\usepackage{mathtools, commath, amssymb, amsthm}
\usepackage{enumitem, tabularx,graphicx,url,xcolor,rotating,multicol,epsfig,colortbl,lipsum}

\setlist{topsep=1pt, itemsep=0em}
\setlength{\parindent}{0pt}
\setlength{\parskip}{6pt}

\usepackage{hyphenat}
\hyphenation{ма-те-ма-ти-ка вос-ста-нав-ли-вать}

\usepackage[math]{anttor}

\newenvironment{talk}[6]{%
\vskip 0pt\nopagebreak%
\vskip 0pt\nopagebreak%
\section*{#1}
\phantomsection
\addcontentsline{toc}{section}{#2. \textit{#1}}
% \addtocontents{toc}{\textit{#1}\par}
\textit{#2}\\\nopagebreak%
#3\\\nopagebreak%
\ifthenelse{\equal{#4}{}}{}{\url{#4}\\\nopagebreak}%
\ifthenelse{\equal{#5}{}}{}{Соавторы: #5\\\nopagebreak}%
\ifthenelse{\equal{#6}{}}{}{Секция: #6\\\nopagebreak}%
}

\definecolor{LovelyBrown}{HTML}{FDFCF5}

\usepackage[pdftex,
breaklinks=true,
bookmarksnumbered=true,
linktocpage=true,
linktoc=all
]{hyperref}

\begin{document}
\pagenumbering{gobble}
\pagestyle{plain}
\pagecolor{LovelyBrown}
\begin{talk}
{Об аттракторе Лоренца и псевдогиперболических аттракторах нового типа}
{Казаков Алексей Олегович}
{Национальный исследовательский университет ``Высшая школа экономики''}
{kazakovdz@yandex.ru}
{}
{Пленарный доклад}

Аттрактор Лоренца является первым примером негрубого, но при этом робастного хаотического поведения. Его негрубость обусловлена тем, что при малых возмущениях в нем возникают бифуркации гомоклинических траекторий к седловому состоянию равновесия. Робастность аттрактора Лоренца заключается в том, что любая его траектория характеризуется положительным максимальным показателем Ляпунова, и это свойство сохраняется при малых возмущениях. В работе [1] выдвинута гипотеза о том, что робастность хаотического аттрактора эквивалентна его псевдогиперболичности. Из этого следует, что установив псевдогиперболичность аттрактора, исследователь может быть уверен, что наблюдаемый в эксперименте динамический режим действительно является хаотическим.

В наших недавних работах были разработаны методы проверки псевдогиперболичности, а также обнаружен ряд новых негрубых псевдогиперболических аттракторов лоренцевского типа. В докладе будут представлены недавние результаты по данной тематике.

Работа подготовлена в рамках Программы фундаментальных исследований Национального исследовательского университета ``Высшая школа экономики''.

\medskip

\begin{enumerate}
\item[{[1]}] Gonchenko S., Kazakov A., Turaev D. Wild pseudohyperbolic attractor in a four-dimensional Lorenz system //Nonlinearity. – 2021. – Т. 34. – №. 4. – С. 2018.
\end{enumerate}
\end{talk}
\end{document}