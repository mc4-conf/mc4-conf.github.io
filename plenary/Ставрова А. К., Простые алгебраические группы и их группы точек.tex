\documentclass[12pt]{article}
\usepackage{hyphsubst}
\usepackage[T2A]{fontenc}
\usepackage[english,main=russian]{babel}
\usepackage[utf8]{inputenc}
\usepackage[letterpaper,top=2cm,bottom=2cm,left=2cm,right=2cm,marginparwidth=2cm]{geometry}
\usepackage{float}
\usepackage{mathtools, commath, amssymb, amsthm}
\usepackage{enumitem, tabularx,graphicx,url,xcolor,rotating,multicol,epsfig,colortbl,lipsum}

\setlist{topsep=1pt, itemsep=0em}
\setlength{\parindent}{0pt}
\setlength{\parskip}{6pt}

\usepackage{hyphenat}
\hyphenation{ма-те-ма-ти-ка вос-ста-нав-ли-вать}

\usepackage[math]{anttor}

\newenvironment{talk}[6]{%
\vskip 0pt\nopagebreak%
\vskip 0pt\nopagebreak%
\section*{#1}
\phantomsection
\addcontentsline{toc}{section}{#2. \textit{#1}}
% \addtocontents{toc}{\textit{#1}\par}
\textit{#2}\\\nopagebreak%
#3\\\nopagebreak%
\ifthenelse{\equal{#4}{}}{}{\url{#4}\\\nopagebreak}%
\ifthenelse{\equal{#5}{}}{}{Соавторы: #5\\\nopagebreak}%
\ifthenelse{\equal{#6}{}}{}{Секция: #6\\\nopagebreak}%
}

\definecolor{LovelyBrown}{HTML}{FDFCF5}

\usepackage[pdftex,
breaklinks=true,
bookmarksnumbered=true,
linktocpage=true,
linktoc=all
]{hyperref}

\begin{document}
\pagenumbering{gobble}
\pagestyle{plain}
\pagecolor{LovelyBrown}
\begin{talk}
{Простые алгебраические группы и их группы точек}
{Ставрова Анастасия Константиновна}
{Санкт-Петербургское отделение Математического института им. В.\,А. Стеклова РАН}
{anastasia.stavrova@gmail.com}
{}
{Пленарный доклад} %

Простые алгебраические группы над полем \(K\) являются аналогами в алгебраической геометрии простых групп Ли
в геометрии дифференциальной. Будучи подмногообразием аффинного пространства, простая алгебраическая группа
\(G\) задается системой полиномиальных уравнений от нескольких переменных,
и множество решений \(G(L)\) этой системы уравнений в произвольном расширении \(L\) поля \(K\)
является группой в обычном смысле и называется группой \(L\)-точек алгебраической группы \(G\). Группа \(G(L)\),
вообще говоря, не является
простой, однако, если \(G\) изотропна (условие, соответствующее не-компактности для простых групп Ли),
то по теореме Ж. Титса (1964) она содержит ``большую'' нормальную подгруппу \(EG(L)\), которая проективно проста.
В. П. Платонов (1975) привел первый пример, показывающий, что фактор-группа \(G(L)/EG(L)\) может быть
нетривиальной, и в настоящее время проблема ее вычисления называется проблемой Кнезера-Титса.
Мы обсудим некоторые результаты по этой проблеме и ее обобщения.
\end{talk}
\end{document}