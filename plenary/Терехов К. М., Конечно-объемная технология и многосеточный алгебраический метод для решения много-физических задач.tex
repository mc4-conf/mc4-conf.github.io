\documentclass[12pt]{article}
\usepackage{hyphsubst}
\usepackage[T2A]{fontenc}
\usepackage[english,main=russian]{babel}
\usepackage[utf8]{inputenc}
\usepackage[letterpaper,top=2cm,bottom=2cm,left=2cm,right=2cm,marginparwidth=2cm]{geometry}
\usepackage{float}
\usepackage{mathtools, commath, amssymb, amsthm}
\usepackage{enumitem, tabularx,graphicx,url,xcolor,rotating,multicol,epsfig,colortbl,lipsum}

\setlist{topsep=1pt, itemsep=0em}
\setlength{\parindent}{0pt}
\setlength{\parskip}{6pt}

\usepackage{hyphenat}
\hyphenation{ма-те-ма-ти-ка вос-ста-нав-ли-вать}

\usepackage[math]{anttor}

\newenvironment{talk}[6]{%
\vskip 0pt\nopagebreak%
\vskip 0pt\nopagebreak%
\section*{#1}
\phantomsection
\addcontentsline{toc}{section}{#2. \textit{#1}}
% \addtocontents{toc}{\textit{#1}\par}
\textit{#2}\\\nopagebreak%
#3\\\nopagebreak%
\ifthenelse{\equal{#4}{}}{}{\url{#4}\\\nopagebreak}%
\ifthenelse{\equal{#5}{}}{}{Соавторы: #5\\\nopagebreak}%
\ifthenelse{\equal{#6}{}}{}{Секция: #6\\\nopagebreak}%
}

\definecolor{LovelyBrown}{HTML}{FDFCF5}

\usepackage[pdftex,
breaklinks=true,
bookmarksnumbered=true,
linktocpage=true,
linktoc=all
]{hyperref}

\begin{document}
\pagenumbering{gobble}
\pagestyle{plain}
\pagecolor{LovelyBrown}
\begin{talk}
{Конечно-объемная технология и многосеточный алгебраический метод для решения много-физических задач}
{Терехов Кирилл Михайлович}
{Институт Вычислительной Математики им. Г.И. Марчука Российской Академии Наук}
{terekhov@inm.ras.ru}
{Коньшин Игорь Николаевич, Василевский Юрий Викторович}
{Пленарный доклад}


В докладе обсуждается устойчивая консервативная конечно-объемная технология для совместного суперкомпьютерного моделирования нескольких физических процессов на динамических адаптивных подвижных сетках общего вида.

Предложенные численные методы отличаются устойчивостью как для задач с преобладающей конвективной составляющей, так и для задач седлового типа, формирующихся в процессе совместного решения нескольких физических процессов [1, 3--10]. Численные методы применены к задачам разной физики, таких как линейная упругость и пороупругость [4, 7, 8], течение несжимаемой жидкости [5], механика жестких тел [1], многофазная фильтрация [1, 10],  взаимодействие электромагнитных полей [1], течение и свертываемость крови [3], а также взаимодействие областей с разными физическими законами [6].

Предложены два подхода решения возникающих систем. Первый подход основан на многоуровневой неполной факторизации второго порядка с переупорядочиванием и масштабированием [10], второй подход основан на блочном алгебраическом многосеточном методе [2]. Алгебраический многосеточный метод на практике показывают линейную зависимость сложности решения от размера задачи, в том числе для систем седлового характера.

Одной из особенностью вычислительной технологии заключается в возможности динамической адаптации расчетных сеток общего вида в параллельном режиме. Динамическая адаптация включает как измельчение и разгрубление многогранных ячеек, так и перемещение узлов сетки в пространстве. Для работы с подвижными сетками был предложен консервативный четырехмерный вариант метода конечных объемов [3]. Динамическая адаптация расчетной сетки позволяет как моделировать процессы в подвижных областях, так и повышать точность расчета при экономии вычислительных ресурсов.

Ряд суперкомпьютерных технологий, образующих основу реализации численных методов, внедрены в открытой программной платформе INMOST (\url{www.inmost.org}, \url{www.inmost.ru}) для распределенного параллельного математического моделирования [10].

\medskip

\begin{enumerate}
\item[{[1]}] K.M. Terekhov. General finite-volume framework for saddle-point problems of various physics // Russian Journal of Numerical Analysis and Mathematical Modelling 36 (6), 359--379, (2021)
\item[{[2]}]  I.N. Konshin, K.M. Terekhov. Block Algebraic Multigrid Method for Saddle-Point Problems of Various Physics // Supercomputing: 9th Russian Supercomputing Days, Springer, 17--34, (2023)
\item[{[3]}] K.M. Terekhov, I.D. Butakov, A.A. Danilov, Yu.V. Vassilevski. Dynamic adaptive moving mesh finite‐volume method for the blood flow and coagulation modeling // International Journal for Numerical Methods in Biomedical Engineering, e3731, (2023)
\item[{[4]}] K.M. Terekhov, Yu. V. Vassilevski. Finite volume method for coupled subsurface flow problems, II: Poroelasticity // Journal of Computational Physics 462, 111225, (2022)
\item[{[5]}] K.M. Terekhov. Collocated finite-volume method for the incompressible Navier–Stokes problem // Journal of Numerical Mathematics 29 (1), 63--79, (2021)
\item[{[6]}] K.M. Terekhov. Multi-physics flux coupling for hydraulic fracturing modelling within INMOST platform // Russian Journal of Numerical Analysis and Mathematical Model\-ling 35 (4), 223--237, (2020)
\item[{[7]}] K.M. Terekhov. Cell-centered finite-volume method for heterogeneous anisotropic poro\-mechanics problem // Journal of Computational and Applied Mathematics 365, 112357, (2020)
\item[{[8]}] K.M. Terekhov, H.A. Tchelepi. Cell-centered finite-volume method for elastic defor\-mation of heterogeneous media with full-tensor properties // Journal of Computational and Applied Mathematics 364, 112331, (2020)
\item[{[9]}] K.M. Terekhov, Yu.V. Vassilevski. Finite volume method for coupled subsurface flow problems, I: Darcy problem // Journal of Computational Physics 395, 298--306, (2019)
\item[{[10]}] Yu. Vassilevski, K. Terekhov, K. Nikitin, I. Kapyrin. Parallel finite volume computation on general meshes // Springer International Publishing, (2020)
\end{enumerate}
\end{talk}
\end{document}