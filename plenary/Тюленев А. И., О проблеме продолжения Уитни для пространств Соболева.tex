\documentclass[12pt]{article}
\usepackage{hyphsubst}
\usepackage[T2A]{fontenc}
\usepackage[english,main=russian]{babel}
\usepackage[utf8]{inputenc}
\usepackage[letterpaper,top=2cm,bottom=2cm,left=2cm,right=2cm,marginparwidth=2cm]{geometry}
\usepackage{float}
\usepackage{mathtools, commath, amssymb, amsthm}
\usepackage{enumitem, tabularx,graphicx,url,xcolor,rotating,multicol,epsfig,colortbl,lipsum}

\setlist{topsep=1pt, itemsep=0em}
\setlength{\parindent}{0pt}
\setlength{\parskip}{6pt}

\usepackage{hyphenat}
\hyphenation{ма-те-ма-ти-ка вос-ста-нав-ли-вать}

\usepackage[math]{anttor}

\newenvironment{talk}[6]{%
\vskip 0pt\nopagebreak%
\vskip 0pt\nopagebreak%
\section*{#1}
\phantomsection
\addcontentsline{toc}{section}{#2. \textit{#1}}
% \addtocontents{toc}{\textit{#1}\par}
\textit{#2}\\\nopagebreak%
#3\\\nopagebreak%
\ifthenelse{\equal{#4}{}}{}{\url{#4}\\\nopagebreak}%
\ifthenelse{\equal{#5}{}}{}{Соавторы: #5\\\nopagebreak}%
\ifthenelse{\equal{#6}{}}{}{Секция: #6\\\nopagebreak}%
}

\definecolor{LovelyBrown}{HTML}{FDFCF5}

\usepackage[pdftex,
breaklinks=true,
bookmarksnumbered=true,
linktocpage=true,
linktoc=all
]{hyperref}

\begin{document}
\pagenumbering{gobble}
\pagestyle{plain}
\pagecolor{LovelyBrown}
\begin{talk}
{О проблеме продолжения Уитни для пространств Соболева}
{Тюленев Александр Иванович}
{МЦМУ МИАН}
{tyulenev-math@yandex.ru, tyulenev@mi-ras.ru}
{}
{Пленарный доклад} %

В 1934 году Х. Уитни поставил следующую задачу. Пусть \(m,n \in \mathbb{N}\), а \(S \subset \mathbb{R}^{n}\) --- непустое замкнутое множество.
Для заданной функции \(f: S \to \mathbb{R}\) требуется найти условия, необходимые и достаточные для существования функции \(F \in C^{m}(\mathbb{R}^{n})\), являющейся продолжением \(f\), т.е. \(F|_{S}=f\). В полной общности эта проблема была решена Ч. Фефферманом в середине 2000-ых. Большой интерес представляет аналогичная задача, сформулированная в контексте пространств Соболева
\(W^{m}_{p}(\mathbb{R}^{n})\), \(p \in [1,\infty]\). Такая задача ещё очень далека от своего окончательного решения.

На данный момент окончательные ответы получены лишь в случае \(m=1\), \(p > n\) в работах П. Шварцмана. Некоторые результаты при \(p > n\) и \(m \in \mathbb{N}\) получены в работах Ч. Феффермана и его учеников.
Также в последние два года Ч. Фефферманом и его учениками предпринимаются попытки продвижения в случае \(m=n=2\) и \(p > 1\), однако и здесь ситуация далека от своего окончательного решения.

Основной фокус доклада -- случай \(m=1\), \(n \geq 2\) и \(p \in (1,n]\). В таком диапазоне параметров ранее задача рассматривалась лишь для регулярных по Альфорсу--Давиду множеств \(S \subset \mathbb{R}^{n}\).
Недавно удалось получить окончательные ответы для существенно более широкого класса ``толстых'' множеств, введённого В. Рычковым. Более того,
в контексте абстрактных метрических пространств с мерой удаётся получить естественное обобщение как этих результатов, так и некоторых результатов П. Шварцмана, относящихся к случаю \(m=1\) и \(p \in (n,\infty)\).
Наконец, если множество \(S \subset \mathbb{R}^{n}\) не удовлетворяет каким-либо дополнительным условиям регулярности, то удаётся получить почти точное описание следа
пространств Соболева на \(S\).
\end{talk}
\end{document}