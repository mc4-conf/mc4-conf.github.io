\documentclass[12pt]{article}
\usepackage{hyphsubst}
\usepackage[T2A]{fontenc}
\usepackage[english,main=russian]{babel}
\usepackage[utf8]{inputenc}
\usepackage[letterpaper,top=2cm,bottom=2cm,left=2cm,right=2cm,marginparwidth=2cm]{geometry}
\usepackage{float}
\usepackage{mathtools, commath, amssymb, amsthm}
\usepackage{enumitem, tabularx,graphicx,url,xcolor,rotating,multicol,epsfig,colortbl,lipsum}

\setlist{topsep=1pt, itemsep=0em}
\setlength{\parindent}{0pt}
\setlength{\parskip}{6pt}

\usepackage{hyphenat}
\hyphenation{ма-те-ма-ти-ка вос-ста-нав-ли-вать}

\usepackage[math]{anttor}

\newenvironment{talk}[6]{%
\vskip 0pt\nopagebreak%
\vskip 0pt\nopagebreak%
\section*{#1}
\phantomsection
\addcontentsline{toc}{section}{#2. \textit{#1}}
% \addtocontents{toc}{\textit{#1}\par}
\textit{#2}\\\nopagebreak%
#3\\\nopagebreak%
\ifthenelse{\equal{#4}{}}{}{\url{#4}\\\nopagebreak}%
\ifthenelse{\equal{#5}{}}{}{Соавторы: #5\\\nopagebreak}%
\ifthenelse{\equal{#6}{}}{}{Секция: #6\\\nopagebreak}%
}

\definecolor{LovelyBrown}{HTML}{FDFCF5}

\usepackage[pdftex,
breaklinks=true,
bookmarksnumbered=true,
linktocpage=true,
linktoc=all
]{hyperref}

\begin{document}
\pagenumbering{gobble}
\pagestyle{plain}
\pagecolor{LovelyBrown}
\begin{talk}
{О моменте вырождения ветвящегося процесса Гальтона-Ватсона при условии большого числа частиц за всю историю}
{Бакай Гавриил Андреевич}
{Математический институт им. В.\,А. Стеклова Российской академии наук}
{gavrik_lur_bakay@mail.ru}
{}
{Теория вероятностей}

Пусть случайные величины \(X, X_{i,j},\ i,j\in\mathbb{N},\) являются независимыми и одинаково распределенными и принимают целые неотрицательные значения. Положим
\[
Z_0:=1,\quad Z_{n}:= \sum_{j=1}^{Z_{n-1}} X_{n,j},\quad n\in\mathbb{N}.
\]
Случайный процесс \(\{Z_n,\ n\ge 0\}\) называют {\it ветвящимся процессом Гальтона-Ватсона}. Пусть
\[
T:= \min\{k\in\mathbb{N}:\ Z_k = 0\},\quad \xi:=\sum_{j=1}^{+\infty}Z_j.
\]
Известно ([1]), что в предположениии \({\bf E}X = 1,\ 0<{\bf D}X =\sigma^2<+\infty \) имеет место сходимость
\[
{\bf P}(T/\sqrt{n} \le x | \xi = n ) \to F(x),\quad n\to\infty,\ x>0,
\]
где \(F(x)\) --- некоторая функция распределения.

Автором доказано выполнение равномерного по \(k/n=k(n)/n\in[a,b]\subset(0,1)\) соотношения
\[
{\bf P}(T = k|\xi = n)\sim n F_0\left(\frac kn\right)\exp\left(-L\left(\frac kn\right)n\right),\quad n\to\infty,
\]
и получены выражения для функций \(F_0\) и \(L\).

\medskip

\begin{enumerate}
\item[{[1]}] D. Aldous, {\it The continuum random tree. II. An overview}, Stochastic analysis, 167 (1990), 23-70.
\end{enumerate}
\end{talk}
\end{document}