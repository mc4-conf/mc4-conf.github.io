\documentclass[12pt]{article}
\usepackage{hyphsubst}
\usepackage[T2A]{fontenc}
\usepackage[english,main=russian]{babel}
\usepackage[utf8]{inputenc}
\usepackage[letterpaper,top=2cm,bottom=2cm,left=2cm,right=2cm,marginparwidth=2cm]{geometry}
\usepackage{float}
\usepackage{mathtools, commath, amssymb, amsthm}
\usepackage{enumitem, tabularx,graphicx,url,xcolor,rotating,multicol,epsfig,colortbl,lipsum}

\setlist{topsep=1pt, itemsep=0em}
\setlength{\parindent}{0pt}
\setlength{\parskip}{6pt}

\usepackage{hyphenat}
\hyphenation{ма-те-ма-ти-ка вос-ста-нав-ли-вать}

\usepackage[math]{anttor}

\newenvironment{talk}[6]{%
\vskip 0pt\nopagebreak%
\vskip 0pt\nopagebreak%
\section*{#1}
\phantomsection
\addcontentsline{toc}{section}{#2. \textit{#1}}
% \addtocontents{toc}{\textit{#1}\par}
\textit{#2}\\\nopagebreak%
#3\\\nopagebreak%
\ifthenelse{\equal{#4}{}}{}{\url{#4}\\\nopagebreak}%
\ifthenelse{\equal{#5}{}}{}{Соавторы: #5\\\nopagebreak}%
\ifthenelse{\equal{#6}{}}{}{Секция: #6\\\nopagebreak}%
}

\definecolor{LovelyBrown}{HTML}{FDFCF5}

\usepackage[pdftex,
breaklinks=true,
bookmarksnumbered=true,
linktocpage=true,
linktoc=all
]{hyperref}

\begin{document}
\pagenumbering{gobble}
\pagestyle{plain}
\pagecolor{LovelyBrown}
\begin{talk}
{Распределение длины очереди в системе обслуживания со смешанной приоритетной дисциплиной в условиях критической загрузки}
{Берговин Алексей Константинович}
{МГУ имени М.\,В. Ломоносова, МЦМУ ``Московский центр фундаментальной и прикладной математики''}
{alexey.bergovin@gmail.com}
{}
{Теория вероятностей} %

В данном докладе рассматривается система обслуживания с приоритетами, функционирующая в условиях критической загрузки, а, именно, исследуется поведение длины очереди, когда одновременно и время стремится к бесконечности, и загрузка к единице, такую постановку предложил Ю.\,В. Прохоров в [1].

Рассматривается система в которой три пуассоновских входящих потока, каждый из которых имеет свою функцию распределения времени обслуживания, в системе есть бесконечное число мест в очереди для ожидания и один обслуживающий прибор. Приоритетная дисциплина является смешанной, установлены следующие приоритеты: относительный приоритет между требованиями первого и второго потоков и между требованиями второго и третьего потоков, а для требований первого и третьего потоков --- абсолютный приоритет с обслуживанием занового прерванного требования. Будем считать, что первый поток имеет наивысший приоритет, а третий --- низший.

Используя соотношения, которым удовлетворяет преобразование Лалпаса совместной производящей функции количества требований каждого приоритетного класса в системе, полученные в работе [2], были найдены предельные распределения для количества требований наименее приоритетного класса системы в условиях критической загрузки.

\medskip

\begin{enumerate}
\item[{[1]}] Прохоров~Ю.\,В. Переходные явления в процессах массового обслуживания // Литовский математический сборник. 1963. Т.3,  № 1. С. 199 -– 206.
\item[{[2]}] Берговин~А.\,К., Ушаков~В.\,Г. Исследование систем обслуживания со смешанными приоритетами // Информатика и ее применения. 2023. Т. 17, Вып. 2. С. 57 -- 61.
\end{enumerate}
\end{talk}
\end{document}