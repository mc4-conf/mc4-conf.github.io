\documentclass[12pt]{article}
\usepackage{hyphsubst}
\usepackage[T2A]{fontenc}
\usepackage[english,main=russian]{babel}
\usepackage[utf8]{inputenc}
\usepackage[letterpaper,top=2cm,bottom=2cm,left=2cm,right=2cm,marginparwidth=2cm]{geometry}
\usepackage{float}
\usepackage{mathtools, commath, amssymb, amsthm}
\usepackage{enumitem, tabularx,graphicx,url,xcolor,rotating,multicol,epsfig,colortbl,lipsum}

\setlist{topsep=1pt, itemsep=0em}
\setlength{\parindent}{0pt}
\setlength{\parskip}{6pt}

\usepackage{hyphenat}
\hyphenation{ма-те-ма-ти-ка вос-ста-нав-ли-вать}

\usepackage[math]{anttor}

\newenvironment{talk}[6]{%
\vskip 0pt\nopagebreak%
\vskip 0pt\nopagebreak%
\section*{#1}
\phantomsection
\addcontentsline{toc}{section}{#2. \textit{#1}}
% \addtocontents{toc}{\textit{#1}\par}
\textit{#2}\\\nopagebreak%
#3\\\nopagebreak%
\ifthenelse{\equal{#4}{}}{}{\url{#4}\\\nopagebreak}%
\ifthenelse{\equal{#5}{}}{}{Соавторы: #5\\\nopagebreak}%
\ifthenelse{\equal{#6}{}}{}{Секция: #6\\\nopagebreak}%
}

\definecolor{LovelyBrown}{HTML}{FDFCF5}

\usepackage[pdftex,
breaklinks=true,
bookmarksnumbered=true,
linktocpage=true,
linktoc=all
]{hyperref}

\begin{document}
\pagenumbering{gobble}
\pagestyle{plain}
\pagecolor{LovelyBrown}
\begin{talk}
{Бесконечномерные компакты и гауссовские процессы}
{Досполова Мария Каиржановна}
{Санкт-Петербургское отделение
Математического института им. В.\,А.Стеклова РАН}
{dospolova.maria@yandex.ru}
{}
{Теория вероятностей} %

Пусть \(K\) --- выпуклое компактное подмножество
евклидова пространства \(\mathbb{R}^d\). У каждого такого компакта \(K\) есть характеристики, которые не зависят от размерности объемлющего пространства \(d\), а зависят только от внутренней геометрии \(K\). Они называются \textit{внутренними объемами} \(K\), обозначаются через \(V_k(K), \ k=0,1,\ldots,d\) и определяются как  коэффициенты в формуле Штейнера. Штейнер показал, что объем \(\lambda\)-окрестности компакта \(K\) представляется многочленом от \(\lambda\) с коэффициентами \(V_k(K)\) (где нормировка подобрана специальным образом).

Известно еще одно, эквивалентное первому, определение внутренних объемов: \(V_k(K)\) --- это средний объем проекции \(K\) на случайное \(k\)-мерное линейное подпространство, выбранное по мере Хаара.

Воспользовавшись независимостью внутренних объемов от размерности, Судаков и Шеве обобщили данное
понятие на бесконечномерные выпуклые множества в сепарабельном гильбертовом пространстве.

Оказалось, что, помимо вышеприведенных определений, у внутренних объемов существует гауссовское представление (Судаков [1] и Цирельсон [2]), которое позволяет
изучать их с вероятностной точки зрения. Обнаружение глубокой связи между внутренними объемами  и гауссовскими процессами позволило решить задачи на стыке теории вероятностей и выпуклой геометрии.

В докладе мы рассмотрим дальнейшие свойства бесконечномерных компактов и их связь с гауссовскими процессами.

\medskip

\begin{enumerate}
\item[{[1]}] В. Н. Судаков, {\it Геометрические проблемы теории бесконечномерных вероятностных распределений}, Труды МИАН, {141} (1976), 3-191.
\item[{[2]}] Б. С. Цирельсон, {\it Геометрический подход к оценке максимального правдоподобия для бесконечномерного гауссовского сдвига. {II}}, Теория вероятн. и ее примен., {30}:4 (1985), 772-779.
\item[{[3]}] R. Schneider, W. Weil, {\it Stochastic and integral geometry}, Springer, 2008.
\item[{[4]}] B. Klartag, V. Milman, {\it The slicing problem by Bourgain. In
\textit{Analysis at large. Dedicated to the life and work of Jean Bourgain}}, Springer, 2022.
\end{enumerate}
\end{talk}
\end{document}