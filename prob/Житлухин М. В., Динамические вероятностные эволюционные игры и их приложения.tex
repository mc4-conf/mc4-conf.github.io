\documentclass[12pt]{article}
\usepackage{hyphsubst}
\usepackage[T2A]{fontenc}
\usepackage[english,main=russian]{babel}
\usepackage[utf8]{inputenc}
\usepackage[letterpaper,top=2cm,bottom=2cm,left=2cm,right=2cm,marginparwidth=2cm]{geometry}
\usepackage{float}
\usepackage{mathtools, commath, amssymb, amsthm}
\usepackage{enumitem, tabularx,graphicx,url,xcolor,rotating,multicol,epsfig,colortbl,lipsum}

\setlist{topsep=1pt, itemsep=0em}
\setlength{\parindent}{0pt}
\setlength{\parskip}{6pt}

\usepackage{hyphenat}
\hyphenation{ма-те-ма-ти-ка вос-ста-нав-ли-вать}

\usepackage[math]{anttor}

\newenvironment{talk}[6]{%
\vskip 0pt\nopagebreak%
\vskip 0pt\nopagebreak%
\section*{#1}
\phantomsection
\addcontentsline{toc}{section}{#2. \textit{#1}}
% \addtocontents{toc}{\textit{#1}\par}
\textit{#2}\\\nopagebreak%
#3\\\nopagebreak%
\ifthenelse{\equal{#4}{}}{}{\url{#4}\\\nopagebreak}%
\ifthenelse{\equal{#5}{}}{}{Соавторы: #5\\\nopagebreak}%
\ifthenelse{\equal{#6}{}}{}{Секция: #6\\\nopagebreak}%
}

\definecolor{LovelyBrown}{HTML}{FDFCF5}

\usepackage[pdftex,
breaklinks=true,
bookmarksnumbered=true,
linktocpage=true,
linktoc=all
]{hyperref}

\begin{document}
\pagenumbering{gobble}
\pagestyle{plain}
\pagecolor{LovelyBrown}
\begin{talk}
{Динамические вероятностные эволюционные игры и их приложения}
{Житлухин Михаил Валентинович}
{МИАН им.~В.\,А.~Стеклова}
{mikhailzh@mi-ras.ru}
{}
{Теории вероятностей} %

Под \emph{динамической вероятностной эволюционной игрой} мы понимаем систему взаимодействующих агентов (игроков), описываемую случайным процессом \(R_t\) со значениями в множестве \(\{r\in \mathbb{R}_+^M : r^1+\ldots+r^M=1\}\), который показывает насколько ``успешны'' стратегии игроков --- чем координата \(R_t^m\) ближе к 1, тем ``успешнее'' стратегия игрока~\(m\).

Главным образом нас будут интересовать стратегии, называемые \emph{выживающими}, для которых процесс \(R_t^m\) остается отделенным от нуля с вероятностью 1 на бесконечном промежутке времени независимо от стратегий, используемых оппонентами.
Интерес изучения выживающих стратегий объясняется тем, что присутствие использующих их игроков позволяет предсказать, как будет развиваться игра при времени \(t\to\infty\).

Доклад будет основан на цикле работ автора за последние несколько лет, в которых изучались вопросы  существования выживающих стратегий и их конструктивного построения в конкретных моделях игр, возникающих в экономических приложениях.
\end{talk}
\end{document}