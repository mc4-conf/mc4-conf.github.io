\documentclass[12pt]{article}
\usepackage{hyphsubst}
\usepackage[T2A]{fontenc}
\usepackage[english,main=russian]{babel}
\usepackage[utf8]{inputenc}
\usepackage[letterpaper,top=2cm,bottom=2cm,left=2cm,right=2cm,marginparwidth=2cm]{geometry}
\usepackage{float}
\usepackage{mathtools, commath, amssymb, amsthm}
\usepackage{enumitem, tabularx,graphicx,url,xcolor,rotating,multicol,epsfig,colortbl,lipsum}

\setlist{topsep=1pt, itemsep=0em}
\setlength{\parindent}{0pt}
\setlength{\parskip}{6pt}

\usepackage{hyphenat}
\hyphenation{ма-те-ма-ти-ка вос-ста-нав-ли-вать}

\usepackage[math]{anttor}

\newenvironment{talk}[6]{%
\vskip 0pt\nopagebreak%
\vskip 0pt\nopagebreak%
\section*{#1}
\phantomsection
\addcontentsline{toc}{section}{#2. \textit{#1}}
% \addtocontents{toc}{\textit{#1}\par}
\textit{#2}\\\nopagebreak%
#3\\\nopagebreak%
\ifthenelse{\equal{#4}{}}{}{\url{#4}\\\nopagebreak}%
\ifthenelse{\equal{#5}{}}{}{Соавторы: #5\\\nopagebreak}%
\ifthenelse{\equal{#6}{}}{}{Секция: #6\\\nopagebreak}%
}

\definecolor{LovelyBrown}{HTML}{FDFCF5}

\usepackage[pdftex,
breaklinks=true,
bookmarksnumbered=true,
linktocpage=true,
linktoc=all
]{hyperref}

\begin{document}
\pagenumbering{gobble}
\pagestyle{plain}
\pagecolor{LovelyBrown}
\begin{talk}
{Свойства семейства случайных внешних мер, ассоциированных с бесконечной урновой схемой}
{Ковалевский Артем Павлович}
{Институт математики им. С.\,Л. Соболева Сибирского отделения РАН}
{artyom.kovalevskii@gmail.com}
{Берхане Абебе Андемикаэль}
{Теория вероятностей} %

Урновой схемой называется вероятностная модель, в которой шары последовательно и независимо друг от друга раскладываются по урнам.
Предполагается, что для всех шаров есть одно и то же распределение вероятностей попадания в каждую из урн. В простейшем случае урн конечное
число, и вероятности попадания в каждую урну одинаковы. Бесконечная урновая схема предполагает, что урн счетное число, в этом случае вероятность
попадания шара в урну обязательно зависит от номера урны. Одна из изучаемых статистик --- число непустых урн после бросания \(n\) шаров.

Итак, будем предполагать, что урн счетное число, и зафиксируем вероятности попадания шара в каждую из урн (одинаковые для всех шаров).
Для произвольного подмножества \(A\) отрезка \([0, \, 1]\) рассмотрим не все номера шаров от 1 до \(n\), а только те из них, которые попали в множество
\(nA\), и изучим число непустых урн после бросания шаров с номерами из множества \(nA\). Это число неотрицательно, и в случае пустого множества \(A\)
тождественно равно нулю. Кроме того, оно удовлетворяет свойству сигма-субаддитивности: если \(A\) является подмножеством счетного объединения
множеств, то число непустых урн для номеров из \(nA\) не превосходит суммы чисел, посчитанных таким же образом по каждому из этих множеств.
Таким образом, число непустых урн для номеров шаров из \(nA\), где \(A\) --- произвольное подмножество из \([0, \, 1]\), удовлетворяет всем свойствам внешней меры на отрезке \([0, \, 1]\).

Изучается сужение этого семейства случайных внешних мер на класс множеств \(A\), образованных  объединением конечного числа промежутков.
Для этого сужения мы доказываем закон больших чисел и центральную предельную теорему.
\end{talk}
\end{document}