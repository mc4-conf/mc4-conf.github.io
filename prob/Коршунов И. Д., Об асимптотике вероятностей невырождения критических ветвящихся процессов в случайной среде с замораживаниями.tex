\documentclass[12pt]{article}
\usepackage{hyphsubst}
\usepackage[T2A]{fontenc}
\usepackage[english,main=russian]{babel}
\usepackage[utf8]{inputenc}
\usepackage[letterpaper,top=2cm,bottom=2cm,left=2cm,right=2cm,marginparwidth=2cm]{geometry}
\usepackage{float}
\usepackage{mathtools, commath, amssymb, amsthm}
\usepackage{enumitem, tabularx,graphicx,url,xcolor,rotating,multicol,epsfig,colortbl,lipsum}

\setlist{topsep=1pt, itemsep=0em}
\setlength{\parindent}{0pt}
\setlength{\parskip}{6pt}

\usepackage{hyphenat}
\hyphenation{ма-те-ма-ти-ка вос-ста-нав-ли-вать}

\usepackage[math]{anttor}

\newenvironment{talk}[6]{%
\vskip 0pt\nopagebreak%
\vskip 0pt\nopagebreak%
\section*{#1}
\phantomsection
\addcontentsline{toc}{section}{#2. \textit{#1}}
% \addtocontents{toc}{\textit{#1}\par}
\textit{#2}\\\nopagebreak%
#3\\\nopagebreak%
\ifthenelse{\equal{#4}{}}{}{\url{#4}\\\nopagebreak}%
\ifthenelse{\equal{#5}{}}{}{Соавторы: #5\\\nopagebreak}%
\ifthenelse{\equal{#6}{}}{}{Секция: #6\\\nopagebreak}%
}

\definecolor{LovelyBrown}{HTML}{FDFCF5}

\usepackage[pdftex,
breaklinks=true,
bookmarksnumbered=true,
linktocpage=true,
linktoc=all
]{hyperref}

\begin{document}
\pagenumbering{gobble}
\pagestyle{plain}
\pagecolor{LovelyBrown}
\begin{talk}
{Об асимптотике вероятностей невырождения критических ветвящихся процессов в случайной среде с замораживаниями}
{Коршунов Иван Дмитриевич}
{МГУ}
{IDKorshunov@mail.com}
{}
{Теория вероятностей} %

Известно, что ветвящийся процесс в случайной среде хорошо описывается соответствующим случайным блужданием
\[
S_n = \xi_1 + \ldots + \xi_n,
\]
где \(\xi_k = \ln \varphi_{\eta_k}'(1)\), \(\varphi_x (t)\) и \(\eta_k\) --- производящая функция числа потомков и случайная среда. В докладе будет рассмотрен вопрос вырождения ветвящегося процесса в случайной среде с заморозками при \(\mathsf{E} \xi_1 = 0\), отличающегося от обычного ВПСС тем, что каждая среда устанавливается на несколько поколений. Оказывается, что данный вопрос так же тесно связан со случайным блужданием
\[
S_n = \tau_1 \xi_1 + \ldots + \tau_n \xi_n,
\]
где \(\xi_k = \ln \varphi_{\eta_k}'(1)\), \(\varphi_x (t)\) и \(\eta_k\) --- производящая функция числа потомков и случайная среда, а \(\tau_k\) --- длительность \(k\)-й заморозки.

В докладе будет показано, что, если число потомков любой частицы имеет геометрическое распределение, а также при определенных условиях на моменты \(\xi\) и на замораживания \(\{ \tau_n \}_{n = 1}^{\infty}\) вероятность выживания всего процесса удовлетворяет асимптотическому соотношению
\[
\mathsf{P} \left( Z_{s_n} > 0 \right) \sim \frac{c}{\sqrt{\tau_1^2 + \ldots + \tau_n^2}},~n \to \infty
\]
для некоторой положительной константы \(c\), где \(s_n = \tau_1 + \ldots + \tau_n\).
\end{talk}
\end{document}