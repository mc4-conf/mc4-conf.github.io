\documentclass[12pt]{article}
\usepackage{hyphsubst}
\usepackage[T2A]{fontenc}
\usepackage[english,main=russian]{babel}
\usepackage[utf8]{inputenc}
\usepackage[letterpaper,top=2cm,bottom=2cm,left=2cm,right=2cm,marginparwidth=2cm]{geometry}
\usepackage{float}
\usepackage{mathtools, commath, amssymb, amsthm}
\usepackage{enumitem, tabularx,graphicx,url,xcolor,rotating,multicol,epsfig,colortbl,lipsum}

\setlist{topsep=1pt, itemsep=0em}
\setlength{\parindent}{0pt}
\setlength{\parskip}{6pt}

\usepackage{hyphenat}
\hyphenation{ма-те-ма-ти-ка вос-ста-нав-ли-вать}

\usepackage[math]{anttor}

\newenvironment{talk}[6]{%
\vskip 0pt\nopagebreak%
\vskip 0pt\nopagebreak%
\section*{#1}
\phantomsection
\addcontentsline{toc}{section}{#2. \textit{#1}}
% \addtocontents{toc}{\textit{#1}\par}
\textit{#2}\\\nopagebreak%
#3\\\nopagebreak%
\ifthenelse{\equal{#4}{}}{}{\url{#4}\\\nopagebreak}%
\ifthenelse{\equal{#5}{}}{}{Соавторы: #5\\\nopagebreak}%
\ifthenelse{\equal{#6}{}}{}{Секция: #6\\\nopagebreak}%
}

\definecolor{LovelyBrown}{HTML}{FDFCF5}

\usepackage[pdftex,
breaklinks=true,
bookmarksnumbered=true,
linktocpage=true,
linktoc=all
]{hyperref}

\begin{document}
\pagenumbering{gobble}
\pagestyle{plain}
\pagecolor{LovelyBrown}
\begin{talk}
{Об асимптотическом поведении среднего значения функционалов от случайного поля частиц, задаваемого ветвящимся случайным блужданием}
{Люлинцев Андрей Валерьевич}
{ПОМИ РАН}
{lav_100k@mail.ru}
{}
{Теория вероятностей} %

Рассматривается однородный марковский процесс с непрерывным временем на фазовом пространстве \(\mathbb{Z}_+=\{0,1,2,\ldots\}\), который мы интерпретируем как движение частицы. Частица может переходить только в соседние точки \(\mathbb{Z}_+\), то есть при каждой смене положения частицы ее координата изменяется на единицу. Процесс снабжен механизмом ветвления. Источники ветвления могут находиться в каждой точке \(\mathbb{Z}_+\). В момент ветвления новые частицы появляются в точке ветвления и дальше начинают эволюционировать независимо друг от друга (и от остальных частиц) по тем же законам, что и начальная частица. В каждый момент времени \(t\) мы имеем случайное поле на \(\mathbb{Z}_+\), состоящее из частиц, имеющихся в системе в этот момент. Рассматриваются функционалы от этого поля вида \(\sum_{(m_j,m_k)}\Phi(m_j,m_k)\), где суммирование происходит по всем упорядоченным парам \((m_j,m_k)\) различных частиц поля. Изучается асимптотическое поведение среднего значения данного функционала при \(t\to +\infty\).
\end{talk}
\end{document}