\documentclass[12pt]{article}
\usepackage{hyphsubst}
\usepackage[T2A]{fontenc}
\usepackage[english,main=russian]{babel}
\usepackage[utf8]{inputenc}
\usepackage[letterpaper,top=2cm,bottom=2cm,left=2cm,right=2cm,marginparwidth=2cm]{geometry}
\usepackage{float}
\usepackage{mathtools, commath, amssymb, amsthm}
\usepackage{enumitem, tabularx,graphicx,url,xcolor,rotating,multicol,epsfig,colortbl,lipsum}

\setlist{topsep=1pt, itemsep=0em}
\setlength{\parindent}{0pt}
\setlength{\parskip}{6pt}

\usepackage{hyphenat}
\hyphenation{ма-те-ма-ти-ка вос-ста-нав-ли-вать}

\usepackage[math]{anttor}

\newenvironment{talk}[6]{%
\vskip 0pt\nopagebreak%
\vskip 0pt\nopagebreak%
\section*{#1}
\phantomsection
\addcontentsline{toc}{section}{#2. \textit{#1}}
% \addtocontents{toc}{\textit{#1}\par}
\textit{#2}\\\nopagebreak%
#3\\\nopagebreak%
\ifthenelse{\equal{#4}{}}{}{\url{#4}\\\nopagebreak}%
\ifthenelse{\equal{#5}{}}{}{Соавторы: #5\\\nopagebreak}%
\ifthenelse{\equal{#6}{}}{}{Секция: #6\\\nopagebreak}%
}

\definecolor{LovelyBrown}{HTML}{FDFCF5}

\usepackage[pdftex,
breaklinks=true,
bookmarksnumbered=true,
linktocpage=true,
linktoc=all
]{hyperref}

\begin{document}
\pagenumbering{gobble}
\pagestyle{plain}
\pagecolor{LovelyBrown}
\begin{talk}
{О неравенствах для объемов случайных симплексов}
{Мосеева Татьяна Дмитриевна}
{СПбГУ, ПОМИ РАН}
{polezina@yandex.ru}
{}
{Теория вероятностей} %

В 1971 году Майлзом была получена явная формула для моментов объема случайного симплекса, часть вершин которого выбирается равномерно внутри единичного шара, а часть --- равномерно на единичной сфере. Из данного результата следует, что при фиксированном количестве вершин симплекса его средний объем возрастает при увеличении количества вершин, выбираемых на границе шара.

Равномерное распреление внутри или на границе единичного шара можно рассматривать как частный случай бета-распределения: в первом случае параметр  равен 0, а равномерное распределение на единичной сфере получается как предел бета-распределений при стремлении параметра к \(-1\).
Рассмотрим случайный симплекс, часть вершин которого выбрана в соответствии с бета-распределением с параметром \(\beta_1\), а часть --- с параметром \(\beta_2 < \beta_1\). Оказывается, чем больше вершин выбирается со вторым параметром (при фиксированном общем количестве вершин), тем больше средний объем рассматриваемого симплекса.

Доклад посвящен данному обобщению результата Майлза, а также более общим условиям на распределение вершин симплексов, при которых мы можем получить аналогичные соотношения на объемы.
\end{talk}
\end{document}