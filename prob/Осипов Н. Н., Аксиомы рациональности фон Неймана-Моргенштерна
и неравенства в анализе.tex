\documentclass[12pt]{article}
\usepackage{hyphsubst}
\usepackage[T2A]{fontenc}
\usepackage[english,main=russian]{babel}
\usepackage[utf8]{inputenc}
\usepackage[letterpaper,top=2cm,bottom=2cm,left=2cm,right=2cm,marginparwidth=2cm]{geometry}
\usepackage{float}
\usepackage{mathtools, commath, amssymb, amsthm}
\usepackage{enumitem, tabularx,graphicx,url,xcolor,rotating,multicol,epsfig,colortbl,lipsum}

\setlist{topsep=1pt, itemsep=0em}
\setlength{\parindent}{0pt}
\setlength{\parskip}{6pt}

\usepackage{hyphenat}
\hyphenation{ма-те-ма-ти-ка вос-ста-нав-ли-вать}

\usepackage[math]{anttor}

\newenvironment{talk}[6]{%
\vskip 0pt\nopagebreak%
\vskip 0pt\nopagebreak%
\section*{#1}
\phantomsection
\addcontentsline{toc}{section}{#2. \textit{#1}}
% \addtocontents{toc}{\textit{#1}\par}
\textit{#2}\\\nopagebreak%
#3\\\nopagebreak%
\ifthenelse{\equal{#4}{}}{}{\url{#4}\\\nopagebreak}%
\ifthenelse{\equal{#5}{}}{}{Соавторы: #5\\\nopagebreak}%
\ifthenelse{\equal{#6}{}}{}{Секция: #6\\\nopagebreak}%
}

\definecolor{LovelyBrown}{HTML}{FDFCF5}

\usepackage[pdftex,
breaklinks=true,
bookmarksnumbered=true,
linktocpage=true,
linktoc=all
]{hyperref}

\begin{document}
\pagenumbering{gobble}
\pagestyle{plain}
\pagecolor{LovelyBrown}
\begin{talk}
{Аксиомы рациональности фон Неймана--Моргенштерна
и неравенства в анализе}
{Осипов Николай Николаевич}
{ПОМИ РАН}
{}
{}
{Теория вероятностей} %

Меняющееся благосостояние агента, делающего ставки на результаты бросков честной монеты, является классическим примером случайного процесса с мартингальным свойством. В такой игре не существует стратегии, которая бы давала положительное математическое ожидание прибыли. Однако задача о том, можно ли рационально выбрать стратегию, отличную от бездействия, %
остается осмысленной и нетривиальной: оказывается, что есть некоторый ``зазор'' между полным отказом от игры
и полностью \emph{не}рациональным экономическим поведением, которое нарушает базовые аксиомы рациональности фон Неймана--Моргенштерна.
Решая задачу об описании этого ``зазора'' и поиске в рамках него оптимальных стратегий, мы придем к функциям Беллмана, которые ранее возникали в решении полностью абстрактных задач о поиске точных констант в неравенствах из анализа.
Тем самым мы %
получим естественную экономическую интерпретацию для этих %
неравенств и связанных с ними функций Беллмана.
\end{talk}
\end{document}