\documentclass[12pt]{article}
\usepackage{hyphsubst}
\usepackage[T2A]{fontenc}
\usepackage[english,main=russian]{babel}
\usepackage[utf8]{inputenc}
\usepackage[letterpaper,top=2cm,bottom=2cm,left=2cm,right=2cm,marginparwidth=2cm]{geometry}
\usepackage{float}
\usepackage{mathtools, commath, amssymb, amsthm}
\usepackage{enumitem, tabularx,graphicx,url,xcolor,rotating,multicol,epsfig,colortbl,lipsum}

\setlist{topsep=1pt, itemsep=0em}
\setlength{\parindent}{0pt}
\setlength{\parskip}{6pt}

\usepackage{hyphenat}
\hyphenation{ма-те-ма-ти-ка вос-ста-нав-ли-вать}

\usepackage[math]{anttor}

\newenvironment{talk}[6]{%
\vskip 0pt\nopagebreak%
\vskip 0pt\nopagebreak%
\section*{#1}
\phantomsection
\addcontentsline{toc}{section}{#2. \textit{#1}}
% \addtocontents{toc}{\textit{#1}\par}
\textit{#2}\\\nopagebreak%
#3\\\nopagebreak%
\ifthenelse{\equal{#4}{}}{}{\url{#4}\\\nopagebreak}%
\ifthenelse{\equal{#5}{}}{}{Соавторы: #5\\\nopagebreak}%
\ifthenelse{\equal{#6}{}}{}{Секция: #6\\\nopagebreak}%
}

\definecolor{LovelyBrown}{HTML}{FDFCF5}

\usepackage[pdftex,
breaklinks=true,
bookmarksnumbered=true,
linktocpage=true,
linktoc=all
]{hyperref}

\begin{document}
\pagenumbering{gobble}
\pagestyle{plain}
\pagecolor{LovelyBrown}
\begin{talk}
{Предельные теоремы для вероятностной модели биржевого стакана}
{Прасолов Тимофей Вячеславович}
{Институт математики им. С.\,Л. Соболева СО РАН}
{t.prasolov@g.nsu.ru}
{Тарасенко А.\,А.}
{Теория вероятностей} %

Биржевой стакан --- это таблица заявок на покупку/продажу определенного количества актива. В нашей работе рассматривается модель, где цены заявок выбираются из фиксированного конечного множества, а размещения и отмены заявок и сделки по заявкам с наилучшей ценой происходят согласно независимым процессам Пуассона.

В работах предшественников рассматривался случай, когда сделки совершаются только для одной заявки. В случае, когда сделка может происходить по нескольким заявкам одновременно, приведены достаточные условия эргодичности системы. Для среднего арифметического между наилучшими ценами на покупку и продажу приводятся оценка вероятности при следующем изменении пойти вверх и результаты численных симуляций.

\medskip

Работа выполнена при поддержке Математического Центра в Академгородке,
соглашение с Министерством науки и высшего образования Российской Федерации №075-15-2022-281.
\end{talk}
\end{document}