\documentclass[12pt]{article}
\usepackage{hyphsubst}
\usepackage[T2A]{fontenc}
\usepackage[english,main=russian]{babel}
\usepackage[utf8]{inputenc}
\usepackage[letterpaper,top=2cm,bottom=2cm,left=2cm,right=2cm,marginparwidth=2cm]{geometry}
\usepackage{float}
\usepackage{mathtools, commath, amssymb, amsthm}
\usepackage{enumitem, tabularx,graphicx,url,xcolor,rotating,multicol,epsfig,colortbl,lipsum}

\setlist{topsep=1pt, itemsep=0em}
\setlength{\parindent}{0pt}
\setlength{\parskip}{6pt}

\usepackage{hyphenat}
\hyphenation{ма-те-ма-ти-ка вос-ста-нав-ли-вать}

\usepackage[math]{anttor}

\newenvironment{talk}[6]{%
\vskip 0pt\nopagebreak%
\vskip 0pt\nopagebreak%
\section*{#1}
\phantomsection
\addcontentsline{toc}{section}{#2. \textit{#1}}
% \addtocontents{toc}{\textit{#1}\par}
\textit{#2}\\\nopagebreak%
#3\\\nopagebreak%
\ifthenelse{\equal{#4}{}}{}{\url{#4}\\\nopagebreak}%
\ifthenelse{\equal{#5}{}}{}{Соавторы: #5\\\nopagebreak}%
\ifthenelse{\equal{#6}{}}{}{Секция: #6\\\nopagebreak}%
}

\definecolor{LovelyBrown}{HTML}{FDFCF5}

\usepackage[pdftex,
breaklinks=true,
bookmarksnumbered=true,
linktocpage=true,
linktoc=all
]{hyperref}

\begin{document}
\pagenumbering{gobble}
\pagestyle{plain}
\pagecolor{LovelyBrown}
\begin{talk}
{О точности оценивания ковариационной матрицы многомерного случайного вектора}
{Пучкин Никита Андреевич}
{Национальный исследовательский университет «Высшая школа экономики»}
{npuchkin@hse.ru}
{Ф. Носков, М. Рахуба, В. Спокойный}
{Теория вероятностей}

Пусть \(X, X_1, \dots, X_n\) --- центрированные независимые одинаково распределенные случайные векторы в \(\mathbb R^d\) с конечным вторым моментом. Одной из классических задач математической статистики является оценивание ковариационной матрицы \(\Sigma = \mathbb E X X^\top\) по конечной выборке \((X_1, \dots, X_n)\). Основной интерес представляет случай, когда значение \(d\) велико. Чтобы избежать проклятия размерности, обычно на распределение случайного вектора \(X\) и, в частности, на вид его ковариационной матрицы накладываются дополнительные ограничения. В рамках доклада будет показано, что при достаточно мягких условиях возможно оценить \(\Sigma\) с точностью, определяемой лишь ее следом и объемом выборки \(n\), а не размерностью пространства \(d\). Более того, результат может быть улучшен, если предположить, что \(\Sigma\) представима в виде суммы нескольких произведений Кронекера матриц меньшего размера. Доклад основан на совместной работе с Ф.~Носковым и В.~Спокойным [1], а также с М. Рахубой [2].

\medskip

\begin{enumerate}
\item[{[1]}] N. Puchkin, F. Noskov, V. Spokoiny, {\it Sharper dimension-free bounds on the Frobenius distance between sample covariance and its expectation}, Bernoulli (to appear),\\ arXiv:2308.14739.
\item[{[2]}] N. Puchkin, M. Rakhuba, {\it Dimension-free structured covariance estimation}, The 37th Annual Conference on Learning Theory (COLT 2024, to appear), arXiv:2402.10032.
\end{enumerate}
\end{talk}
\end{document}