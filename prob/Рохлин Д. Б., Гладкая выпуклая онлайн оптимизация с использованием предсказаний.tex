\documentclass[12pt]{article}
\usepackage{hyphsubst}
\usepackage[T2A]{fontenc}
\usepackage[english,main=russian]{babel}
\usepackage[utf8]{inputenc}
\usepackage[letterpaper,top=2cm,bottom=2cm,left=2cm,right=2cm,marginparwidth=2cm]{geometry}
\usepackage{float}
\usepackage{mathtools, commath, amssymb, amsthm}
\usepackage{enumitem, tabularx,graphicx,url,xcolor,rotating,multicol,epsfig,colortbl,lipsum}

\setlist{topsep=1pt, itemsep=0em}
\setlength{\parindent}{0pt}
\setlength{\parskip}{6pt}

\usepackage{hyphenat}
\hyphenation{ма-те-ма-ти-ка вос-ста-нав-ли-вать}

\usepackage[math]{anttor}

\newenvironment{talk}[6]{%
\vskip 0pt\nopagebreak%
\vskip 0pt\nopagebreak%
\section*{#1}
\phantomsection
\addcontentsline{toc}{section}{#2. \textit{#1}}
% \addtocontents{toc}{\textit{#1}\par}
\textit{#2}\\\nopagebreak%
#3\\\nopagebreak%
\ifthenelse{\equal{#4}{}}{}{\url{#4}\\\nopagebreak}%
\ifthenelse{\equal{#5}{}}{}{Соавторы: #5\\\nopagebreak}%
\ifthenelse{\equal{#6}{}}{}{Секция: #6\\\nopagebreak}%
}

\definecolor{LovelyBrown}{HTML}{FDFCF5}

\usepackage[pdftex,
breaklinks=true,
bookmarksnumbered=true,
linktocpage=true,
linktoc=all
]{hyperref}

\begin{document}
\pagenumbering{gobble}
\pagestyle{plain}
\pagecolor{LovelyBrown}
\begin{talk}
{Гладкая выпуклая онлайн оптимизация с использованием предсказаний}
{Рохлин Дмитрий Борисович}
{Институт математики, механики и компьютерных наук Южного федерального университета и
Региональный научно-образовательный математический центр Южного федерального университета}
{dbrohlin@sfedu.ru}
{}
{Теория вероятностей} %

Рассмотрим последовательность выпуклых \(L\)-гладких функций \(f_t(w)\), \(w\in\mathcal W\). Базовая версия задачи онлайн оптимизации состоит отыскании последовательности \(w_t\), обеспечивающей равномерную оценку сожаления \(R_T(w)=\sum_{t=1}^T(f_t(w_t)-f_t(w))=o(T)\). В градиентных методах на каждом шаге алгоритм получает информацию о градиенте  \(g_t=\nabla f_t(w_t)\)  в запрашиваемой точке. %
В докладе рассматривается случай, когда доступны предсказания \(\hat g_t\) градиента. Большинство соответствующих оценок сожаления содержат слагаемые вида \(\|g_t-\hat g_t\|\) и являются неявными, так как точка \(w_t\) неизвестна в момент использования предсказания \(\hat g_t\). В докладе обсуждаются, в частности, результаты недавней работы [1], где получены явные оценки сожаления в терминах ошибок предсказания градиента. В случае когда \(f_t(w)=\ell(w,z_t)\), где \(z_t\) --- эргодический марковский процесс с неизвестным переходным ядром, и \(\ell\) является липшицевой по второму аргументу, данный результат непосредственно приводит к оценке сожаления в терминах \(\varepsilon_t=\|z_t-\mathsf E(z_t|z_{t-1})\|\). Аппроксимация условного математического ожидания \(\mathsf E(z_t|z_{t-1})\) на основе выборки \(z_1,\dots,z_{t-1}\), и оценка математического ожидания \(\varepsilon_t\) о инвариантной мере также могут быть найдены с использованием онлайн-оптимизации: [2]. Одним из приложений является задача лог-оптимального инвестирования.

\medskip

\begin{enumerate}
\item[{[1]}] P.Z. Scroccaro, A.S. Kolarijani,  P.M. Esfahani, {\it Adaptive composite online optimization: Predictions in static and dynamic environments}, IEEE Transactions on Automatic Control, 68 (2023), 2906–2921.
\item[{[2]}] A. Agarwal, J.C. Duchi, {\it The generalization ability of online algorithms for dependent data}, IEEE Transactions on Information Theory, 59 (2012), 573–587.
\end{enumerate}
\end{talk}
\end{document}