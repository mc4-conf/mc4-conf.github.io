\documentclass[12pt]{article}
\usepackage{hyphsubst}
\usepackage[T2A]{fontenc}
\usepackage[english,main=russian]{babel}
\usepackage[utf8]{inputenc}
\usepackage[letterpaper,top=2cm,bottom=2cm,left=2cm,right=2cm,marginparwidth=2cm]{geometry}
\usepackage{float}
\usepackage{mathtools, commath, amssymb, amsthm}
\usepackage{enumitem, tabularx,graphicx,url,xcolor,rotating,multicol,epsfig,colortbl,lipsum}

\setlist{topsep=1pt, itemsep=0em}
\setlength{\parindent}{0pt}
\setlength{\parskip}{6pt}

\usepackage{hyphenat}
\hyphenation{ма-те-ма-ти-ка вос-ста-нав-ли-вать}

\usepackage[math]{anttor}

\newenvironment{talk}[6]{%
\vskip 0pt\nopagebreak%
\vskip 0pt\nopagebreak%
\section*{#1}
\phantomsection
\addcontentsline{toc}{section}{#2. \textit{#1}}
% \addtocontents{toc}{\textit{#1}\par}
\textit{#2}\\\nopagebreak%
#3\\\nopagebreak%
\ifthenelse{\equal{#4}{}}{}{\url{#4}\\\nopagebreak}%
\ifthenelse{\equal{#5}{}}{}{Соавторы: #5\\\nopagebreak}%
\ifthenelse{\equal{#6}{}}{}{Секция: #6\\\nopagebreak}%
}

\definecolor{LovelyBrown}{HTML}{FDFCF5}

\usepackage[pdftex,
breaklinks=true,
bookmarksnumbered=true,
linktocpage=true,
linktoc=all
]{hyperref}

\begin{document}
\pagenumbering{gobble}
\pagestyle{plain}
\pagecolor{LovelyBrown}
\begin{talk}
{О формуле Ито для винеровского процесса}
{Смородина Наталия Васильевна}
{ПОМИ РАН}
{smorodina@pdmi.ras.ru}
{}
{Теория вероятностей}

Показано, что в классической формуле Ито для винеровского процесса \(w(t)\) можно  заменить вторую производную,
понимаемую в смысле обычного дифференцирования, на вторую
производную в смысле дифференцирования
обобщенных функций. Это можно сделать в
случае, когда первая производная принадлежит классу \(L_{2,loc}(\mathbb{R})\).
Ранее, в работе [1] при тех же условиях была получена другая
форма последнего слагаемого в формуле Ито.

\textbf{Теорема.} {\it Пусть функция \(v\in L_{2,loc}(\mathbb{R})\),
функция \(V\) есть первообразная \(v\) (т.е. \(V^\prime=v\)), а
обобщенная функция \(v^\prime\) есть производная \(v\) в смысле дифференцирования обобщенных функций. Пусть
\(\{v_\varepsilon\}\) --- произвольное семейство абсолютно непрерывных
функций,  такое, что для любого \(N>0\) выполнено
\(\lim_{\varepsilon\to 0}\|v_\varepsilon-v\|_{L_2[-N,N]}=0\).
Тогда
\begin{enumerate}
\item Существует предел по вероятности \(\lim_{\varepsilon\to
0+}\int_0^tv^\prime_\varepsilon(w(\tau))\,d\tau,\) и этот предел не
зависит от выбора семейства \(v_\varepsilon\). Для данного предела
используем обозначение \(\int_0^tv^\prime(w(\tau))\,d\tau\).
\item Справедлива формула Ито
\(V(w(t))=V(w(0))+\int_0^tv(w(\tau))\,dw(\tau)+\frac{1}{2}\int_0^tv^\prime(w(\tau))\,d\tau\).
\end{enumerate}}

\medskip

\begin{enumerate}
\item[{[1]}] H. F\"{o}llmer , Ph.  Protter , A. N.  Shiryayev,
{\it Quadratic Covariation and an Extension of Ito's Formula , }
Bernoulli, 1/2(1995), 149--169.
\end{enumerate}
\end{talk}
\end{document}