\documentclass[12pt]{article}
\usepackage{hyphsubst}
\usepackage[T2A]{fontenc}
\usepackage[english,main=russian]{babel}
\usepackage[utf8]{inputenc}
\usepackage[letterpaper,top=2cm,bottom=2cm,left=2cm,right=2cm,marginparwidth=2cm]{geometry}
\usepackage{float}
\usepackage{mathtools, commath, amssymb, amsthm}
\usepackage{enumitem, tabularx,graphicx,url,xcolor,rotating,multicol,epsfig,colortbl,lipsum}

\setlist{topsep=1pt, itemsep=0em}
\setlength{\parindent}{0pt}
\setlength{\parskip}{6pt}

\usepackage{hyphenat}
\hyphenation{ма-те-ма-ти-ка вос-ста-нав-ли-вать}

\usepackage[math]{anttor}

\newenvironment{talk}[6]{%
\vskip 0pt\nopagebreak%
\vskip 0pt\nopagebreak%
\section*{#1}
\phantomsection
\addcontentsline{toc}{section}{#2. \textit{#1}}
% \addtocontents{toc}{\textit{#1}\par}
\textit{#2}\\\nopagebreak%
#3\\\nopagebreak%
\ifthenelse{\equal{#4}{}}{}{\url{#4}\\\nopagebreak}%
\ifthenelse{\equal{#5}{}}{}{Соавторы: #5\\\nopagebreak}%
\ifthenelse{\equal{#6}{}}{}{Секция: #6\\\nopagebreak}%
}

\definecolor{LovelyBrown}{HTML}{FDFCF5}

\usepackage[pdftex,
breaklinks=true,
bookmarksnumbered=true,
linktocpage=true,
linktoc=all
]{hyperref}

\begin{document}
\pagenumbering{gobble}
\pagestyle{plain}
\pagecolor{LovelyBrown}
\begin{talk}
{О приближениях для сумм слабо зависимых случайных величин}
{Ульянов Владимир Васильевич}
{МГУ имени М\,В. Ломоносова и НИУ ВШЭ}
{}
{}
{Теория вероятностей} %

Пусть \((X_i, i \in J)\) - семейство локально зависимых неотрицательных целочисленных случайных величин. Рассмотрим сумму \(W =\sum_{i\in J} X_i\). Сначала мы устанавливаем общую верхнюю границу для \(d_{TV} (W,M)\), используя метод Стейна, где целевая переменная
\(M\) является  смесью распределения Пуассона и биномиального или отрицательного биномиального
распределения.В качестве приложений мы получаем оптимальный порядок   \(O(|J|^{-1})\)   ошибки приближения для распределений \((k_1, k_2)\)- серий и \(k\)-серий. Наши результаты значительно улучшают существующие результаты
порядка \(O(|J|^{-1/2})\). Более того, используя недавний результат Бобкова и Ульянова [1] по уточнению центральной предельной теоремы для независимых слагаемых с целыми значениями, мы получаем асимптотические разложения для функции распределения \(W\). Доклад основан на
препринте [2].

\medskip

\begin{enumerate}
\item[{[1]}] С. Г. Бобков, В. В. Ульянов, {\it Поправка Чебышёва–Эджворта в центральной предельной теореме для целочисленных независимых слагаемых}, Теория вероятностей и ее применения, 66(2022), вып.4, 676--692.
\item[{[2]}] Zhonggen Su, Vladimir V. Ulyanov, Xiaolin Wang, {\it Approximation of Sums of Locally Dependent Random Variables via Perturbation of Stein Operator}, \\ https://doi.org/10.48550/arXiv.2209.09770.
\end{enumerate}
\end{talk}
\end{document}