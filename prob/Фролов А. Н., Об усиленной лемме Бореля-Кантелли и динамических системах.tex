\documentclass[12pt]{article}
\usepackage{hyphsubst}
\usepackage[T2A]{fontenc}
\usepackage[english,main=russian]{babel}
\usepackage[utf8]{inputenc}
\usepackage[letterpaper,top=2cm,bottom=2cm,left=2cm,right=2cm,marginparwidth=2cm]{geometry}
\usepackage{float}
\usepackage{mathtools, commath, amssymb, amsthm}
\usepackage{enumitem, tabularx,graphicx,url,xcolor,rotating,multicol,epsfig,colortbl,lipsum}

\setlist{topsep=1pt, itemsep=0em}
\setlength{\parindent}{0pt}
\setlength{\parskip}{6pt}

\usepackage{hyphenat}
\hyphenation{ма-те-ма-ти-ка вос-ста-нав-ли-вать}

\usepackage[math]{anttor}

\newenvironment{talk}[6]{%
\vskip 0pt\nopagebreak%
\vskip 0pt\nopagebreak%
\section*{#1}
\phantomsection
\addcontentsline{toc}{section}{#2. \textit{#1}}
% \addtocontents{toc}{\textit{#1}\par}
\textit{#2}\\\nopagebreak%
#3\\\nopagebreak%
\ifthenelse{\equal{#4}{}}{}{\url{#4}\\\nopagebreak}%
\ifthenelse{\equal{#5}{}}{}{Соавторы: #5\\\nopagebreak}%
\ifthenelse{\equal{#6}{}}{}{Секция: #6\\\nopagebreak}%
}

\definecolor{LovelyBrown}{HTML}{FDFCF5}

\usepackage[pdftex,
breaklinks=true,
bookmarksnumbered=true,
linktocpage=true,
linktoc=all
]{hyperref}

\begin{document}
\pagenumbering{gobble}
\pagestyle{plain}
\pagecolor{LovelyBrown}
\begin{talk}
{Об усиленной лемме Бореля--Кантелли и динамических системах}
{Фролов Андрей Николаевич}
{Санкт-Петербургский государственный университет}
{a.frolov@spbu.ru}
{}
{Теория вероятностей} % [6] название секции

Пусть $\{A_n\}$-- последовательность событий таких, что ряд из их
вероятностей расходится. Пусть $S_n$-- сумма индикаторов первых
$n$ событий из этой последовательности и $E_n = E S_n$. Сумму
$S_n$ можно рассматривать как число успехов в первых n зависимых
испытаниях с меняющейся вероятностью успеха. Таким образом,
мы имеем дело с обобщением схемы Бернулли. Естественными задачами
для рассматриваемой схемы являются получение усиленного закона
больших чисел и оценивание скорости сходимости в нем.
При этом сумму $S_n$ центрируют ее средним и нормируют некоторой
функцией от него. Подобные результаты называют усиленными формами
леммы Бореля--Кантелли. Если в качестве вероятностного
пространства выступает пространство с мерой и сохраняющим эту
меру преобразованием, то подобные результаты называют
динамическими вариантами леммы Бореля--Кантелли. Они дают
представление о статистических свойствах соответствующих
динамических систем. В докладе обсуждаются варианты усиленной
леммы Бореля--Кантелли и их применения к динамическим системам со
степенным убыванием корреляций (не обязательно суммируемых).
\end{talk}
\end{document}