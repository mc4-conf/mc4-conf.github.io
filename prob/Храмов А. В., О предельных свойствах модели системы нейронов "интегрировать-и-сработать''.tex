\documentclass[12pt]{article}
\usepackage{hyphsubst}
\usepackage[T2A]{fontenc}
\usepackage[english,main=russian]{babel}
\usepackage[utf8]{inputenc}
\usepackage[letterpaper,top=2cm,bottom=2cm,left=2cm,right=2cm,marginparwidth=2cm]{geometry}
\usepackage{float}
\usepackage{mathtools, commath, amssymb, amsthm}
\usepackage{enumitem, tabularx,graphicx,url,xcolor,rotating,multicol,epsfig,colortbl,lipsum}

\setlist{topsep=1pt, itemsep=0em}
\setlength{\parindent}{0pt}
\setlength{\parskip}{6pt}

\usepackage{hyphenat}
\hyphenation{ма-те-ма-ти-ка вос-ста-нав-ли-вать}

\usepackage[math]{anttor}

\newenvironment{talk}[6]{%
\vskip 0pt\nopagebreak%
\vskip 0pt\nopagebreak%
\section*{#1}
\phantomsection
\addcontentsline{toc}{section}{#2. \textit{#1}}
% \addtocontents{toc}{\textit{#1}\par}
\textit{#2}\\\nopagebreak%
#3\\\nopagebreak%
\ifthenelse{\equal{#4}{}}{}{\url{#4}\\\nopagebreak}%
\ifthenelse{\equal{#5}{}}{}{Соавторы: #5\\\nopagebreak}%
\ifthenelse{\equal{#6}{}}{}{Секция: #6\\\nopagebreak}%
}

\definecolor{LovelyBrown}{HTML}{FDFCF5}

\usepackage[pdftex,
breaklinks=true,
bookmarksnumbered=true,
linktocpage=true,
linktoc=all
]{hyperref}

\begin{document}
\pagenumbering{gobble}
\pagestyle{plain}
\pagecolor{LovelyBrown}
\begin{talk}
{О предельных свойствах модели системы нейронов ``ин\-те\-гри\-ро\-вать-и-сра\-бо\-тать''}
{Храмов Александр Вадимович}
{ИМ СО РАН}
{a.khramov@g.nsu.ru}
{}
{Теория вероятностей} %

В работе изучается многомерный случайный процесс в непрерывном времени
\[ Z(t)=(Z_1(t),\ldots,Z_N(t)),\ t>0, \]
где \(N\ -\) число нейронов, \(Z_i\ -\) положительный потенциал \(i\)-го нейрона и процесс \(Z(t)\) непрерывен справа и имеет конечные пределы слева. С течением времени потенциалы нейронов линейно снижаются, а при достижении нуля нейрон посылает сигналы всем остальным нейронам.

В предшествующих работах (см. [1, 2, 3]) подробно изучен случай, когда все сигналы между нейронами неотрицательны. Случай с наличием отрицательных связей рассмотрен лишь с жёсткими ограничениями на взаимосвязь между нейронами ([4]).

В данной работе рассмотрен случай двух нейронов, когда один из них (ингибитор) посылает неотрицательные импульсы, а другой (возбудитель) --- неположительные. Рассматривается положительная возвратность процесса.

\medskip

\begin{enumerate}
\item[{[1]}] Cottrell M., {\it Mathematical analysis of a neural network with inhibitory coupling},\\ Stochastic Processes and their applications, 40.1 (1992), 103-126.
\item[{[2]}] Karpelevich F., Malyshev V. A., Rybko A. N., {\it Stochastic evolution of neural networks},\\ Markov Processes and Related Fields, 1.1 (1995), 141-161.
\item[{[3]}] Prasolov T. V., {\it Stochastic stability of a system of perfect integrate-and-fire inhibitory neurons}, Сибирские электронные математические известия. 17.0 (2020), 971-987.
\item[{[4]}] Cottrell M., Turova T. S., {\it Use of an hourglass model in neuronal coding}, Journal of applied probability, 37.1 (2000), 168-186.
\end{enumerate}
\end{talk}
\end{document}