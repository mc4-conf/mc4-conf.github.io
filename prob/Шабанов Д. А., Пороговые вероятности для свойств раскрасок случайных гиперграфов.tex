\documentclass[12pt]{article}
\usepackage{hyphsubst}
\usepackage[T2A]{fontenc}
\usepackage[english,main=russian]{babel}
\usepackage[utf8]{inputenc}
\usepackage[letterpaper,top=2cm,bottom=2cm,left=2cm,right=2cm,marginparwidth=2cm]{geometry}
\usepackage{float}
\usepackage{mathtools, commath, amssymb, amsthm}
\usepackage{enumitem, tabularx,graphicx,url,xcolor,rotating,multicol,epsfig,colortbl,lipsum}

\setlist{topsep=1pt, itemsep=0em}
\setlength{\parindent}{0pt}
\setlength{\parskip}{6pt}

\usepackage{hyphenat}
\hyphenation{ма-те-ма-ти-ка вос-ста-нав-ли-вать}

\usepackage[math]{anttor}

\newenvironment{talk}[6]{%
\vskip 0pt\nopagebreak%
\vskip 0pt\nopagebreak%
\section*{#1}
\phantomsection
\addcontentsline{toc}{section}{#2. \textit{#1}}
% \addtocontents{toc}{\textit{#1}\par}
\textit{#2}\\\nopagebreak%
#3\\\nopagebreak%
\ifthenelse{\equal{#4}{}}{}{\url{#4}\\\nopagebreak}%
\ifthenelse{\equal{#5}{}}{}{Соавторы: #5\\\nopagebreak}%
\ifthenelse{\equal{#6}{}}{}{Секция: #6\\\nopagebreak}%
}

\definecolor{LovelyBrown}{HTML}{FDFCF5}

\usepackage[pdftex,
breaklinks=true,
bookmarksnumbered=true,
linktocpage=true,
linktoc=all
]{hyperref}

\begin{document}
\pagenumbering{gobble}
\pagestyle{plain}
\pagecolor{LovelyBrown}
\begin{talk}
{Пороговые вероятности для свойств раскрасок случайных гиперграфов}
{Шабанов Дмитрий Александрович}
{МФТИ}
{}
{}
{Теория вероятностей} %

Доклад посвящен исследованию пороговых вероятностей для свойств раскрасок случайных гиперграфов в классической биномиальной модели \(H(n,k,p)\). Данная модель представляет собой схему Бернулли на множестве \(k\)-подмножеств \(n\)-элементного множества (вершин): каждое такое подмножество включается в \(H(n,k,p)\) в качестве ребра независимо от других с вероятностью \(p\in(0,1)\). Мы рассматриваем ситуацию, когда \(k\ge 2\) фиксировано, \(n\to\infty\), а \(p=p(n)\) некоторым образом зависит от \(n\). Цель работы --- поиск точных пороговых вероятностей для свойств дробной \((a:b)\)-раскрашивамости случайного гиперграфа \(H(n,k,p)\). Напомним, что для фиксированных целых чисел \(a>b\ge 1\) функция \(\widehat{p}_{a,b}=\widehat{p}_{a,b}(n)\) является точной пороговой вероятностью для свойств дробной \((a:b)\)-раскрашивамости в модели \(H(n,k,p)\), если для любого \(\varepsilon>0\) выполнено
\[\lim_{n\to\infty}P(H(n,k,p)\mbox{ --- \((a:b)\)-раскрашиваем})=\begin{cases}1,& \forall n p(n)\le (1-\varepsilon)\widehat{p}_{a,b}(n),\\ 0,& \forall n p(n)\ge (1+\varepsilon)\widehat{p}_{a,b}(n).\end{cases}\]
В докладе мы представим ряд результатов об оценках \(\widehat{p}_{a,b}\) для некоторых значений пар \((a,b)\).
\end{talk}
\end{document}