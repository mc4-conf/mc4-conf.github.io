\documentclass[12pt]{article}
\usepackage{hyphsubst}
\usepackage[T2A]{fontenc}
\usepackage[english,main=russian]{babel}
\usepackage[utf8]{inputenc}
\usepackage[letterpaper,top=2cm,bottom=2cm,left=2cm,right=2cm,marginparwidth=2cm]{geometry}
\usepackage{float}
\usepackage{mathtools, commath, amssymb, amsthm}
\usepackage{enumitem, tabularx,graphicx,url,xcolor,rotating,multicol,epsfig,colortbl,lipsum}

\setlist{topsep=1pt, itemsep=0em}
\setlength{\parindent}{0pt}
\setlength{\parskip}{6pt}

\usepackage{hyphenat}
\hyphenation{ма-те-ма-ти-ка вос-ста-нав-ли-вать}

\usepackage[math]{anttor}

\newenvironment{talk}[6]{%
\vskip 0pt\nopagebreak%
\vskip 0pt\nopagebreak%
\section*{#1}
\phantomsection
\addcontentsline{toc}{section}{#2. \textit{#1}}
% \addtocontents{toc}{\textit{#1}\par}
\textit{#2}\\\nopagebreak%
#3\\\nopagebreak%
\ifthenelse{\equal{#4}{}}{}{\url{#4}\\\nopagebreak}%
\ifthenelse{\equal{#5}{}}{}{Соавторы: #5\\\nopagebreak}%
\ifthenelse{\equal{#6}{}}{}{Секция: #6\\\nopagebreak}%
}

\definecolor{LovelyBrown}{HTML}{FDFCF5}

\usepackage[pdftex,
breaklinks=true,
bookmarksnumbered=true,
linktocpage=true,
linktoc=all
]{hyperref}

\begin{document}
\pagenumbering{gobble}
\pagestyle{plain}
\pagecolor{LovelyBrown}
\begin{talk}
{Существование марковского равновесия для игр среднего поля}
{Шатилович Дмитрий Вячеславович}
{МГУ имени М.\,В.Ломоносова}
{dmitriy.shatilovich@math.msu.ru}
{}
{Теория вероятностей} %

Задача оптимального управления, которая соответствует исследованию марковского равновесия в игре среднего поля с неуправляемой диффузией и линейным по управлению сносом, имеет вид
\[
\begin{cases}
J(\alpha(\cdot), \mu_\cdot) = \displaystyle{\mathbb{E}\left[\int_0^Tf\left(t, X_t, \mu_t, \alpha(t, X_t)\right)dt + g(X_T, \mu_T)\right] \longrightarrow \inf_{\alpha(\cdot)\in\mathcal{A}}},\\
dX_t = \sigma\left(t, X_t, \mu_t\right)dW_t +  \left(b\left(t, X_t, \mu_t,\right) + \alpha(t, X_t)\right)dt,\\
\mu_t = \operatorname{Low}(X_t).
\end{cases}
\]

Игры среднего поля были введены независимо в работах [1], [2] и заключались в исследовании определенного класса стохастических дифференциальных игр с большим числом игроков. В играх среднего поля каждый игрок учитывает поведение других агентов через эмпирическое распределение. В литературе по статистической физике это распределение принято называть средним полем.

Три основных вопроса, которые рассматриваются в теории игр среднего поля, касаются существования решения, единственности решения и сходимости равновесия в случае конечного числа игроков. Доклад будет посвящен вопросу существования решения. Существование марковского равновесия для игр среднего поля было доказано в работе [3]. В докладе будет рассмотрен новый метод исследования игр среднего поля, позволяющий обосновать существование марковского равновесия при предположениях, более слабых, чем в работе [3]. Так условие липшицевости коэффициентов уравнения будет заменено условием непрерывности. Ключевую роль в данном подходе играют уравнения Колмогорова, которые подробно исследовались в работе [4].

\medskip

\begin{enumerate}
\item[{[1]}] M. Huang, R. Malham\'e, P. Caines, Large population stochastic dynamic games: closed-loop McKean–Vlasov systems and the Nash certainty equivalence principle, Commun. Inf. Syst. 6 (3) (2006) 221–252.
\item[{[2]}] J.M. Lasry, P.L. Lions, Mean field games, Jpn. J. Math. 2 (2007) 229–260.
\item[{[3]}] Lacker, D. Mean field games via controlled martingale problems: existence of
markovian equilibria. Stochastic Processes and their Applications, 125(7):2856–2894.
\item[{[4]}] V.\,I.~Bogachev, M.~R\"ockner, S.\,V.~Shaposhnikov. Uniqueness Problems for Degenerate Fokker--Planck--Kolmogorov Equations. Journal of Mathematical Sciences,
vol. 207, p. 147--165.
\end{enumerate}
\end{talk}
\end{document}