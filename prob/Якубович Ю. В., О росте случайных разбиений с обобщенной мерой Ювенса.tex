\documentclass[12pt]{article}
\usepackage{hyphsubst}
\usepackage[T2A]{fontenc}
\usepackage[english,main=russian]{babel}
\usepackage[utf8]{inputenc}
\usepackage[letterpaper,top=2cm,bottom=2cm,left=2cm,right=2cm,marginparwidth=2cm]{geometry}
\usepackage{float}
\usepackage{mathtools, commath, amssymb, amsthm}
\usepackage{enumitem, tabularx,graphicx,url,xcolor,rotating,multicol,epsfig,colortbl,lipsum}

\setlist{topsep=1pt, itemsep=0em}
\setlength{\parindent}{0pt}
\setlength{\parskip}{6pt}

\usepackage{hyphenat}
\hyphenation{ма-те-ма-ти-ка вос-ста-нав-ли-вать}

\usepackage[math]{anttor}

\newenvironment{talk}[6]{%
\vskip 0pt\nopagebreak%
\vskip 0pt\nopagebreak%
\section*{#1}
\phantomsection
\addcontentsline{toc}{section}{#2. \textit{#1}}
% \addtocontents{toc}{\textit{#1}\par}
\textit{#2}\\\nopagebreak%
#3\\\nopagebreak%
\ifthenelse{\equal{#4}{}}{}{\url{#4}\\\nopagebreak}%
\ifthenelse{\equal{#5}{}}{}{Соавторы: #5\\\nopagebreak}%
\ifthenelse{\equal{#6}{}}{}{Секция: #6\\\nopagebreak}%
}

\definecolor{LovelyBrown}{HTML}{FDFCF5}

\usepackage[pdftex,
breaklinks=true,
bookmarksnumbered=true,
linktocpage=true,
linktoc=all
]{hyperref}

\begin{document}
\pagenumbering{gobble}
\pagestyle{plain}
\pagecolor{LovelyBrown}
\begin{talk}
{О росте случайных разбиений с обобщенной мерой Ювенса}
{Якубович Юрий Владимирович}
{Санкт-Петербургский государственный университет}
{y.yakubovich@spbu.ru}
{}
{Теория вероятностей}

Мерой Ювенса с параметром \(\theta>0\) называют вероятностную меру на разбиениях целого числа \(n\), получаемую как образ
вероятностной меры на перестановках \(n\) объектов, для которой мера пропорциональна параметру \(\theta\) в степени количества
циклов,  при отображении, сопоставляющей перестановке разбиение на длины ее циклов. Под обобщенной мерой Ювенса мы понимаем
меру, параметризованную последовательностью неотрицательных чисел \((\theta_k)_{k\ge 1}\), где вес \(\theta_k\) отвечает циклам
длины \(k\). Эти меры активно изучаются последнее время, см. [1, 2] и ссылки в этих статьях.
В докладе описывается явная конструкция построения случайных разбиений путем последовательного марковского добавления в него
слагаемых, начиная с пустого разбиения нуля, со следующим свойством: при условии, что на некотором шаге получилось разбиение
числа \(n\), это будет случайное разбиение с обобщенным распределением Ювенса. Приводятся точные и асимптотические результаты
об этой модели, а также некоторые следствия из них.

\medskip

\begin{enumerate}
\item[{[1]}] N. M. Ercolani,  D. Ueltschi, {\it Cycle structure of random permutations with cycle weights}, Random Struct. Alg., 44 (2014), 109--133.
\item[{[2]}] А. Л. Якымив, {\it О порядке случайной подстановки с весами циклов}, Теория вероятн. и ее примен., 63:2 (2018), 261--283.
\end{enumerate}
\end{talk}
\end{document}