\documentclass[12pt]{article}
\usepackage{hyphsubst}
\usepackage[T2A]{fontenc}
\usepackage[english,main=russian]{babel}
\usepackage[utf8]{inputenc}
\usepackage[letterpaper,top=2cm,bottom=2cm,left=2cm,right=2cm,marginparwidth=2cm]{geometry}
\usepackage{float}
\usepackage{mathtools, commath, amssymb, amsthm}
\usepackage{enumitem, tabularx,graphicx,url,xcolor,rotating,multicol,epsfig,colortbl,lipsum}

\setlist{topsep=1pt, itemsep=0em}
\setlength{\parindent}{0pt}
\setlength{\parskip}{6pt}

\usepackage{hyphenat}
\hyphenation{ма-те-ма-ти-ка вос-ста-нав-ли-вать}

\usepackage[math]{anttor}

\newenvironment{talk}[6]{%
\vskip 0pt\nopagebreak%
\vskip 0pt\nopagebreak%
\section*{#1}
\phantomsection
\addcontentsline{toc}{section}{#2. \textit{#1}}
% \addtocontents{toc}{\textit{#1}\par}
\textit{#2}\\\nopagebreak%
#3\\\nopagebreak%
\ifthenelse{\equal{#4}{}}{}{\url{#4}\\\nopagebreak}%
\ifthenelse{\equal{#5}{}}{}{Соавторы: #5\\\nopagebreak}%
\ifthenelse{\equal{#6}{}}{}{Секция: #6\\\nopagebreak}%
}

\definecolor{LovelyBrown}{HTML}{FDFCF5}

\usepackage[pdftex,
breaklinks=true,
bookmarksnumbered=true,
linktocpage=true,
linktoc=all
]{hyperref}

\begin{document}
\pagenumbering{gobble}
\pagestyle{plain}
\pagecolor{LovelyBrown}
\begin{talk}
{Спектральные методы и их применения в стохастическом анализе}
{Яровая Елена Борисовна}
{МГУ имени М.\,В. Ломоносова, механико-математический факультет, кафедра теории вероятностей,
МИАН им. В. А. Стеклова РАН}
{elena.yarovaya@math.msu.ru}
{}
{Теория вероятностей} %

Развитие спектральной техники позволит получить предельные теоремы о численностях частиц в ветвящемся случайном блуждании по точкам многомерной решетки в предположении существования  источников ветвления (т.е. точек решетки, в которых возможны размножение и гибель частиц) как с положительной, так и с  отрицательной интенсивностью ветвления. Будут представлены результаты о связи между структурой спектра эволюционного оператора и геометрическим расположением источников ветвления на многомерной решетке. Как правило, в более ранних исследованиях  случайное блуждание, лежащее в основе процесса предполагалось симметричным  Показано, что полученные результаты остаются справедливыми при замене условия самосопряженности оператора, задающего случайное блуждание, на более слабое условие подобия самосопряженному. Таким образом, c привлечением  спектральной техники решены задачи, связанные с многоточечными возмущениями операторов, возникающих в эволюционных уравнениях для первых моментов численностей частиц в многотипных ветвящихся случайных блужданиях и доказан ряд новых предельных теорем о поведении популяций и субпопуляций частиц в ветвящемся случайном блуждании.

\medskip

Работа выполнена при поддержке РНФ, проект 23-11-00375, в Математическом институте им. В. А. Стеклова РАН.
\end{talk}
\end{document}