\documentclass[12pt]{article}
\usepackage{hyphsubst}
\usepackage[T2A]{fontenc}
\usepackage[english,main=russian]{babel}
\usepackage[utf8]{inputenc}
\usepackage[letterpaper,top=2cm,bottom=2cm,left=2cm,right=2cm,marginparwidth=2cm]{geometry}
\usepackage{float}
\usepackage{mathtools, commath, amssymb, amsthm}
\usepackage{enumitem, tabularx,graphicx,url,xcolor,rotating,multicol,epsfig,colortbl,lipsum}

\setlist{topsep=1pt, itemsep=0em}
\setlength{\parindent}{0pt}
\setlength{\parskip}{6pt}

\usepackage{hyphenat}
\hyphenation{ма-те-ма-ти-ка вос-ста-нав-ли-вать}

\usepackage[math]{anttor}

\newenvironment{talk}[6]{%
\vskip 0pt\nopagebreak%
\vskip 0pt\nopagebreak%
\section*{#1}
\phantomsection
\addcontentsline{toc}{section}{#2. \textit{#1}}
% \addtocontents{toc}{\textit{#1}\par}
\textit{#2}\\\nopagebreak%
#3\\\nopagebreak%
\ifthenelse{\equal{#4}{}}{}{\url{#4}\\\nopagebreak}%
\ifthenelse{\equal{#5}{}}{}{Соавторы: #5\\\nopagebreak}%
\ifthenelse{\equal{#6}{}}{}{Секция: #6\\\nopagebreak}%
}

\definecolor{LovelyBrown}{HTML}{FDFCF5}

\usepackage[pdftex,
breaklinks=true,
bookmarksnumbered=true,
linktocpage=true,
linktoc=all
]{hyperref}

\begin{document}
\pagenumbering{gobble}
\pagestyle{plain}
\pagecolor{LovelyBrown}
\begin{talk}
{Двойственные фреймы всплесков на множестве \(M\)-по\-ло\-жи\-тель\-ных векторов}
{Бабушкин Максим Владимирович}
{СПбГУ, кафедра высшей математики; Университет ИТМО, научно-образовательный центр математики}
{m.v.babushkin@yandex.ru}
{Скопина Мария Александровна}
{Вещественный и функциональный анализ} %

Множество \(M\)-положительных векторов является многомерным аналогом положительной полупрямой. Это пространство порождается матрицей \(M\) и снабжается операцией сложения по модулю \(M\). На основе функций Уолша можно построить гармонический анализ на этом пространстве, аналогичный анализу Уолша на полупрямой. Особенностью этой теории является наличие класса функций с компактным носителем (``тест-функций''), преобразование Фурье которых также имеет компактный носитель. В рамках нашего исследования предложен способ построения двойственных фреймов всплесков, состоящих из тест-функций. Приводится несколько простых конкретных примеров таких систем всплесков.
\end{talk}
\end{document}