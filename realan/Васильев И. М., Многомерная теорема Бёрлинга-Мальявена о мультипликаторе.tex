\documentclass[12pt]{article}
\usepackage{hyphsubst}
\usepackage[T2A]{fontenc}
\usepackage[english,main=russian]{babel}
\usepackage[utf8]{inputenc}
\usepackage[letterpaper,top=2cm,bottom=2cm,left=2cm,right=2cm,marginparwidth=2cm]{geometry}
\usepackage{float}
\usepackage{mathtools, commath, amssymb, amsthm}
\usepackage{enumitem, tabularx,graphicx,url,xcolor,rotating,multicol,epsfig,colortbl,lipsum}

\setlist{topsep=1pt, itemsep=0em}
\setlength{\parindent}{0pt}
\setlength{\parskip}{6pt}

\usepackage{hyphenat}
\hyphenation{ма-те-ма-ти-ка вос-ста-нав-ли-вать}

\usepackage[math]{anttor}

\newenvironment{talk}[6]{%
\vskip 0pt\nopagebreak%
\vskip 0pt\nopagebreak%
\section*{#1}
\phantomsection
\addcontentsline{toc}{section}{#2. \textit{#1}}
% \addtocontents{toc}{\textit{#1}\par}
\textit{#2}\\\nopagebreak%
#3\\\nopagebreak%
\ifthenelse{\equal{#4}{}}{}{\url{#4}\\\nopagebreak}%
\ifthenelse{\equal{#5}{}}{}{Соавторы: #5\\\nopagebreak}%
\ifthenelse{\equal{#6}{}}{}{Секция: #6\\\nopagebreak}%
}

\definecolor{LovelyBrown}{HTML}{FDFCF5}

\usepackage[pdftex,
breaklinks=true,
bookmarksnumbered=true,
linktocpage=true,
linktoc=all
]{hyperref}

\begin{document}
\pagenumbering{gobble}
\pagestyle{plain}
\pagecolor{LovelyBrown}
\begin{talk}
{Многомерная теорема Бёрлинга--Мальявена о мультипликаторе}
{Васильев Иоанн Михайлович}
{Université Paris Cergy и ПОМИ РАН}
{ioann.vasilyev@cyu.fr}
{}
{Вещественный и функциональный анализ} %

Доклад будет посвящен новому многомерному обобщению теоремы о Бёрлинга и Мальявена о мультипликаторе. Более подробно, мы увидим, как получить новое достаточное условие на то, чтобы радиальная функция являлась мажорантой Бёрлинга и Мальявена в многомерном случае (это означает, что рассматриваемая функция может быть оценена снизу ненулевой, квадратично интегрируемой функцией, которая имеет носитель преобразования Фурье, заключенный в шаре произвольно малого радиуса). Мы также объясним, как отсюда вывести новое точное достаточное условие и в нерадиальном случае. Наши результаты дают частичный ответ на вопрос, поставленный Л. Хёрмандером. Если позволит время, то мы также обсудим некоторые связанные одномерные результаты. Доклад будет основан на результатах статей [1], [2] и [3].

\medskip

\begin{enumerate}
\item[{[1]}] I. Vasilyev, \emph{On the multidimensional Nazarov lemma}, Proceedings of American Ma\-the\-ma\-ti\-cal Society, 11 p., (2022) (DOI: https://doi.org/10.1090/proc/15805).
\item[{[2]}] I. Vasilyev, \emph{A generalization of the First Beurling--Malliavin theorem}, (2022), 16 p., to appear in Analysis and PDE, preprint: \texttt{https://arxiv.org/pdf/2109.04123.pdf},\\\texttt{https://msp.org/soon/coming.php?jpath=apde}.
\item[{[3]}] I. Vasilyev, \emph{The Beurling and Malliavin Theorem in Several Dimensions}, preprint: \texttt{https://arxiv.org/pdf/2306.12397.pdf}.
\end{enumerate}
\end{talk}
\end{document}