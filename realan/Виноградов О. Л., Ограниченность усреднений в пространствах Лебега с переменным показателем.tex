\documentclass[12pt]{article}
\usepackage{hyphsubst}
\usepackage[T2A]{fontenc}
\usepackage[english,main=russian]{babel}
\usepackage[utf8]{inputenc}
\usepackage[letterpaper,top=2cm,bottom=2cm,left=2cm,right=2cm,marginparwidth=2cm]{geometry}
\usepackage{float}
\usepackage{mathtools, commath, amssymb, amsthm}
\usepackage{enumitem, tabularx,graphicx,url,xcolor,rotating,multicol,epsfig,colortbl,lipsum}

\setlist{topsep=1pt, itemsep=0em}
\setlength{\parindent}{0pt}
\setlength{\parskip}{6pt}

\usepackage{hyphenat}
\hyphenation{ма-те-ма-ти-ка вос-ста-нав-ли-вать}

\usepackage[math]{anttor}

\newenvironment{talk}[6]{%
\vskip 0pt\nopagebreak%
\vskip 0pt\nopagebreak%
\section*{#1}
\phantomsection
\addcontentsline{toc}{section}{#2. \textit{#1}}
% \addtocontents{toc}{\textit{#1}\par}
\textit{#2}\\\nopagebreak%
#3\\\nopagebreak%
\ifthenelse{\equal{#4}{}}{}{\url{#4}\\\nopagebreak}%
\ifthenelse{\equal{#5}{}}{}{Соавторы: #5\\\nopagebreak}%
\ifthenelse{\equal{#6}{}}{}{Секция: #6\\\nopagebreak}%
}

\definecolor{LovelyBrown}{HTML}{FDFCF5}

\usepackage[pdftex,
breaklinks=true,
bookmarksnumbered=true,
linktocpage=true,
linktoc=all
]{hyperref}

\begin{document}
\pagenumbering{gobble}
\pagestyle{plain}
\pagecolor{LovelyBrown}
\begin{talk}
{Ограниченность усреднений в пространствах Лебега с переменным показателем}
{Виноградов Олег Леонидович}
{Санкт-Петербургский государственный университет}
{olvin@math.spbu.ru}
{}
{Вещественный и функциональный анализ} %

Если нормированное пространство~\(X\), состоящее из
заданных на~\(\mathbb{R}^n\) функций, вместе с
каждой функцией содержит ее средние Стеклова \(S_hf\) и
\(\sup_{h>0}\|S_h\|<+\infty\), то \(X\)~называется
пространством с ограниченнным усреднением. Ранее автором
были установлены прямые и обратные теоремы
теории приближений тригонометрическими многочленами и целыми функциями
конечной степени в банаховых идеальных пространствах с ограниченным
усреднением. Эти теоремы во многом аналогичны таковым в обычных пространствах~\(L_p\).
Ограниченности максимального оператора в этих вопросах
не требуется. Особая роль средних Стеклова состоит
в том, что их ограниченность влечет ограниченность сверток с любыми ядрами,
имеющими суммируемую горбатую мажоранту.
Единственный известный критерий ограниченности усреднений в пространствах, не инвариантных относительно сдвига, относится к весовым пространствам Лебега: ограниченность усреднений равносильна условию Макенхаупта.
В других случаях известные достаточные условия не совпадают с
необходимыми. В докладе обсуждаются условия ограниченности усреднений в пространствах
Лебега с переменным показателем.
\end{talk}
\end{document}