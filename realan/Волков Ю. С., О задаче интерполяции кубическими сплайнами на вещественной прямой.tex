\documentclass[12pt]{article}
\usepackage{hyphsubst}
\usepackage[T2A]{fontenc}
\usepackage[english,main=russian]{babel}
\usepackage[utf8]{inputenc}
\usepackage[letterpaper,top=2cm,bottom=2cm,left=2cm,right=2cm,marginparwidth=2cm]{geometry}
\usepackage{float}
\usepackage{mathtools, commath, amssymb, amsthm}
\usepackage{enumitem, tabularx,graphicx,url,xcolor,rotating,multicol,epsfig,colortbl,lipsum}

\setlist{topsep=1pt, itemsep=0em}
\setlength{\parindent}{0pt}
\setlength{\parskip}{6pt}

\usepackage{hyphenat}
\hyphenation{ма-те-ма-ти-ка вос-ста-нав-ли-вать}

\usepackage[math]{anttor}

\newenvironment{talk}[6]{%
\vskip 0pt\nopagebreak%
\vskip 0pt\nopagebreak%
\section*{#1}
\phantomsection
\addcontentsline{toc}{section}{#2. \textit{#1}}
% \addtocontents{toc}{\textit{#1}\par}
\textit{#2}\\\nopagebreak%
#3\\\nopagebreak%
\ifthenelse{\equal{#4}{}}{}{\url{#4}\\\nopagebreak}%
\ifthenelse{\equal{#5}{}}{}{Соавторы: #5\\\nopagebreak}%
\ifthenelse{\equal{#6}{}}{}{Секция: #6\\\nopagebreak}%
}

\definecolor{LovelyBrown}{HTML}{FDFCF5}

\usepackage[pdftex,
breaklinks=true,
bookmarksnumbered=true,
linktocpage=true,
linktoc=all
]{hyperref}

\begin{document}
\pagenumbering{gobble}
\pagestyle{plain}
\pagecolor{LovelyBrown}
\begin{talk}
{О задаче интерполяции кубическими сплайнами на вещественной прямой}
{Волков Юрий Степанович}
{Институт математики им.\,С.\,Л. Соболева СО РАН}
{e2vol@ya.ru}
{}
{Вещественный и функциональный анализ} %

Интерполяция сплайнами с равномерными узлами на всей вещественной прямой (так называемая кардинальная интерполяция) достаточно хорошо изучена.
Если же сетка точек интерполяции бесконечная и неравномерная, то таких исследований немного. К.~де~Бор~(1976) доказал существование единственного ограниченного сплайна для ограниченных данных
при условии ограниченности отношения наибольшего и наименьшего шагов сетки, а в случае кубических сплайнов установил разрешимость этой задачи при ограничении на отношение соседних шагов сетки.
Также в терминах ограничений на отношения соседних шагов сетки Фридланд и Миккелли (1978) рассмотрели задачу существования и единственности ограниченного сплайна произвольной степени.
Некоторые достаточные условия существования и единственности сплайна полиномиального роста при интерполировании данных полиномиального роста были получены Якимовским (1984), и это также было сделано при некоторых ограничениях на сетку.

Нами рассмотрена задача интерполяции кубическими сплайнами функций линейного и квадратичного роста на произвольных неравномерных сетках на всей вещественной оси.
Установлено [1], что в этом случае всегда существует единственный кубический сплайн линейного или квадратичного роста соответственно, причём ограничения на сетку не требуются.
Кроме того, мы приводим оценки погрешности на классах интерполируемых функций \(W_{\infty}^{4}(\mathbb{R})\) и при этом оказывается, что оценки для бесконечных сеток на оси совпадают с известными оценками погрешности в случае конечного отрезка.

Исследование основано на изучении решений систем линейных алгебраических уравнений с бесконечными в обе стороны матрицами коэффициентов.

\medskip

\begin{enumerate}
\item[{[1]}] Ю. С. Волков, С. И. Новиков, {\it Оценки решений бесконечных систем линейных уравнений и задача интерполяции кубическими сплайнами на прямой}, Сиб. матем. журн., 63 (2022), 814--830.
\end{enumerate}
\end{talk}
\end{document}