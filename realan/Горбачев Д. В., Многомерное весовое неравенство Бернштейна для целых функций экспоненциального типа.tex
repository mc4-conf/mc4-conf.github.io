\documentclass[12pt]{article}
\usepackage{hyphsubst}
\usepackage[T2A]{fontenc}
\usepackage[english,main=russian]{babel}
\usepackage[utf8]{inputenc}
\usepackage[letterpaper,top=2cm,bottom=2cm,left=2cm,right=2cm,marginparwidth=2cm]{geometry}
\usepackage{float}
\usepackage{mathtools, commath, amssymb, amsthm}
\usepackage{enumitem, tabularx,graphicx,url,xcolor,rotating,multicol,epsfig,colortbl,lipsum}

\setlist{topsep=1pt, itemsep=0em}
\setlength{\parindent}{0pt}
\setlength{\parskip}{6pt}

\usepackage{hyphenat}
\hyphenation{ма-те-ма-ти-ка вос-ста-нав-ли-вать}

\usepackage[math]{anttor}

\newenvironment{talk}[6]{%
\vskip 0pt\nopagebreak%
\vskip 0pt\nopagebreak%
\section*{#1}
\phantomsection
\addcontentsline{toc}{section}{#2. \textit{#1}}
% \addtocontents{toc}{\textit{#1}\par}
\textit{#2}\\\nopagebreak%
#3\\\nopagebreak%
\ifthenelse{\equal{#4}{}}{}{\url{#4}\\\nopagebreak}%
\ifthenelse{\equal{#5}{}}{}{Соавторы: #5\\\nopagebreak}%
\ifthenelse{\equal{#6}{}}{}{Секция: #6\\\nopagebreak}%
}

\definecolor{LovelyBrown}{HTML}{FDFCF5}

\usepackage[pdftex,
breaklinks=true,
bookmarksnumbered=true,
linktocpage=true,
linktoc=all
]{hyperref}

\begin{document}
\pagenumbering{gobble}
\pagestyle{plain}
\pagecolor{LovelyBrown}
\begin{talk}
{Многомерное весовое неравенство Бернштейна для целых функций экспоненциального типа}
{Горбачев Дмитрий Викторович}
{ООО ``Горизонт''}
{dvgmail@mail.ru}
{}
{Вещественный и функциональный анализ} % [6] название секции

Неравенство Бернштейна для целых функций экспоненциального типа является
классическим в теории приближений. В стандартной постановке оно устанавливает
порядок роста $L^{p}$-нормы линейного дифференциального оператора на классе
полиномов или целых функций экспоненциального типа. Интересен случай весовой
$L^{p}$-нормы при $p>0$, задаваемой мерой, удовлетворяющей условию удвоения. Он
разобран, например, для тригонометрических и сферических полиномов. Однако для
целых функций экспоненциального типа, особенно в многомерной постановке,
известно меньше. Целью доклада будет привести новые результаты в этом
направлении. Особенно интересны весовые неравенства, отвечающие степенным весам
Данкля и $(\kappa,a)$-обобщенного преобразования Фурье, где есть
дифференциально-разностные аналоги классических дифференциальных операторов.
Одно из приложений данных результатов состоит в доказательстве обратных теорем
теории приближений в соответствующих весовых пространствах $L^{p}$.
\end{talk}
\end{document}