\documentclass[12pt]{article}
\usepackage{hyphsubst}
\usepackage[T2A]{fontenc}
\usepackage[english,main=russian]{babel}
\usepackage[utf8]{inputenc}
\usepackage[letterpaper,top=2cm,bottom=2cm,left=2cm,right=2cm,marginparwidth=2cm]{geometry}
\usepackage{float}
\usepackage{mathtools, commath, amssymb, amsthm}
\usepackage{enumitem, tabularx,graphicx,url,xcolor,rotating,multicol,epsfig,colortbl,lipsum}

\setlist{topsep=1pt, itemsep=0em}
\setlength{\parindent}{0pt}
\setlength{\parskip}{6pt}

\usepackage{hyphenat}
\hyphenation{ма-те-ма-ти-ка вос-ста-нав-ли-вать}

\usepackage[math]{anttor}

\newenvironment{talk}[6]{%
\vskip 0pt\nopagebreak%
\vskip 0pt\nopagebreak%
\section*{#1}
\phantomsection
\addcontentsline{toc}{section}{#2. \textit{#1}}
% \addtocontents{toc}{\textit{#1}\par}
\textit{#2}\\\nopagebreak%
#3\\\nopagebreak%
\ifthenelse{\equal{#4}{}}{}{\url{#4}\\\nopagebreak}%
\ifthenelse{\equal{#5}{}}{}{Соавторы: #5\\\nopagebreak}%
\ifthenelse{\equal{#6}{}}{}{Секция: #6\\\nopagebreak}%
}

\definecolor{LovelyBrown}{HTML}{FDFCF5}

\usepackage[pdftex,
breaklinks=true,
bookmarksnumbered=true,
linktocpage=true,
linktoc=all
]{hyperref}

\begin{document}
\pagenumbering{gobble}
\pagestyle{plain}
\pagecolor{LovelyBrown}
\begin{talk}
{Об одном классе интегральных операторов на интервале \((-1, 1)\)}
{Данелян Елена Дмитриевна}
{Южный федеральный университет}
{danelian@sfedu.ru}
{Карапетянц А.\,Н.}
{Вещественный и функциональный анализ} %

Рассматриваются интегральные операторы типа Хаусдорфа на интервале \((-1,1)\), которые естественным образом возникают в некоторых задачах теории интегральных уравнений и математической физики. А именно, при наличии измеримой функции \(k\) (нашего интегрального ядра) на интервале \((-1,1)\) рассматривается интегральный оператор
\[
K_\mu f(x)=\int_{-1}^{1}k(t)f(\varphi_x(t))d\mu(t),
\]
где \(\mu \) -- произвольная положительная мера Радона на \((-1,1)\) и \(\varphi_x(t)=\frac{x-t}{1-xt},\) \(x, t\in (-1,1)\) инволютивный автоморфизм на \((-1,1)\) на весовых пространствах Лебега.

Рассматриваются алгебраические свойства для частного случая изучаемых операторов. Устанавливаются достаточные, а затем и необходимые условия ограниченности операторов в пространствах \(L^p(v)\), в качестве следствий приводятся некоторые важные частные случаи операторов и некоторых весов. Применяется техника операторов с однородными ядрами с получением принципиально иных условий ограниченности, которые, тем не менее, дают схожие результаты в специальных частных случаях.  Строятся аппроксимационные конструкции в рамках обсуждаемых операторов.
\end{talk}
\end{document}