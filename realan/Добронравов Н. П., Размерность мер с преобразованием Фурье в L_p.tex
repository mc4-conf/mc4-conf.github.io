\documentclass[12pt]{article}
\usepackage{hyphsubst}
\usepackage[T2A]{fontenc}
\usepackage[english,main=russian]{babel}
\usepackage[utf8]{inputenc}
\usepackage[letterpaper,top=2cm,bottom=2cm,left=2cm,right=2cm,marginparwidth=2cm]{geometry}
\usepackage{float}
\usepackage{mathtools, commath, amssymb, amsthm}
\usepackage{enumitem, tabularx,graphicx,url,xcolor,rotating,multicol,epsfig,colortbl,lipsum}

\setlist{topsep=1pt, itemsep=0em}
\setlength{\parindent}{0pt}
\setlength{\parskip}{6pt}

\usepackage{hyphenat}
\hyphenation{ма-те-ма-ти-ка вос-ста-нав-ли-вать}

\usepackage[math]{anttor}

\newenvironment{talk}[6]{%
\vskip 0pt\nopagebreak%
\vskip 0pt\nopagebreak%
\section*{#1}
\phantomsection
\addcontentsline{toc}{section}{#2. \textit{#1}}
% \addtocontents{toc}{\textit{#1}\par}
\textit{#2}\\\nopagebreak%
#3\\\nopagebreak%
\ifthenelse{\equal{#4}{}}{}{\url{#4}\\\nopagebreak}%
\ifthenelse{\equal{#5}{}}{}{Соавторы: #5\\\nopagebreak}%
\ifthenelse{\equal{#6}{}}{}{Секция: #6\\\nopagebreak}%
}

\definecolor{LovelyBrown}{HTML}{FDFCF5}

\usepackage[pdftex,
breaklinks=true,
bookmarksnumbered=true,
linktocpage=true,
linktoc=all
]{hyperref}

\begin{document}
\pagenumbering{gobble}
\pagestyle{plain}
\pagecolor{LovelyBrown}
\begin{talk}
{Размерность мер с преобразованием Фурье в \(L_p\)}
{Добронравов Никита Петрович}
{СПбГУ}
{dobronravov1999@mail.ru}
{}
{Вещественный и функциональный анализ}

Принцип неопределённости в математическом анализе --- это семейство фактов о том, что  функция и её преобразование Фурье не могут быть одновременно малы. Одной из версий это принципа является следующая теорема.

{\bf Теорема.} {\it Пусть \(S\subset \mathbb{R}^d\) --- компакт, такой что \(\mathcal{H}_{\alpha}(S)<\infty\). Пусть обобщённая функция \(\zeta\) такая что \(supp(\zeta)\subset S\) и \(\hat{\zeta}\in L_p(\mathbb{R}^d)\) для некоторого \(p<\frac{2d}{\alpha}\). Тогда \(\zeta=0\).}

Здесь \(\mathcal{H}_{\alpha}\) --- это \(\alpha\)-мера Хаусдорфа.
Мы разобрали, что происходит в предельном случае \(p=\frac{2d}{\alpha}\). Оказалось, что в этом случае принцип неопределённости неверен, а именно удалось доказать следующую теорему:

{\bf Теорема.}
{\it Существуют компакт \(S\subset\mathbb{R}^d\) и такая вероятностная мера \(\mu\), что \(supp(\mu)\subset S\), \(\hat{\mu}\in L_p(\mathbb{R}^d)\) и \(\mathcal{H}_{\frac{2d}{p}}(S)=0\).}
\end{talk}
\end{document}