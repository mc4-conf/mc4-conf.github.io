\documentclass[12pt]{article}
\usepackage{hyphsubst}
\usepackage[T2A]{fontenc}
\usepackage[english,main=russian]{babel}
\usepackage[utf8]{inputenc}
\usepackage[letterpaper,top=2cm,bottom=2cm,left=2cm,right=2cm,marginparwidth=2cm]{geometry}
\usepackage{float}
\usepackage{mathtools, commath, amssymb, amsthm}
\usepackage{enumitem, tabularx,graphicx,url,xcolor,rotating,multicol,epsfig,colortbl,lipsum}

\setlist{topsep=1pt, itemsep=0em}
\setlength{\parindent}{0pt}
\setlength{\parskip}{6pt}

\usepackage{hyphenat}
\hyphenation{ма-те-ма-ти-ка вос-ста-нав-ли-вать}

\usepackage[math]{anttor}

\newenvironment{talk}[6]{%
\vskip 0pt\nopagebreak%
\vskip 0pt\nopagebreak%
\section*{#1}
\phantomsection
\addcontentsline{toc}{section}{#2. \textit{#1}}
% \addtocontents{toc}{\textit{#1}\par}
\textit{#2}\\\nopagebreak%
#3\\\nopagebreak%
\ifthenelse{\equal{#4}{}}{}{\url{#4}\\\nopagebreak}%
\ifthenelse{\equal{#5}{}}{}{Соавторы: #5\\\nopagebreak}%
\ifthenelse{\equal{#6}{}}{}{Секция: #6\\\nopagebreak}%
}

\definecolor{LovelyBrown}{HTML}{FDFCF5}

\usepackage[pdftex,
breaklinks=true,
bookmarksnumbered=true,
linktocpage=true,
linktoc=all
]{hyperref}

\begin{document}
\pagenumbering{gobble}
\pagestyle{plain}
\pagecolor{LovelyBrown}
\begin{talk}
{Сверхгладкие самоподобные B-сплайны}
{Зайцева Татьяна Ивановна}
{МГУ имени М.\,В. Ломоносова}
{zaitsevatanja@gmail.com}
{}
{Вещественный и функциональный анализ} %

Тайлы это самоподобные компакты, порождённые одной матрицей растяжения, т. е. матрицей, у которой все собственные значения по модулю больше единицы.
На основе этих множеств по аналогии с кардинальными B-сплайнами строятся тайловые B-сплайны --- свёртки индикаторов тайлов, при этом многие свойства B-сплайнов сохраняются. В частности, они являются решениями специальных разностных уравнений со сжатием аргумента, что позволяет применять их в прикладных алгоритмах.
Будет представлен новый способ вычисления их гладкости в \(L_2\). По результатам вычислений обнаружилось несколько семейств ``сверхгладких'' сплайнов, гладкость которых превышает гладкость стандартных сплайнов соответствующих порядков.

\medskip

\begin{enumerate}
\item[{[1]}] M. Charina, V. Yu. Protasov, {\it Regularity of anisotropic refinable functions}, Appl. Comput. Harmon. Anal., 47 (2019), 795 – 821.
\item[{[2]}] T. Eirola, {\it Sobolev characterization of solutions of dilation equations}, SIAM J. Math. Anal., 23 (1992), 1015 – 1030.
\item[{[3]}] A. Cohen, I. Daubechies, {\it A new technique to estimate the regularity of refinable functions}, Revista Mathematica Iberoamericana, 12 (1996), 527 – 591.
\end{enumerate}
\end{talk}
\end{document}