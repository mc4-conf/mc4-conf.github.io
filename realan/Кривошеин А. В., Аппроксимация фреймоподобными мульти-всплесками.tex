\documentclass[12pt]{article}
\usepackage{hyphsubst}
\usepackage[T2A]{fontenc}
\usepackage[english,main=russian]{babel}
\usepackage[utf8]{inputenc}
\usepackage[letterpaper,top=2cm,bottom=2cm,left=2cm,right=2cm,marginparwidth=2cm]{geometry}
\usepackage{float}
\usepackage{mathtools, commath, amssymb, amsthm}
\usepackage{enumitem, tabularx,graphicx,url,xcolor,rotating,multicol,epsfig,colortbl,lipsum}

\setlist{topsep=1pt, itemsep=0em}
\setlength{\parindent}{0pt}
\setlength{\parskip}{6pt}

\usepackage{hyphenat}
\hyphenation{ма-те-ма-ти-ка вос-ста-нав-ли-вать}

\usepackage[math]{anttor}

\newenvironment{talk}[6]{%
\vskip 0pt\nopagebreak%
\vskip 0pt\nopagebreak%
\section*{#1}
\phantomsection
\addcontentsline{toc}{section}{#2. \textit{#1}}
% \addtocontents{toc}{\textit{#1}\par}
\textit{#2}\\\nopagebreak%
#3\\\nopagebreak%
\ifthenelse{\equal{#4}{}}{}{\url{#4}\\\nopagebreak}%
\ifthenelse{\equal{#5}{}}{}{Соавторы: #5\\\nopagebreak}%
\ifthenelse{\equal{#6}{}}{}{Секция: #6\\\nopagebreak}%
}

\definecolor{LovelyBrown}{HTML}{FDFCF5}

\usepackage[pdftex,
breaklinks=true,
bookmarksnumbered=true,
linktocpage=true,
linktoc=all
]{hyperref}

\begin{document}
\pagenumbering{gobble}
\pagestyle{plain}
\pagecolor{LovelyBrown}
\begin{talk}
{Аппроксимация фреймоподобными мульти-всплесками}
{Кривошеин Александр Владимирович}
{Санкт-Петербургский государственный университет}
{krivosheinav@gmail.com}
{}
{Вещественный и функциональный анализ} % [6] название секции

Квазипроекционный оператор, порождённый парой вектор-функций $\Phi$, $\widetilde\Phi: {\mathbb R}^d \to {\mathbb C}^r$, имеет вид
$$
Q_j(\Phi, \widetilde\Phi, f) =\sum_{k\in{\mathbb Z}^d} \langle f, \widetilde\Phi_{jk} \rangle \Phi_{jk},
%=\sum_{\nu = 1}^r\sum_{k\in\zd} \langle f, \w\phi_{\nu,j,k} \rangle \phi_{\nu,j,k}
$$
где $\Phi_{jk} = |\det M|^{j/2} \Phi(M^j \cdot + k)$, $j\in{\mathbb Z}$, $k\in {\mathbb Z}^d$, $M$ -- матрица растяжения.
Изучены аппроксимационные свойства таких операторов и получены оценки погрешности в $L_2$-норме для широкого класса таких операторов.

Для масштабирующих вектор-функций $\Phi$, $\widetilde\Phi$ квазипроекционные операторы $Q_j(f,\Phi,\widetilde\Phi)$
связаны с двойственными системами мульти-всплесков. Хотя общая схема построения двойственных фреймов мульти-всплесков в многомерном случае известна, ее реализация на практике является сложной задачей из-за необходимости обеспечения некоторых дополнительных свойств. Предложена конструкция фреймоподобных мульти-всплесков с отказом от фреймовости,  но с сохранением возможности разложения функций аналогичного разложению по фреймам. Это упрощает задачу построения фреймоподобных мульти-всплесков. Установлены аппроксимационные свойства фреймоподобных мульти-всплесков. Предложены алгоритмы построения фреймоподобных мульти-всплесков с заданным порядком аппроксимации.

\medskip
\end{talk}
\end{document}