\documentclass[12pt]{article}
\usepackage{hyphsubst}
\usepackage[T2A]{fontenc}
\usepackage[english,main=russian]{babel}
\usepackage[utf8]{inputenc}
\usepackage[letterpaper,top=2cm,bottom=2cm,left=2cm,right=2cm,marginparwidth=2cm]{geometry}
\usepackage{float}
\usepackage{mathtools, commath, amssymb, amsthm}
\usepackage{enumitem, tabularx,graphicx,url,xcolor,rotating,multicol,epsfig,colortbl,lipsum}

\setlist{topsep=1pt, itemsep=0em}
\setlength{\parindent}{0pt}
\setlength{\parskip}{6pt}

\usepackage{hyphenat}
\hyphenation{ма-те-ма-ти-ка вос-ста-нав-ли-вать}

\usepackage[math]{anttor}

\newenvironment{talk}[6]{%
\vskip 0pt\nopagebreak%
\vskip 0pt\nopagebreak%
\section*{#1}
\phantomsection
\addcontentsline{toc}{section}{#2. \textit{#1}}
% \addtocontents{toc}{\textit{#1}\par}
\textit{#2}\\\nopagebreak%
#3\\\nopagebreak%
\ifthenelse{\equal{#4}{}}{}{\url{#4}\\\nopagebreak}%
\ifthenelse{\equal{#5}{}}{}{Соавторы: #5\\\nopagebreak}%
\ifthenelse{\equal{#6}{}}{}{Секция: #6\\\nopagebreak}%
}

\definecolor{LovelyBrown}{HTML}{FDFCF5}

\usepackage[pdftex,
breaklinks=true,
bookmarksnumbered=true,
linktocpage=true,
linktoc=all
]{hyperref}

\begin{document}
\pagenumbering{gobble}
\pagestyle{plain}
\pagecolor{LovelyBrown}
\begin{talk}
{Односторонние неравенства дискретизации и восстановление по выборке}
{Лимонова Ирина Викторовна}
{Математический институт им. В.\,А. Стеклова Российской академии наук, \\МЦМУ МИАН}
{limonova_irina@rambler.ru}
{Ю.\,В.~Малыхин, В.\,Н.~Темляков}
{Вещественный и функциональный анализ} %

В 2017 г. В.\,Н.~Темляковым было начато систематическое изучение дискретизации по значениям в точках \(L_p\)-норм функций из конечномерных подпространств. Первые результаты в этом направлении были получены в 1930-е годы С.\,Н.~Бернштейном, Й.~Марцинкевичем и А.~Зигмундом для одномерных тригонометрических полиномов. В настоящее время это обширная и активно развивающаяся область исследований, имеющая глубокие связи с другими важными направлениями (см. [1], [2]). %
В литературе основное внимание уделяется двусторонним неравенствам, которые показывают, что дискретная норма вектора--выборки ограничена снизу и сверху интегральной \(L_p\)-нормой функции, умноженной на некоторые константы.
В последнее время в ряде работ результаты о дискретизации по значениям в точках успешно применялись в задачах восстановления по выборке.  Более того, оказалось, что для некоторых из этих приложений достаточно иметь односторонние неравенства дискретизации, о которых и пойдет речь в докладе. Мы также рассмотрим приложения этих неравенств к задачам восстановления по выборке. Доклад основан на совместной работе с Ю.\,В.~Малыхиным и В.\,Н.~Темляковым~[3].

\medskip

\begin{enumerate}
\item[{[1]}] Ф.~Дай, A.~Примак,  В.\,Н.~Темляков,  С.\,Ю.Тихонов, {\it Дискретизация интегральной нормы и близкие задачи},
УМН, 74:4(448) (2019), 3–-58.
\item[{[2]}] B.~Kashin, E.~Kosov, I.~Limonova, V.~Temlyakov, {\it Sampling discretization and related problems},
J. Complexity, 71 (2022), Paper No. 101653.
\item[{[3]}] И.\,В.~Лимонова,  Ю.\,В.~Малыхин, В.\,Н.~Темляков, {\it Односторонние неравенства дискретизации и восстановление по выборке}, УМН, 79:3(477) (2024), 149--180.
\end{enumerate}
\end{talk}
\end{document}