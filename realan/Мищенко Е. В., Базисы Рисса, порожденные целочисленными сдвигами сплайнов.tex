\documentclass[12pt]{article}
\usepackage{hyphsubst}
\usepackage[T2A]{fontenc}
\usepackage[english,main=russian]{babel}
\usepackage[utf8]{inputenc}
\usepackage[letterpaper,top=2cm,bottom=2cm,left=2cm,right=2cm,marginparwidth=2cm]{geometry}
\usepackage{float}
\usepackage{mathtools, commath, amssymb, amsthm}
\usepackage{enumitem, tabularx,graphicx,url,xcolor,rotating,multicol,epsfig,colortbl,lipsum}

\setlist{topsep=1pt, itemsep=0em}
\setlength{\parindent}{0pt}
\setlength{\parskip}{6pt}

\usepackage{hyphenat}
\hyphenation{ма-те-ма-ти-ка вос-ста-нав-ли-вать}

\usepackage[math]{anttor}

\newenvironment{talk}[6]{%
\vskip 0pt\nopagebreak%
\vskip 0pt\nopagebreak%
\section*{#1}
\phantomsection
\addcontentsline{toc}{section}{#2. \textit{#1}}
% \addtocontents{toc}{\textit{#1}\par}
\textit{#2}\\\nopagebreak%
#3\\\nopagebreak%
\ifthenelse{\equal{#4}{}}{}{\url{#4}\\\nopagebreak}%
\ifthenelse{\equal{#5}{}}{}{Соавторы: #5\\\nopagebreak}%
\ifthenelse{\equal{#6}{}}{}{Секция: #6\\\nopagebreak}%
}

\definecolor{LovelyBrown}{HTML}{FDFCF5}

\usepackage[pdftex,
breaklinks=true,
bookmarksnumbered=true,
linktocpage=true,
linktoc=all
]{hyperref}

\begin{document}
\pagenumbering{gobble}
\pagestyle{plain}
\pagecolor{LovelyBrown}
\begin{talk}
{Базисы Рисса, порожденные целочисленными сдвигами сплайнов}
{Мищенко Евгения Васильевна}
{Институт математики им. С.\,Л.Соболева}
{e.mishchenko@g.nsu.ru}
{}
{Вещественный и функциональный анализ} %

Исследована так называемая устойчивость семейств целочисленных сдвигов \(B_m\)-сплай\-нов, представляющих собой \(m\)-кратную свертку функции-индикатора единичного отрезка, и экспоненциальных сплайнов \(U_{m,p}\), являющихся свертками  некоторой финитной функции экспоненциального вида и \(B_m\)-сплайна. Установить устойчивость  семейства функций из гильбертова пространства \(H\) означает найти  ненулевые конечные константы \(A\)  и \(B\), с помощью которых норма любой линейной комбинации из элементов этого семейства c  коэффициентами  из \(l_2\)  оценивается снизу и сверху через \(l_2\) норму последовательности этих коэффицентов. Такие константы также называются границами Рисса.  Если семейство функций устойчиво и вдобавок полно в \(H\), то оно образует базис Рисса. При \(A=B\) базис Рисса обращается в ортнормированный базис.

В рассматриваемом случае были найдены  постоянные \(A\)  и \(B\) для любых \(m \in N, p \in R\), а также установлены некоторые предельные свойства экспоненциальных сплайнов при \(p \rightarrow 0, \pm \infty .\)

\medskip

Работа выполнена в рамках государственного задания в Институте математики им. С.\,Л.Соболева СО РАН (проект №FWNF-2022-0008).

\begin{enumerate}
\item[{[1]}] M.A. Unser, {\it Splines: a perfect fit for medical imaging},  Proc. SPIE 4684, Medical Imaging 2002: Image Processing, (9 May 2002).
\item[{[2]}] Ch. Chui,
{\it An Introduction to Wavelets}, Academic Press, San Diego, 1992.
\item[{[3]}] E.V. Mishchenko,
{\it Determination of Riesz bounds for the spline basis with the help of trigonometric polynomials}, Siberian Mathematical Journal,  {51}, (2010), 660--666.
\end{enumerate}
\end{talk}
\end{document}