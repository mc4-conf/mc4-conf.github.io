\documentclass[12pt]{article}
\usepackage{hyphsubst}
\usepackage[T2A]{fontenc}
\usepackage[english,main=russian]{babel}
\usepackage[utf8]{inputenc}
\usepackage[letterpaper,top=2cm,bottom=2cm,left=2cm,right=2cm,marginparwidth=2cm]{geometry}
\usepackage{float}
\usepackage{mathtools, commath, amssymb, amsthm}
\usepackage{enumitem, tabularx,graphicx,url,xcolor,rotating,multicol,epsfig,colortbl,lipsum}

\setlist{topsep=1pt, itemsep=0em}
\setlength{\parindent}{0pt}
\setlength{\parskip}{6pt}

\usepackage{hyphenat}
\hyphenation{ма-те-ма-ти-ка вос-ста-нав-ли-вать}

\usepackage[math]{anttor}

\newenvironment{talk}[6]{%
\vskip 0pt\nopagebreak%
\vskip 0pt\nopagebreak%
\section*{#1}
\phantomsection
\addcontentsline{toc}{section}{#2. \textit{#1}}
% \addtocontents{toc}{\textit{#1}\par}
\textit{#2}\\\nopagebreak%
#3\\\nopagebreak%
\ifthenelse{\equal{#4}{}}{}{\url{#4}\\\nopagebreak}%
\ifthenelse{\equal{#5}{}}{}{Соавторы: #5\\\nopagebreak}%
\ifthenelse{\equal{#6}{}}{}{Секция: #6\\\nopagebreak}%
}

\definecolor{LovelyBrown}{HTML}{FDFCF5}

\usepackage[pdftex,
breaklinks=true,
bookmarksnumbered=true,
linktocpage=true,
linktoc=all
]{hyperref}

\begin{document}
\pagenumbering{gobble}
\pagestyle{plain}
\pagecolor{LovelyBrown}
\begin{talk}
{Вариационные задачи нелинейной теории упругости на группах Карно}
{Павлов Степан Валерьевич}
{Новосибирский Государственный Университет}
{s.pavlov4254@gmail.com}
{Водопьянов Сергей Константинович}
{Вещественный и функциональный анализ} %

Один из подходов к поиску положения, занимаемого гиперупругим телом \(\Omega\) в результате воздействия на него известных внешних сил, состоит в нахождении отображения \(\varphi:\Omega\to \mathbb{R}^n\), доставляющего минимум функционала энергии
\[I(\varphi)=\int\limits_\Omega W(x,D\varphi(x))\, dx.\]
В прошлом веке Дж. Боллом были найдены соответствующие реальным материалам математические условия, при которых удается получить теорему о существовании минимума функционала \(I\) в некотором классе непрерывных отображений с обобщенными производными.

В работе [1] представлено приложение методов современного квазиконформного анализа к данной задаче~--- с их помощью в классе отображений с интегрируемым искажением установлено существование экстремального отображения, являющего взаимно однозначным. В настоящей работе этот подход развивается на группах Карно, обладающих существенно более сложной геометрией по сравнению с евклидовым пространством. Более подробные историческая справка и литература могут быть найдены в [1].

\medskip

\begin{enumerate}
\item[{[1]}] Molchanova A., Vodopyanov S., {\it Injectivity almost everywhere and mappings with finite distortion in nonlinear elasticity}, Calc. Var., 59, №17 (2019).
\item[{[2]}] Водопьянов С.\,К., Павлов С.\,В., {\it Функциональные свойства пределов соболевских гомеоморфизмов с интегрируемым искажением}, Современная математика. Фундаментальные направления, 2024, Том 7, №3 (в печати).
\end{enumerate}
\end{talk}
\end{document}