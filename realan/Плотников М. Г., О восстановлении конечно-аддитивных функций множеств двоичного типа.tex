\documentclass[12pt]{article}
\usepackage{hyphsubst}
\usepackage[T2A]{fontenc}
\usepackage[english,main=russian]{babel}
\usepackage[utf8]{inputenc}
\usepackage[letterpaper,top=2cm,bottom=2cm,left=2cm,right=2cm,marginparwidth=2cm]{geometry}
\usepackage{float}
\usepackage{mathtools, commath, amssymb, amsthm}
\usepackage{enumitem, tabularx,graphicx,url,xcolor,rotating,multicol,epsfig,colortbl,lipsum}

\setlist{topsep=1pt, itemsep=0em}
\setlength{\parindent}{0pt}
\setlength{\parskip}{6pt}

\usepackage{hyphenat}
\hyphenation{ма-те-ма-ти-ка вос-ста-нав-ли-вать}

\usepackage[math]{anttor}

\newenvironment{talk}[6]{%
\vskip 0pt\nopagebreak%
\vskip 0pt\nopagebreak%
\section*{#1}
\phantomsection
\addcontentsline{toc}{section}{#2. \textit{#1}}
% \addtocontents{toc}{\textit{#1}\par}
\textit{#2}\\\nopagebreak%
#3\\\nopagebreak%
\ifthenelse{\equal{#4}{}}{}{\url{#4}\\\nopagebreak}%
\ifthenelse{\equal{#5}{}}{}{Соавторы: #5\\\nopagebreak}%
\ifthenelse{\equal{#6}{}}{}{Секция: #6\\\nopagebreak}%
}

\definecolor{LovelyBrown}{HTML}{FDFCF5}

\usepackage[pdftex,
breaklinks=true,
bookmarksnumbered=true,
linktocpage=true,
linktoc=all
]{hyperref}

\begin{document}
\pagenumbering{gobble}
\pagestyle{plain}
\pagecolor{LovelyBrown}
\begin{talk}
{О восстановлении конечно-аддитивных функций множеств двоичного типа}
{Плотников Михаил Геннадьевич}
{МГУ им. М.\,В. Ломоносова, Московский центр фундаментальной и прикладной математики}
{mikhail.plotnikov@math.msu.ru}
{}
{Вещественный и функциональный анализ} %

Пусть \(\mathcal{B}\) --- множество двоичных полуоткрытых кубов из \([ 0, 1 )^d\), \(QM\) --- множество конечно-аддитивных функций из \(\mathcal{B}\) в \(\mathbb{C}\) (называемых {\it квазимерами}).
Рассмотрена задача о том, при каких условиях можно полностью восстановить квазимеру \(\tau\) из некоторого подкласса \(QM\), если знать значения \(\tau\) на всех двоичных кубах, лежащих в заданном открытом множестве \(G \subset [ 0, 1 )^d\). В качестве ингредиентов этой задачи мы берем обобщенные двоичные классы Коробова, состоящие из квазимер, чьи коэффициенты Фурье по системе Уолша имеют не более чем степенную скорость убывания, а также модельные множества \(G\) типа Шапиро, обладающие определенной двоичной структурой.

Изучен вопрос о взаимоотношениях между параметром обобщенного двоичного класса Коробова и энтропией ``слоев'' \(d\)-мерного множества \(G\) типа Шапиро, при которых любая квазимера из данного класса может быть распознана по своим значениям на лежащих в~\(G\) двоичных кубах. В этом направлении найдены условия, являющиеся в определенном смысле окончательными.

\medskip

\begin{enumerate}
\item[{[1]}] Б.\,И.~Голубов, А.\,В.~Ефимов, В.\,А.~Скворцов, {\it Ряды и преобразования Уолша\(:\) теория и применение}, М.: Наука, 1987.
\item[{[2]}] F.~Schipp, W.\,R.~Wade, P.~Simon, {\it Walsh Series. An Introduction to Dyadic Harmonic Analysis}, Budapest: Academiai Kiado, 1990.
\end{enumerate}
\end{talk}
\end{document}