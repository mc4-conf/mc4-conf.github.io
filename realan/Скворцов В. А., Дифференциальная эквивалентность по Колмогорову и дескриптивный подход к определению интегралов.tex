\documentclass[12pt]{article}
\usepackage{hyphsubst}
\usepackage[T2A]{fontenc}
\usepackage[english,main=russian]{babel}
\usepackage[utf8]{inputenc}
\usepackage[letterpaper,top=2cm,bottom=2cm,left=2cm,right=2cm,marginparwidth=2cm]{geometry}
\usepackage{float}
\usepackage{mathtools, commath, amssymb, amsthm}
\usepackage{enumitem, tabularx,graphicx,url,xcolor,rotating,multicol,epsfig,colortbl,lipsum}

\setlist{topsep=1pt, itemsep=0em}
\setlength{\parindent}{0pt}
\setlength{\parskip}{6pt}

\usepackage{hyphenat}
\hyphenation{ма-те-ма-ти-ка вос-ста-нав-ли-вать}

\usepackage[math]{anttor}

\newenvironment{talk}[6]{%
\vskip 0pt\nopagebreak%
\vskip 0pt\nopagebreak%
\section*{#1}
\phantomsection
\addcontentsline{toc}{section}{#2. \textit{#1}}
% \addtocontents{toc}{\textit{#1}\par}
\textit{#2}\\\nopagebreak%
#3\\\nopagebreak%
\ifthenelse{\equal{#4}{}}{}{\url{#4}\\\nopagebreak}%
\ifthenelse{\equal{#5}{}}{}{Соавторы: #5\\\nopagebreak}%
\ifthenelse{\equal{#6}{}}{}{Секция: #6\\\nopagebreak}%
}

\definecolor{LovelyBrown}{HTML}{FDFCF5}

\usepackage[pdftex,
breaklinks=true,
bookmarksnumbered=true,
linktocpage=true,
linktoc=all
]{hyperref}

\begin{document}
\pagenumbering{gobble}
\pagestyle{plain}
\pagecolor{LovelyBrown}
\begin{talk}
{Дифференциальная эквивалентность по Колмогорову и дескриптивный подход к определению интегралов}
{Скворцов Валентин Анатольевич}
{Московский государственный университет имени М.\,В. Ломоносова, Московский центр фундаментальной и прикладной математики}
{vaskvor2000@yahoo.com}
{}
{Вещественный и функциональный анализ} %

Рассматриваются обобщения конструкции интеграла Колмогорова (см. [1]) на случай весьма общего дифференциального базиса. Поскольку конструкция интеграла Колмогорова базируется на  обобщенных суммах Римана, то наибольшее влияние эти идеи оказали на теорию неабсолютных интегралов римановского типа и, прежде всего, на теорию интегралов Хенстока--Курцвейля.
В качестве одного из примеров идеи, нашедшей очень широкое применение в современной теории, является колмогоровское понятие дифференциальной эквивалентности, получившее в теории  Хенстока--Курцвейля название вариационной эквивалентности, которую мы рассматриваем в применении к базисам в абстрактном пространстве с мерой.  Вариационный интеграл, определяемый с помощью  понятия вариационной эквивалентности, выявляет тесную связь между интегрированием и дифференцированием относительно базиса.

С понятием вариационной эквивалентности связано понятие вариационной меры, что в свою очередь позволяет получать дескриптивные характеристики неабсолютных интегралов в стиле теорем типа Радона--Никодима.

Рассматриваются примеры применения введенных интегралов в гармоническом анализе на компактных группах специального вида.

\medskip

\begin{enumerate}
\item[{[1]}] А.\,Н. Колмогоров, {\it Исследование понятия интеграла // Избранные труды. Математика и механика}, М.: Наука, 1985.
\end{enumerate}
\end{talk}
\end{document}