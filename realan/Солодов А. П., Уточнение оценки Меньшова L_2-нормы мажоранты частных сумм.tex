\documentclass[12pt]{article}
\usepackage{hyphsubst}
\usepackage[T2A]{fontenc}
\usepackage[english,main=russian]{babel}
\usepackage[utf8]{inputenc}
\usepackage[letterpaper,top=2cm,bottom=2cm,left=2cm,right=2cm,marginparwidth=2cm]{geometry}
\usepackage{float}
\usepackage{mathtools, commath, amssymb, amsthm}
\usepackage{enumitem, tabularx,graphicx,url,xcolor,rotating,multicol,epsfig,colortbl,lipsum}

\setlist{topsep=1pt, itemsep=0em}
\setlength{\parindent}{0pt}
\setlength{\parskip}{6pt}

\usepackage{hyphenat}
\hyphenation{ма-те-ма-ти-ка вос-ста-нав-ли-вать}

\usepackage[math]{anttor}

\newenvironment{talk}[6]{%
\vskip 0pt\nopagebreak%
\vskip 0pt\nopagebreak%
\section*{#1}
\phantomsection
\addcontentsline{toc}{section}{#2. \textit{#1}}
% \addtocontents{toc}{\textit{#1}\par}
\textit{#2}\\\nopagebreak%
#3\\\nopagebreak%
\ifthenelse{\equal{#4}{}}{}{\url{#4}\\\nopagebreak}%
\ifthenelse{\equal{#5}{}}{}{Соавторы: #5\\\nopagebreak}%
\ifthenelse{\equal{#6}{}}{}{Секция: #6\\\nopagebreak}%
}

\definecolor{LovelyBrown}{HTML}{FDFCF5}

\usepackage[pdftex,
breaklinks=true,
bookmarksnumbered=true,
linktocpage=true,
linktoc=all
]{hyperref}

\begin{document}
\pagenumbering{gobble}
\pagestyle{plain}
\pagecolor{LovelyBrown}
\begin{talk}
{Уточнение оценки Меньшова \(L_2\)-нормы мажоранты частных сумм}
{Солодов Алексей Петрович}
{Московский государственный университет имени М.\,В. Ломоносова, Московский центр фундаментальной и прикладной математики}
{apsolodov@mail.ru}
{}
{Вещественный и функциональный анализ} %

Теорема Меньшова--Радемахера о точном множителе Вейля для сходимости почти всюду по любой ортонормированной системе основана на оценке нормы максимального оператора. Меньшов показал, что \(L_2\)-норма мажоранты частных сумм с коэффициентами  \(c_n\) по любой ортонормированной системе, состоящей из \(N\) функций, не превосходит \(L_2\)-нормы самой частной суммы с теми же коэффициентами, умноженной на величину \(\log_2 N + 1\). Впоследствии Беннетт показал, что \(L_1\)-норма мажоранты частных сумм с коэффициентами, равными единице,  по любой ортонормированной системе, состоящей из \(N\) функций, не превосходит \(L_\infty\)-нормы самой частной суммы с некоторыми коэффициентами, по модулю не превосходящими единицы, умноженной на величину \(\pi^{-1}\ln N+o(1)\), причем постоянная \(\pi^{-1}\)~--- точная. Этот результат  Беннетт использовал  для усиления неравенства  Меньшова, получив асипмтотически  точную оценку.

Предложен метод усиления доказательства Меньшова, позволяющий уточнить оценку \(L_2\)-нормы мажоранты частных сумм непосредственно. В частности, показано, что  множитель \(\log_2 N + 1\) в оценке Меньшова можно заменить на \(0.5 \log_2 N + 1\). Развитие этого метода позволит, на наш взгляд, лучше изучить структуру ортонормированных систем с экстремально большой нормой мажоранты частных сумм.
\end{talk}
\end{document}