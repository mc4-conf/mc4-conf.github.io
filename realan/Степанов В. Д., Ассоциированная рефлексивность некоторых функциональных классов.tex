\documentclass[12pt]{article}
\usepackage{hyphsubst}
\usepackage[T2A]{fontenc}
\usepackage[english,main=russian]{babel}
\usepackage[utf8]{inputenc}
\usepackage[letterpaper,top=2cm,bottom=2cm,left=2cm,right=2cm,marginparwidth=2cm]{geometry}
\usepackage{float}
\usepackage{mathtools, commath, amssymb, amsthm}
\usepackage{enumitem, tabularx,graphicx,url,xcolor,rotating,multicol,epsfig,colortbl,lipsum}

\setlist{topsep=1pt, itemsep=0em}
\setlength{\parindent}{0pt}
\setlength{\parskip}{6pt}

\usepackage{hyphenat}
\hyphenation{ма-те-ма-ти-ка вос-ста-нав-ли-вать}

\usepackage[math]{anttor}

\newenvironment{talk}[6]{%
\vskip 0pt\nopagebreak%
\vskip 0pt\nopagebreak%
\section*{#1}
\phantomsection
\addcontentsline{toc}{section}{#2. \textit{#1}}
% \addtocontents{toc}{\textit{#1}\par}
\textit{#2}\\\nopagebreak%
#3\\\nopagebreak%
\ifthenelse{\equal{#4}{}}{}{\url{#4}\\\nopagebreak}%
\ifthenelse{\equal{#5}{}}{}{Соавторы: #5\\\nopagebreak}%
\ifthenelse{\equal{#6}{}}{}{Секция: #6\\\nopagebreak}%
}

\definecolor{LovelyBrown}{HTML}{FDFCF5}

\usepackage[pdftex,
breaklinks=true,
bookmarksnumbered=true,
linktocpage=true,
linktoc=all
]{hyperref}

\begin{document}
\pagenumbering{gobble}
\pagestyle{plain}
\pagecolor{LovelyBrown}
\begin{talk}
{Ассоциированная рефлексивность некоторых функциональных классов}
{Степанов Владимир Дмитриевич}
{ВЦ ДВО РАН, МИАН}
{stepanov@mi-ras.ru}
{}
{Вещественный и функциональный анализ} %

В докладе рассматривается задача об описании ассоциированных и дважды
ассоциированных пространств к функциональным классам, включающим как идеальные, так и
неидеальные структуры. Последние включают в себя двухвесовые пространства Соболева первого
порядка на положительной полуоси [1]. Показано, что, в отличие от понятия двойственности,
ассоциированность может быть "сильной" и "слабой". В то же время дважды ассоциированные
пространства делятся еще на три типа. В этом контексте установлено, что пространство функций
Соболева с компактным носителем обладает слабо ассоциированной рефлексивностью, а сильно
ассоциированное к слабо ассоциированному пространству состоит только из нуля [2].
Аналогичными свойствами обладают весовые пространства типа Чезаро и Копсона, для которых
проблема полностью изучена и установлена их связь с пространствами Соболева со степенными
весами~[3]. В качестве приложения рассматривается проблема ограниченности преобразования
Гильберта из весового пространства Соболева в весовое пространство Лебега~[4].

\medskip

Работа поддержана Российским Научным Фондом (https://rscf.ru/project/24-11-00170/, Project 19-11-00087).

\begin{enumerate}
\item[{[1]}] Д.~В. Прохоров, В. Д. Степанов, Е. П. Ушакова,
{\it Характеризация функциональных пространств, ассоциированных с весовыми пространствами
Соболева первого порядка на действительной оси}, Успехи матем. наук, 74:6 (2019), 119--158.
\item[{[2]}] В. Д. Степанов, Е. П. Ушакова,
{\it О сильной и слабой ассоциированности весовых пространств Соболева первого порядка},
Успехи матем. наук, 78:1 (2023), 167--204.
\item[{[3]}] V. D. Stepanov,
{\it On Ces\`{a}ro and Copson type function spaces. Reflexivity},
J. Math. Anal. Appl., 507:1 (2022), Paper No. 125764, 18 pp.
\item[{[4]}] V. D. Stepanov,
{\it On the boundedness of the Hilbert transform from weighted Sobolev space to
weighted Lebesgue space},
J. Fourier Anal. Appl., {28} (2022), Paper No. 46, 17 pp.
\end{enumerate}
\end{talk}
\end{document}