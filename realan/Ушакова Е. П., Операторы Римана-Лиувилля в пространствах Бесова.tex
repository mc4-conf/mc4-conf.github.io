\documentclass[12pt]{article}
\usepackage{hyphsubst}
\usepackage[T2A]{fontenc}
\usepackage[english,main=russian]{babel}
\usepackage[utf8]{inputenc}
\usepackage[letterpaper,top=2cm,bottom=2cm,left=2cm,right=2cm,marginparwidth=2cm]{geometry}
\usepackage{float}
\usepackage{mathtools, commath, amssymb, amsthm}
\usepackage{enumitem, tabularx,graphicx,url,xcolor,rotating,multicol,epsfig,colortbl,lipsum}

\setlist{topsep=1pt, itemsep=0em}
\setlength{\parindent}{0pt}
\setlength{\parskip}{6pt}

\usepackage{hyphenat}
\hyphenation{ма-те-ма-ти-ка вос-ста-нав-ли-вать}

\usepackage[math]{anttor}

\newenvironment{talk}[6]{%
\vskip 0pt\nopagebreak%
\vskip 0pt\nopagebreak%
\section*{#1}
\phantomsection
\addcontentsline{toc}{section}{#2. \textit{#1}}
% \addtocontents{toc}{\textit{#1}\par}
\textit{#2}\\\nopagebreak%
#3\\\nopagebreak%
\ifthenelse{\equal{#4}{}}{}{\url{#4}\\\nopagebreak}%
\ifthenelse{\equal{#5}{}}{}{Соавторы: #5\\\nopagebreak}%
\ifthenelse{\equal{#6}{}}{}{Секция: #6\\\nopagebreak}%
}

\definecolor{LovelyBrown}{HTML}{FDFCF5}

\usepackage[pdftex,
breaklinks=true,
bookmarksnumbered=true,
linktocpage=true,
linktoc=all
]{hyperref}

\begin{document}
\pagenumbering{gobble}
\pagestyle{plain}
\pagecolor{LovelyBrown}
\begin{talk}
{Операторы Римана--Лиувилля в пространствах Бесова}
{Ушакова Елена Павловна}
{Институт проблем управления им. В.\,А. Трапезникова РАН; Математический институт им. В.\,А. Стеклова РАН}
{elenau@inbox.ru}
{}
{Вещественный и функциональный анализ} %

Рассматриваются свойства операторов интегрирования Римана--Лиувилля \(I^\alpha_\pm\) положительных порядков \(\alpha\) [1] в пространствах Бесова с весовыми функциями типа Мукенхоупта на \(\mathbb{R}\). Найдены условия для выполнения неравенств, связывающих нормы образов и прообразов \(I^\alpha_\pm\). В качестве инструментов решения задачи используются системы сплайновых всплесков и соответствующие им теоремы декомпозиции. Полученные результаты применяются к исследованию поведения последовательностей аппроксимативных и энтропийных чисел \(I^\alpha_\pm\), а также к изучению свойств преобразования Гильберта.

\medskip

Доклад основан на результатах публикаций [2-5]. Исследование поддержано  грантом Российского научного фонда № 24-11-00170, https://rscf.ru/project/24-11-00170/.

\begin{enumerate}
\item[{[1]}]~С.~Г. Самко,~А.~А. Килбас,~О.~И. Маричев, {\it Интегралы и производные дробного порядка и некоторые приложения}, М.: Наука и техника, 1987.
\item[{[2]}] Е. П. Ушакова, {\it Образы операторов интегрирования в весовых функциональных пространствах}, Сибирский математический журнал, 63:6 (2022), 1382--1410.
\item[{[3]}] Е. П. Ушакова, К. Э. Ушакова, {\it Неравенства для норм с дробными интегралами}, Алгебра и анализ, 35:3 (2023), 185--219.
\item[{[4]}] E. P. Ushakova, {\it Boundedness of the Hilbert transform in Besov spaces}, Anal. Math., 49:4 (2023), 1137–1174.
\item[{[5]}] E. P. Ushakova, {\it The study by splines of norm inequalities for Riemann--LIouville operators in weighted Besov spaces}, Journal of Mathematical Sciences, (2023), accepted.
\end{enumerate}
\end{talk}
\end{document}