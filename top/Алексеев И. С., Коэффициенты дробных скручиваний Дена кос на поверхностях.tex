\documentclass[12pt]{article}
\usepackage{hyphsubst}
\usepackage[T2A]{fontenc}
\usepackage[english,main=russian]{babel}
\usepackage[utf8]{inputenc}
\usepackage[letterpaper,top=2cm,bottom=2cm,left=2cm,right=2cm,marginparwidth=2cm]{geometry}
\usepackage{float}
\usepackage{mathtools, commath, amssymb, amsthm}
\usepackage{enumitem, tabularx,graphicx,url,xcolor,rotating,multicol,epsfig,colortbl,lipsum}

\setlist{topsep=1pt, itemsep=0em}
\setlength{\parindent}{0pt}
\setlength{\parskip}{6pt}

\usepackage{hyphenat}
\hyphenation{ма-те-ма-ти-ка вос-ста-нав-ли-вать}

\usepackage[math]{anttor}

\newenvironment{talk}[6]{%
\vskip 0pt\nopagebreak%
\vskip 0pt\nopagebreak%
\section*{#1}
\phantomsection
\addcontentsline{toc}{section}{#2. \textit{#1}}
% \addtocontents{toc}{\textit{#1}\par}
\textit{#2}\\\nopagebreak%
#3\\\nopagebreak%
\ifthenelse{\equal{#4}{}}{}{\url{#4}\\\nopagebreak}%
\ifthenelse{\equal{#5}{}}{}{Соавторы: #5\\\nopagebreak}%
\ifthenelse{\equal{#6}{}}{}{Секция: #6\\\nopagebreak}%
}

\definecolor{LovelyBrown}{HTML}{FDFCF5}

\usepackage[pdftex,
breaklinks=true,
bookmarksnumbered=true,
linktocpage=true,
linktoc=all
]{hyperref}

\begin{document}
\pagenumbering{gobble}
\pagestyle{plain}
\pagecolor{LovelyBrown}
\begin{talk}
{Коэффициенты дробных скручиваний Дена кос на поверхностях}
{Алексеев Илья Сергеевич}
{СПбГУ}
{ilyaalekseev@yahoo.com}
{Малютин Андрей Валерьевич и Ионин Василий Андреевич}
{Топология} %

{\it Коэффициент дробного скручивания Дена} (или {\it закрученность}) — это рациональнозначный инвариант сопряженности в группе классов отображений~$\textrm{Mod}(\Sigma)$ компактной поверхности~$\Sigma$ с непустым краем (и, возможно, конечным числом отмеченных точек), который измеряет то, насколько автогомеоморфизм поверхности «перекручивает» окрестность заданной граничной компоненты~$\partial$ этой поверхности.
Закрученность является псевдохарактером (отображением, «похожим» на гомоморфизм), принимает единичное значение на скручивании Дена~$T_\partial$ вдоль компоненты~$\partial$ и неотрицательные значения на всех тех элементах группы~$\textrm{Mod}(\Sigma)$, которые «переводят направо» каждую дугу с началом на компоненте~$\partial$.

Коэффициенты дробных скручиваний Дена играют заметную роль в маломерной топологии и динамике. В литературе они впервые появились в работе~[1], и с тех пор получили ряд различных эквивалентных определений: в терминах теории Нильсена--Тёрстона~[2,5], в терминах левоинвариантных порядков на группах классов отображений~[3,15,16], в терминах гомологий Хегора--Флоера~[9,11,14], в терминах введённых А.~Пуанкаре числа переноса и числа вращения~[2,10].
Подобное разнообразие позволяет закрученности проникать в контактную топологию~[6,7], теорию слоений на трёхмерных многообразиях~[8,12], теорию узлов~[2,4,11,14], теорию пространств модулей~[13].

Первые исследования закрученности относятся к теории кос. Так, группа классов отображений диска с~$n$ отмеченными точками изоморфна {\it группе кос Артина~$B_n$}, то есть фундаментальной группе конфигурационного пространства наборов из $n$~различных точек на диске~$D^2$, и в этом случае коэффициент дробного скручивания Дена показывает, насколько сильно коса «закручена» или «перекручена»~[2]. В случае же произвольной поверхности~$\Sigma$ имеется точная последовательность Бирман, которая устанавливает связь между группой кос~$B_n(\Sigma)$ этой поверхности и группой классов отображений~$\textrm{Mod}(\Sigma; \{x_1,\ldots,x_n\})$ поверхности, получающейся из~$\Sigma$ добавлением~$n$ отмеченных точек~$x_1,\ldots, x_n$. Данная последовательность позволяет задать коэффициенты дробных скручиваний Дена для кос на произвольных поверхностях с непустым краем, однако такой подход не охватывает важный случай групп кос замкнутых поверхностей.

В работе~[5] были предложены аналоги коэффициентов дробных скручиваний \hbox{Дена} на группах кос Артина $B_n$, которые измеряют «закрученность косы вокруг той или иной её нити». Мы исследуем возможность переноса данных понятий с диска на произвольные поверхности. Наш подход предполагает обращение к задаче поднятия {\it гомоморфизма-протаскивания} (от англ. {\it pushing}) относительно {\it гомоморфизма-заклеивания} (от англ. {\it capping}), фигурирующих в длинной точной последовательности Бирман. Мы показываем, что данная задача поднятия имеет решение в случае сферы, проективной плоскости, тора, бутылки Клейна и всех поверхностей с непустым краем, что позволяет построить искомые аналоги коэффициентов дробного скручивания Дена для кос на этих поверхностях. Кроме того, мы устанавливаем неразрешимость указанной задачи поднятия на всех остальных замкнутых поверхностях.

Данный круг идей приводит нас к аналогам известной в теории кос проблемы расщепления последовательности Фаделла--Нойвирта, проблеме непрерывного построения векторных полей с заданными нулями и ряду родственных задач.

\begin{enumerate}
\item[{[1]}] D. Gabai, U. Oertel, {\it Essential laminations in} 3{\it-manifolds}, The Annals of Mathematics {\bf 130} (1989), no.~1, 41--73.
\item[{[2]}] А. В. Малютин, {\it Закрученность (замкнутых) кос}, Алгебра и анализ, {\bf 16} (2004), \textnumero~5, 59--91.
\item[{[3]}] K. Honda, W. H. Kazez, G. Mati{\'c}, {\it Right-veering diffeomorphisms of compact surfaces with boundary}, Inventiones Mathematicae {\bf 169} (2007), no.~2, 427--49.
\item[{[4]}] T. Ito, {\it Braid ordering and the geometry of closed braid}, Geometry \& Topology, {\bf 15} (2011), no.~1, 473--98.
\item[{[5]}] И. А. Дынников, В. А. Шастин, {\it О независимости некоторых псевдохарактеров на группах кос}, Алгебра и анализ, {\bf 24} (2012), \textnumero~6, 21--41.
\item[{[6]}] W. H. Kazez, R. Roberts, {\it Fractional Dehn twists in knot theory and contact topology}, Algebraic \& Geometric Topology {\bf 13} (2013), no.~6, 3603--37.
\item[{[7]}] J. B. Etnyre, J. Van Horn-Morris, {\it Monoids in the mapping class group}, Geometry \& Topology Monographs {\bf 19} (2015), no.~1, 319--65.
\item[{[8]}] T. Ito, K. Kawamuro, {\it Essential open book foliations and fractional Dehn twist coef\-ficient}, Geometriae Dedicata {\bf 187} (2017), no.~1, 17--67.
\item[{[9]}] M. Hedden, T. E. Mark, {\it Floer homology and fractional Dehn twists}, Advances in Mathematics {\bf 324} (2018), 1--39.
\item[{[10]}] А. В. Малютин, {\it Эффект целочисленного квантования числа вращения в группах кос}, Труды МИАН {\bf 305} (2019), 197--210.
\item[{[11]}] P. Feller, D. Hubbard, {\it Braids with as many full twists as strands realize the braid index}, Journal of Topology {\bf 12} (2019), no.~4, 1069--1092.
\item[{[12]}] D. Hubbard et al, {\it Braids, fibered knots, and concordance questions}, in: Research directions in symplectic and contact geometry and topology (2021), 293--24.
\item[{[13]}] X. L. Liu, {\it Fractional Dehn twists and modular invariants}, Science China Mathematics {\bf 64} (2021), no.~8, 1735--44.
\item[{[14]}] P. Feller, {\it The slice-Bennequin inequality for the fractional Dehn twist coefficient}, \url{arxiv.org/abs/2204.05288v2}.
\item[{[15]}] P. Feller, D. Hubbard, H. Turner, {\it The Dehn twist coefficient for big and small mapping class groups}, \url{arxiv.org/abs/2308.06214v1}.
\item[{[16]}] A. Clay, T. Ghaswala, {\it Cofinal elements and fractional Dehn twist coefficients}, Interna\-tional Mathematics Research Notices {\bf 2024} (2024), no.~9, 7401--20.
\end{enumerate}
\end{talk}
\end{document}