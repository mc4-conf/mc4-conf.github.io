\documentclass[12pt]{article}
\usepackage{hyphsubst}
\usepackage[T2A]{fontenc}
\usepackage[english,main=russian]{babel}
\usepackage[utf8]{inputenc}
\usepackage[letterpaper,top=2cm,bottom=2cm,left=2cm,right=2cm,marginparwidth=2cm]{geometry}
\usepackage{float}
\usepackage{mathtools, commath, amssymb, amsthm}
\usepackage{enumitem, tabularx,graphicx,url,xcolor,rotating,multicol,epsfig,colortbl,lipsum}

\setlist{topsep=1pt, itemsep=0em}
\setlength{\parindent}{0pt}
\setlength{\parskip}{6pt}

\usepackage{hyphenat}
\hyphenation{ма-те-ма-ти-ка вос-ста-нав-ли-вать}

\usepackage[math]{anttor}

\newenvironment{talk}[6]{%
\vskip 0pt\nopagebreak%
\vskip 0pt\nopagebreak%
\section*{#1}
\phantomsection
\addcontentsline{toc}{section}{#2. \textit{#1}}
% \addtocontents{toc}{\textit{#1}\par}
\textit{#2}\\\nopagebreak%
#3\\\nopagebreak%
\ifthenelse{\equal{#4}{}}{}{\url{#4}\\\nopagebreak}%
\ifthenelse{\equal{#5}{}}{}{Соавторы: #5\\\nopagebreak}%
\ifthenelse{\equal{#6}{}}{}{Секция: #6\\\nopagebreak}%
}

\definecolor{LovelyBrown}{HTML}{FDFCF5}

\usepackage[pdftex,
breaklinks=true,
bookmarksnumbered=true,
linktocpage=true,
linktoc=all
]{hyperref}

\begin{document}
\pagenumbering{gobble}
\pagestyle{plain}
\pagecolor{LovelyBrown}
\begin{talk}
{Проблема Кервера в стабильной теории гомотопий и ее обобщение}
{Ахметьев Петр Михайлович}
{ИЗМИРАН}
{pmakhmet@mail.ru}
{}
{Топология} %

Проблема Кервера в стабильной теории гомотопий состоит в перечислении списка размерностей, в которых существует погружение коразмерности \(1\) замкнутого ориентированного многообразия с Арф-инвариантом \(1\).
Первый интересный пример имеется в размерности \(30\). Цель доклада --- геометрическая конструкция указанного многообразия как в работе \([1]\).

Далее мы обобщим конструкцию и построим бесконечную серию погружений замкнутых многообразий размерностей
\(2^l-2\) в коразмерности  \(2^{l-1}-1\), \(l\ge 5\), которые оснащены с коразмерности \(2^{l-1}\) (\(1\)-стабильно-оснащенные погружения) со скрученным Арф-инвариантом \(1\).

Кроме того, мы докажем, что список размерностей, в которых существует оснащенные многообразия с Арф-инвариантом \(1\) конечен, а также представим доказательство того, что в размерности \(126\) не существует оснащенного многообразия, представляющего элемент стабильной гомотопической группы сфер \(\pi_{126+128}(S^{128})\), если такой элемент денадстраивается в группу \(\pi_{126+128-9}(S^{128-9})\).

\medskip

\begin{enumerate}
\item[{[1]}]
J.D.S. Jones, {\it The Kervaire invariant of extended power manifolds} Topology
17 (1978) 249-266.
\end{enumerate}
\end{talk}
\end{document}