\documentclass[12pt]{article}
\usepackage{hyphsubst}
\usepackage[T2A]{fontenc}
\usepackage[english,main=russian]{babel}
\usepackage[utf8]{inputenc}
\usepackage[letterpaper,top=2cm,bottom=2cm,left=2cm,right=2cm,marginparwidth=2cm]{geometry}
\usepackage{float}
\usepackage{mathtools, commath, amssymb, amsthm}
\usepackage{enumitem, tabularx,graphicx,url,xcolor,rotating,multicol,epsfig,colortbl,lipsum}

\setlist{topsep=1pt, itemsep=0em}
\setlength{\parindent}{0pt}
\setlength{\parskip}{6pt}

\usepackage{hyphenat}
\hyphenation{ма-те-ма-ти-ка вос-ста-нав-ли-вать}

\usepackage[math]{anttor}

\newenvironment{talk}[6]{%
\vskip 0pt\nopagebreak%
\vskip 0pt\nopagebreak%
\section*{#1}
\phantomsection
\addcontentsline{toc}{section}{#2. \textit{#1}}
% \addtocontents{toc}{\textit{#1}\par}
\textit{#2}\\\nopagebreak%
#3\\\nopagebreak%
\ifthenelse{\equal{#4}{}}{}{\url{#4}\\\nopagebreak}%
\ifthenelse{\equal{#5}{}}{}{Соавторы: #5\\\nopagebreak}%
\ifthenelse{\equal{#6}{}}{}{Секция: #6\\\nopagebreak}%
}

\definecolor{LovelyBrown}{HTML}{FDFCF5}

\usepackage[pdftex,
breaklinks=true,
bookmarksnumbered=true,
linktocpage=true,
linktoc=all
]{hyperref}

\begin{document}
\pagenumbering{gobble}
\pagestyle{plain}
\pagecolor{LovelyBrown}
\begin{talk}
{Инварианты заузленных тел с ручками}
{Бардаков Валерий Георгиевич}
{ведущий научный сотрудник Регионального научно-образовательного математического центра Томского государственного университета}
{bardakova@math.nsc.ru}
{Федосеев Денис Александрович}
{Топология} %

Одним из обобщений теории узлов является теория пространственных графов,
в которой под пространственным графом понимается вложение графа в
трехмерное пространство. Два пространственных графа называются
эквивалентными если существует сохраняющий ориентацию гомеоморфизм
объемлющего пространства, переводящий один граф в другой. Если
рассмотреть граф, состоящий из одной вершины и одного ребра, то
полученная теория будет совпадать с теорией узлов. Та же теория
получится если заменить такой граф его регулярной окрестностью
(полноторием). Если же взять тело с двумя ручками и изучать его
вложения, то получим теорию, отличную от теории вложения
соответствующего графа.

Разницу между вложениями тел с ручками и вложениями соответствующих
графов хорошо видно на примере топологического человечка [1, рисунок
306], который может распутать пальцы. Если же взять соответствующий
пространственный граф, то он не эквивалентен плоскому графу. Что
произойдет если у человечка есть часы? Рисунок 307 показывает, что трюк,
используемый ранее не годится. Тем не менее остается такой вопрос: может
ли человечек с часами распутать пальцы и снять часы?

В докладе мы расскажем о так называемых G-системах квандлов, введенных в
[2] для построения инвариантов заузленных тел с ручками, введем
алгебраическую систему, дающую инвариант пространственного графа и дадим
ответ на вопрос, сформулированный выше.

\medskip

\begin{enumerate}

\item[{[1]}] С. В. Матвеев, А. Т. Фоменко,
{\it Алгоритмические и компьютерные методы в трехмерной топологии}, Изд-во МГУ, 1991.
\item[{[2]}] A.Ishii, {\it Moves and invariants for knotted
handlebodies}, Algebr. Geom. Topol. 8 (2008), 1403-1418.
\end{enumerate}
\end{talk}
\end{document}