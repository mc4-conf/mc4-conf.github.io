\documentclass[12pt]{article}
\usepackage{hyphsubst}
\usepackage[T2A]{fontenc}
\usepackage[english,main=russian]{babel}
\usepackage[utf8]{inputenc}
\usepackage[letterpaper,top=2cm,bottom=2cm,left=2cm,right=2cm,marginparwidth=2cm]{geometry}
\usepackage{float}
\usepackage{mathtools, commath, amssymb, amsthm}
\usepackage{enumitem, tabularx,graphicx,url,xcolor,rotating,multicol,epsfig,colortbl,lipsum}

\setlist{topsep=1pt, itemsep=0em}
\setlength{\parindent}{0pt}
\setlength{\parskip}{6pt}

\usepackage{hyphenat}
\hyphenation{ма-те-ма-ти-ка вос-ста-нав-ли-вать}

\usepackage[math]{anttor}

\newenvironment{talk}[6]{%
\vskip 0pt\nopagebreak%
\vskip 0pt\nopagebreak%
\section*{#1}
\phantomsection
\addcontentsline{toc}{section}{#2. \textit{#1}}
% \addtocontents{toc}{\textit{#1}\par}
\textit{#2}\\\nopagebreak%
#3\\\nopagebreak%
\ifthenelse{\equal{#4}{}}{}{\url{#4}\\\nopagebreak}%
\ifthenelse{\equal{#5}{}}{}{Соавторы: #5\\\nopagebreak}%
\ifthenelse{\equal{#6}{}}{}{Секция: #6\\\nopagebreak}%
}

\definecolor{LovelyBrown}{HTML}{FDFCF5}

\usepackage[pdftex,
breaklinks=true,
bookmarksnumbered=true,
linktocpage=true,
linktoc=all
]{hyperref}

\begin{document}
\pagenumbering{gobble}
\pagestyle{plain}
\pagecolor{LovelyBrown}
\begin{talk}
{О двух проблемах Ролфсена}
{Мелихов Сергей Александрович}
{МИАН}
{melikhov@mi-ras.ru}
{}
{Топология} %

50 лет назад Д. Ролфсен поставил две проблемы [1]: (а) Всякий ли узел в \(S^3\) изотопен (=гомотопен в классе вложений) кусочно-линейному или, эквивалентно,
тривиальному узлу? В частности, изотопен ли кусочно-линейному узлу слинг Бинга? (б) Если два кусочно-линейных зацепления в \(S^3\) изотопны,
будут ли они кусочно-линейно изотопны?

Ответ на вопрос (б) утвердителен, если инварианты конечного порядка дают полную классификацию кусочно-линейных зацеплений [2].
Cлинг Бинга не изотопен никакому кусочно-линейному узлу: (i) изотопией, продолжающейся до изотопии двухкомпонентного зацепления с
коэффициентом зацепления 1; (ii) в классе узлов, являющихся пересечениями вложенных цепочек полноториев [4].
Причём результат (i) сохраняет силу, если дополнительной компоненте разрешить самопересекаться и даже заменяться на новую, если она представляет
тот же класс сопряжённости в \(G/[G',G'']\), где \(G\) --- фундаментальная группа дополнения к исходной компоненте [4].
Доказательства основаны на прояснении геометрического смысла определяемых с помощью полинома Конвея от двух переменных формальных аналогов
производных инвариантов Кохрана для двухкомпонентных зацеплений с коэффициентом зацепления 1 [3].

\medskip

\begin{enumerate}
\item[{[1]}] D. Rolfsen, {\it Some counterexamples in link theory}, Canadian J. Math. 26 (1974)
\item[{[2]}] S. A. Melikhov, {\it Topological isotopy and finite type invariants}, arXiv:2406.09331
\item[{[3]}] S. A. Melikhov, {\it Two-variable Conway polynomial and Cochran's derived invariants}, arXiv:math/0312007v3 (2024).
\item[{[4]}] S. A. Melikhov, {\it Is every knot isotopic to the unknot?}, arXiv:2406.09365
\end{enumerate}
\end{talk}
\end{document}