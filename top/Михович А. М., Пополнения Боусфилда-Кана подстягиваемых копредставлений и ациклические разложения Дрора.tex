\documentclass[12pt]{article}
\usepackage{hyphsubst}
\usepackage[T2A]{fontenc}
\usepackage[english,main=russian]{babel}
\usepackage[utf8]{inputenc}
\usepackage[letterpaper,top=2cm,bottom=2cm,left=2cm,right=2cm,marginparwidth=2cm]{geometry}
\usepackage{float}
\usepackage{mathtools, commath, amssymb, amsthm}
\usepackage{enumitem, tabularx,graphicx,url,xcolor,rotating,multicol,epsfig,colortbl,lipsum}

\setlist{topsep=1pt, itemsep=0em}
\setlength{\parindent}{0pt}
\setlength{\parskip}{6pt}

\usepackage{hyphenat}
\hyphenation{ма-те-ма-ти-ка вос-ста-нав-ли-вать}

\usepackage[math]{anttor}

\newenvironment{talk}[6]{%
\vskip 0pt\nopagebreak%
\vskip 0pt\nopagebreak%
\section*{#1}
\phantomsection
\addcontentsline{toc}{section}{#2. \textit{#1}}
% \addtocontents{toc}{\textit{#1}\par}
\textit{#2}\\\nopagebreak%
#3\\\nopagebreak%
\ifthenelse{\equal{#4}{}}{}{\url{#4}\\\nopagebreak}%
\ifthenelse{\equal{#5}{}}{}{Соавторы: #5\\\nopagebreak}%
\ifthenelse{\equal{#6}{}}{}{Секция: #6\\\nopagebreak}%
}

\definecolor{LovelyBrown}{HTML}{FDFCF5}

\usepackage[pdftex,
breaklinks=true,
bookmarksnumbered=true,
linktocpage=true,
linktoc=all
]{hyperref}

\begin{document}
\pagenumbering{gobble}
\pagestyle{plain}
\pagecolor{LovelyBrown}
\begin{talk}
{Пополнения Боусфилда-Кана подстягиваемых копредставлений и ациклические разложения Дрора}
{Михович Андрей Михайлович}
{Московский центр фундаментальной и прикладной математики}
{amikhovich@gmail.com}
{}
{Топология} %

Как показали Беррик и Хилман, для любого стягиваемого копредставления его конечное копредставление асферично тогда и только тогда, когда верна гипотеза асферичности Уайтхеда.
При этом, как известно, если гипотеза Уайтхеда не верна, то накрытие, соответствующее радикалу Адамса, является ацикличным 2-комплексом.
В начале 70-х Дрор показал, как можно исследовать ациклические пространства с помощью ациклических разложений и их алгебраических инвариантов.
Для подстягиваемых копредставлений удобно использовать относительные ациклические разложения, которые строятся с использованием целочисленного пополнения Боусфилда-Кана.
Мы показываем, что целочисленное пополнение Боусфилда-Кана конечного подстягиваемого копредставления асферично и проводим вычисления в его разложении Дрора.

\medskip

\begin{enumerate}
\item[{[1]}] Andrey M. Mikhovich. Bousfield-Kan completions of subcontractible presentations, doi:10.13140/rg.2.2.14640.78088/2
\item[{[2]}] A. J. Berrick and J. A. Hillman. Whitehead’s asphericity question and its relation to other open problems. In Algebraic
topology and related topics, Trends Math., pages 27–49. Birkhauser/Springer, Singapore, 2019.
\item[{[3]}] Emmanuel Dror. Homology spheres. Isr. J. Math., 15:115–129, 1973.
\end{enumerate}
\end{talk}
\end{document}