\documentclass[12pt]{article}
\usepackage{hyphsubst}
\usepackage[T2A]{fontenc}
\usepackage[english,main=russian]{babel}
\usepackage[utf8]{inputenc}
\usepackage[letterpaper,top=2cm,bottom=2cm,left=2cm,right=2cm,marginparwidth=2cm]{geometry}
\usepackage{float}
\usepackage{mathtools, commath, amssymb, amsthm}
\usepackage{enumitem, tabularx,graphicx,url,xcolor,rotating,multicol,epsfig,colortbl,lipsum}

\setlist{topsep=1pt, itemsep=0em}
\setlength{\parindent}{0pt}
\setlength{\parskip}{6pt}

\usepackage{hyphenat}
\hyphenation{ма-те-ма-ти-ка вос-ста-нав-ли-вать}

\usepackage[math]{anttor}

\newenvironment{talk}[6]{%
\vskip 0pt\nopagebreak%
\vskip 0pt\nopagebreak%
\section*{#1}
\phantomsection
\addcontentsline{toc}{section}{#2. \textit{#1}}
% \addtocontents{toc}{\textit{#1}\par}
\textit{#2}\\\nopagebreak%
#3\\\nopagebreak%
\ifthenelse{\equal{#4}{}}{}{\url{#4}\\\nopagebreak}%
\ifthenelse{\equal{#5}{}}{}{Соавторы: #5\\\nopagebreak}%
\ifthenelse{\equal{#6}{}}{}{Секция: #6\\\nopagebreak}%
}

\definecolor{LovelyBrown}{HTML}{FDFCF5}

\usepackage[pdftex,
breaklinks=true,
bookmarksnumbered=true,
linktocpage=true,
linktoc=all
]{hyperref}

\begin{document}
\pagenumbering{gobble}
\pagestyle{plain}
\pagecolor{LovelyBrown}
\begin{talk}
{Полином Ямады K4-кривых и полином Джонса ассоциированных зацеплений}
{Ошмарина Ольга Андреевна}
{ТГУ, НГУ}
{o.oshmarina@g.nsu.ru}
{Веснин А.\,Ю.}
{Топология} %

В теории заузленных графов нередко используются методы, пришедшие из теории узлов. Так, для графов строятся полиномиальные инварианты, наиболее известными из которых являются полином Ямады [1] и полином Егера [2].

В работе [3] была доказана эквивалентность полиномов Ямады и Егера для планарных графов, а также была изучена связь, возникающая между полиномом Ямады тета-кривой и полиномом Джонса зацепления, однозначно строящегося по заузленному тета-графу. В данном докладе мы представим аналогичные результаты для заузленных $\mathbb{K}_4$-графов [4].

\medskip

\begin{enumerate}
\item[{[1]}] S. Yamada, {\it An invariant of spatial graphs}, Graph Theory, 13 (1989), 537–551.
\item[{[2]}] F. Jaeger, {\it On some graph invariants related to the Kauffman polynomial}, Progress in knot theory and related topics, 56 (1997), 69–82.
\item[{[3]}] Y. Huh, {\it Yamada polynomial and associated link of theta-curves}, Discrete Mathematics, 347 (2024).
\item[{[4]}]  O. Oshmarina, A. Vesnin, {\it Polynomials of complete spatial graphs and Jones polynomial of related links}, 2024, preprint arXiv:2404.12264.
\end{enumerate}
\end{talk}
\end{document}