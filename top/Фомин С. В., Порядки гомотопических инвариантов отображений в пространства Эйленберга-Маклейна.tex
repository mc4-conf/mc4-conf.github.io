\documentclass[12pt]{article}
\usepackage{hyphsubst}
\usepackage[T2A]{fontenc}
\usepackage[english,main=russian]{babel}
\usepackage[utf8]{inputenc}
\usepackage[letterpaper,top=2cm,bottom=2cm,left=2cm,right=2cm,marginparwidth=2cm]{geometry}
\usepackage{float}
\usepackage{mathtools, commath, amssymb, amsthm}
\usepackage{enumitem, tabularx,graphicx,url,xcolor,rotating,multicol,epsfig,colortbl,lipsum}

\setlist{topsep=1pt, itemsep=0em}
\setlength{\parindent}{0pt}
\setlength{\parskip}{6pt}

\usepackage{hyphenat}
\hyphenation{ма-те-ма-ти-ка вос-ста-нав-ли-вать}

\usepackage[math]{anttor}

\newenvironment{talk}[6]{%
\vskip 0pt\nopagebreak%
\vskip 0pt\nopagebreak%
\section*{#1}
\phantomsection
\addcontentsline{toc}{section}{#2. \textit{#1}}
% \addtocontents{toc}{\textit{#1}\par}
\textit{#2}\\\nopagebreak%
#3\\\nopagebreak%
\ifthenelse{\equal{#4}{}}{}{\url{#4}\\\nopagebreak}%
\ifthenelse{\equal{#5}{}}{}{Соавторы: #5\\\nopagebreak}%
\ifthenelse{\equal{#6}{}}{}{Секция: #6\\\nopagebreak}%
}

\definecolor{LovelyBrown}{HTML}{FDFCF5}

\usepackage[pdftex,
breaklinks=true,
bookmarksnumbered=true,
linktocpage=true,
linktoc=all
]{hyperref}

\begin{document}
\pagenumbering{gobble}
\pagestyle{plain}
\pagecolor{LovelyBrown}
\begin{talk}
{Порядки гомотопических инвариантов отображений в пространства Эйленберга--Маклейна}
{Фомин Сергей Вадимович}
{СПбГУ}
{sf2902@mail.ru}
{}
{Топология} %

Пусть \(X, Y\) --- топологические пространства, \(A\) --- абелева группа, тогда на множестве функций \([X,Y]\rightarrow A\) (гомотопических инвариантов) можно определить меру сложности, называемую порядком. Инварианты конечного порядка можно понимать как гомотопические аналоги инвариантов Васильева узлов (см. предложение 2 в [1]). Пусть \(A, B\) --- абелевы группы, тогда у функции из \(A\) в \(B\) можно определить её степень. Это непосредственный аналог степени многочлена.

Если \(Y\) --- это \(H\)-пространство, то множество \([X,Y]\) --- это абелева группа. В статье [2] доказано, что, если \(Y=S^1\), порядок гомотопического инварианта равен его степени как отображения между абелевыми группами. В дипломной работе докладчика доказано двойное неравенство на порядок в терминах степени, если \(X\) --- конечный CW-комплекс, \(Y\) --- \(K(G,n)\)-пространство (\(G\) абелева), и исследован вопрос достижения верхнего и нижнего пределов в этом неравенстве.  Доклад будет посвящён результатам этой работы.

\medskip

\begin{enumerate}
\item[{[1]}]  Подкорытов С. С. Об отображениях сферы в односвязное пространство //Записки научных семинаров ПОМИ. – 2005. – Т. 329. – №. 0. – С. 159-194.
\item[{[2]}]  Подкорытов С. С. О гомотопических инвариантах отображений в окружность //Записки научных семинаров ПОМИ. – 2009. – Т. 372. – С. 187-202.
\end{enumerate}
\end{talk}
\end{document}