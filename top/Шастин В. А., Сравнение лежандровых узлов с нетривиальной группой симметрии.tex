\documentclass[12pt]{article}
\usepackage{hyphsubst}
\usepackage[T2A]{fontenc}
\usepackage[english,main=russian]{babel}
\usepackage[utf8]{inputenc}
\usepackage[letterpaper,top=2cm,bottom=2cm,left=2cm,right=2cm,marginparwidth=2cm]{geometry}
\usepackage{float}
\usepackage{mathtools, commath, amssymb, amsthm}
\usepackage{enumitem, tabularx,graphicx,url,xcolor,rotating,multicol,epsfig,colortbl,lipsum}

\setlist{topsep=1pt, itemsep=0em}
\setlength{\parindent}{0pt}
\setlength{\parskip}{6pt}

\usepackage{hyphenat}
\hyphenation{ма-те-ма-ти-ка вос-ста-нав-ли-вать}

\usepackage[math]{anttor}

\newenvironment{talk}[6]{%
\vskip 0pt\nopagebreak%
\vskip 0pt\nopagebreak%
\section*{#1}
\phantomsection
\addcontentsline{toc}{section}{#2. \textit{#1}}
% \addtocontents{toc}{\textit{#1}\par}
\textit{#2}\\\nopagebreak%
#3\\\nopagebreak%
\ifthenelse{\equal{#4}{}}{}{\url{#4}\\\nopagebreak}%
\ifthenelse{\equal{#5}{}}{}{Соавторы: #5\\\nopagebreak}%
\ifthenelse{\equal{#6}{}}{}{Секция: #6\\\nopagebreak}%
}

\definecolor{LovelyBrown}{HTML}{FDFCF5}

\usepackage[pdftex,
breaklinks=true,
bookmarksnumbered=true,
linktocpage=true,
linktoc=all
]{hyperref}

\begin{document}
\pagenumbering{gobble}
\pagestyle{plain}
\pagecolor{LovelyBrown}
\begin{talk}
{Сравнение лежандровых узлов с нетривиальной группой симметрии}
{Шастин Владимир Алексеевич}
{МГУ им. М.\,В. Ломоносова}
{vashast@gmail.com}
{М.\,В. Прасолов}
{Топология} %

В работе [1] был построен алгоритм сравнения лежандровых узлов. Если группа симметрий узла тривиальна, соответствующий алгоритм значительно упрощается (см. [2]).  В случае нетривиальной группы симметрии возникают дополнительные трудности: нужно проанализировать подгруппу группы симметрий, порождённую лежандровыми изотопиями. В докладе будут предъявлены порождающие лежандровы изотопии в случае узлов \(7_4\), \(9_{48}\), \(10_{136}\), что позволяет завершить классификацию лежандровых узлов сложности не выше 9. Доклад основан на совместной работе с М.\,В. Прасоловым [3].

\medskip

Исследование выполнено за счет гранта Российского научного фонда № 22-11-00299, https://rscf.ru/project/22-11-00299/.

\begin{enumerate}
\item[{[1]}] I.\,Dynnikov, M.\,Prasolov.
An algorithm for comparing Legendrian knots. \emph{Preprint} \\ arXiv:2309.05087
\item[{[2]}] I.\,Dynnikov, V.\,Shastin. Distinguishing Legendrian knots with trivial orientation-\\-preserving symmetry group. \emph{Algebraic \& Geometric Topology} {\bf 23-4} (2023), 1849--1889. \\ arXiv:1810.06460.
\item[{[3]}] M.\,Prasolov, V.\,Shastin Distinguishing Legendrian Knots in Topological Types \(7_4\), \(9_{48}\), \(10_{136}\) with maximal Thurston-Benequin number, \emph{Journal of Knot Theory and Its Ramifications}, {\bf 33-01} (2024). arXiv:2306.15461
\end{enumerate}
\end{talk}
\end{document}