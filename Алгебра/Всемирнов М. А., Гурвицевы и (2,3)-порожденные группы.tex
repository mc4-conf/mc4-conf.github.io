\documentclass[12pt, a4paper, figuresright]{book}
\usepackage{mathtools, commath, amssymb, amsthm}
\usepackage{tabularx,graphicx,url,xcolor,rotating,multicol,epsfig,colortbl,lipsum}
\usepackage[T2A]{fontenc}
\usepackage[english,main=russian]{babel}

\setlength{\textheight}{25.2cm}
\setlength{\textwidth}{16.5cm}
\setlength{\voffset}{-1.6cm}
\setlength{\hoffset}{-0.3cm}
\setlength{\evensidemargin}{-0.3cm} 
\setlength{\oddsidemargin}{0.3cm}
\setlength{\parindent}{0cm} 
\setlength{\parskip}{0.3cm}

\newenvironment{talk}[6]{%
\vskip 0pt\nopagebreak%
\vskip 0pt\nopagebreak%
\textbf{#1}\vspace{3mm}\\\nopagebreak%
\textit{#2}\\\nopagebreak%
#3\\\nopagebreak%
\url{#4}\vspace{3mm}\\\nopagebreak%
\ifthenelse{\equal{#5}{}}{}{Соавторы: #5\vspace{3mm}\\\nopagebreak}%
\ifthenelse{\equal{#6}{}}{}{Секция: #6\quad \vspace{3mm}\\\nopagebreak}%
}

\pagestyle{empty}

\begin{document}
\begin{talk}
  {Гурвицевы и (2,3)-порожденные группы} % [1] название доклада
  {Всемирнов Максим Александрович} % [2] имя докладчика
  {ПОМИ РАН}% [3] аффилиация
  {vsemir@pdmi.ras.ru} % [4] адрес электронной почты (НЕОБЯЗАТЕЛЬНО)
  {} % [5] соавторы (НЕОБЯЗАТЕЛЬНО)
  {Алгебра} % [6] название секции

В докладе будет рассказано о задаче порождения групп с условиями на порядки образующих и их произведений. Причем нас будет интересовать не только вопрос о возможности или невозможности такого порождения,  но и задача явного нахождения соответствующих образующих. Эти задачи относятся к области так называемого эффективного порождения в группах. Многолетние усилия многих авторов привели к практически полному ответу на вопрос, какие конечные простые группы могут быть порождены инволюцией и элементом порядка три ((2,3)-порождение). Последние существенные результаты в этом направлении были недавно получены Тамбурини и Пеллегрини. С другой стороны, аналогичный вопрос для для другого важного класса, а именно для гурвицевых групп (конечных (2,3,7)-порожденных групп), еще далек от полного разрешения. В докладе будет дан обзор современного состояния этих проблем и сформулированы открытые вопросы.
\end{talk}
\end{document}

