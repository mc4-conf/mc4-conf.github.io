\documentclass[12pt, a4paper, figuresright]{book}
\usepackage{mathtools, commath, amssymb, amsthm}
\usepackage{tabularx,graphicx,url,xcolor,rotating,multicol,epsfig,colortbl,lipsum}
\usepackage[T2A]{fontenc}
\usepackage[english,main=russian]{babel}

\setlength{\textheight}{25.2cm}
\setlength{\textwidth}{16.5cm}
\setlength{\voffset}{-1.6cm}
\setlength{\hoffset}{-0.3cm}
\setlength{\evensidemargin}{-0.3cm} 
\setlength{\oddsidemargin}{0.3cm}
\setlength{\parindent}{0cm} 
\setlength{\parskip}{0.3cm}

\newenvironment{talk}[6]{%
	\vskip 0pt\nopagebreak%
	\vskip 0pt\nopagebreak%
	\textbf{#1}\vspace{3mm}\\\nopagebreak%
	\textit{#2}\\\nopagebreak%
	#3\\\nopagebreak%
	\url{#4}\vspace{3mm}\\\nopagebreak%
	\ifthenelse{\equal{#5}{}}{}{Соавторы: #5\vspace{3mm}\\\nopagebreak}%
	\ifthenelse{\equal{#6}{}}{}{Секция: #6\quad \vspace{3mm}\\\nopagebreak}%
}

\pagestyle{empty}

\begin{document}
\begin{talk} 
{О дистанционно регулярных графах \(\Gamma\) диаметра 3, содержащих максимальный 1-код, и с сильно регулярными графами \(\Gamma_2\) и \(\Gamma_3\)}
{Голубятников Михаил Петрович}
{ИММ УрО РАН}
{mike_ru1@mail.ru} 
{А.\,А. Махнев, Минчжу Чень}
{Алгебра}

Наши терминология и обозначения стандартны, их можно найти в [1].

В докладе рассматриваются дистанционно регулярные графы диаметра \(d=2e+1\), содержащие \(e\)-код \(C\). 
Для \(e\)-кода \(C\) справедлива оценка \(|C|\le p^d_{dd}+2\) (см [2]).

Если равенство достигается, то \(C\) называется максимальным кодом. В случае равенства \(v=|C|(k+1)\) код \(C\) называется совершенным. 

Аналогично, справедлива оценка 
\[|C| \le \frac{k_d}{\sum_{i=0}^e p^d_{id}} + 1.\]

Если равенство достигается в этой границе, то код \(C\) называется совершенным относительно последней окрестности.

Для дистанционно регулярных графов диаметра \(3\), содержащих максимальный локально регулярный 1-код, совершенный относительно последней окрестности, Юришич и Видали нашли возможные массивы пересечений (см [2]). Оказалось, что такой граф \(\Gamma\) имеет массив пересечений \(\{a(p+1),cp,a+1;1,c,ap\}\) (и сильно регулярный граф \(\Gamma_3\)) или \(\{a(p+1),(a+1)p,c;1,c,ap\}\), где \(a=a_3\), \(p=p^3_{33}\), \(c=c_2\). В первом случае при \(a=c+1\) граф \(\Gamma_3\) --- псевдогеометрический граф для 
\(GQ(p+1,c_2+1)\), а \(\bar \Gamma_2\) --- псевдогеометрический граф для \(pG_2(p+1,2c_2+2)\).

В работе рассматриваются дистанционно регулярным графом диаметра \(3\) с сильно регулярными графами \(\Gamma_2\) и \(\Gamma_3\), содержащими максимальный \(1\)-код. 

Основные результаты доклада формулируются в следующих двух теоремах:

\textbf{Теорема 1.}
\textit{Пусть \(\Gamma\) является дистанционно регулярным графом диаметра  
\(3\) с сильно регулярными графами \(\Gamma_2\) и \(\Gamma_3\). Если \(\Gamma\) содержит максимальный \(1\)-код, то
\(a_3=c_2+1\) и \(\Gamma\) имеет массив пересечений \(\{(p+1)(c_2+1),pc_2,c_2+2;1,c_2,p(c_2+1)\}\), где \(p=p^3_{33}\).}

\textbf{Теорема 2.}
\textit{Дистанционно регулярный граф с массивом пересечений 
\[\{(p+1)(c_2+1),pc_2,c_2+2;1,c_2,p(c_2+1)\}\] не существует.}

\medskip

\begin{enumerate}
\item[{[1]}] Brouwer~A.\,E.,  Cohen~A.\,M.,  Neumaier~A. {\it Distance-Regular Graphs}. Berlin; Heidelberg; New York: Springer-Verlag, 1989. 495~p.

\item[{[2]}] A. Jurishich, J. Vidali, {\it Extremal 1-codes in distance-regular graphs of diameter 3},  Des. Codes Cryptogr, 65 (2012), 29-47.
\end{enumerate}
\end{talk}
\end{document}