\documentclass[12pt, a4paper, figuresright]{book}
\usepackage{mathtools, commath, amssymb, amsthm}
\usepackage{tabularx,graphicx,url,xcolor,rotating,multicol,epsfig,colortbl,lipsum}
\usepackage[T2A]{fontenc}
\usepackage[english,main=russian]{babel}

\setlength{\textheight}{25.2cm}
\setlength{\textwidth}{16.5cm}
\setlength{\voffset}{-1.6cm}
\setlength{\hoffset}{-0.3cm}
\setlength{\evensidemargin}{-0.3cm} 
\setlength{\oddsidemargin}{0.3cm}
\setlength{\parindent}{0cm} 
\setlength{\parskip}{0.3cm}

\newenvironment{talk}[6]{%
	\vskip 0pt\nopagebreak%
	\vskip 0pt\nopagebreak%
	\textbf{#1}\vspace{3mm}\\\nopagebreak%
	\textit{#2}\\\nopagebreak%
	#3\\\nopagebreak%
	\url{#4}\vspace{3mm}\\\nopagebreak%
	\ifthenelse{\equal{#5}{}}{}{Соавторы: #5\vspace{3mm}\\\nopagebreak}%
	\ifthenelse{\equal{#6}{}}{}{Секция: #6\quad \vspace{3mm}\\\nopagebreak}%
}

\pagestyle{empty}

\begin{document}
\begin{talk}
{Группы с ограничениями на множество размеров классов сопряженности}
{Горшков Илья Борисович}
{Институт математики им. С.\,Л. Соболева СОРАН}
{ilygor8@gmail.com}
{Алгебра}

Пусть \(G\) --- конечная группа и \(N(G)\) --- множество размеров ее классов сопряженности, исключая 1. Определим ориентированный граф \(\Gamma(G)\), множество вершин этого графа есть \(N(G)\) и вершины \(x\) и \(y\) соединены направленным ребром от \(x\) к \(y\), если \(x\) делит \(y\) и \(N(G)\) не содержать числа \(z\), отличного от \(x\) и \(y\), такого, что \(x\) делит \(z\), а \(z\) делит \(y\). Мы будем называть граф \(\Gamma(G)\) сопряженным графом группы \(G\). Мы изучили конечные группы, сопряженный граф которых не содержит ребер.
\end{talk}
\end{document}