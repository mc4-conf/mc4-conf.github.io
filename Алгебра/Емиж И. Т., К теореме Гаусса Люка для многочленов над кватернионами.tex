\documentclass[12pt, a4paper, figuresright]{book}
\usepackage{mathtools, commath, amssymb, amsthm}
\usepackage{tabularx,graphicx,url,xcolor,rotating,multicol,epsfig,colortbl,lipsum}
\usepackage[T2A]{fontenc}
\usepackage[english,main=russian]{babel}

\setlength{\textheight}{25.2cm}
\setlength{\textwidth}{16.5cm}
\setlength{\voffset}{-1.6cm}
\setlength{\hoffset}{-0.3cm}
\setlength{\evensidemargin}{-0.3cm} 
\setlength{\oddsidemargin}{0.3cm}
\setlength{\parindent}{0cm} 
\setlength{\parskip}{0.3cm}

\newenvironment{talk}[6]{%
\vskip 0pt\nopagebreak%
\vskip 0pt\nopagebreak%
\textbf{#1}\vspace{3mm}\\\nopagebreak%
\textit{#2}\\\nopagebreak%
#3\\\nopagebreak%
\url{#4}\vspace{3mm}\\\nopagebreak%
\ifthenelse{\equal{#5}{}}{}{Соавторы: #5\vspace{3mm}\\\nopagebreak}%
\ifthenelse{\equal{#6}{}}{}{Секция: #6\quad \vspace{3mm}\\\nopagebreak}%
}

\pagestyle{empty}

\begin{document}
\begin{talk}
{К теореме Гаусса Люка для многочленов над кватернионами}
{Емиж Ислам Тимурович}
{Кавказский математический центр Адыгейского государственного университета}
{iemizh@bk.ru}
{А.\,Э. Гутерман}
{Алгебра}

Получено усиление кватернионной теоремы Гаусса-Люка, доказанной Гилони и Перотти в 2018 г. Пусть I --- кватернион единичной нормы без действительной части и P ---
многочлен с кватернионными коэффициентами. Возьмем многочлены полученные из
P путем ортогонального проектирования его коэффциентов на и вдоль C --- плоскости
порожденной 1 и I. Ограничим проекции на данную плоскость, соответственно будем
рассматривать только те корни, которые принадлежат C. Рассмотрим множество, которое является пересечением выпуклых оболочек корней данных проекций. Доказано,
что корни производной многочлена P принадлежат объединению по всем возможным
I таких множеств.

\medskip

\begin{enumerate}
\item[{[1]}] R. Ghiloni, A. Perotti, The quaternionic Gauss–Lucas theorem. — Annali di Matematica Pura ed Applicata (1923-) 197 (2018), 1679-1686.
\end{enumerate}
\end{talk}
\end{document}