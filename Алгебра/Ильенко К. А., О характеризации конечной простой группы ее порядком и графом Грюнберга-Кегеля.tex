\documentclass[12pt, a4paper, figuresright]{book}
\usepackage{mathtools, commath, amssymb, amsthm}
\usepackage{tabularx,graphicx,url,xcolor,rotating,multicol,epsfig,colortbl,lipsum}
\usepackage[T2A]{fontenc}
\usepackage[english,main=russian]{babel}

\setlength{\textheight}{25.2cm}
\setlength{\textwidth}{16.5cm}
\setlength{\voffset}{-1.6cm}
\setlength{\hoffset}{-0.3cm}
\setlength{\evensidemargin}{-0.3cm}
\setlength{\oddsidemargin}{0.3cm}
\setlength{\parindent}{0cm}
\setlength{\parskip}{0.3cm}

\newenvironment{talk}[6]{%
\vskip 0pt\nopagebreak%
\vskip 0pt\nopagebreak%
\textbf{#1}\vspace{3mm}\\\nopagebreak%
\textit{#2}\\\nopagebreak%
#3\\\nopagebreak%
\url{#4}\vspace{3mm}\\\nopagebreak%
\ifthenelse{\equal{#5}{}}{}{Соавторы: #5\vspace{3mm}\\\nopagebreak}%
\ifthenelse{\equal{#6}{}}{}{Секция: #6\quad \vspace{3mm}\\\nopagebreak}%
}

\pagestyle{empty}

\begin{document}
\begin{talk}
{О характеризации конечной простой группы ее порядком и графом Грюнберга--Кегеля}
{Ильенко Кристина Альбертовна}
{Институт математики и механики им. Н.\,Н. Красовского УрО РАН}
{christina.ilyenko@yandex.ru}
{}
{Алгебра}

Пусть \(G\) --- конечная группа. {\it Спектром} \(\omega(G)\) называется множество всех порядков элементов группы \(G\). Под {\it простым спектром} \(\pi(G)\) понимают множество всех простых чисел из \(\omega(G)\). Неориентированный граф без петель и кратных рёбер \(\Gamma(G)\), множество вершин которого совпадает с \(\pi(G)\), и две различные вершины \(p\) и \(q\) в котором смежны тогда и только тогда, когда \(pq \in \omega(G)\), называется {\it графом Грюнберга--Кегеля} или {\it графом простых чисел} группы \(G\). 

В. Ши поставил вопрос характеризации конечной простой группы ее спектром и порядком. Этот вопрос исследовался в серии работ, и 2009 г. А.В. Васильевым, М.А. Гречкосеевой и В.Д. Мазуровым [1] было доказано, что если \(G\) --- конечная простая группа и \(H\)~--- конечная группа такая, что \(\omega(H)=\omega(G)\) и \(|H|=|G|\), то \(H \cong G\). 

Поскольку понятие графа Грюнберга--Кегеля широко обобщает понятие спектра конечной группы, то вопрос характеризации конечной простой группы ее порядком и графом Грюнберга--Кегеля возникает естественным образом. Историю постановки этого вопроса можно найти, например, в [2].

В этой работе мы обсуждаем вопрос характеризации порядком и графом Грюнберга--Кегеля конечной простой группы, граф Грюнберга--Кегеля которой содержит не менее трех компонент связности. В частности, мы исправляем неточности, допущенные в работах [3] и [4].

\medskip

\begin{enumerate}
\item[{[1]}] А.~В.~Васильев, М.~А.~Гречкосеева, В.~Д.~Мазуров, {\it Характеризация конечных простых групп спектром и порядком}, Алгебра и логика, \textbf{48}:6 (2009), 685–728.
\item[{[2]}] Peter~Cameron, Natalia~Maslova, {\it Criterion of unrecognizability of a finite group by its Gruenberg-Kegel graph}, J. Algebra, \textbf{607}:Part A (2022), 186-213.
\item[{[3]}] Bahman~Khosravi, Behnam~Khosravi, Behrooz~Khosravi, {\it On the prime graph of \(PSL(2, p)\) where \(p > 3\) is a prime number}, Acta Math Hungar, \textbf{116}:4 (2007), 295.
\item[{[4]}] Q.~Zhang, W.~Shi, R.~Shen, {\it Quasirecognition by prime graph of the simple groups \(G_2(q)\) and \({^2}B_2(q)\)}, J. Algebra Appl., \textbf{10}:2 (2011), 309--317.
\end{enumerate}
\end{talk}
\end{document} 