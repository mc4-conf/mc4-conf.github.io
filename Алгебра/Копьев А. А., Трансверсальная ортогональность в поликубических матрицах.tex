\documentclass[12pt, a4paper, figuresright]{book}
\usepackage{mathtools, commath, amssymb, amsthm}
\usepackage{tabularx,graphicx,url,xcolor,rotating,multicol,epsfig,colortbl,lipsum}
\usepackage[T2A]{fontenc}
\usepackage[english,main=russian]{babel}

\setlength{\textheight}{25.2cm}
\setlength{\textwidth}{16.5cm}
\setlength{\voffset}{-1.6cm}
\setlength{\hoffset}{-0.3cm}
\setlength{\evensidemargin}{-0.3cm} 
\setlength{\oddsidemargin}{0.3cm}
\setlength{\parindent}{0cm} 
\setlength{\parskip}{0.3cm}

\newenvironment{talk}[6]{%
\vskip 0pt\nopagebreak%
\vskip 0pt\nopagebreak%
\textbf{#1}\vspace{3mm}\\\nopagebreak%
\textit{#2}\\\nopagebreak%
#3\\\nopagebreak%
\url{#4}\vspace{3mm}\\\nopagebreak%
\ifthenelse{\equal{#5}{}}{}{Соавторы: #5\vspace{3mm}\\\nopagebreak}%
\ifthenelse{\equal{#6}{}}{}{Секция: #6\quad \vspace{3mm}\\\nopagebreak}%
}

\pagestyle{empty}

\begin{document}
\begin{talk}
{Трансверсальная ортогональность в поликубических матрицах}
{Копьев Алексей Александрович}
{Студент 4 курса бакалавриата, Казанский Федеральный Университет}
{AAKopev@stud.kpfu.ru}
{Хмельницкая Алена Владимировна, Яшагин Евгений Иванович}
{Алгебра}

Мы определяем на множестве поликубически матриц умножения -- расширения классического не путем тензорного умножения двумерных, а по принципу: строка на столбец, столбец на ряд и т. д.

В работе получена формула числа новых умножений, введено понятие \(k\)-мерной трансверсальной ортогональности в \(n\)-размерных поликубических матрицах и приведены формулы для поиска алгебр им удовлетворяющих. На примерах трехмерных кубических матриц продемонстрированы групповые структуры и выделены те, которые сохраняют трансверсальную ортогональность. Понятие трансверсальной ортогональности обязано своим появлением необходимости описания с помощью поликубических матриц многочастичных квантовых взаимодействий элементарных частиц со множеством квантовых характеристик. Полученные формулы вводят для этого новые понятия полиэрмитовых алгебр.
\end{talk}
\end{document}