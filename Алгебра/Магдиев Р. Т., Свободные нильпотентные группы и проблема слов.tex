\documentclass[12pt, a4paper, figuresright]{book}
\usepackage{mathtools, commath, amssymb, amsthm}
\usepackage{tabularx,graphicx,url,xcolor,rotating,multicol,epsfig,colortbl,lipsum}
\usepackage[T2A]{fontenc}
\usepackage[english,main=russian]{babel}

\setlength{\textheight}{25.2cm}
\setlength{\textwidth}{16.5cm}
\setlength{\voffset}{-1.6cm}
\setlength{\hoffset}{-0.3cm}
\setlength{\evensidemargin}{-0.3cm} 
\setlength{\oddsidemargin}{0.3cm}
\setlength{\parindent}{0cm} 
\setlength{\parskip}{0.3cm}

\newenvironment{talk}[6]{%
\vskip 0pt\nopagebreak%
\vskip 0pt\nopagebreak%
\textbf{#1}\vspace{3mm}\\\nopagebreak%
\textit{#2}\\\nopagebreak%
#3\\\nopagebreak%
\url{#4}\vspace{3mm}\\\nopagebreak%
\ifthenelse{\equal{#5}{}}{}{Соавторы: #5\vspace{3mm}\\\nopagebreak}%
\ifthenelse{\equal{#6}{}}{}{Секция: #6\quad \vspace{3mm}\\\nopagebreak}%
}

\pagestyle{empty}

\begin{document}
\begin{talk}
{Свободные нильпотентные группы и проблема слов}
{Магдиев Руслан Тимурович}
{Магистрант фМКН СПБГУ}
{rus.magdy@mail.ru}
{Семидетнов Артём Алексеевич}
{Алгебра}

Доклад посвящен геометрической интерпретации решения проблемы слов для конечно-порождённых свободных нильпотентных групп. В ходе описания решения будет описана обнаруженная нами геометрическая конструкция, позволяющая построить ранее неизвестные представления для групп из данного класса. Также будет уделено время обобщениям на случай произвольной конечно-порождённой нильпотентной группы. Доклад основан на [1, 2] и на новых, ранее неопубликованных результатах в этой области математики.

\medskip

\begin{enumerate}
\item[{[1]}] I. Alekseev, R. Magdiev, {\it The language of geodesics for the discrete Heisenberg group}, {\tt arXiv:1905.03226}, 2019.
\item[{[2]}] R. Magdiev, A. Semidetnov, \textit{On the geometry of free nilpotent groups}, \texttt{arXiv:2106.00095}, 2021.
\end{enumerate}
\end{talk}
\end{document}
