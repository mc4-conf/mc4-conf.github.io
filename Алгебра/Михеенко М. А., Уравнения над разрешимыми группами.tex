\documentclass[12pt, a4paper, figuresright]{book}
\usepackage{mathtools, commath, amssymb, amsthm}
\usepackage{tabularx,graphicx,url,xcolor,rotating,multicol,epsfig,colortbl,lipsum}
\usepackage[T2A]{fontenc}
\usepackage[english,main=russian]{babel}

\setlength{\textheight}{25.2cm}
\setlength{\textwidth}{16.5cm}
\setlength{\voffset}{-1.6cm}
\setlength{\hoffset}{-0.3cm}
\setlength{\evensidemargin}{-0.3cm} 
\setlength{\oddsidemargin}{0.3cm}
\setlength{\parindent}{0cm} 
\setlength{\parskip}{0.3cm}

\newenvironment{talk}[6]{%
\vskip 0pt\nopagebreak%
\vskip 0pt\nopagebreak%
\textbf{#1}\vspace{3mm}\\\nopagebreak%
\textit{#2}\\\nopagebreak%
#3\\\nopagebreak%
\url{#4}\vspace{3mm}\\\nopagebreak%
\ifthenelse{\equal{#5}{}}{}{Соавторы: #5\vspace{3mm}\\\nopagebreak}%
\ifthenelse{\equal{#6}{}}{}{Секция: #6\quad \vspace{3mm}\\\nopagebreak}%
}

\pagestyle{empty}

\begin{document}
\begin{talk}
{Уравнения над разрешимыми группами}
{Михеенко Михаил Александрович}
{Механико-математический факультет МГУ имени М. В. Ломоносова;
Московский Центр фундаментальной и прикладной математики}
{mamikheenko@mail.ru}
{А.\,А. Клячко, В.\,А. Романьков}
{Алгебра}

Доклад посвящён уравнениям и системам уравнений над разрешимыми группами.

Будет приведено условие, при котором группа
вкладывается в разрешимую группу, содержащую
решения всех невырожденных систем уравнений над собой,
и аналог этого условия для систем,
невырожденных по некоторому простому модулю.

Также будет обсуждён минимальный по порядку
пример метабелевой группы,
над которой есть унимодулярное уравнение,
не имеющее решений в метабелевых группах.
\end{talk}
\end{document}