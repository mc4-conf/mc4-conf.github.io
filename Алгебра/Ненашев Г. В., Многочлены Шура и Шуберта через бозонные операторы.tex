\documentclass[12pt, a4paper, figuresright]{book}
\usepackage{mathtools, commath, amssymb, amsthm}
\usepackage{tabularx,graphicx,url,xcolor,rotating,multicol,epsfig,colortbl,lipsum}
\usepackage[T2A]{fontenc}
\usepackage[english,main=russian]{babel}

\setlength{\textheight}{25.2cm}
\setlength{\textwidth}{16.5cm}
\setlength{\voffset}{-1.6cm}
\setlength{\hoffset}{-0.3cm}
\setlength{\evensidemargin}{-0.3cm} 
\setlength{\oddsidemargin}{0.3cm}
\setlength{\parindent}{0cm} 
\setlength{\parskip}{0.3cm}

\newenvironment{talk}[6]{%
\vskip 0pt\nopagebreak%
\vskip 0pt\nopagebreak%
\textbf{#1}\vspace{3mm}\\\nopagebreak%
\textit{#2}\\\nopagebreak%
#3\\\nopagebreak%
\url{#4}\vspace{3mm}\\\nopagebreak%
\ifthenelse{\equal{#5}{}}{}{Соавторы: #5\vspace{3mm}\\\nopagebreak}%
\ifthenelse{\equal{#6}{}}{}{Секция: #6\quad \vspace{3mm}\\\nopagebreak}%
}

\pagestyle{empty}

\begin{document}
\begin{talk}
{Многочлены Шура и Шуберта через бозонные операторы}
{Ненашев Глеб Вячеславович}
{Санкт-Петербургский государственный университет}
{glebnen@gmail.com}
{}
{Алгебра}

В этом докладе будет рассказан новый способ работать с функциями Шура и их обобщением полиномами Шуберта. Полиномы Шуберта представляют классы когомологий циклов Шуберта в многообразиях полных флагов, когда как функции Шура представляют классы для Грассманова многообразия. 
Мы введем два дифференциальных оператора, которые однозначно определяют коэффициенты Литтлвуда-Ричардсона и структурные константы для полиномов Шуберта. В частности этот способ позволяет работать с многочленами Шура и Шуберта без использования понятия многочленов.
\end{talk}
\end{document}