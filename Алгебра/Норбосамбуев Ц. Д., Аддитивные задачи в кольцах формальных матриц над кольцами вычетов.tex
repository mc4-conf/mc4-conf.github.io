\documentclass[12pt, a4paper, figuresright]{book}
\usepackage{mathtools, commath, amssymb, amsthm}
\usepackage{tabularx,graphicx,url,xcolor,rotating,multicol,epsfig,colortbl,lipsum}
\usepackage[T2A]{fontenc}
\usepackage[english,main=russian]{babel}

\setlength{\textheight}{25.2cm}
\setlength{\textwidth}{16.5cm}
\setlength{\voffset}{-1.6cm}
\setlength{\hoffset}{-0.3cm}
\setlength{\evensidemargin}{-0.3cm} 
\setlength{\oddsidemargin}{0.3cm}
\setlength{\parindent}{0cm} 
\setlength{\parskip}{0.3cm}

\newenvironment{talk}[6]{%
\vskip 0pt\nopagebreak%
\vskip 0pt\nopagebreak%
\textbf{#1}\vspace{3mm}\\\nopagebreak%
\textit{#2}\\\nopagebreak%
#3\\\nopagebreak%
\url{#4}\vspace{3mm}\\\nopagebreak%
\ifthenelse{\equal{#5}{}}{}{Соавторы: #5\vspace{3mm}\\\nopagebreak}%
\ifthenelse{\equal{#6}{}}{}{Секция: #6\quad \vspace{3mm}\\\nopagebreak}%
}

\pagestyle{empty}

\begin{document}
\begin{talk}
{Аддитивные задачи в кольцах формальных матриц над кольцами вычетов}
{Норбосамбуев Цырендоржи Дашацыренович}
{НОМЦ ТГУ, Томск}
{nstsddts@yandex.ru}
{}
{Алгебра}

Изучение колец, порождаемых аддитивно своими специальными элементами, --- под ``специальными'' подразумеваем: обратимые, инволюции, идемпотенты, \(q\)-потенты, нильпотенты и т.п. --- хорошо известная задача, давно привлекающая внимание многих алгебраистов. 
Часто данное направление исследований называют «аддитивными задачами в кольцах». 

Нас интересуют аддитивные задачи в кольцах формальных матриц над кольцами вычетов (см. [1-5]). Такое кольцо будем обозначать буквой \(K\).
Группа обратимых матриц в \(K\)  была описана в [2]. 
Ответ на вопрос о хорошести формальных матриц из \(K\) дан в статье [3]. 
Нильпотентные формальные матрицы из \(K\)  описаны в[ 4]. К слову, нильпотенты в \(K\) образуют идеал.
В \(K\) есть матрицы, непредставимые в виде суммы нильпотентной и обратимой матриц. 
Другими словами, \(K\) --- не изящное кольцо [4]. 
Более того, оно не будет ниль-хорошим [4]. 
При \(p>2\) в \(K\) найдутся также матрицы, которые нельзя записать как сумму нильпотентной и идемпотентной матриц, то есть \(K\) --- не ниль-чистое кольцо [5].

Кольца формальных матриц над кольцами вычетов интересны как сами по себе, так и как возможная основа для построения некоммутативного протокола шифрования данных (см. литературу в [3]).
Этот раздел криптографии --- некоммутативная алгебраическая криптография --- в последнее время развивается особенно бурно.

\medskip

\begin{enumerate}
\item[{[1]}] Крылов П.А., Туганбаев А.А. Кольца формальных матриц и модули над ними. М.: МЦНМО, 2017.
\item[{[2]}] Степанова А.Ю., Тимошенко Е.А. Матричное представление эндоморфизмов примарных групп малых рангов // Вестник Томского госуниверситета. Математика и механика. 2021. № 74. С. 30–42. DOI: {\tt 10.17223/19988621/74/4}.
\item[{[3]}] Норбосамбуев Ц.Д. Хорошие кольца формальных матриц над кольцами вычетов // Вестник Томского госуниверситета. Математика и механика. 2023. № 85. С. 32–42. DOI: {\tt 10.17223/19988621/85/3}.
\item[{[4]}] Елфимова А.М., Норбосамбуев Ц.Д., Подкорытов М.В. Нильпотентные, ниль-хорошие и ниль-чистые формальные матрицы над кольцами вычетов // Вестник Томского госуниверситета. Математика и механика (в печати).
\item[{[5]}] Елфимова А.М., Норбосамбуев Ц.Д., Подкорытов М.В. Идемпотентные и ниль-чистые формальные матрицы над кольцами вычетов // Вестник Томского госуниверситета. Математика и механика (готовится к печати).
\end{enumerate}
\end{talk}
\end{document}
