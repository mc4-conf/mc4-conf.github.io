\documentclass[12pt, a4paper, figuresright]{book}
\usepackage{mathtools, commath, amssymb, amsthm}
\usepackage{tabularx,graphicx,url,xcolor,rotating,multicol,epsfig,colortbl,lipsum}
\usepackage[T2A]{fontenc}
\usepackage[english,main=russian]{babel}

\setlength{\textheight}{25.2cm}
\setlength{\textwidth}{16.5cm}
\setlength{\voffset}{-1.6cm}
\setlength{\hoffset}{-0.3cm}
\setlength{\evensidemargin}{-0.3cm} 
\setlength{\oddsidemargin}{0.3cm}
\setlength{\parindent}{0cm} 
\setlength{\parskip}{0.3cm}

\newenvironment{talk}[6]{%
\vskip 0pt\nopagebreak%
\vskip 0pt\nopagebreak%
\textbf{#1}\vspace{3mm}\\\nopagebreak%
\textit{#2}\\\nopagebreak%
#3\\\nopagebreak%
\url{#4}\vspace{3mm}\\\nopagebreak%
\ifthenelse{\equal{#5}{}}{}{Соавторы: #5\vspace{3mm}\\\nopagebreak}%
\ifthenelse{\equal{#6}{}}{}{Секция: #6\quad \vspace{3mm}\\\nopagebreak}%
}

\pagestyle{empty}

\begin{document}
\begin{talk}
{Порождающие множества инволюций почти простых групп и их приложения}
{Нужин Яков Нифантьевич}
{Сибирский федеральный университет}
{nuzhin2008@rambler.ru}
{}
{Алгебра}

В докладе речь пойдет о порождающих множествах инволюций малой мощности с определенными свойствами и их приложенииях. В частности, будут отмечены результаты по следующему вопросу из Коуровской тетради, записанному автором в 1999 г.

{\it Для каждой конечной простой неабелевой группы найти минимум числа порождающих инволюций, удовлетворяющих дополнительному условию, в каждом из следующих случаев.

а) Произведение порождающих инволюций равно 1.

б) (Малле-Саксл-Вайгель). Порождающие инволюции сопряжены.

в) (Малле-Саксл-Вайгель). Выполнены свойства а) и б).

г) Инволюции сопряжены и две из них перестановочны.}
\end{talk}
\end{document}
