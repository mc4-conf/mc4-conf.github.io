\documentclass[12pt, a4paper, figuresright]{book}
\usepackage{mathtools, commath, amssymb, amsthm}
\usepackage{tabularx,graphicx,url,xcolor,rotating,multicol,epsfig,colortbl,lipsum}
\usepackage[T2A]{fontenc}
\usepackage[english,main=russian]{babel}

\setlength{\textheight}{25.2cm}
\setlength{\textwidth}{16.5cm}
\setlength{\voffset}{-1.6cm}
\setlength{\hoffset}{-0.3cm}
\setlength{\evensidemargin}{-0.3cm} 
\setlength{\oddsidemargin}{0.3cm}
\setlength{\parindent}{0cm} 
\setlength{\parskip}{0.3cm}

\newenvironment{talk}[6]{%
\vskip 0pt\nopagebreak%
\vskip 0pt\nopagebreak%
\textbf{#1}\vspace{3mm}\\\nopagebreak%
\textit{#2}\\\nopagebreak%
#3\\\nopagebreak%
\url{#4}\vspace{3mm}\\\nopagebreak%
\ifthenelse{\equal{#5}{}}{}{Соавторы: #5\vspace{3mm}\\\nopagebreak}%
\ifthenelse{\equal{#6}{}}{}{Секция: #6\quad \vspace{3mm}\\\nopagebreak}%
}

\pagestyle{empty}

\begin{document}
\begin{talk}
{k-замыкания групп подстановок}
{Скресанов Савелий Вячеславович}
{Новосибирский государственный университет; Институт математики им. С.\,Л. Соболева СО РАН}
{skresan@math.nsc.ru}
{}
{Алгебра}

Группа подстановок \(G\) на конечном множестве \(\Omega\) также действует
естественным образом на декартовой степени \(\Omega^k\),  \(k \geq 1\).
Наибольшая группа подстановок на  \(\Omega\), имеющая такие же орбиты на \(\Omega^k\), что и \(G\), называется \(k\)-замыканием группы \(G\). Это понятие,
предложенное Х.~Виландом в связи с изучением порядков примитивных групп
подстановок, нашло впоследствии многочисленные применения для изучения
автоморфизмов комбинаторных структур --- например, группа автоморфизмов любого
графа будет совпадать со своим \(2\)-замыканием. Особо важным оказалось изучение
абстрактных свойств групп, сохраняемых при \(k\)-замыканиях. В докладе планируется
рассказать о недавних продвижениях в этом направлении, а также затронуть
алгоритмические вопросы вычисления \(k\)-замыканий и их связь с проблемой
изоморфизма графов из теории сложности.

\medskip

Исследование выполнено за счет гранта Российского научного фонда \textnumero 24-11-00127, \url{https://rscf.ru/project/24-11-00127/}.
\end{talk}
\end{document}
