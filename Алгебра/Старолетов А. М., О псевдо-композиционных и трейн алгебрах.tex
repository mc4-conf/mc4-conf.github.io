\documentclass[12pt, a4paper, figuresright]{book}
\usepackage{mathtools, commath, amssymb, amsthm}
\usepackage{tabularx,graphicx,url,xcolor,rotating,multicol,epsfig,colortbl,lipsum}
\usepackage[T2A]{fontenc}
\usepackage[english,main=russian]{babel}

\setlength{\textheight}{25.2cm}
\setlength{\textwidth}{16.5cm}
\setlength{\voffset}{-1.6cm}
\setlength{\hoffset}{-0.3cm}
\setlength{\evensidemargin}{-0.3cm}
\setlength{\oddsidemargin}{0.3cm}
\setlength{\parindent}{0cm}
\setlength{\parskip}{0.3cm}

\newenvironment{talk}[6]{%
\vskip 0pt\nopagebreak%
\vskip 0pt\nopagebreak%
\textbf{#1}\vspace{3mm}\\\nopagebreak%
\textit{#2}\\\nopagebreak%
#3\\\nopagebreak%
\url{#4}\vspace{3mm}\\\nopagebreak%
\ifthenelse{\equal{#5}{}}{}{Соавторы: #5\vspace{3mm}\\\nopagebreak}%
\ifthenelse{\equal{#6}{}}{}{Секция: #6\quad \vspace{3mm}\\\nopagebreak}%
}

\pagestyle{empty}

\begin{document}
\begin{talk}
{О псевдо-композиционных и трейн алгебрах}
{Старолетов Алексей Михайлович}
{Институт математики имени С. Л. Соболева СО РАН}
{staroletov@math.nsc.ru}
{}
{Алгебра}

Коммутативная неассоциативная алгебра имеет ранг \(r\), если каждый её элемент порождает подалгебру размерности не более \(r-1\). Нас будут интересовать следующие два класса алгебр ранга 3.
Предположим, что \(\mathbb{F}\) -- поле характеристики, отличной от 2 и 3.
Коммутативная \(\mathbb{F}\)-алгебра \(A\), на которой задана ненулевая симметрическая билинейная форма \(\varphi\), называется псевдо-композицинной, если
\(x^3=\varphi(x, x)x\) для всех \(x\in A\). Эти алгебры активно изучались в прошлом, в частности Мейберг и Осборн получили классификацию, при некоторых ограничениях, в [1].

Второй класс --- это трейн алгебры ранга 3. Пусть, как и ранее, \(A\) --- коммутативная \(\mathbb{F}\)-алгебра, где \(\operatorname{char}\mathbb{F}\neq 2,3\). Главные степени элементов в \(A\) определяются следующим образом: \(x^1=x\) и \(x^i=x^{i-1}x\) при \(i \geq 2\).
Если существует ненулевой гомоморфизм алгебр \(\omega:A\rightarrow\mathbb{F}\), то
\(A\) называется барической. В этом случае пара \((A,\omega)\) называется трейн алгеброй ранга \(r\), если существуют такие элементы \(\lambda_1,\ldots,\lambda_{r-1}\in\mathbb{F}\), что каждый \(x\in A\) удовлетворяет равенству
\(x^r+\lambda_1\omega(x)x^{r-1}+\ldots+\lambda_{r-1}\omega(x)^{r-1}x=0\).
Эти алгебры были введены Этерингтоном в 1939 году как часть алгебраического формализма генетики в его фундаментальной работе [2].

Предположим, что \(A\) --- коммутативная \(\mathbb{F}\)-алгебра и \(a\in A\). Если \(\lambda\in\mathbb{F}\), то обозначим \(A_\lambda(a)=\{u\in A~|~au=\lambda u\}\) и для \(L\subseteq F\) определим \(A_L(a):= \bigoplus A_\lambda(a)\).

Псевдо-композиционные алгебры и трейн алгебры ранга 3 обладают следующим общим свойством: для алгебры \(A\) найдётся такой элемент \(\eta\in\mathbb{F}\setminus\{\frac{1}{2},1\}\), что для каждого идемпотента \(e\in A\) справедливо разложение Пирса
\[A=A_1(e)\oplus A_{\eta}(e)\oplus A_\frac{1}{2}(e),\]
где \(A_1(e)=\langle e\rangle\), с правилами умножения
\[A_\eta(e)^2\subseteq A_1(e), A_{1/2}(e)^2\subseteq A_1(e)\oplus A_\eta(e), A_\frac{1}{2}(e)A_\eta(e)\subseteq A_\frac{1}{2}(e).\]
Будем называть такой идемпотент \(\eta\)-осью в алгебре \(A\).
Оказывается это свойство характеризует два упомянутых класса алгебр в следующем смысле.

{\bf Теорема.}
{\it Пусть \(\mathbb{F}\) --- поле характеристики, отличной от \(2\) и \(3\).
Предположим, что \(\eta\in\mathbb{F}\setminus\{\frac{1}{2},1\}\) и
\(A\) --- коммутативная \(\mathbb{F}\)-алгебра, порождённая множеством \(\eta\)-осей.
Справедливы следующие утверждения.
\begin{enumerate}
\item[(a)] если \(\eta=-1\), то \(A\) --- псевдо-композиционная алгебра;
\item[(b)] если \(\eta\neq-1\), то \(A\) --- трейн алгебра ранга \(3\).
\end{enumerate}
}

При доказательстве используются методы, развитые ранее при исследовании аксиальных алгебр, определённых в [3]. Результаты доступны в виде препринта {\tt arXiv:2309.05237}.

\medskip

\begin{enumerate}
\item[{[1]}]
K.~Meyberg, J.M.~Osborn, {\it Pseudo-composition algebras}, Math Z., {\bf214}:1 (1993), 67--77.
\item[{[2]}] I.M.H.~Etherington, {\it Genetic algebras},
Proc. Roy. Soc. Edinburgh, {\bf59} (1939), 242--258.
\item[{[3]}] J.I.~Hall, F.~Rehren and S.~Shpectorov, {\it Universal axial algebras and a theorem of Sakuma}, J. Algebra, {\bf421} (2015), 394--424.
\end{enumerate}
\end{talk}
\end{document}
