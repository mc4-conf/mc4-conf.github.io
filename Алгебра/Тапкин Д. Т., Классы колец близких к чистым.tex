\documentclass[12pt, a4paper, figuresright]{book}
\usepackage{mathtools, commath, amssymb, amsthm}
\usepackage{tabularx,graphicx,url,xcolor,rotating,multicol,epsfig,colortbl,lipsum}
\usepackage[T2A]{fontenc}
\usepackage[english,main=russian]{babel}

\setlength{\textheight}{25.2cm}
\setlength{\textwidth}{16.5cm}
\setlength{\voffset}{-1.6cm}
\setlength{\hoffset}{-0.3cm}
\setlength{\evensidemargin}{-0.3cm} 
\setlength{\oddsidemargin}{0.3cm}
\setlength{\parindent}{0cm} 
\setlength{\parskip}{0.3cm}

\newenvironment{talk}[6]{%
\vskip 0pt\nopagebreak%
\vskip 0pt\nopagebreak%
\textbf{#1}\vspace{3mm}\\\nopagebreak%
\textit{#2}\\\nopagebreak%
#3\\\nopagebreak%
\url{#4}\vspace{3mm}\\\nopagebreak%
\ifthenelse{\equal{#5}{}}{}{Соавторы: #5\vspace{3mm}\\\nopagebreak}%
\ifthenelse{\equal{#6}{}}{}{Секция: #6\quad \vspace{3mm}\\\nopagebreak}%
}

\pagestyle{empty}

\begin{document}
\begin{talk}
{Классы колец близких к чистым}
{Тапкин Даниль Тагирзянович}
{Казанский (Приволжский) федеральный университет}
{danil.tapkin@yandex.ru}
{Абызов А.Н.}
{Алгебра}

Под чистым кольцом понимается кольцо, в котором каждый элемент является суммой идемпотента и обратимого элемента.
Данные кольца были введены Никольсоном в 1977\,г. в связи с изучением колец со свойством замены.
К примеру, чистым является кольцо эндоморфизмов всякого инъективного модуля. В 2013\,г. Дизл ввел понятие ниль-чистого кольца --- это кольцо, в котором каждый элемент является суммой илемпотента и нильпотента. Вполне очевидно, что каждое ниль-чистое кольцо является чистым (но обратное, вообще говоря, неверно).
В 2017\,г. Матзук установил интересную связь между свойствами ниль-чистых колец и проблемой Кете.

Естественным обобщением понятия ниль-чистого кольца являются \(q\)-ниль-чистые кольца --- кольца, в которых каждый элемент представим в виде суммы \(q\)-потента (элемента, который равен самому себе при возведении в натуральную степень \(q\)) и нильпотента.
Также представляют интерес кольца, в которых каждый элемент является суммой \(q_{1}\)-потента и \(q_{2}\)-потента. Все указанные выше кольца мы называем кольцами близкими к чистым. Эти классы колец в последнее время исследовались во многих работах.

В докладе будет приведен обзор результатов связанных с кольцами близкими к чистым, включая новые результаты полученные авторами.
\end{talk}
\end{document}