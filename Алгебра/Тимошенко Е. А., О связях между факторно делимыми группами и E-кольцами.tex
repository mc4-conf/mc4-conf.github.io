\documentclass[12pt, a4paper, figuresright]{book}
\usepackage{mathtools, commath, amssymb, amsthm}
\usepackage{tabularx,graphicx,url,xcolor,rotating,multicol,epsfig,colortbl,lipsum}
\usepackage[T2A]{fontenc}
\usepackage[english,main=russian]{babel}

\setlength{\textheight}{25.2cm}
\setlength{\textwidth}{16.5cm}
\setlength{\voffset}{-1.6cm}
\setlength{\hoffset}{-0.3cm}
\setlength{\evensidemargin}{-0.3cm}
\setlength{\oddsidemargin}{0.3cm}
\setlength{\parindent}{0cm}
\setlength{\parskip}{0.3cm}

\newenvironment{talk}[6]{%
\vskip 0pt\nopagebreak%
\vskip 0pt\nopagebreak%
\textbf{#1}\vspace{3mm}\\\nopagebreak%
\textit{#2}\\\nopagebreak%
#3\\\nopagebreak%
\url{#4}\vspace{3mm}\\\nopagebreak%
\ifthenelse{\equal{#5}{}}{}{Соавторы: #5\vspace{3mm}\\\nopagebreak}%
\ifthenelse{\equal{#6}{}}{}{Секция: #6\quad \vspace{3mm}\\\nopagebreak}%
}

\pagestyle{empty}

\begin{document}
\begin{talk}
{О связях между факторно делимыми группами и \(E\)-кольцами}
{Тимошенко Егор Александрович}
{НОМЦ Томского государственного университета}
{tea471@mail.tsu.ru}
{М.\,Н. Зонов}
{Алгебра}

Факторно делимые группы без кручения были введены Бьюмонтом и Пирсом в 1961 году.
Позднее Фомин и Уиклесс распространили определение на случай произвольных абелевых групп:

{\bf Определение.}
Пусть \(n\) --- неотрицательное целое число. Группу \(G\) называют {\it факторно делимой группой ранга \(n\)}, если ее периодическая часть редуцированна и существует свободная
подгруппа \(F \subset G\) ранга \(n\) такая, что \(G/F\) --- делимая периодическая группа.

Важность изучения факторно делимых групп связана, в частности, с тем фактом, что
образуемая ими категория двойственна категории групп без кручения конечного ранга.

С другой стороны, \(E\)-кольца были введены Шульцем в 1973 году как попытка решения одной
из проблем Фукса и быстро стали предметом активного изучения:

{\bf Определение.}
Кольцо \(R\) называется {\it \(E\)-кольцом}, если оно коммутативно и всякий аддитивный гомоморфизм
\(R \to R\) представляет собой умножение на некоторый элемент из \(R\).

Авторы доклада исследовали и систематизировали связи между факторно делимыми группами
и \(E\)-кольцами. В частности, получен критерий того, чтобы аддитивная группа \(E\)-кольца
была факторно делимой.
Также дано отрицательное решение проблемы Боушелла и Шульца о квазиразложениях \(E\)-колец.
\end{talk}
\end{document}







