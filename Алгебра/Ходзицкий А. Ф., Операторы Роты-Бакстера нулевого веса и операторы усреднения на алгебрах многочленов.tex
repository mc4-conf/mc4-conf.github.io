\documentclass[12pt, a4paper, figuresright]{book}
\usepackage{mathtools, commath, amssymb, amsthm}
\usepackage{tabularx,graphicx,url,xcolor,rotating,multicol,epsfig,colortbl,lipsum}
\usepackage[T2A]{fontenc}
\usepackage[english,main=russian]{babel}

\setlength{\textheight}{25.2cm}
\setlength{\textwidth}{16.5cm}
\setlength{\voffset}{-1.6cm}
\setlength{\hoffset}{-0.3cm}
\setlength{\evensidemargin}{-0.3cm} 
\setlength{\oddsidemargin}{0.3cm}
\setlength{\parindent}{0cm} 
\setlength{\parskip}{0.3cm}

\newenvironment{talk}[6]{%
\vskip 0pt\nopagebreak%
\vskip 0pt\nopagebreak%
\textbf{#1}\vspace{3mm}\\\nopagebreak%
\textit{#2}\\\nopagebreak%
#3\\\nopagebreak%
\url{#4}\vspace{3mm}\\\nopagebreak%
\ifthenelse{\equal{#5}{}}{}{Соавторы: #5\vspace{3mm}\\\nopagebreak}%
\ifthenelse{\equal{#6}{}}{}{Секция: #6\quad \vspace{3mm}\\\nopagebreak}%
}

\pagestyle{empty}

\begin{document}
\begin{talk}
{Операторы Роты--Бакстера нулевого веса и операторы усреднения на алгебрах многочленов}
{Ходзицкий Артем Федорович}
{НГУ, Новосибирск}
{a.khodzitskii@g.nsu.ru}
{}
{Алгебра}

Пусть \(A\) --- алгебра над полем \(F\).
Линейный оператор \(T\) на \(A\) называется оператором усреднения,
если выполнены соотношения 
\(T(a)T(b) = T(T(a)b) = T(a T(b))\) 
для всех \(a,b\in A\).
Линейный оператор \(R\) на \(A\)
называется оператором Роты--Бакстера, если 
\[R(a)R(b) = R\big(R(a)b + a R(b) + \lambda a b\big)\]
выполнено для всех~\(a,b \in A\).
Здесь \(\lambda\in F\) --- фиксированный скаляр, вес оператора~\(R\).

Линейный оператор \(L\) на алгебре многочленов называется мономиальным, 
если для любого монома \(t\) найдутся 
моном \(z_t\) и скаляр \(\alpha_t\) такие, что \(L(t) = \alpha_t z_t\).
Мономиальные операторы Роты--Бакстера на \(F[x]\) были введены в~[1] 
и описаны на \(F[x]\) в~[2].

В работе~[3] найдена взаимосвязь между
операторами Роты--Бакстера и операторами усреднения.
В этой работе был описан класс операторов Роты--Бакстера ненулевого веса,
построенных по гомоморфным операторам усреднения на \(F[x,y]\).
Мы классифицировали операторы Роты--Бакстера нулевого веса,
построенные по операторам усреднения 
с~линейными функциями в степенях мономов из образа на \(F[x,y]\).

\medskip

\begin{enumerate}
\item[{[1]}] L. Guo, M. Rosenkranz, and S.H. Zheng. 
\textit{Rota---Baxter operators on the polynomial algebras,~integration~and~averaging operators,~Pacific~J.\,Math.\,(2)~275~(2015),\,481--507.}
\item[{[2]}] H. Yu. 
\textit{Classification of monomial Rota---Baxter operators on \(k[x]\),
J. Algebra Appl. 15 (2016), 1650087.}
\item[{[3]}] A. Khodzitskii, 
\textit{Monomial Rota---Baxter Operators of Nonzero Weight on \(F[x, y]\) Coming from Averaging Operators, 
Mediterr. J. Math. 20 (2023), No~251.}
\end{enumerate}
\end{talk}
\end{document}