\documentclass[12pt, a4paper, figuresright]{book}
\usepackage{mathtools, commath, amssymb, amsthm}
\usepackage{tabularx,graphicx,url,xcolor,rotating,multicol,epsfig,colortbl,lipsum}
\usepackage[T2A]{fontenc}
\usepackage[english,main=russian]{babel}

\setlength{\textheight}{25.2cm}
\setlength{\textwidth}{16.5cm}
\setlength{\voffset}{-1.6cm}
\setlength{\hoffset}{-0.3cm}
\setlength{\evensidemargin}{-0.3cm} 
\setlength{\oddsidemargin}{0.3cm}
\setlength{\parindent}{0cm} 
\setlength{\parskip}{0.3cm}

\newenvironment{talk}[6]{%
\vskip 0pt\nopagebreak%
\vskip 0pt\nopagebreak%
\textbf{#1}\vspace{3mm}\\\nopagebreak%
\textit{#2}\\\nopagebreak%
#3\\\nopagebreak%
\url{#4}\vspace{3mm}\\\nopagebreak%
\ifthenelse{\equal{#5}{}}{}{Соавторы: #5\vspace{3mm}\\\nopagebreak}%
\ifthenelse{\equal{#6}{}}{}{Секция: #6\quad \vspace{3mm}\\\nopagebreak}%
}

\pagestyle{empty}

\begin{document}
\begin{talk}
{Двойственные фреймы всплесков на множестве \(M\)-положительных векторов} %
{Бабушкин Максим Владимирович} %
{СПбГУ, кафедра высшей математики; Университет ИТМО, научно-образовательный центр математики}%
{m.v.babushkin@yandex.ru} %
{Скопина Мария Александровна} %
{Вещественный и функциональный анализ} %

Множество \(M\)-положительных векторов является многомерным аналогом положительной полупрямой. Это пространство порождается матрицей \(M\) и снабжается операцией сложения по модулю \(M\). На основе функций Уолша можно построить гармонический анализ на этом пространстве, аналогичный анализу Уолша на полупрямой. Особенностью этой теории является наличие класса функций с компактным носителем (``тест-функций''), преобразование Фурье которых также имеет компактный носитель. В рамках нашего исследования предложен способ построения двойственных фреймов всплесков, состоящих из тест-функций. Приводится несколько простых конкретных примеров таких систем всплесков.
\end{talk}
\end{document}
