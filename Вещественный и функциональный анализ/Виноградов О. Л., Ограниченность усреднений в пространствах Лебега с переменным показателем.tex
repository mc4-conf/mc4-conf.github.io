\documentclass[12pt, a4paper, figuresright]{book}
\usepackage{mathtools, commath, amssymb, amsthm}
\usepackage{tabularx,graphicx,url,xcolor,rotating,multicol,epsfig,colortbl,lipsum}
\usepackage[T2A]{fontenc}
\usepackage[utf8]{inputenc}
\usepackage[english,main=russian]{babel}

\setlength{\textheight}{25.2cm}
\setlength{\textwidth}{16.5cm}
\setlength{\voffset}{-1.6cm}
\setlength{\hoffset}{-0.3cm}
\setlength{\evensidemargin}{-0.3cm} 
\setlength{\oddsidemargin}{0.3cm}
\setlength{\parindent}{0cm} 
\setlength{\parskip}{0.3cm}

\newenvironment{talk}[6]{%
\vskip 0pt\nopagebreak%
\vskip 0pt\nopagebreak%
\textbf{#1}\vspace{3mm}\\\nopagebreak%
\textit{#2}\\\nopagebreak%
#3\\\nopagebreak%
\url{#4}\vspace{3mm}\\\nopagebreak%
\ifthenelse{\equal{#5}{}}{}{Соавторы: #5\vspace{3mm}\\\nopagebreak}%
\ifthenelse{\equal{#6}{}}{}{Секция: #6\quad \vspace{3mm}\\\nopagebreak}%
}

\pagestyle{empty}

\begin{document}
\begin{talk}
{Ограниченность усреднений в пространствах Лебега с переменным показателем} %
{Виноградов Олег Леонидович} %
{Санкт-Петербургский государственный университет}%
{olvin@math.spbu.ru} %
{} %
{Вещественный и функциональный анализ} %

Если нормированное пространство~\(X\), состоящее из
заданных на~\(\mathbb{R}^n\) функций, вместе с
каждой функцией содержит ее средние Стеклова \(S_hf\) и
\(\sup_{h>0}\|S_h\|<+\infty\), то \(X\)~называется
пространством с ограниченнным усреднением. Ранее автором
были установлены прямые и обратные теоремы
теории приближений тригонометрическими многочленами и целыми функциями
конечной степени в банаховых идеальных пространствах с ограниченным
усреднением. Эти теоремы во многом аналогичны таковым в обычных пространствах~\(L_p\). 
Ограниченности максимального оператора в этих вопросах
не требуется. Особая роль средних Стеклова состоит
в том, что их ограниченность влечет ограниченность сверток с любыми ядрами,
имеющими суммируемую горбатую мажоранту.
Единственный известный критерий ограниченности усреднений в пространствах, не инвариантных относительно сдвига, относится к весовым пространствам Лебега: ограниченность усреднений равносильна условию Макенхаупта.
В других случаях известные достаточные условия не совпадают с
необходимыми. В докладе обсуждаются условия ограниченности усреднений в пространствах
Лебега с переменным показателем.
\end{talk}
\end{document}
