\documentclass[12pt, a4paper, figuresright]{book}
\usepackage{mathtools, commath, amssymb, amsthm}
\usepackage{tabularx,graphicx,url,xcolor,rotating,multicol,epsfig,colortbl,lipsum}
\usepackage[T2A]{fontenc}
\usepackage[english,main=russian]{babel}

\setlength{\textheight}{25.2cm}
\setlength{\textwidth}{16.5cm}
\setlength{\voffset}{-1.6cm}
\setlength{\hoffset}{-0.3cm}
\setlength{\evensidemargin}{-0.3cm}
\setlength{\oddsidemargin}{0.3cm}
\setlength{\parindent}{0cm}
\setlength{\parskip}{0.3cm}

\newenvironment{talk}[6]{%
\vskip 0pt\nopagebreak%
\vskip 0pt\nopagebreak%
\textbf{#1}\vspace{3mm}\\\nopagebreak%
\textit{#2}\\\nopagebreak%
#3\\\nopagebreak%
\url{#4}\vspace{3mm}\\\nopagebreak%
\ifthenelse{\equal{#5}{}}{}{Соавторы: #5\vspace{3mm}\\\nopagebreak}%
\ifthenelse{\equal{#6}{}}{}{Секция: #6\quad \vspace{3mm}\\\nopagebreak}%
}

\pagestyle{empty}

\begin{document}
\begin{talk}
{Оценки жесткости кручения выпуклой области} %
{Гафиятуллина Лилия Ильгизяровна} %
{Казанский (Приволжский) федеральный университет}%
{gafiyat@gmail.com} %
{Салахудинов Рустем Гумерович} %
{Вещественный и функциональный анализ} %

Работа посвящена оценкам жесткости кручения выпуклой области через ее геометрические характеристики. В работе определены две новые характеристики выпуклой области с конечной длиной границы, а также изучены их свойства и приведен алгоритм их вычисления. Получена верхняя оценка жесткости кручения через новые геометрические функционалы области. 

\begin{enumerate}
\item[{[1]}] Г. Полиа, Г. Сегё, {\it Изопериметрические неравенства в математической физике}, М.: Физматгиз, 1962, 336 с.
\end{enumerate}
\end{talk}
\end{document}