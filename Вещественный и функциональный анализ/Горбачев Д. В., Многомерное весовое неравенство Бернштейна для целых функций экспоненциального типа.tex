\documentclass[12pt, a4paper, figuresright]{book}
\usepackage{mathtools, commath, amssymb, amsthm}
\usepackage{tabularx,graphicx,url,xcolor,rotating,multicol,epsfig,colortbl,lipsum}
\usepackage[T2A]{fontenc}
\usepackage[english,main=russian]{babel}

\setlength{\textheight}{25.2cm}
\setlength{\textwidth}{16.5cm}
\setlength{\voffset}{-1.6cm}
\setlength{\hoffset}{-0.3cm}
\setlength{\evensidemargin}{-0.3cm} 
\setlength{\oddsidemargin}{0.3cm}
\setlength{\parindent}{0cm} 
\setlength{\parskip}{0.3cm}

\newenvironment{talk}[6]{%
\vskip 0pt\nopagebreak%
\vskip 0pt\nopagebreak%
\textbf{#1}\vspace{3mm}\\\nopagebreak%
\textit{#2}\\\nopagebreak%
#3\\\nopagebreak%
\url{#4}\vspace{3mm}\\\nopagebreak%
\ifthenelse{\equal{#5}{}}{}{Соавторы: #5\vspace{3mm}\\\nopagebreak}%
\ifthenelse{\equal{#6}{}}{}{Секция: #6\quad \vspace{3mm}\\\nopagebreak}%
}

\pagestyle{empty}

\begin{document}
\begin{talk}
  {Многомерное весовое неравенство Бернштейна для целых функций экспоненциального типа} % [1] название доклада
  {Горбачев Дмитрий Викторович} % [2] имя докладчика
  {ООО ``Горизонт''}% [3] аффилиация
  {dvgmail@mail.ru} % [4] адрес электронной почты (НЕОБЯЗАТЕЛЬНО)
  {} % [5] соавторы (НЕОБЯЗАТЕЛЬНО)
  {Вещественный и функциональный анализ} % [6] название секции

Неравенство Бернштейна для целых функций экспоненциального типа является
классическим в теории приближений. В стандартной постановке оно устанавливает
порядок роста $L^{p}$-нормы линейного дифференциального оператора на классе
полиномов или целых функций экспоненциального типа. Интересен случай весовой
$L^{p}$-нормы при $p>0$, задаваемой мерой, удовлетворяющей условию удвоения. Он
разобран, например, для тригонометрических и сферических полиномов. Однако для
целых функций экспоненциального типа, особенно в многомерной постановке,
известно меньше. Целью доклада будет привести новые результаты в этом
направлении. Особенно интересны весовые неравенства, отвечающие степенным весам
Данкля и $(\kappa,a)$-обобщенного преобразования Фурье, где есть
дифференциально-разностные аналоги классических дифференциальных операторов.
Одно из приложений данных результатов состоит в доказательстве обратных теорем
теории приближений в соответствующих весовых пространствах $L^{p}$.
\end{talk}
\end{document}