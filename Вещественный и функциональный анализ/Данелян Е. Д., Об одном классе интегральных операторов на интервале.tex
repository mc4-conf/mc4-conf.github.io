\documentclass[12pt, a4paper, figuresright]{book}
\usepackage{mathtools, commath, amssymb, amsthm}
\usepackage{tabularx,graphicx,url,xcolor,rotating,multicol,epsfig,colortbl,lipsum}
\usepackage[T2A]{fontenc}
\usepackage[english,main=russian]{babel}

\setlength{\textheight}{25.2cm}
\setlength{\textwidth}{16.5cm}
\setlength{\voffset}{-1.6cm}
\setlength{\hoffset}{-0.3cm}
\setlength{\evensidemargin}{-0.3cm} 
\setlength{\oddsidemargin}{0.3cm}
\setlength{\parindent}{0cm} 
\setlength{\parskip}{0.3cm}

\newenvironment{talk}[6]{%
\vskip 0pt\nopagebreak%
\vskip 0pt\nopagebreak%
\textbf{#1}\vspace{3mm}\\\nopagebreak%
\textit{#2}\\\nopagebreak%
#3\\\nopagebreak%
\url{#4}\vspace{3mm}\\\nopagebreak%
\ifthenelse{\equal{#5}{}}{}{Соавторы: #5\vspace{3mm}\\\nopagebreak}%
\ifthenelse{\equal{#6}{}}{}{Секция: #6\quad \vspace{3mm}\\\nopagebreak}%
}

\pagestyle{empty}

\begin{document}
\begin{talk}
{Об одном классе интегральных операторов на интервале \((-1, 1)\)} %
{Данелян Елена Дмитриевна} %
{Южный федеральный университет}%
{danelian@sfedu.ru} %
{Карапетянц А.\,Н.} %
{Вещественный и функциональный анализ} %

Рассматриваются интегральные операторы типа Хаусдорфа на интервале \((-1,1)\), которые естественным образом возникают в некоторых задачах теории интегральных уравнений и математической физики. А именно, при наличии измеримой функции \(k\) (нашего интегрального ядра) на интервале \((-1,1)\) рассматривается интегральный оператор
\[
K_\mu f(x)=\int_{-1}^{1}k(t)f(\varphi_x(t))d\mu(t),
\]
где \(\mu \) -- произвольная положительная мера Радона на \((-1,1)\) и \(\varphi_x(t)=\frac{x-t}{1-xt},\) \(x, t\in (-1,1)\) инволютивный автоморфизм на \((-1,1)\) на весовых пространствах Лебега.

Рассматриваются алгебраические свойства для частного случая изучаемых операторов. Устанавливаются достаточные, а затем и необходимые условия ограниченности операторов в пространствах \(L^p(v)\), в качестве следствий приводятся некоторые важные частные случаи операторов и некоторых весов. Применяется техника операторов с однородными ядрами с получением принципиально иных условий ограниченности, которые, тем не менее, дают схожие результаты в специальных частных случаях.  Строятся аппроксимационные конструкции в рамках обсуждаемых операторов.
\end{talk}
\end{document}
