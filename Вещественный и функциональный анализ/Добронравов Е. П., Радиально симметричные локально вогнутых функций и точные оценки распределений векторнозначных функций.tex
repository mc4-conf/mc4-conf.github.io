\documentclass[12pt, a4paper, figuresright]{book}
\usepackage{mathtools, commath, amssymb, amsthm}
\usepackage{tabularx,graphicx,url,xcolor,rotating,multicol,epsfig,colortbl,lipsum}
\usepackage[T2A]{fontenc}
\usepackage[english,main=russian]{babel}

\setlength{\textheight}{25.2cm}
\setlength{\textwidth}{16.5cm}
\setlength{\voffset}{-1.6cm}
\setlength{\hoffset}{-0.3cm}
\setlength{\evensidemargin}{-0.3cm} 
\setlength{\oddsidemargin}{0.3cm}
\setlength{\parindent}{0cm} 
\setlength{\parskip}{0.3cm}

\newenvironment{talk}[5]{%
\vskip 0pt\nopagebreak%
\vskip 0pt\nopagebreak%
\textbf{#1}\vspace{3mm}\\\nopagebreak%
\textit{#2}\\\nopagebreak%
#3\\\nopagebreak%
\url{#4}\vspace{3mm}\\\nopagebreak%
\ifthenelse{\equal{#5}{}}{}{Секция: #5\quad \vspace{3mm}\\\nopagebreak}%
}

\pagestyle{empty}

\begin{document}
\begin{talk}
{Радиально симметричные локально вогнутых функций и точные оценки распределений векторнозначных функций} %
{Добронравов Егор Петрович} %
{СПбГУ}%
{yegordobronravov@mail.ru} %
{Вещественный и функциональный анализ} %

Мы построим теорию позволяющую вычилсять радиально симметричные минимальные локально вогнутые функции по их аналогам меньшей размерности, а так же приложения этой теории к построению точных оценок распределений векторнозначных функций. Так же с помощью полученной теории построим несколько точных функций Беллмана и соответственно получим точные неравенства оценивающие распределения функций.
\end{talk}
\end{document}