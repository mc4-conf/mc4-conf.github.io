\documentclass[12pt, a4paper, figuresright]{book}
\usepackage{mathtools, commath, amssymb, amsthm}
\usepackage{tabularx,graphicx,url,xcolor,rotating,multicol,epsfig,colortbl,lipsum}
\usepackage[T2A]{fontenc}
\usepackage[english,main=russian]{babel}

\setlength{\textheight}{25.2cm}
\setlength{\textwidth}{16.5cm}
\setlength{\voffset}{-1.6cm}
\setlength{\hoffset}{-0.3cm}
\setlength{\evensidemargin}{-0.3cm} 
\setlength{\oddsidemargin}{0.3cm}
\setlength{\parindent}{0cm} 
\setlength{\parskip}{0.3cm}


\newenvironment{talk}[5]{%
	\vskip 0pt\nopagebreak%
	\vskip 0pt\nopagebreak%
	\textbf{#1}\vspace{3mm}\\\nopagebreak%
	\textit{#2}\\\nopagebreak%
	#3\\\nopagebreak%
	\url{#4}\vspace{3mm}\\\nopagebreak%
	\ifthenelse{\equal{#5}{}}{}{Секция: #5\quad \vspace{3mm}\\\nopagebreak}%
}

\pagestyle{empty}

\begin{document}
\begin{talk}
{Размерность мер с преобразованием Фурье в \(L_p\)} %
{Добронравов Никита Петрович} %
{СПбГУ}%
{dobronravov1999@mail.ru} %
{Вещественный и функциональный анализ} %

Принцип неопределённости в математическом анализе --- это семейство фактов о том, что  функция и её преобразование Фурье не могут быть одновременно малы. Одной из версий это принципа является следующая теорема. 

{\bf Теорема.}
{\it Пусть \(S\subset \mathbb{R}^d\) --- компакт, такой что \(\mathcal{H}_{\alpha}(S)<\infty\). Пусть обобщённая функция \(\zeta\) такая что \(supp(\zeta)\subset S\) и \(\hat{\zeta}\in L_p(\mathbb{R}^d)\) для некоторого \(p<\frac{2d}{\alpha}\). Тогда \(\zeta=0\).}



Здесь \(\mathcal{H}_{\alpha}\) --- это \(\alpha\)-мера Хаусдорфа.	
Мы разобрали, что происходит в предельном случае \(p=\frac{2d}{\alpha}\). Оказалось, что в этом случае принцип неопределённости неверен, а именно удалось доказать следующую теорему:

{\bf Теорема.}
{\it Существуют компакт \(S\subset\mathbb{R}^d\) и такая вероятностная мера \(\mu\), что \(supp(\mu)\subset S\), \(\hat{\mu}\in L_p(\mathbb{R}^d)\) и \(\mathcal{H}_{\frac{2d}{p}}(S)=0\).}
\end{talk}
\end{document}