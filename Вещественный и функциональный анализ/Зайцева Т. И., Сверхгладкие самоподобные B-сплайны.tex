\documentclass[12pt, a4paper, figuresright]{book}
\usepackage{mathtools, commath, amssymb, amsthm}
\usepackage{tabularx,graphicx,url,xcolor,rotating,multicol,epsfig,colortbl,lipsum}
\usepackage[T2A]{fontenc}
\usepackage[english,main=russian]{babel}

\setlength{\textheight}{25.2cm}
\setlength{\textwidth}{16.5cm}
\setlength{\voffset}{-1.6cm}
\setlength{\hoffset}{-0.3cm}
\setlength{\evensidemargin}{-0.3cm} 
\setlength{\oddsidemargin}{0.3cm}
\setlength{\parindent}{0cm} 
\setlength{\parskip}{0.3cm}

\newenvironment{talk}[6]{%
\vskip 0pt\nopagebreak%
\vskip 0pt\nopagebreak%
\textbf{#1}\vspace{3mm}\\\nopagebreak%
\textit{#2}\\\nopagebreak%
#3\\\nopagebreak%
\url{#4}\vspace{3mm}\\\nopagebreak%
\ifthenelse{\equal{#5}{}}{}{Соавторы: #5\vspace{3mm}\\\nopagebreak}%
\ifthenelse{\equal{#6}{}}{}{Секция: #6\quad \vspace{3mm}\\\nopagebreak}%
}

\pagestyle{empty}

\begin{document}
\begin{talk}
{Сверхгладкие самоподобные B-сплайны} %
{Зайцева Татьяна Ивановна} %
{МГУ имени М.\,В. Ломоносова}%
{zaitsevatanja@gmail.com} %
{} %
{Вещественный и функциональный анализ} %

Тайлы это самоподобные компакты, порождённые одной матрицей растяжения, т. е. матрицей, у которой все собственные значения по модулю больше единицы. 
На основе этих множеств по аналогии с кардинальными B-сплайнами строятся тайловые B-сплайны --- свёртки индикаторов тайлов, при этом многие свойства B-сплайнов сохраняются. В частности, они являются решениями специальных разностных уравнений со сжатием аргумента, что позволяет применять их в прикладных алгоритмах. 
Будет представлен новый способ вычисления их гладкости в \(L_2\). По результатам вычислений обнаружилось несколько семейств ``сверхгладких'' сплайнов, гладкость которых превышает гладкость стандартных сплайнов соответствующих порядков. 

\medskip

\begin{enumerate}
\item[{[1]}] M. Charina, V. Yu. Protasov, {\it Regularity of anisotropic refinable functions}, Appl. Comput. Harmon. Anal., 47 (2019), 795 – 821. 
\item[{[2]}] T. Eirola, {\it Sobolev characterization of solutions of dilation equations}, SIAM J. Math. Anal., 23 (1992), 1015 – 1030. 
\item[{[3]}] A. Cohen, I. Daubechies, {\it A new technique to estimate the regularity of refinable functions}, Revista Mathematica Iberoamericana, 12 (1996), 527 – 591. 
\end{enumerate}
\end{talk}
\end{document}
