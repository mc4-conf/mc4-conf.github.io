\documentclass[12pt, a4paper, figuresright]{book}
\usepackage{mathtools, commath, amssymb, amsthm}
\usepackage{tabularx,graphicx,url,xcolor,rotating,multicol,epsfig,colortbl,lipsum}
\usepackage[T2A]{fontenc}
\usepackage[english,main=russian]{babel}

\setlength{\textheight}{25.2cm}
\setlength{\textwidth}{16.5cm}
\setlength{\voffset}{-1.6cm}
\setlength{\hoffset}{-0.3cm}
\setlength{\evensidemargin}{-0.3cm} 
\setlength{\oddsidemargin}{0.3cm}
\setlength{\parindent}{0cm} 
\setlength{\parskip}{0.3cm}

\newenvironment{talk}[6]{%
	\vskip 0pt\nopagebreak%
	\vskip 0pt\nopagebreak%
	\textbf{#1}\vspace{3mm}\\\nopagebreak%
	\textit{#2}\\\nopagebreak%
	#3\\\nopagebreak%
	\url{#4}\vspace{3mm}\\\nopagebreak%
	\ifthenelse{\equal{#5}{}}{}{Соавторы: #5\vspace{3mm}\\\nopagebreak}%
	\ifthenelse{\equal{#6}{}}{}{Секция: #6\quad \vspace{3mm}\\\nopagebreak}%
}

\pagestyle{empty}

\begin{document}
\begin{talk}
{Пространство почти сходящихся последовательностей и инвариантные банаховы пределы} %
{Зволинский Роман Евгеньевич} %
{Воронежский государственный университет}%
{roman.zvolinskiy@gmail.com} %
{} %
{Вещественный и функциональный анализ} %

Банахов предел --- положительный  линейный ограниченный функционал, инвариантный относительно
оператора сдвига, являющийся продолжением предела последовательности с пространства
сходящихся последовательностей на пространство ограниченных последовательностей с сохранением нормы.
Ограниченная последовательность действительных чисел называется почти сходящейся, если все банаховы
пределы принимают на ней постоянное значение. Приводятся новые результаты о банаховых пределах, в частности, аналог
критерия почти сходимости Г. Лоренца [1, Теорема 2]. Рассматриваются множества банаховых пределов, инвариантных относительно операторов растяжения, а также функциональные характеристики множества банаховых пределов.

\medskip

\begin{enumerate}
\item[{[1]}] G. G. Lorentz, {\it A contribution to the theory of divergent sequences}, Acta mathematica, 80(1), 1948, 167--190.
\end{enumerate}
\end{talk}
\end{document}