\documentclass[12pt, a4paper, figuresright]{book}
\usepackage{mathtools, commath, amssymb, amsthm}
\usepackage{tabularx,graphicx,url,xcolor,rotating,multicol,epsfig,colortbl,lipsum}
\usepackage[T2A]{fontenc}
\usepackage[english,main=russian]{babel}

\setlength{\textheight}{25.2cm}
\setlength{\textwidth}{16.5cm}
\setlength{\voffset}{-1.6cm}
\setlength{\hoffset}{-0.3cm}
\setlength{\evensidemargin}{-0.3cm} 
\setlength{\oddsidemargin}{0.3cm}
\setlength{\parindent}{0cm} 
\setlength{\parskip}{0.3cm}

\newenvironment{talk}[6]{%
\vskip 0pt\nopagebreak%
\vskip 0pt\nopagebreak%
\textbf{#1}\vspace{3mm}\\\nopagebreak%
\textit{#2}\\\nopagebreak%
#3\\\nopagebreak%
\url{#4}\vspace{3mm}\\\nopagebreak%
\ifthenelse{\equal{#5}{}}{}{Соавторы: #5\vspace{3mm}\\\nopagebreak}%
\ifthenelse{\equal{#6}{}}{}{Секция: #6\quad \vspace{3mm}\\\nopagebreak}%
}

\pagestyle{empty}

\begin{document}
\begin{talk}
{Точная оценка приближения оператором Саса--Миракьяна через второй модуль непрерывности} %
{Ихсанов Лев Назарович} %
{Санкт-Петербургский государственный университет}%
{lv.ikhs@gmail.com} %
{вещественный и функциональный анализ} %

Оператором Саса--Миракьяна называется оператор 
\(M_nf(x)=e^{-nx}\sum\limits_{k=0}^\infty\frac{(nx)^k}{k!}f\left(\frac{k}n\right).\)

Этот классический положительный оператор, впервые рассмотренный Сасом в работе 1950 года
(в работе Миракьяна 1941-го рассматриваются соответствующие частичные суммы),
интерпретируется как математическое ожидание случайной величины \(f\left(\frac{\xi}n\right)\),
где \(\xi\) -- случайная величина, имеющая распределение Пуассона с параметром \(nx\).
В силу этой интерпретации, а также совпадения многих аппроксимационных свойств, он часто называется аналогом оператора Бернштейна на полуоси
(оператор Бернштейна ассоциируется с биномиальным распределением).
Применяется в статистических моделях, численном решении интегральных уравнений и других задачах.

Нас будет интересовать оценка приближения \(|M_nf(x)-f(x)|\), где \(x>0\), в терминах конструктивных свойств функции \(f\).
Лучшая  известная ранее оценка была получена Палтаней (1988):
\[|M_nf(x)-f(x)|\,\le\,\frac{5}2\omega_2\left(f,\,\sqrt{\frac{x}n}\right).\]

{\bf Основной результат} нашей работы представляет собой оценка
\begin{equation*}
|M_nf(x)-f(x)|\,\le\,\omega_2\left(f,\,4\max\left\{\frac1n,\,\sqrt{\frac{x}n}\right\}\right).
\end{equation*}

Речь в докладе пойдёт о свойствах оператора Саса--Миракьяна, точности основного результата, а так же о методе, использованном в исследовании.
\end{talk}
\end{document}