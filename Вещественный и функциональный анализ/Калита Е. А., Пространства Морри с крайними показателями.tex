\documentclass[12pt, a4paper, figuresright]{book}
\usepackage{mathtools, commath, amssymb, amsthm}
\usepackage{tabularx,graphicx,url,xcolor,rotating,multicol,epsfig,colortbl,lipsum}
\usepackage[T2A]{fontenc}
\usepackage[english,main=russian]{babel}

\setlength{\textheight}{25.2cm}
\setlength{\textwidth}{16.5cm}
\setlength{\voffset}{-1.6cm}
\setlength{\hoffset}{-0.3cm}
\setlength{\evensidemargin}{-0.3cm} 
\setlength{\oddsidemargin}{0.3cm}
\setlength{\parindent}{0cm} 
\setlength{\parskip}{0.3cm}

\newenvironment{talk}[6]{%
	\vskip 0pt\nopagebreak%
	\vskip 0pt\nopagebreak%
	\textbf{#1}\vspace{3mm}\\\nopagebreak%
	\textit{#2}\\\nopagebreak%
	#3\\\nopagebreak%
	\url{#4}\vspace{3mm}\\\nopagebreak%
	\ifthenelse{\equal{#5}{}}{}{Соавторы: #5\vspace{3mm}\\\nopagebreak}%
	\ifthenelse{\equal{#6}{}}{}{Секция: #6\quad \vspace{3mm}\\\nopagebreak}%
}

\pagestyle{empty}

\begin{document}
\begin{talk}
{Пространства Морри с крайними показателями \(1, \infty\)} %
{Калита Евгений Александрович} %
{Институт прикладной математики и механики}%
{ekalita@mail.ru} %
{}%
{Вещественный и функциональный анализ} %

Рассматриваются пространства Морри \(L_{1, a}\) и дуальные пространства Морри \(L_{\infty, a}\). 
Устанавливается, что сингулярные интегральные операторы и максимальные операторы в этих пространствах сильно ограничены --- в отличие от пространств Лебега \(L_1\), \(L_\infty\). 
Также получены некоторые теоремы двойственности для этих пространств.
\end{talk}
\end{document}