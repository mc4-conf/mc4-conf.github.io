\documentclass[12pt, a4paper, figuresright]{book}
\usepackage{mathtools, commath, amssymb, amsthm}
\usepackage{tabularx,graphicx,url,xcolor,rotating,multicol,epsfig,colortbl,lipsum}
\usepackage[T2A]{fontenc}
\usepackage[english,main=russian]{babel}

\setlength{\textheight}{25.2cm}
\setlength{\textwidth}{16.5cm}
\setlength{\voffset}{-1.6cm}
\setlength{\hoffset}{-0.3cm}
\setlength{\evensidemargin}{-0.3cm} 
\setlength{\oddsidemargin}{0.3cm}
\setlength{\parindent}{0cm} 
\setlength{\parskip}{0.3cm}

\newenvironment{talk}[6]{%
\vskip 0pt\nopagebreak%
\vskip 0pt\nopagebreak%
\textbf{#1}\vspace{3mm}\\\nopagebreak%
\textit{#2}\\\nopagebreak%
#3\\\nopagebreak%
\url{#4}\vspace{3mm}\\\nopagebreak%
\ifthenelse{\equal{#5}{}}{}{Соавторы: #5\vspace{3mm}\\\nopagebreak}%
\ifthenelse{\equal{#6}{}}{}{Секция: #6\quad \vspace{3mm}\\\nopagebreak}%
}

\pagestyle{empty}

\begin{document}
\begin{talk}
  {Аппроксимация фреймоподобными мульти-всплесками} % [1] название доклада
  {Кривошеин Александр Владимирович} % [2] имя докладчика
  {Санкт-Петербургский государственный университет}% [3] аффилиация
  {krivosheinav@gmail.com} % [4] адрес электронной почты (НЕОБЯЗАТЕЛЬНО)
  {} % [5] соавторы (НЕОБЯЗАТЕЛЬНО)
  {Вещественный и функциональный анализ} % [6] название секции

Квазипроекционный оператор, порождённый парой вектор-функций $\Phi$, $\widetilde\Phi: {\mathbb R}^d \to {\mathbb C}^r$, имеет вид
$$
Q_j(\Phi, \widetilde\Phi, f) =\sum_{k\in{\mathbb Z}^d} \langle f, \widetilde\Phi_{jk} \rangle \Phi_{jk},
%=\sum_{\nu = 1}^r\sum_{k\in\zd} \langle f, \w\phi_{\nu,j,k} \rangle \phi_{\nu,j,k}
$$
где $\Phi_{jk} = |\det M|^{j/2} \Phi(M^j \cdot + k)$, $j\in{\mathbb Z}$, $k\in {\mathbb Z}^d$, $M$ -- матрица растяжения.
Изучены аппроксимационные свойства таких операторов и получены оценки погрешности в $L_2$-норме для широкого класса таких операторов.

Для масштабирующих вектор-функций $\Phi$, $\widetilde\Phi$ квазипроекционные операторы $Q_j(f,\Phi,\widetilde\Phi)$
связаны с двойственными системами мульти-всплесков. Хотя общая схема построения двойственных фреймов мульти-всплесков в многомерном случае известна, ее реализация на практике является сложной задачей из-за необходимости обеспечения некоторых дополнительных свойств. Предложена конструкция фреймоподобных мульти-всплесков с отказом от фреймовости,  но с сохранением возможности разложения функций аналогичного разложению по фреймам. Это упрощает задачу построения фреймоподобных мульти-всплесков. Установлены аппроксимационные свойства фреймоподобных мульти-всплесков. Предложены алгоритмы построения фреймоподобных мульти-всплесков с заданным порядком аппроксимации.

\medskip
\end{talk}
\end{document}