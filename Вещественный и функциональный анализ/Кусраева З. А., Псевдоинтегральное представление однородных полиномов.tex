\documentclass[12pt, a4paper, figuresright]{book}
\usepackage{mathtools, commath, amssymb, amsthm}
\usepackage{tabularx,graphicx,url,xcolor,rotating,multicol,epsfig,colortbl,lipsum}
\usepackage[T2A]{fontenc}
\usepackage[english,main=russian]{babel}

\setlength{\textheight}{25.2cm}
\setlength{\textwidth}{16.5cm}
\setlength{\voffset}{-1.6cm}
\setlength{\hoffset}{-0.3cm}
\setlength{\evensidemargin}{-0.3cm}
\setlength{\oddsidemargin}{0.3cm}
\setlength{\parindent}{0cm}
\setlength{\parskip}{0.3cm}

\newenvironment{talk}[6]{%
\vskip 0pt\nopagebreak%
\vskip 0pt\nopagebreak%
\textbf{#1}\vspace{3mm}\\\nopagebreak%
\textit{#2}\\\nopagebreak%
#3\\\nopagebreak%
\url{#4}\vspace{3mm}\\\nopagebreak%
\ifthenelse{\equal{#5}{}}{}{Соавторы: #5\vspace{3mm}\\\nopagebreak}%
\ifthenelse{\equal{#6}{}}{}{Секция: #6\quad \vspace{3mm}\\\nopagebreak}%
}

\pagestyle{empty}

\begin{document}
\begin{talk}
{Псевдоинтегральное представление однородных полиномов} %
{Кусраева Залина Анатольевна} %
{Региональный научно-образовательный математический центр Южного федерального университета}%
{zali13@mail.ru} %
{} %
{Вещественный и функциональный анализ} %

В настоящем докладе вводятся классы полилинейных операторов и однородных полиномов в пространствах измеримых функций, допускающих псевдоинтегральное представление. Для таких нелинейных операторов устанавливается теорема типа Радона--Никодима. Этот факт обобщает аналогичный результат, доказанный ранее для линейных операторов в работе [1].

\medskip

\begin{enumerate}
\item[{[1]}] Grobler, J.~J., de Pagter, B., Rambane, D.T. {\it Lattice properties of operators defined by random measures},  Quaestiones Math. (2003). V. 26, N 3. P. 307--319.
\end{enumerate}
\end{talk}
\end{document}