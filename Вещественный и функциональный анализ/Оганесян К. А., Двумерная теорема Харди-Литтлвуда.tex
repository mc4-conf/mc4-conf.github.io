\documentclass[12pt, a4paper, figuresright]{book}
\usepackage{mathtools, commath, amssymb, amsthm}
\usepackage{tabularx,graphicx,url,xcolor,rotating,multicol,epsfig,colortbl,lipsum}
\usepackage[T2A]{fontenc}
\usepackage[english,main=russian]{babel}

\setlength{\textheight}{25.2cm}
\setlength{\textwidth}{16.5cm}
\setlength{\voffset}{-1.6cm}
\setlength{\hoffset}{-0.3cm}
\setlength{\evensidemargin}{-0.3cm} 
\setlength{\oddsidemargin}{0.3cm}
\setlength{\parindent}{0cm} 
\setlength{\parskip}{0.3cm}

\newenvironment{talk}[6]{%
\vskip 0pt\nopagebreak%
\vskip 0pt\nopagebreak%
\textbf{#1}\vspace{3mm}\\\nopagebreak%
\textit{#2}\\\nopagebreak%
#3\\\nopagebreak%
\url{#4}\vspace{3mm}\\\nopagebreak%
\ifthenelse{\equal{#5}{}}{}{Соавторы: #5\vspace{3mm}\\\nopagebreak}%
\ifthenelse{\equal{#6}{}}{}{Секция: #6\quad \vspace{3mm}\\\nopagebreak}%
}

\pagestyle{empty}

\begin{document}
\begin{talk}
{Двумерная теорема Харди--Литтлвуда} %
{Оганесян Кристина Артаковна} %
{Московский центр фундаментальной и прикладной математики (отделение МГУ)}%
{oganchris@gmail.com} %
{}
{Вещественный и функциональный анализ} %

Мы докажем теорему Харди--Литтлвуда в двумерном случае для функций с обобщённо монотонными коэффициентами Фурье произвольного знака. Кроме того, мы приведём контрпример, устанавливающий точность результата в том смысле, что если естественным образом расширить рассматриваемый класс последовательностей, то соотношения Харди--Литтлвуда перестанут выполняться.
\end{talk}
\end{document}