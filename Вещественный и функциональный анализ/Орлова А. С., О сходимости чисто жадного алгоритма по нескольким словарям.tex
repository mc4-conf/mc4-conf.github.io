\documentclass[12pt, a4paper, figuresright]{book}
\usepackage{mathtools, commath, amssymb, amsthm}
\usepackage{tabularx,graphicx,url,xcolor,rotating,multicol,epsfig,colortbl,lipsum}
\usepackage[T2A]{fontenc}
\usepackage[english,main=russian]{babel}

\setlength{\textheight}{25.2cm}
\setlength{\textwidth}{16.5cm}
\setlength{\voffset}{-1.6cm}
\setlength{\hoffset}{-0.3cm}
\setlength{\evensidemargin}{-0.3cm} 
\setlength{\oddsidemargin}{0.3cm}
\setlength{\parindent}{0cm} 
\setlength{\parskip}{0.3cm}

\newenvironment{talk}[6]{%
\vskip 0pt\nopagebreak%
\vskip 0pt\nopagebreak%
\textbf{#1}\vspace{3mm}\\\nopagebreak%
\textit{#2}\\\nopagebreak%
#3\\\nopagebreak%
\url{#4}\vspace{3mm}\\\nopagebreak%
\ifthenelse{\equal{#5}{}}{}{Соавторы: #5\vspace{3mm}\\\nopagebreak}%
\ifthenelse{\equal{#6}{}}{}{Секция: #6\quad \vspace{3mm}\\\nopagebreak}%
}

\pagestyle{empty}

\begin{document}
\begin{talk}
{О сходимости чисто жадного алгоритма по нескольким словарям}%
{Орлова Анастасия Сергеевна} %
{МГУ имени М.\,В. Ломоносова, Московский центр фундаментальной и прикладной математики}%
{Anastasia-Orlova1@ya.ru} %
{} %
{Вещественный и функциональный анализ} %

Обобщением  чисто жадного алгоритма является чисто жадный алгоритм по нескольким словарям, 
на каждом шаге которого выбирается следующий приближающий вектор 
локально оптимальным образом из соответствующего словаря.
Порядок словарей задаётся специальной последовательностью.
Для классического случая чисто жадного алгоритма по одному словарю известно [1], что алгоритм сходится к приближаемому вектору.
В работе [2] доказана слабая сходимость нового алгоритма,
но вопрос сильной сходимости открыт.
Для почти периодических последовательностей показано, что имеет место сильная сходимость чисто жадных приближений по нескольким словарям.

\medskip

\begin{enumerate}
\item[{[1]}] L. K. Jones, {\it A simple lemma on greedy approximation in Hilbert space and convergence rates for projection pursuit regression and neural network training}, Ann. Statist, 20 (1992), 608–613.
\item[{[2]}] П. А. Бородин, Е. Копецка, {\it Слабые пределы последовательных проекций и жадных шагов}, Труды МИАН, 319 (2022),  64–72.
\end{enumerate}
\end{talk}
\end{document}