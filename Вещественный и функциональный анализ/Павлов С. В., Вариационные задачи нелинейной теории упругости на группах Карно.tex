\documentclass[12pt, a4paper, figuresright]{book}
\usepackage{mathtools, commath, amssymb, amsthm}
\usepackage{tabularx,graphicx,url,xcolor,rotating,multicol,epsfig,colortbl,lipsum}
\usepackage[T2A]{fontenc}
\usepackage[english,main=russian]{babel}

\setlength{\textheight}{25.2cm}
\setlength{\textwidth}{16.5cm}
\setlength{\voffset}{-1.6cm}
\setlength{\hoffset}{-0.3cm}
\setlength{\evensidemargin}{-0.3cm} 
\setlength{\oddsidemargin}{0.3cm}
\setlength{\parindent}{0cm} 
\setlength{\parskip}{0.3cm}

\newenvironment{talk}[6]{%
\vskip 0pt\nopagebreak%
\vskip 0pt\nopagebreak%
\textbf{#1}\vspace{3mm}\\\nopagebreak%
\textit{#2}\\\nopagebreak%
#3\\\nopagebreak%
\url{#4}\vspace{3mm}\\\nopagebreak%
\ifthenelse{\equal{#5}{}}{}{Соавторы: #5\vspace{3mm}\\\nopagebreak}%
\ifthenelse{\equal{#6}{}}{}{Секция: #6\quad \vspace{3mm}\\\nopagebreak}%
}

\pagestyle{empty}

\begin{document}
\begin{talk}
{Вариационные задачи нелинейной теории упругости на группах Карно} %
{Павлов Степан Валерьевич} %
{Новосибирский Государственный Университет}%
{s.pavlov4254@gmail.com} %
{Водопьянов Сергей Константинович} %
{Вещественный и функциональный анализ} %

Один из подходов к поиску положения, занимаемого гиперупругим телом \(\Omega\) в результате воздействия на него известных внешних сил, состоит в нахождении отображения \(\varphi:\Omega\to \mathbb{R}^n\), доставляющего минимум функционала энергии 
\[I(\varphi)=\int\limits_\Omega W(x,D\varphi(x))\, dx.\]
В прошлом веке Дж. Боллом были найдены соответствующие реальным материалам математические условия, при которых удается получить теорему о существовании минимума функционала \(I\) в некотором классе непрерывных отображений с обобщенными производными. 

В работе [1] представлено приложение методов современного квазиконформного анализа к данной задаче~--- с их помощью в классе отображений с интегрируемым искажением установлено существование экстремального отображения, являющего взаимно однозначным. В настоящей работе этот подход развивается на группах Карно, обладающих существенно более сложной геометрией по сравнению с евклидовым пространством. Более подробные историческая справка и литература могут быть найдены в [1].

\medskip

\begin{enumerate}
\item[{[1]}] Molchanova A., Vodopyanov S., {\it Injectivity almost everywhere and mappings with finite distortion in nonlinear elasticity}, Calc. Var., 59, №17 (2019).
\item[{[2]}] Водопьянов С.\,К., Павлов С.\,В., {\it Функциональные свойства пределов соболевских гомеоморфизмов с интегрируемым искажением}, Современная математика. Фундаментальные направления, 2024, Том 7, №3 (в печати). 
\end{enumerate}
\end{talk}
\end{document}
