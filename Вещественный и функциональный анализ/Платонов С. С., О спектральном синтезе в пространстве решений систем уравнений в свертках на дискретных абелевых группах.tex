\documentclass[12pt, a4paper, figuresright]{book}
%\usepackage{mathtools, commath, amssymb, amsthm}
\usepackage{tabularx,graphicx,url,xcolor,rotating,multicol,epsfig,colortbl,lipsum}
\usepackage[T2A]{fontenc}
\usepackage[english,main=russian]{babel}

\setlength{\textheight}{25.2cm}
\setlength{\textwidth}{16.5cm}
\setlength{\voffset}{-1.6cm}
\setlength{\hoffset}{-0.3cm}
\setlength{\evensidemargin}{-0.3cm} 
\setlength{\oddsidemargin}{0.3cm}
\setlength{\parindent}{0cm} 
\setlength{\parskip}{0.3cm}

\newenvironment{talk}[6]{%
\vskip 0pt\nopagebreak%
\vskip 0pt\nopagebreak%
\textbf{#1}\vspace{3mm}\\\nopagebreak%
\textit{#2}\\\nopagebreak%
#3\\\nopagebreak%
\url{#4}\vspace{3mm}\\\nopagebreak%
\ifthenelse{\equal{#5}{}}{}{Соавторы: #5\vspace{3mm}\\\nopagebreak}%
\ifthenelse{\equal{#6}{}}{}{Секция: #6\quad \vspace{3mm}\\\nopagebreak}%
}

\pagestyle{empty}

\begin{document}
\begin{talk}
{О спектральном синтезе в пространстве решений систем уравнений в свертках на дискретных абелевых группах} %
{Платонов Сергей Сергеевич} %
{Петрозаводский государственный университет}%
{ssplatonov@yandex.ru} %
{} %
{Вещественный и функциональный анализ} %

Рассматриваются задачи о спектральном синтезе в топологическом векторном пространстве \(\mathcal{M}(G)\) функций  медленного роста на дискретной абелевой группе \(G\). Доказывается, что в пространстве \(\mathcal{M}(G)\) линейные подпространства, состоящие из решений систем уравнений в свертках, допускают спектральный синтез, т. е., что подпространство всех решений системы уравнений в свертках совпадает с замыканием  в \(\mathcal{M}(G)\) линейной оболочки  всех экспоненциальных мономиальных решений этой системы.

\medskip

\begin{enumerate}
\item[{[1]}] L. Sz\'ekelyhidi, {\it On the principal ideal theorem and spectral synthesis on discrete Abelian groups}, Acta Math. Hung., 150:1 (2016), 228--233.
\item[{[2]}] S. S. Platonov, {\it On spectral analysis and spectral synthesis in the
space of tempered functions on discrete abelian groups},   J. Fourier Anal. Appl.  24 (2018), 1340–1376.
\end{enumerate}
\end{talk}
\end{document}
