\documentclass[12pt, a4paper, figuresright]{book}
\usepackage{mathtools, commath, amssymb, amsthm}
\usepackage{tabularx,graphicx,url,xcolor,rotating,multicol,epsfig,colortbl,lipsum}
\usepackage[T2A]{fontenc}
\usepackage[english,main=russian]{babel}

\setlength{\textheight}{25.2cm}
\setlength{\textwidth}{16.5cm}
\setlength{\voffset}{-1.6cm}
\setlength{\hoffset}{-0.3cm}
\setlength{\evensidemargin}{-0.3cm} 
\setlength{\oddsidemargin}{0.3cm}
\setlength{\parindent}{0cm} 
\setlength{\parskip}{0.3cm}

\newenvironment{talk}[6]{%
\vskip 0pt\nopagebreak%
\vskip 0pt\nopagebreak%
\textbf{#1}\vspace{3mm}\\\nopagebreak%
\textit{#2}\\\nopagebreak%
#3\\\nopagebreak%
\url{#4}\vspace{3mm}\\\nopagebreak%
\ifthenelse{\equal{#5}{}}{}{Соавторы: #5\vspace{3mm}\\\nopagebreak}%
\ifthenelse{\equal{#6}{}}{}{Секция: #6\quad \vspace{3mm}\\\nopagebreak}%
}

\pagestyle{empty}

\begin{document}
\begin{talk}
{Вещественная интерполяция пространств типа Харди} %
{Руцкий Дмитрий Владимирович} %
{Санкт-Петербургское отделение Математического институтаим. В.\,А.Стеклова РАН}%
{rutsky@pdmi.ras.ru} %
{} %
{Вещественный и функциональный анализ} %

Пространства типа Харди для квазинормированных решёток измеримых функций~\(X\) на единичной окружности~\(\mathbb T\) задаются как~\(X_A = X \cap N^+\), где \(N^+\) --- граничный класс Смирнова.  В частности, для пространств Лебега получаются обычные пространства Харди~\((L_p)_A = H_p\).

В докладе будет рассмотрен вопрос о характеризации K-замкнутости и устойчивости вещественной интерполяции пар \((X_A, Y_A)\) таких пространств в достаточно общей ситуации.
Первое свойство означает, что произвольные измеримые разложения в~\(X + Y\) функций из~\(X_A + Y_A\) могут быть исправлены до аналитических с оценками норм слагаемых через исходное разложение.
Второе свойство ---
это формула~\((X_A, Y_A)_{\theta, q} = \left[(X, Y)_{\theta, q}\right]_A\).

Оказывается, оба эти свойства можно охарактеризовать в терминах свойства ограниченной BMO-регулярности: для всяких функций \(f \in X\), \(g \in Y\) существуют некоторые функции \(u \in X\) и~\(v \in Y\) с оценками норм через~\(f\) и~\(g\), такие, что \(u + v \geqslant |f| + |g|\) и \(\log u/v\) лежит в BMO с подходящей оценкой нормы.  Это свойство обобщает известное свойство BMO-регулярности, где вместо условия мажорирования в совокупности имеется индивидуальное мажорирование: \(u \geqslant |f|\) и~\(v \geqslant |g|\).  Такую характеризацию удаётся установить для общей ситуации, где от решёток требуется лишь свойство Фату, хотя технически она получается довольно сложной и громоздкой.  С другой стороны, в важном частном случае~\(Y = L_\infty\) этот результат можно получить довольно элементарными средствами.

\medskip

\begin{enumerate}
\item [{[1]}] S.~V. Kisliakov, {\it Interpolation of {\(H^p\)}-spaces: some recent
developments}, Function spaces, interpolation spaces, and related topics
({H}aifa, 1995), Israel Math. Conf. Proc., vol.~13, Bar-Ilan Univ., Ramat
Gan, 1999, pp.~102--140.
\item [{[2]}] D.~V. Rutsky, {\it Real Interpolation of Hardy-type Spaces and BMO-regularity}, Journal Fourier Anal. Appl, 26 (2020), no.~4, 1--40.
\item [{[3]}] Д.~В.~Руцкий, {\it Вещественная интерполяция пространств типа Харди: анонс и некоторые замечания}, Зап. научн. сем. ПОМИ, 480 (2019), 170--190.
\end{enumerate}
\end{talk}
\end{document}
