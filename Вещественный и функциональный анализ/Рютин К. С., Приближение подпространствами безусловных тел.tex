\documentclass[12pt, a4paper, figuresright]{book}
\usepackage{mathtools, commath, amssymb, amsthm}
\usepackage{tabularx,graphicx,url,xcolor,rotating,multicol,epsfig,colortbl,lipsum}
\usepackage[T2A]{fontenc}
\usepackage[english,main=russian]{babel}

\setlength{\textheight}{25.2cm}
\setlength{\textwidth}{16.5cm}
\setlength{\voffset}{-1.6cm}
\setlength{\hoffset}{-0.3cm}
\setlength{\evensidemargin}{-0.3cm} 
\setlength{\oddsidemargin}{0.3cm}
\setlength{\parindent}{0cm} 
\setlength{\parskip}{0.3cm}

\newenvironment{talk}[6]{%
\vskip 0pt\nopagebreak%
\vskip 0pt\nopagebreak%
\textbf{#1}\vspace{3mm}\\\nopagebreak%
\textit{#2}\\\nopagebreak%
#3\\\nopagebreak%
\url{#4}\vspace{3mm}\\\nopagebreak%
\ifthenelse{\equal{#5}{}}{}{Соавторы: #5\vspace{3mm}\\\nopagebreak}%
\ifthenelse{\equal{#6}{}}{}{Секция: #6\quad \vspace{3mm}\\\nopagebreak}%
}

\pagestyle{empty}

\begin{document}
\begin{talk}
{Приближение подпространствами безусловных тел} %
{Рютин Константин Сергеевич} %
{Московский Центр фундаментальной и прикладной математики, механико-математический факультет МГУ}%
{kriutin@yahoo.com} %
{Ю.\,В. Малыхин} %
{Вещественный и функциональный анализ} %

Будут представлены результаты, продолжающие исследования начатые Ю.\,В.~Малыхиным, по жесткости (несжимаемости) случайных векторов и выпуклых множеств.  
Авторами было доказано, что любое безусловное множество в  \(\mathbb{R}^N\) инвариантное относительно циклических перестановок координат является жестким в метрике  \(\ell_q^N\), \(1\le q\le 2\), т. е. не может быть хорошо приближено линейными подпространствами размерности существенно меньшей \(N\). 
Стартовой точкой была задача о поперечнике по Колмогорову множества, полученного из заданного вектора \(x\in \mathbb{R}^N\) всевозможными перестановками координат и  расстановками знаков 

Доказательство развивает подход, предложенный  Е.\,Д. Глускиным для  оценки поперечника множества \(B_\infty^N \cap kB_1^N\) --- пересечения куба и октаэдра, на общий случай усредненных поперечников по Колмогорову безусловных случайных векторов или же случайных векторов с независимыми, нулевыми в среднем компонентами.   Доказана  общая  оценка снизу для усредненного поперечника по Колмогорову через слабые моменты биортогонального случайного вектора. 

Получены следствия для безусловных множеств, инвариантных относительно транзитивного действия некоторой группы перестановок координат. Установлены нижние оценки поперечников шаров в смешанных нормах \(B_{q_1,q_2}^{s,b}\) и некоторых множеств матриц.
Поперечники шаров в смешанных нормах активно изучаются различными авторами, начиная с 80х годов. Поперечники же так называемых транзитивных множеств (орбит конечной подгруппы ортогональной группы) исследовались в недавних работах B.~ Green, A. Sah, M. Sawhney, Y. Zhao. 
\end{talk}
\end{document}