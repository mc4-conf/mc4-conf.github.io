\documentclass[12pt, a4paper, figuresright]{book}
\usepackage{mathtools, commath, amssymb, amsthm}
\usepackage{tabularx,graphicx,url,xcolor,rotating,multicol,epsfig,colortbl,lipsum}
\usepackage[T2A]{fontenc}
\usepackage[english,main=russian]{babel}

\setlength{\textheight}{25.2cm}
\setlength{\textwidth}{16.5cm}
\setlength{\voffset}{-1.6cm}
\setlength{\hoffset}{-0.3cm}
\setlength{\evensidemargin}{-0.3cm} 
\setlength{\oddsidemargin}{0.3cm}
\setlength{\parindent}{0cm} 
\setlength{\parskip}{0.3cm}

\newenvironment{talk}[6]{%
\vskip 0pt\nopagebreak%
\vskip 0pt\nopagebreak%
\textbf{#1}\vspace{3mm}\\\nopagebreak%
\textit{#2}\\\nopagebreak%
#3\\\nopagebreak%
\url{#4}\vspace{3mm}\\\nopagebreak%
\ifthenelse{\equal{#5}{}}{}{Соавторы: #5\vspace{3mm}\\\nopagebreak}%
\ifthenelse{\equal{#6}{}}{}{Секция: #6\quad \vspace{3mm}\\\nopagebreak}%
}

\pagestyle{empty}

\begin{document}
\begin{talk}
{Плотность сумм сдвигов одной функции для действия компактной группы} %
{Скворцов Юрий Александрович} %
{Московский государственный университет им. М.\,В. Ломоносова}%
{iura.skvortsov2@gmail.com} %
{} %
{Вещественный и функциональный анализ} %

Пусть связная компактная топологическая группа \(G\) со счетной базой непрерывно и транзитивно действует на хаусдорфовом топологическом пространстве \(X\). Доказывается существование такой функции \(f\) на \(X\), для которой суммы \(\sum_{k=1}^{n}f(g_k x)\) плотны в пространствах с нулевым средним \(C^0(X)\) и \(L_p^0(X), \ 1 \leq p < \infty\).
\end{talk}
\end{document}