\documentclass[12pt, a4paper, figuresright]{book}
\usepackage{mathtools, commath, amssymb, amsthm}
\usepackage{tabularx,graphicx,url,xcolor,rotating,multicol,epsfig,colortbl,lipsum}
\usepackage[T2A]{fontenc}
\usepackage[english,main=russian]{babel}

\setlength{\textheight}{25.2cm}
\setlength{\textwidth}{16.5cm}
\setlength{\voffset}{-1.6cm}
\setlength{\hoffset}{-0.3cm}
\setlength{\evensidemargin}{-0.3cm} 
\setlength{\oddsidemargin}{0.3cm}
\setlength{\parindent}{0cm} 
\setlength{\parskip}{0.3cm}

\newenvironment{talk}[6]{%
	\vskip 0pt\nopagebreak%
	\vskip 0pt\nopagebreak%
	\textbf{#1}\vspace{3mm}\\\nopagebreak%
	\textit{#2}\\\nopagebreak%
	#3\\\nopagebreak%
	\url{#4}\vspace{3mm}\\\nopagebreak%
	\ifthenelse{\equal{#5}{}}{}{Соавторы: #5\vspace{3mm}\\\nopagebreak}%
	\ifthenelse{\equal{#6}{}}{}{Секция: #6\quad \vspace{3mm}\\\nopagebreak}%
}

\pagestyle{empty}
\begin{document}
\begin{talk}
{Уточнение оценки Меньшова \(L_2\)-нормы мажоранты частных сумм} %
{Солодов Алексей Петрович} %
{Московский государственный университет имени М.\,В. Ломоносова, Московский центр фундаментальной и прикладной математики}%
{apsolodov@mail.ru} %
{} %
{Вещественный и функциональный анализ} %

Теорема Меньшова--Радемахера о точном множителе Вейля для сходимости почти всюду по любой ортонормированной системе основана на оценке нормы максимального оператора. Меньшов показал, что \(L_2\)-норма мажоранты частных сумм с коэффициентами  \(c_n\) по любой ортонормированной системе, состоящей из \(N\) функций, не превосходит \(L_2\)-нормы самой частной суммы с теми же коэффициентами, умноженной на величину \(\log_2 N + 1\). Впоследствии Беннетт показал, что \(L_1\)-норма мажоранты частных сумм с коэффициентами, равными единице,  по любой ортонормированной системе, состоящей из \(N\) функций, не превосходит \(L_\infty\)-нормы самой частной суммы с некоторыми коэффициентами, по модулю не превосходящими единицы, умноженной на величину \(\pi^{-1}\ln N+o(1)\), причем постоянная \(\pi^{-1}\)~--- точная. Этот результат  Беннетт использовал  для усиления неравенства  Меньшова, получив асипмтотически  точную оценку.

Предложен метод усиления доказательства Меньшова, позволяющий уточнить оценку \(L_2\)-нормы мажоранты частных сумм непосредственно. В частности, показано, что  множитель \(\log_2 N + 1\) в оценке Меньшова можно заменить на \(0.5 \log_2 N + 1\). Развитие этого метода позволит, на наш взгляд, лучше изучить структуру ортонормированных систем с экстремально большой нормой мажоранты частных сумм.
\end{talk}
\end{document}