\documentclass[12pt, a4paper, figuresright]{book}
\usepackage{mathtools, commath, amssymb, amsthm}
\usepackage{tabularx,graphicx,url,xcolor,rotating,multicol,epsfig,colortbl,lipsum}
\usepackage[T2A]{fontenc}
\usepackage[english,main=russian]{babel}

\setlength{\textheight}{25.2cm}
\setlength{\textwidth}{16.5cm}
\setlength{\voffset}{-1.6cm}
\setlength{\hoffset}{-0.3cm}
\setlength{\evensidemargin}{-0.3cm} 
\setlength{\oddsidemargin}{0.3cm}
\setlength{\parindent}{0cm} 
\setlength{\parskip}{0.3cm}

\newenvironment{talk}[6]{%
\vskip 0pt\nopagebreak%
\vskip 0pt\nopagebreak%
\textbf{#1}\vspace{3mm}\\\nopagebreak%
\textit{#2}\\\nopagebreak%
#3\\\nopagebreak%
\url{#4}\vspace{3mm}\\\nopagebreak%
%\ifthenelse{\equal{#5}{}}{}{Соавторы: #5\vspace{3mm}\\\nopagebreak}%
\ifthenelse{\equal{#6}{}}{}{Секция: #6\quad \vspace{3mm}\\\nopagebreak}%
}

\pagestyle{empty}

\begin{document}
\begin{talk}
{Ассоциированная рефлексивность некоторых функциональных классов} %
{Степанов Владимир Дмитриевич} 
{ВЦ ДВО РАН, МИАН}%
{stepanov@mi-ras.ru} %
{} %
{Вещественный и функциональный анализ} %

В докладе рассматривается задача об описании ассоциированных и дважды 
ассоциированных пространств к функциональным классам, включающим как идеальные, так и 
неидеальные структуры. Последние включают в себя двухвесовые пространства Соболева первого 
порядка на положительной полуоси [1]. Показано, что, в отличие от понятия двойственности, 
ассоциированность может быть "сильной" и "слабой". В то же время дважды ассоциированные 
пространства делятся еще на три типа. В этом контексте установлено, что пространство функций 
Соболева с компактным носителем обладает слабо ассоциированной рефлексивностью, а сильно 
ассоциированное к слабо ассоциированному пространству состоит только из нуля [2]. 
Аналогичными свойствами обладают весовые пространства типа Чезаро и Копсона, для которых 
проблема полностью изучена и установлена их связь с пространствами Соболева со степенными 
весами~[3]. В качестве приложения рассматривается проблема ограниченности преобразования 
Гильберта из весового пространства Соболева в весовое пространство Лебега~[4].

\medskip

Работа поддержана Российским Научным Фондом (https://rscf.ru/project/24-11-00170/, Project 19-11-00087).

\begin{enumerate}
\item[{[1]}] Д.~В. Прохоров, В. Д. Степанов, Е. П. Ушакова, 
{\it Характеризация функциональных пространств, ассоциированных с весовыми пространствами 
Соболева первого порядка на действительной оси}, Успехи матем. наук, 74:6 (2019), 119--158.
\item[{[2]}] В. Д. Степанов, Е. П. Ушакова, 
{\it О сильной и слабой ассоциированности весовых пространств Соболева первого порядка}, 
Успехи матем. наук, 78:1 (2023), 167--204.
\item[{[3]}] V. D. Stepanov, 
{\it On Ces\`{a}ro and Copson type function spaces. Reflexivity}, 
J. Math. Anal. Appl., 507:1 (2022), Paper No. 125764, 18 pp.
\item[{[4]}] V. D. Stepanov, 
{\it On the boundedness of the Hilbert transform from weighted Sobolev space to 
weighted Lebesgue space}, 
J. Fourier Anal. Appl., {28} (2022), Paper No. 46, 17 pp.
\end{enumerate}
\end{talk}
\end{document}