\documentclass[12pt, a4paper, figuresright]{book}
\usepackage{mathtools, commath, amssymb, amsthm}
\usepackage{tabularx,graphicx,url,xcolor,rotating,multicol,epsfig,colortbl,lipsum}
\usepackage[T2A]{fontenc}
\usepackage[english,main=russian]{babel}

\setlength{\textheight}{25.2cm}
\setlength{\textwidth}{16.5cm}
\setlength{\voffset}{-1.6cm}
\setlength{\hoffset}{-0.3cm}
\setlength{\evensidemargin}{-0.3cm} 
\setlength{\oddsidemargin}{0.3cm}
\setlength{\parindent}{0cm} 
\setlength{\parskip}{0.3cm}

\newenvironment{talk}[6]{%
\vskip 0pt\nopagebreak%
\vskip 0pt\nopagebreak%
\textbf{#1}\vspace{3mm}\\\nopagebreak%
\textit{#2}\\\nopagebreak%
#3\\\nopagebreak%
\url{#4}\vspace{3mm}\\\nopagebreak%
\ifthenelse{\equal{#5}{}}{}{Соавторы: #5\vspace{3mm}\\\nopagebreak}%
\ifthenelse{\equal{#6}{}}{}{Секция: #6\quad \vspace{3mm}\\\nopagebreak}%
}

\pagestyle{empty}

\begin{document}
\begin{talk}
{Теорема Арвесона о продолжении для условно унитальных вполне положительных отображений} %
{Яшин Всеволод Игоревич} %
{Математический институт им. Стеклова РАН, Российский Квантовый Центр} %
{yashin.vi@mi-ras.ru} %
{} %
{Вещественный и функциональный анализ} %

Условно унитальные вполне положительные отображения используются для характеризации генераторов унитальных вполне положительных динамических полугрупп на \(C^*\)‑алгебрах. В работе предложено обобщение этого понятия на случай отображений между операторными системами. При таком обобщении условно унитальные вполне положительные отображения оказываются инфинитезимальными приращениями унитальных вполне положительных отображений. Изучаются базовые свойства условно унитальных вполне положительных отображений, доказываются двойственность Чоя–Ямиолковского для таких отображений и теорема типа Арвесона о продолжении для вполне ограниченных условно унитальных вполне положительных отображений в конечномерные \(C^*\)-алгебры.

\medskip

\begin{enumerate}
\item[{[1]}] В. И. Яшин, {\it Теорема Арвесона о продолжении для условно унитальных вполне положительных отображений}, Труды МИАН, 324, МИАН, М., 2024, 277–291
\end{enumerate}
\end{talk}
\end{document}
