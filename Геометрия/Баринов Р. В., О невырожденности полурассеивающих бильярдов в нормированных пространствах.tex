\documentclass[12pt, a4paper, figuresright]{book}
\usepackage{mathtools, commath, amssymb, amsthm}
\usepackage{tabularx,graphicx,url,xcolor,rotating,multicol,epsfig,colortbl,lipsum}
\usepackage[T2A]{fontenc}
\usepackage[english,main=russian]{babel}

\setlength{\textheight}{25.2cm}
\setlength{\textwidth}{16.5cm}
\setlength{\voffset}{-1.6cm}
\setlength{\hoffset}{-0.3cm}
\setlength{\evensidemargin}{-0.3cm} 
\setlength{\oddsidemargin}{0.3cm}
\setlength{\parindent}{0cm} 
\setlength{\parskip}{0.3cm}

\newenvironment{talk}[6]{%
\vskip 0pt\nopagebreak%
\vskip 0pt\nopagebreak%
\textbf{#1}\vspace{3mm}\\\nopagebreak%
\textit{#2}\\\nopagebreak%
#3\\\nopagebreak%
\url{#4}\vspace{3mm}\\\nopagebreak%
\ifthenelse{\equal{#5}{}}{}{Соавторы: #5\vspace{3mm}\\\nopagebreak}%
\ifthenelse{\equal{#6}{}}{}{Секция: #6\quad \vspace{3mm}\\\nopagebreak}%
}

\pagestyle{empty}

\begin{document}

\begin{talk}
{О невырожденности  полурассеивающих бильярдов в нормированных\\ пространствах} %
{Баринов Роман Васильевич} %
{МКН СПбГУ}%
{rbarinov2013@yandex.ru} %
{} %
{Геометрия} %

Известно, что бильярдные траектории в дополнении нескольких выпуклых множеств (т.\,е. полурассеивающие бильярды) в евклидовом пространстве имеют локально конечное число отражений от стенок, а при некоторых условиях невырожденности --- и глобально конечное, но построенная теория не обобщается на неевклидов случай. 

В докладе представлены аналогичные вопросы для бильярдов в неевклидовых нормированных пространствах, а также доказано, что в полурассеивающем бильярде в нормированном пространстве любая бильярдная траектория конечной длины содержит конечное число отражений, откуда следует локальная конечность отражений. 


\medskip



\end{talk}
\end{document}
