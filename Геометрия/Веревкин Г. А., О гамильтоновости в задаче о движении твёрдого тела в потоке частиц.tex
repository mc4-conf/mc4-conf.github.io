\documentclass[12pt, a4paper, figuresright]{book}
\usepackage{mathtools, commath, amssymb, amsthm}
\usepackage{tabularx,graphicx,url,xcolor,rotating,multicol,epsfig,colortbl,lipsum}
\usepackage[T2A]{fontenc}
\usepackage[english,main=russian]{babel}

\setlength{\textheight}{25.2cm}
\setlength{\textwidth}{16.5cm}
\setlength{\voffset}{-1.6cm}
\setlength{\hoffset}{-0.3cm}
\setlength{\evensidemargin}{-0.3cm} 
\setlength{\oddsidemargin}{0.3cm}
\setlength{\parindent}{0cm} 
\setlength{\parskip}{0.3cm}

\newenvironment{talk}[6]{%
\vskip 0pt\nopagebreak%
\vskip 0pt\nopagebreak%
\textbf{#1}\vspace{3mm}\\\nopagebreak%
\textit{#2}\\\nopagebreak%
#3\\\nopagebreak%
\url{#4}\vspace{3mm}\\\nopagebreak%
\ifthenelse{\equal{#5}{}}{}{Соавторы: #5\vspace{3mm}\\\nopagebreak}%
\ifthenelse{\equal{#6}{}}{}{Секция: #6\quad \vspace{3mm}\\\nopagebreak}%
}

\pagestyle{empty}

\begin{document}
	
\begin{talk}
{О гамильтоновости в задаче о движении твёрдого тела в потоке частиц} %
{Верёвкин Григорий Александрович} %
{МЦМУ Московский центр фундаментальной и прикладной математики}%
{} %
{} %
{Геометрия} %

В работе обсуждается вопрос о гамильтоновости задачи о движении твердого тела с точкой закрепления (неподвижной точкой) в потоке частиц. Динамическая система не гамильтонова, если рассматривать тела произвольной формы, однако при выполнении некоторых условий на форму тела и расположение точки закрепления система может быть гамильтоновой. Примеры таких тел можно найти в работе А.\,А.Бурова и А.\,В.Карапетяна, где также были выписаны некоторые достаточные условия для того, чтобы рассматриваемая система была гамильтоновой. Как оказалось, эти условия можно ослабить, получив тем самым критерий гамильтоновости рассматриваемой системы с заданным гамильтонианом. Также для этой задачи получены условия на расположение точки закрепления в прямоугольном параллелепипеде, при которых уравнения движения будут гамильтоновы.  Рассматривается прямоугольная призма, расположение в ней точки закрепления. 

\medskip

\begin{enumerate}
\item[{[1]}] Гаджиев М.М., Кулешов А.С. О движении твёрдого тела с неподвижной точкой в потоке частиц // Вестник Московского университета. Серия 1. Математика. Механика. №3, с. 58-68. 
\item[{[2]}] Буров А.А., Карапетян А.В. О движении твердого тела в потоке частиц // Прикладная математика и механика. 1993. Т. 57. Вып. 2. С. 77-81.
\item[{[3]}] Сазонов В.В. Об одном механизме потери устойчивости режима гравитационной ориентации спутника // Институт прикладной математики АН СССР. 1988. Препринт №107. 23 с.
\item[{[4]}] Рыбникова Т.А.. Трещев Д.В. Существование инвариантных торов в задаче о движении спутника с солнечным парусом // Космические исследования. 1990. Т. 28. №2. С. 309-312.
\end{enumerate}
\end{talk}
\end{document}



