\documentclass[12pt, a4paper, figuresright]{book}
\usepackage{mathtools, commath, amssymb, amsthm}
\usepackage{tabularx,graphicx,url,xcolor,rotating,multicol,epsfig,colortbl,lipsum}
\usepackage[T2A]{fontenc}
\usepackage[english,main=russian]{babel}

\setlength{\textheight}{25.2cm}
\setlength{\textwidth}{16.5cm}
\setlength{\voffset}{-1.6cm}
\setlength{\hoffset}{-0.3cm}
\setlength{\evensidemargin}{-0.3cm} 
\setlength{\oddsidemargin}{0.3cm}
\setlength{\parindent}{0cm} 
\setlength{\parskip}{0.3cm}

\newenvironment{talk}[6]{%
\vskip 0pt\nopagebreak%
\vskip 0pt\nopagebreak%
\textbf{#1}\vspace{3mm}\\\nopagebreak%
\textit{#2}\\\nopagebreak%
#3\\\nopagebreak%
\url{#4}\vspace{3mm}\\\nopagebreak%
\ifthenelse{\equal{#5}{}}{}{Соавторы: #5\vspace{3mm}\\\nopagebreak}%
\ifthenelse{\equal{#6}{}}{}{Секция: #6\quad \vspace{3mm}\\\nopagebreak}%
}

\pagestyle{empty}

\begin{document}
	
\begin{talk}
{О статических многообразиях с краем} %
{Медведев Владимир Олегович} %
{Национальный исследовательский университет ``Высшая школа экономики''}%
{} %
{} 
{Геометрия} %

Статические многообразия с краем были введены в геометрию совсем недавно и сразу же вызвали живой интерес в связи с их приложениями к различным вопросам геометрической теории относительности. В римановой геометрии данные многообразия возникают естественным образом при изучении вопросов деформации скалярной кривизны многообразий с краем. Родственным понятием является понятие статической тройки, чья важность была осознана в классических работах Кобаяши, Ляфонтэна и Бургиньона. В докладе будут обсуждаться свойства статических многообразий с краем и их приложения.

\medskip

\begin{enumerate}
\item[{[1]}] L. Ambrozio. On static three-manifolds with positive scalar curvature. Journal of Differen\-tial Geometry, 107(1):1–45, 2017.
\item[{[2]}] T. Cruz and F. Vitorio. Prescribing the curvature of Riemannian manifolds with boundary. Calculus of Variations and Partial Differential Equations, 58(4):124, 2019.
\item[{[3]}] T. Cruz and I. Nunes. On static manifolds satisfying an overdetermined Robin type condition on the boundary. Proceedings of the American Mathematical Society, 151(11): 4971–4982, 2023.
\item[{[4]}] V.Medvedev. On static manifolds with boundary. In progress.
\end{enumerate}
\end{talk}
\end{document}

