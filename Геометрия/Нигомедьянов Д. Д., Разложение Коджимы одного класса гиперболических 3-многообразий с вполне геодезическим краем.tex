\documentclass[12pt, a4paper, figuresright]{book}
\usepackage{mathtools, commath, amssymb, amsthm}
\usepackage{tabularx,graphicx,url,xcolor,rotating,multicol,epsfig,colortbl,lipsum}
\usepackage[T2A]{fontenc}
\usepackage[english,main=russian]{babel}

\setlength{\textheight}{25.2cm}
\setlength{\textwidth}{16.5cm}
\setlength{\voffset}{-1.6cm}
\setlength{\hoffset}{-0.3cm}
\setlength{\evensidemargin}{-0.3cm} 
\setlength{\oddsidemargin}{0.3cm}
\setlength{\parindent}{0cm} 
\setlength{\parskip}{0.3cm}

\newenvironment{talk}[6]{%
	\vskip 0pt\nopagebreak%
	\vskip 0pt\nopagebreak%
	\textbf{#1}\vspace{3mm}\\\nopagebreak%
	\textit{#2}\\\nopagebreak%
	#3\\\nopagebreak%
	\url{#4}\vspace{3mm}\\\nopagebreak%
	\ifthenelse{\equal{#5}{}}{}{Соавторы: #5\vspace{3mm}\\\nopagebreak}%
	\ifthenelse{\equal{#6}{}}{}{Секция: #6\quad \vspace{3mm}\\\nopagebreak}%
}

\pagestyle{empty}

\begin{document}
\begin{talk}
{Разложение Коджимы одного класса гиперболических 3-многообразий с вполне геодезическим краем} %
{Нигомедьянов Даниил Дамирович} %
{МЦМУ им. Леонарда Эйлера}%
{danil.nig1@gmail.com} %
{Фоминых Евгений Анатольевич} %
{Геометрия} %

Коджима доказал, что всякое гиперболическое многообразие с вполне геодезическим краем допускает каноническое разложение на выпуклые гиперболические многогранники. В размерности три это разложение двойственно катлокусу края многообразия. Этот инвариант играет ключевую роль в табулировании гиперболических 3-многообразий. 
Доклад будет посвящен классу гиперболических 3-многообразий с вполне геодезическим краем, триангуляционная сложность которых равняется первому числу Бетти этих многообразий с коэффициентами в группе \(\mathbb{Z}/2\mathbb{Z}\), и каноническому разложению таких многообразий. 

\medskip

Исследование выполнено за счёт гранта Российского научного фонда No  22-11-00299.
\end{talk}
\end{document}