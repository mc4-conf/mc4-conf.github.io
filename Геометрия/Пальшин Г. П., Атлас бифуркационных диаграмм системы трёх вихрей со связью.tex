\documentclass[12pt, a4paper, figuresright]{book}
\usepackage{mathtools, commath, amssymb, amsthm}
\usepackage{tabularx,graphicx,url,xcolor,rotating,multicol,epsfig,colortbl,lipsum}
\usepackage[T2A]{fontenc}
\usepackage[english,main=russian]{babel}

\setlength{\textheight}{25.2cm}
\setlength{\textwidth}{16.5cm}
\setlength{\voffset}{-1.6cm}
\setlength{\hoffset}{-0.3cm}
\setlength{\evensidemargin}{-0.3cm}
\setlength{\oddsidemargin}{0.3cm}
\setlength{\parindent}{0cm}
\setlength{\parskip}{0.3cm}

\newenvironment{talk}[6]{%
\vskip 0pt\nopagebreak%
\vskip 0pt\nopagebreak%
\textbf{#1}\vspace{3mm}\\\nopagebreak%
\textit{#2}\\\nopagebreak%
#3\\\nopagebreak%
\url{#4}\vspace{3mm}\\\nopagebreak%
\ifthenelse{\equal{#5}{}}{}{Соавторы: #5\vspace{3mm}\\\nopagebreak}%
\ifthenelse{\equal{#6}{}}{}{Секция: #6\quad \vspace{3mm}\\\nopagebreak}%
}

\pagestyle{empty}

\begin{document}
\begin{talk}
{Атлас бифуркационных диаграмм системы трёх вихрей со связью}
{Пальшин Глеб Павлович}
{Московский физико-технический институт}
{palshin.gp@phystech.edu}
{}
{Геометрия}

Полностью изучена грубая топология многопараметрического семейства интегрируемых систем гамильтоновой механики, которое описывает движение двух свободных вихрей при наличии третьего, закреплённого в начале координат (см.~[1],~[2]). Модель обобщает два частных случая: динамику гидродинамических параллельных вихревых нитей в безграничной идеальной жидкости при наличии топографической неоднородности (в виде горы, острова и т.\,д.) и динамику магнитных вихрей в ферромагнетиках при наличии закреплённой завихренности (вызванной, например, дефектом среды).

Для данной модели получен явный вид бифуркационной диаграммы отображения момента, вычислены индексы критических подмногообразий, изучена топология изоэнергетических поверхностей, построены инварианты Фоменко слоения Лиувилля (грубые молекулы). Результаты представлены в виде атласа пополненных бифуркационных диаграмм, который каждому набору значений параметров системы ставит в соответствие определённый тип диаграммы, оснащённый топологическими инвариантами.

\medskip

\begin{enumerate}
\item[{[1]}] Г.\,П.~Пальшин, {\it О некомпактной бифуркации в одной обобщенной модели вихревой динамики}, ТМФ, 212:1 (2022), 972–983.
\item[{[2]}] Г.\,П.~Пальшин, {\it Топология слоения Лиувилля в~обобщенной задаче трех вихрей со связью}, Матем. сб., 215:5 (2024), 106–145.
\end{enumerate}
\end{talk}
\end{document} 