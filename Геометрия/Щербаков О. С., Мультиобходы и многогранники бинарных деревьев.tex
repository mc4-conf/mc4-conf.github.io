\documentclass[12pt, a4paper, figuresright]{book}
\usepackage{mathtools, commath, amssymb, amsthm}
\usepackage{tabularx,graphicx,url,xcolor,rotating,multicol,epsfig,colortbl,lipsum}
\usepackage[T2A]{fontenc}
\usepackage[english,main=russian]{babel}

\setlength{\textheight}{25.2cm}
\setlength{\textwidth}{16.5cm}
\setlength{\voffset}{-1.6cm}
\setlength{\hoffset}{-0.3cm}
\setlength{\evensidemargin}{-0.3cm} 
\setlength{\oddsidemargin}{0.3cm}
\setlength{\parindent}{0cm} 
\setlength{\parskip}{0.3cm}

\newenvironment{talk}[6]{%
\vskip 0pt\nopagebreak%
\vskip 0pt\nopagebreak%
\textbf{#1}\vspace{3mm}\\\nopagebreak%
\textit{#2}\\\nopagebreak%
#3\\\nopagebreak%
\url{#4}\vspace{3mm}\\\nopagebreak%
\ifthenelse{\equal{#5}{}}{}{Соавторы: #5\vspace{3mm}\\\nopagebreak}%
\ifthenelse{\equal{#6}{}}{}{Секция: #6\quad \vspace{3mm}\\\nopagebreak}%
}

\pagestyle{empty}

\begin{document}
	
\begin{talk}
{Мультиобходы и многогранники бинарных деревьев} %
{Щербаков Олег Сергеевич} %
{МГУ им. М.\,В. Ломоносова, МГТУ им. Н.\,Э. Баумана}%
{shcherbakovos@yandex.ru} %
{}
{Геометрия} %

Одномерная задача Громова о минимальном заполнении конечного метрического пространства [1] возникла как обобщение  задачи Штейнера о кратчайшей сети и задачи Громова о  минимальном заполнении гладкого риманова многообразия. Задача заключается в поиске взвешенного дерева наименьшего веса, соединяющего данное метрическое пространство так, что для любых точек метрического пространства вес единственного пути, соединяющего их в дереве, был не меньше расстояния между ними в метрическом пространстве.

Формула веса миимального заполнения [2] использует т.\,н. мультиобходы бинарного дерева, в частности, неприводимые мультиобходы.  Другой подход --- рассмотреть задачу с точки зрения линейного программирования [3]. При таком подходе возникают так называемые многогранники бинарных деревьев. 

Оказывается, между вершинами многогранников бинарных деревьев и неприводимыми мультиобходами есть естественная биекция [4]. Автору удалось получить оценки на кратности неприводимых мультиобходов, найти их нормальную форму и описать для некоторых типов бинарных деревьев их многогранники.

\medskip

\begin{enumerate}
\item[{[1]}]  {Иванов А.О., Тужилин А.А.}  {\it Одномерная проблема Громова о минимальном заполнении.} // Матем. сб. 2012. Т. 203, № 5.С. 65-118.
\item[{[2]}] Еремин А.Ю. {\it Формула веса минимального заполнения конечного метрического пространства.} Матем. сб., 2013. Т.204, №9. С.51-72.
\item[{[3]}] Ivanov A., Tuzhilin A. {\it Dual Linear Programming Problem and One-Dimensional Gromov Minimal Fillings of Finite Metric Space} //  Differential Equations on Manifolds and Mathematical Physics. Trends in Mathematics. Birkhauser, Cham. 2022. pp. 165-182.
\item[{[4]}]	Щербаков О.С. {\it Многогранники бинарных деревьев, строение многогранника дерева типа ``змея''.} Чебышёвский сб., 2022. Т.23, №85. С.136-151.
\end{enumerate}
\end{talk}
\end{document}
