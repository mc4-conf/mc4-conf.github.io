\documentclass[12pt, a4paper, figuresright]{book}
\usepackage{mathtools, commath, amssymb, amsthm}
\usepackage{tabularx,graphicx,url,xcolor,rotating,multicol,epsfig,colortbl,lipsum}
\usepackage[T2A]{fontenc}
\usepackage[english,main=russian]{babel}

\setlength{\textheight}{25.2cm}
\setlength{\textwidth}{16.5cm}
\setlength{\voffset}{-1.6cm}
\setlength{\hoffset}{-0.3cm}
\setlength{\evensidemargin}{-0.3cm} 
\setlength{\oddsidemargin}{0.3cm}
\setlength{\parindent}{0cm} 
\setlength{\parskip}{0.3cm}

\newenvironment{talk}[6]{%
\vskip 0pt\nopagebreak%
\vskip 0pt\nopagebreak%
\textbf{#1}\vspace{3mm}\\\nopagebreak%
\textit{#2}\\\nopagebreak%
#3\\\nopagebreak%
\url{#4}\vspace{3mm}\\\nopagebreak%
\ifthenelse{\equal{#5}{}}{}{Соавторы: #5\vspace{3mm}\\\nopagebreak}%
\ifthenelse{\equal{#6}{}}{}{Секция: #6\quad \vspace{3mm}\\\nopagebreak}%
}

\pagestyle{empty}

\begin{document}
\begin{talk}
{В поисках теней: загадки Московской топологической конференции}
{Апушкинская Дарья Евгеньевна и Назаров Александр Ильич}
{РУДН; ПОМИ РАН и СПбГУ}
{al.il.nazarov@gmail.com, apushkinskaya@gmail.com}
{Г.\,И. Синкевич}
{История математики}

Первая международная топологическая конференция проходила в Москве с 4 по 10 сентября 1935 года. Это была вторая (после конференции по дифференциальной геометрии 1934 г.) специализированная конференция в истории международного математического сообщества, собравшая выдающийся состав участников из 10 стран Европы и Америки. Представленные на ней результаты оказали колоссальное влияние на развитие топологии.

Работа конференции была широко освещена как в официальных публикациях, так и в многочисленных воспоминаниях участников. Тем не менее фактическая информация (список докладчиков, количество докладов и т. д.) почти во всех источниках была дана неполно или неточно, а зачастую и противоречиво, что довольно загадочно для события, произошедшего сравнительно недавно.

Основываясь на доступных источниках, мы попытались представить полную и непротиворечивую картину событий. В частности, мы приводим полный список докладов и докладчиков, а также даем полное описание фотографии участников конференции.

Доклад основан на статье [1].

\medskip

\begin{enumerate}
\item[{[1]}] D.E. Apushkinskaya, A.I. Nazarov, G.I. Sinkevich, {\it In Search of Shadows: The First Topological Conference, Moscow 1935}, The Mathematical Intelligencer, 41:4 (2019), 37-42.
\end{enumerate}
\end{talk}
\end{document}