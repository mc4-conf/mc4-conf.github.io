\documentclass[12pt, a4paper, figuresright]{book}
\usepackage{mathtools, commath, amssymb, amsthm}
\usepackage{tabularx,graphicx,url,xcolor,rotating,multicol,epsfig,colortbl,lipsum}
\usepackage[T2A]{fontenc}
\usepackage[english,main=russian]{babel}

\setlength{\textheight}{25.2cm}
\setlength{\textwidth}{16.5cm}
\setlength{\voffset}{-1.6cm}
\setlength{\hoffset}{-0.3cm}
\setlength{\evensidemargin}{-0.3cm} 
\setlength{\oddsidemargin}{0.3cm}
\setlength{\parindent}{0cm} 
\setlength{\parskip}{0.3cm}

\newenvironment{talk}[6]{%
\vskip 0pt\nopagebreak%
\vskip 0pt\nopagebreak%
\textbf{#1}\vspace{3mm}\\\nopagebreak%
\textit{#2}\\\nopagebreak%
#3\\\nopagebreak%
\url{#4}\vspace{3mm}\\\nopagebreak%
\ifthenelse{\equal{#5}{}}{}{Соавторы: #5\vspace{3mm}\\\nopagebreak}%
\ifthenelse{\equal{#6}{}}{}{Секция: #6\quad \vspace{3mm}\\\nopagebreak}%
}

\pagestyle{empty}

\begin{document}
\begin{talk}
{Два портрета на фоне эпохи} %
{Апушкинская Дарья Евгеньевна и Назаров Александр Ильич} %
{РУДН; ПОМИ РАН и СПбГУ}%
{apushkinskaya@gmail.com, al.il.nazarov@gmail.com} %
{} %
{История математики} %

Начало XX века в России. Эпоха расцвета промышленности, науки, культуры, и в то же время - эпоха нарастания общественных противоречий, вылившихся в кровавую кашу гражданской войны. Эпоха неустойчивости и ``экспоненциального разбегания'' человеческих судеб. Мы рассмотрим этот период на примере двух ``мировых линий'' с близкими начальными условиями. 

Владимир Иванович Смирнов и Яков Давидович Тамаркин - ровесники, друзья, выпускники Петербургского университета. Выдающиеся математики, ставшие и выдающимися организаторами науки и создателями научных школ – один в Советском Союзе, другой в Соединенных Штатах Америки. Драматических, а порой и трагических поворотов в их биографиях хватило бы на пару голливудских боевиков. Синóпсис такого боевика мы и хотим представить слушателям.
\end{talk}
\end{document}
