\documentclass[12pt, a4paper, figuresright]{book}
\usepackage{mathtools, commath, amssymb, amsthm}
\usepackage{tabularx,graphicx,url,xcolor,rotating,multicol,epsfig,colortbl,lipsum}
\usepackage[T2A]{fontenc}
\usepackage[english,main=russian]{babel}

\setlength{\textheight}{25.2cm}
\setlength{\textwidth}{16.5cm}
\setlength{\voffset}{-1.6cm}
\setlength{\hoffset}{-0.3cm}
\setlength{\evensidemargin}{-0.3cm} 
\setlength{\oddsidemargin}{0.3cm}
\setlength{\parindent}{0cm} 
\setlength{\parskip}{0.3cm}

\newenvironment{talk}[6]{%
\vskip 0pt\nopagebreak%
\vskip 0pt\nopagebreak%
\textbf{#1}\vspace{3mm}\\\nopagebreak%
\textit{#2}\\\nopagebreak%
#3\\\nopagebreak%
\url{#4}\vspace{3mm}\\\nopagebreak%
\ifthenelse{\equal{#5}{}}{}{Соавторы: #5\vspace{3mm}\\\nopagebreak}%
\ifthenelse{\equal{#6}{}}{}{Секция: #6\quad \vspace{3mm}\\\nopagebreak}%
}

\pagestyle{empty}

\begin{document}
\begin{talk}
{Об отечественных научных школах нелинейного функционального анализа в 1940-х-1960-х гг.} %
{Богатов Егор Михайлович} %
{ГФ НИТУ МИСИС; СТИ НИТУ МИСИС}%
{embogatov@inbox.ru} %
{}
{История математики} %

Рассмотрение истории математики через призму научных школ даёт дополнительное представление о  математике и её истории [1]. Математические мыслительные средства, выработанные в рамках какой-либо научной школы, являются продуктом коллективной деятельности, что многократно увеличивает скорость их приведения в систему, генерацию продуктивных методов их использования и распространение в пределах национальных и международных научных сообществ.  

Основным результатом работы является:
\begin{enumerate} 
\item введение в историко-научный оборот в области  математики XX в. нового материала - истории отечественных школ нелинейного функционального анализа (НФА);
\item характеризация научных школ НФА, функционирующих в СССР в 1940-х-1960-х гг. с выделением времени и места их основания, руководителей и основных представителей, продолжительности функционирования и конкретизацией области исследований;
\item определение вклада отечественных школ НФА в развитие следующих его разделов - вариационного исчисления в целом, теории ветвления и бифуркаций, теории положительных операторов, топологических методов нелинейного анализа, вариационных и приближённых  методов решения нелинейных операторных уравнений.
\end{enumerate}

\medskip

\begin{enumerate} 
\item[{[1]}] Demidov S. L'histoire des mathématiques en Russie et l'URSS en tant qu'histoire des écoles // ИМИ. 2-я серия. Спец. выпуск. М., 1997. С.9-21.
\end{enumerate}
\end{talk}
\end{document}
