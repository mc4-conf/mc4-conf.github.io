\documentclass[12pt, a4paper, figuresright]{book}
\usepackage{mathtools, commath, amssymb, amsthm}
\usepackage{tabularx,graphicx,url,xcolor,rotating,multicol,epsfig,colortbl,lipsum}
\usepackage[T2A]{fontenc}
\usepackage[english,main=russian]{babel}

\setlength{\textheight}{25.2cm}
\setlength{\textwidth}{16.5cm}
\setlength{\voffset}{-1.6cm}
\setlength{\hoffset}{-0.3cm}
\setlength{\evensidemargin}{-0.3cm} 
\setlength{\oddsidemargin}{0.3cm}
\setlength{\parindent}{0cm} 
\setlength{\parskip}{0.3cm}

\newenvironment{talk}[6]{%
\vskip 0pt\nopagebreak%
\vskip 0pt\nopagebreak%
\textbf{#1}\vspace{3mm}\\\nopagebreak%
\textit{#2}\\\nopagebreak%
#3\\\nopagebreak%
\url{#4}\vspace{3mm}\\\nopagebreak%
\ifthenelse{\equal{#5}{}}{}{Соавторы: #5\vspace{3mm}\\\nopagebreak}%
\ifthenelse{\equal{#6}{}}{}{Секция: #6\quad \vspace{3mm}\\\nopagebreak}%
}

\pagestyle{empty}

\begin{document}
\begin{talk}
{Вспоминая Людвига Прандтля} %
{Богданов Андрей Николаевич и Кондратьев Игорь Михайлович} %
{НИИ механики МГУ имени М.В.Ломоносова, МГТУ имени Н.\,Э. Баумана}%
{bogdanov@imec.msu.ru} %
{} %
{История математики} %

Работы Людвига Прандтля (1875–1953) в области механики жидкости и газа (МЖГ) давно известны в нашей стране и получили признание отечественных специалистов. Однако наряду с фундаментальным вкладом в МЖГ научные интересы Прандтля реализовались также в теории упругости, теории пластичности и строительной механики~[1].

Регулярно отмечавшиеся юбилеи Людвига Прандтля и связанные с его именем даты становились поводом к анализу его научного наследия и вклада в науку его учеников и последователей. Примечательно, что хотя подобный анализ проводился с разных позиций, заслуги Прандтля не подвергались сомнению, а многие, даже не знавшие ученого лично,
называли его своим научным наставником [2, 3]. Имя Прандтля часто вспоминали и наши соотечественники, в том числе и на проводившихся 
у нас юбилейных научных мероприятиях [4].

Отрадно, что разрешение многих принципиальных проблем МЖГ в развитие идей Прандтля на качественно более высоком уровне было получено в трудах советских и российских ученых-механиков [5–10].

\medskip

\begin{enumerate}
\item[{[1]}] А.Н. Боголюбов, {\it Математики. Механики. Биографический справочник}, Киев: Наукова думка, 1983. 639 с.
\item[{[2]}] {\it Говорят руководители ЦАГИ. Нейланд Владимир Яковлевич} / В.В. Лазарев ЦАГИ. Цаговцы. Время. М.: Капитал-Пресс, 2021. 192 с.
\item[{[3]}] Е.А. Гаев, {\it Людвиг Прандтль в гидромеханике прошлого и будущего} // Прикладна гiдромеханiка. 2014. Т.16. С.3–16.
\item[{[4]}] О.А. Олейник, {\it К математической теории пограничного слоя. Доклад прочитан 20 октября 1967 г. на Юбилейной сессии Общего собрания Отделения математики АН СССР} // Математические заметки. 1968. Т.3, № 4. С.473–480.
\item[{[5]}] В.В. Сычев, А.И. Рубан , Вик.В. Сычев, Г.Л. Королев , {\it Асимптотическая теория отрывных течений}, Москва: Наука, 1987.
\item[{[6]}] В.Я. Нейланд, В.В. Боголепов, Г.Н. Дудин, И.И. Липатов, {\it Асимптотическая теория сверхзвуковых течений вязкого газа}, М.: Физматлит, 2003. 456 с.
\item[{[7]}] В.И. Жук, {\it Волны Толлмина–Шлихтинга и солитоны}, М.: Наука, 2001. 167 с.
\item[{[8]}] О.С. Рыжов, {\it О нестационарном пограничном слое с самоиндуцированным давлением при околозвуковых скоростях внешнего потока} // Докл. АН СССР. 1977. Т.236. № 5. С.1091–1094.
\item[{[9]}] А.Н. Богданов, В.Н. Диесперов, В.И. Жук, {\it Неклассические трансзвуковые пограничные слои. К преодолению некоторых тупиковых ситуаций в аэродинамике больших скоростей} // ЖВММФ. 2018, т. 58, № 2, с. 270–280.
\item[{[10]}] А.Н. Богданов, {\it К математическому моделированию взаимодействующего трансзвукового пограничного слоя с нелинейным профилем невозмущенной скорости} // Докл. РАН. 2021. Т.501. № 1. С.29–32.     
\end{enumerate}
\end{talk}
\end{document}
