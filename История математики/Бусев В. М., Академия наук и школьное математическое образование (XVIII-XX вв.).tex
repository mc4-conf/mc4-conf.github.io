\documentclass[12pt, a4paper, figuresright]{book}
\usepackage{mathtools, commath, amssymb, amsthm}
\usepackage{tabularx,graphicx,url,xcolor,rotating,multicol,epsfig}
\usepackage[T2A]{fontenc}
\usepackage[english,main=russian]{babel}

\setlength{\textheight}{25.2cm}
\setlength{\textwidth}{16.5cm}
\setlength{\voffset}{-1.6cm}
\setlength{\hoffset}{-0.3cm}
\setlength{\evensidemargin}{-0.3cm} 
\setlength{\oddsidemargin}{0.3cm}
\setlength{\parindent}{0cm} 
\setlength{\parskip}{0.3cm}

\newenvironment{talk}[6]{%
\vskip 0pt\nopagebreak%
\vskip 0pt\nopagebreak%
\textbf{#1}\vspace{3mm}\\\nopagebreak%
\textit{#2}\\\nopagebreak%
#3\\\nopagebreak%
\url{#4}\vspace{3mm}\\\nopagebreak%
\ifthenelse{\equal{#5}{}}{}{Соавторы: #5\vspace{3mm}\\\nopagebreak}%
\ifthenelse{\equal{#6}{}}{}{Секция: #6\quad \vspace{3mm}\\\nopagebreak}%
}

\pagestyle{empty}

\begin{document}
	
\begin{talk}
{Академия наук и школьное математическое образование (XVIII--XX вв.)} %
{Бусев Василий Михайлович} %
{}%
{vbusev@yandex.ru} %
{} %
{История математики} %

В докладе планируется осветить вклад Академии наук и ее членов в развитие школьного математического образования. Математические науки изучались в академических гимназии и университете (XVIII в.), ряд учебников школьного уровня был написан\linebreak Л. Эйлером, 
С.\,Я. Румовским, Г.\,В. Крафтом, а ближе к концу XVIII столетия -- выпускниками гимназии и университета Я.\,П. Козельским, М.\,Е. Головиным. В первой четверти XIX века основными учебниками математики для гимназий были руководства Н.И. Фусса, составленные под влиянием идей Л. Эйлера.

Позднее влияние на постановку преподавания математики в средних учебных заведениях оказали Д.\,М. Перевощиков, М.\,В. Остроградский, В.\,Я. Буняковский, Н.\,В. Бугаев, О.\,И. Сомов (авторы программ и учебников), П.Л. Чебышев (как член Ученого комитета Министерства народного просвещения).

Традиция участия математического сообщества в развитии школьного математического образования возрождается в середине 1930-х годов. Большое внимание вопросам преподавания математики в школе уделял А.\,Я. Хинчин.

Наибольшее влияние на школьное образование (в том числе, математическое) Академия наук СССР оказала в период с начала 1960-х по конец 1970-х годов, когда под руководством ученых было проведено обновление содержания предметов преподавания и созданы новые учебники. Тогда же по инициативе ученых были созданы специализированные физико-математические школы, научно-популярный журнал ``Квант''.
В 1980-е годы появились учебники, написанные при участии А.\,Д. Александрова, А.\,В. Погорелова, А.\,Н. Тихонова, С.\,М. Никольского.
В 1990-е годы по ряду причин ученые в значительной степени утратили влияние на школьное образование (если не принимать во внимание влияние посредством учебников, которые продолжали использоваться в школах и после распада СССР).
\end{talk}
\end{document}


