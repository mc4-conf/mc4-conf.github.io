\documentclass[12pt, a4paper, figuresright]{book}
\usepackage{mathtools, commath, amssymb, amsthm}
\usepackage{tabularx,graphicx,url,xcolor,rotating,multicol,epsfig}
\usepackage[T2A]{fontenc}
\usepackage[english,main=russian]{babel}

\setlength{\textheight}{25.2cm}
\setlength{\textwidth}{16.5cm}
\setlength{\voffset}{-1.6cm}
\setlength{\hoffset}{-0.3cm}
\setlength{\evensidemargin}{-0.3cm} 
\setlength{\oddsidemargin}{0.3cm}
\setlength{\parindent}{0cm} 
\setlength{\parskip}{0.3cm}

\newenvironment{talk}[6]{%
\vskip 0pt\nopagebreak%
\vskip 0pt\nopagebreak%
\textbf{#1}\vspace{3mm}\\\nopagebreak%
\textit{#2}\\\nopagebreak%
#3\\\nopagebreak%
\url{#4}\vspace{3mm}\\\nopagebreak%
\ifthenelse{\equal{#5}{}}{}{Соавторы: #5\vspace{3mm}\\\nopagebreak}%
\ifthenelse{\equal{#6}{}}{}{Секция: #6\quad \vspace{3mm}\\\nopagebreak}%
}

\pagestyle{empty}

\begin{document}
	
\begin{talk}
{Н.\,Н. Лузин и математика ХХ века} %
{Демидов Сергей Сергеевич} %
{МГУ им. М.В. Ломоносова}%
{serd42@mail.ru} %
{} %
{История математики} %

Н.\,Н. Лузин (1883 – 1950) известен выдающимися результатами в теории множеств и теории функций. Ему принадлежат важные достижения в теории поверхностей, в теории дифференциальных уравнений и других разделах математики. Высокую оценку в научном сообществе получили его философские идеи, в частности, соображения, касающиеся оснований математики, а также его оригинальные историко-математические исследования. Особенным образом следует отметить его педагогический дар, позволивший ему внести важный вклад в отечественную науку и образование и создать одну из важнейших научных школ ХХ столетия --- Московскую школу теории функций, ставшую одним из краеугольных камней Советской математической школы, одной из наиболее влиятельных в математической науке столетия. 

В докладе будут обсуждаться обстоятельства выбора Н.\,Н. Лузиным направлений его исследований, а также их воздействие на развитие математики в нашей стране и в мире. 
\end{talk}
\end{document}


