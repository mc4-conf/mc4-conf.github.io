\documentclass[12pt, a4paper, figuresright]{book}
\usepackage{mathtools, commath, amssymb, amsthm}
\usepackage{tabularx,graphicx,url,xcolor,rotating,multicol,epsfig}
\usepackage[T2A]{fontenc}
\usepackage[english,main=russian]{babel}

\setlength{\textheight}{25.2cm}
\setlength{\textwidth}{16.5cm}
\setlength{\voffset}{-1.6cm}
\setlength{\hoffset}{-0.3cm}
\setlength{\evensidemargin}{-0.3cm} 
\setlength{\oddsidemargin}{0.3cm}
\setlength{\parindent}{0cm} 
\setlength{\parskip}{0.3cm}

\newenvironment{talk}[6]{%
\vskip 0pt\nopagebreak%
\vskip 0pt\nopagebreak%
\textbf{#1}\vspace{3mm}\\\nopagebreak%
\textit{#2}\\\nopagebreak%
#3\\\nopagebreak%
\url{#4}\vspace{3mm}\\\nopagebreak%
\ifthenelse{\equal{#5}{}}{}{Соавторы: #5\vspace{3mm}\\\nopagebreak}%
\ifthenelse{\equal{#6}{}}{}{Секция: #6\quad \vspace{3mm}\\\nopagebreak}%
}

\pagestyle{empty}

\begin{document}
\begin{talk}
{Первые академики России по прикладной математике} %
{Демидова Ирина Ивановна} %
{СПбГУ}%
{maria_ib@mail.ru} %
{} %
{История математики} %

Для развития российской науки в 1724 году Пётр 1 пригласил иностранных учёных, среди которых был  Даниил Бернулли (1700--1782). Он имел медицинское образование, защитил диссертацию  по физиологии дыхания, выпустил ``Математические упражнения''. Это позволило ему получить звание академика Болонской академии наук. В Петербургской АН Д. Бернулли одним из первых изучал движения мышц. В монографии  ``Гидродинамика или записки о силах и движениях жидкостей'' (1738) он первым вывел основное уравнение стационарного движения идеальной жидкости. Показал возможность изучения кровообращения и предложил способ измерения давления крови в сосудах.

Позднее при изучении взаимосвязи между скоростью, с которой течёт кровь, и ее давлением к Даниилу присоединился Леонард Эйлер (1707--1783).  Эйлер впервые сформулировал задачу о движении крови по сосудам с учётом изменения  свойств стенок сосудов. Эйлер решал задачу о критической нагрузке при  сжатии стержня.  Спустя почти сто лет его формула была применена для расчета прочности опор железнодорожных мостов. В биомеханике эту формулу можно применить для оценки допустимой нагрузки на кости в зависимости от состояния тканей, т.е. при учете изменения величины модуля Юнга от времени, возраста, соотношения компонентов структуры и других параметров. 
В своих исследованиях Бернулли и Эйлер показали возможные применения математики к решению биомеханических задач. 

Почти через 100 лет выпускник École Polytechnique Габриель Ламе (1795--1870) был приглашён Александром I для чтения лекций по инженерным наукам. Ламе также включился в решение разнообразных технических задач, среди них задача о действии внешнего и внутреннего давлений на сферу или цилиндр. В ХХ веке задача Ламе была использована для определения напряжённого состояния и  разрушения биоконструкций сферической и цилиндрической форм.

\end{talk}
\end{document}

