\documentclass[12pt, a4paper, figuresright]{book}
\usepackage{mathtools, commath, amssymb, amsthm}
\usepackage{tabularx,graphicx,url,xcolor,rotating,multicol,epsfig,colortbl,lipsum}
\usepackage[T2A]{fontenc}
\usepackage[english,main=russian]{babel}

\setlength{\textheight}{25.2cm}
\setlength{\textwidth}{16.5cm}
\setlength{\voffset}{-1.6cm}
\setlength{\hoffset}{-0.3cm}
\setlength{\evensidemargin}{-0.3cm} 
\setlength{\oddsidemargin}{0.3cm}
\setlength{\parindent}{0cm} 
\setlength{\parskip}{0.3cm}

\newenvironment{talk}[6]{%
\vskip 0pt\nopagebreak%
\vskip 0pt\nopagebreak%
\textbf{#1}\vspace{3mm}\\\nopagebreak%
\textit{#2}\\\nopagebreak%
#3\\\nopagebreak%
\url{#4}\vspace{3mm}\\\nopagebreak%
\ifthenelse{\equal{#5}{}}{}{Соавторы: #5\vspace{3mm}\\\nopagebreak}%
\ifthenelse{\equal{#6}{}}{}{Секция: #6\quad \vspace{3mm}\\\nopagebreak}%
}

\pagestyle{empty}

\begin{document}
\begin{talk}
{Из истории высшего математического образования в Оренбургском крае} %
{Зубова Инна Каримовна} %
{Оренбургский государственный университет}%
{zubova-inna@yandtx.ru} %
{Игнатушина Инесса Васильевна} %
{История математики} %

Говоря о предыстории высшего образования в Оренбургском крае, нельзя не вспомнить о тех средних учебных заведениях, в которых уже начиная с 30-х гг. XIX в.
работали выпускники лучших университетов страны, приезжавшие сюда по назначению и, как правило, сразу начинавшие играть значительную роль в общественной и 
культурной жизни города. Эти преподаватели сыграли большую роль в формировании предпосылок для создания в городе высших учебных заведений. 
Авторы останавливаются на деятельности некоторых преподавателей математики, которые не только добросовестно выполняли служебные обязанности, но и все свои знания,
энтузиазм и творческие возможности отдавали делу просвещения, организуя научные кружки, читая публичные лекции, совершенствуясь в методике преподавания своих
предметов. Достойными преемниками этих преподавателей являются специалисты советского времени, которых можно назвать создателями современных вузов города.
Оренбургский государственный университет, возникший на базе политехнического института --– сегодня крупнейший вуз области. Богатую историю имеет и
Оренбургский педагогический университет, который в этом году отмечает свое 105-летие. В докладе представлен краткий обзор истории этих высших учебных заведений Оренбурга
и деятельности преподавателей математики, на протяжении многих лет в них трудившихся. 
\end{talk}
\end{document} 
