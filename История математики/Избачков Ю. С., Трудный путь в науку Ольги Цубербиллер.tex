\documentclass[12pt, a4paper, figuresright]{book}
\usepackage{mathtools, commath, amssymb, amsthm}
\usepackage{tabularx,graphicx,url,xcolor,rotating,multicol,epsfig,colortbl,lipsum}
\usepackage[T2A]{fontenc}
\usepackage[english,main=russian]{babel}

\setlength{\textheight}{25.2cm}
\setlength{\textwidth}{16.5cm}
\setlength{\voffset}{-1.6cm}
\setlength{\hoffset}{-0.3cm}
\setlength{\evensidemargin}{-0.3cm} 
\setlength{\oddsidemargin}{0.3cm}
\setlength{\parindent}{0cm} 
\setlength{\parskip}{0.3cm}

\newenvironment{talk}[6]{%
\vskip 0pt\nopagebreak%
\vskip 0pt\nopagebreak%
\textbf{#1}\vspace{3mm}\\\nopagebreak%
\textit{#2}\\\nopagebreak%
#3\\\nopagebreak%
\url{#4}\vspace{3mm}\\\nopagebreak%
\ifthenelse{\equal{#5}{}}{}{Соавторы: #5\vspace{3mm}\\\nopagebreak}%
\ifthenelse{\equal{#6}{}}{}{Секция: #6\quad \vspace{3mm}\\\nopagebreak}%
}

\pagestyle{empty}

\begin{document}
\begin{talk}
{Трудный путь в науку Ольги Цубербиллер}
{Избачков Юрий Сергеевич}
{Российский научно-исследовательский институт культурного и природного наследия имени Д.\,С. Лихачёва}
{strax5nature@gmail.com}
{Рыбак Кирилл Евгеньевич, доктор культурологии}
{История математики} %

Ольга Николаевна Губонина (в замужестве Цубербиллер) (1885-1975) --- известный математик и педагог, автор задачника ``Задачи и упражнения по аналитической геометрии'', выдержавшего внушительное количество переизданий, в том числе на иностранных языках, в течение нескольких десятилетий была заведующей кафедрой Института тонких химических технологий (Московского государственного университета тонких химических технологий имени М.В.Ломоносова). Кроме того, Ольга Николаевна известна своими дружескими связями с представителями творческой интеллигенции Серебряного века, театральными деятелями, художниками и даже адептами советских эзотерических кружков.
Становление и развитие ученого обуславливаются причинами, которые побудили его выбрать научное поприще. У Цубербиллер путь в науку был гораздо более сложным, нежели его описывали советские биографы.
Ольга Николаевна в советское время по понятным причинам была вынуждена скрывать свое родство с купцами-миллионщиками, дворянское происхождение и связь с деятельностью партии эсеров. В 1906-1911 годах она была вовлечена в работу Междупартийного Красного Креста (структура помощи политическим арестантам и ссыльным).
Ее дедушка Петр Ионович Губонин из крепостных через благотворительность дослужится до чина тайного советника. Ни коим образом не умаляя математические способности Ольги Николаевны, отметим, что учреждённая Петром Ионовичем в 1870-е гг. стипендия очевидно способствовала ее обучению на математическом отделении Московских Высших Женских Курсах (один из ее преподавателей был получателем стипендии еще до ее рождения).
В 1908 году родственники выдали Ольгу Николаевну замуж за товарища прокурора Московской Судебной Палаты Владимира Владимировича Цубербиллера (1866-1910). При этом сын Владимира Владимировича от первого брака --- Владимир был младше Ольги Николаевны всего на семь лет. В 1911 году после смерти Владимира Владимировича его сын Владимир и вдова Ольга Николаевна были арестованы в рамках уголовного дела о распространении нелегальной литературы и участии в запрещенном преступном сообществе. Владимир Цубербиллер взял вину на себя, и очевидно не без усилий бывших коллег мужа, Ольга Николаевна была представлена случайным свидетелем развернутой пасынком революционной работы. По результатам рассмотрения дела суд приговорил Владимира Цубербиллера к тюремному заключению сроком на один год, а его соучастников Эмиля Нордштрема и Николая Витка (в будущем известного советского ученого-урбаниста) на 8 месяцев. Полагаем, такое серьезное потрясение заставило Ольгу Николаевну пересмотреть свое участие в революционной деятельности и сконцентрироваться на научной работе, к которой у нее были очевидные способности.
Принимала участие в работе Московского математического общества.
\end{talk}
\end{document}
