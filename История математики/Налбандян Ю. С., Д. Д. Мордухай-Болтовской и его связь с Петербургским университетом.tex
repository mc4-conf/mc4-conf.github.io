\documentclass[12pt, a4paper, figuresright]{book}
\usepackage{mathtools, commath, amssymb, amsthm}
\usepackage{tabularx,graphicx,url,xcolor,rotating,multicol,epsfig}
\usepackage[T2A]{fontenc}
\usepackage[english,main=russian]{babel}

\setlength{\textheight}{25.2cm}
\setlength{\textwidth}{16.5cm}
\setlength{\voffset}{-1.6cm}
\setlength{\hoffset}{-0.3cm}
\setlength{\evensidemargin}{-0.3cm} 
\setlength{\oddsidemargin}{0.3cm}
\setlength{\parindent}{0cm} 
\setlength{\parskip}{0.3cm}

\newenvironment{talk}[6]{%
\vskip 0pt\nopagebreak%
\vskip 0pt\nopagebreak%
\textbf{#1}\vspace{3mm}\\\nopagebreak%
\textit{#2}\\\nopagebreak%
#3\\\nopagebreak%
\url{#4}\vspace{3mm}\\\nopagebreak%
\ifthenelse{\equal{#5}{}}{}{Соавторы: #5\vspace{3mm}\\\nopagebreak}%
\ifthenelse{\equal{#6}{}}{}{Секция: #6\quad \vspace{3mm}\\\nopagebreak}%
}

\pagestyle{empty}

\begin{document}
\begin{talk}
{Д.\,Д. Мордухай-Болтовской и его связь с Петербургским университетом} %
{Налбандян Юлия Сергеевна} %
{Южный федеральный университет, Институт математики, механики и компьютерных наук имени И.\,И. Воровича}%
{ysnalbandyan@sfedu.ru} %
{} %
{История математики} %

Дмитрий Дмитриевич Мордухай-Болтовской, по праву считающийся одним из основателей ростовской математической школы, учился в Санкт-Петербургском университете в 1894--1898 гг. По рекомендации К.А. Поссе и А.А. Маркова по окончании университета он был направлен в Варшаву, в Варшавский политехнический институт, ``штатным преподавателем с функциями ассистента'' при Г.Ф. Вороном. Впоследствии Мордухай-Болтовской сдавал в Санкт-Петербурге магистерские экзамены и в 1906 г. защитил магистерскую диссертацию.  Став профессором Варшавского Императорского университета, переехав вместе с университетом в Ростов-на-Дону, он всегда тепло вспоминал своих учителей, которые, по его образному выражению, ``жили под солнцем Чебышёва''. К этой школе на правах внука он причислял и себя.

В докладе предполагается остановиться на студенческих годах выдающегося математика, проанализировать трудности, с которыми он столкнулся в первые годы своей работы (например, упор на практические занятия в политехническом институте), а также рассмотреть его взаимоотношения с такими учёными как А.\,А. Марков, К.\,А. Поссе, Г.\,Ф. Вороной, И.\,Л. Пташицкий. Используются материалы из опубликованных статей, архивные документы и письма Д.\,Д. Мордухай-Болтовского, адресованные сыну.
\end{talk}
\end{document}
