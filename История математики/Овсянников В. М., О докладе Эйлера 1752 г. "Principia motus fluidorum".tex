\documentclass[12pt, a4paper, figuresright]{book}
\usepackage{mathtools, commath, amssymb, amsthm}
\usepackage{tabularx,graphicx,url,xcolor,rotating,multicol,epsfig}
\usepackage[T2A]{fontenc}
\usepackage[english,main=russian]{babel}

\setlength{\textheight}{25.2cm}
\setlength{\textwidth}{16.5cm}
\setlength{\voffset}{-1.6cm}
\setlength{\hoffset}{-0.3cm}
\setlength{\evensidemargin}{-0.3cm} 
\setlength{\oddsidemargin}{0.3cm}
\setlength{\parindent}{0cm} 
\setlength{\parskip}{0.3cm}

\newenvironment{talk}[6]{%
\vskip 0pt\nopagebreak%
\vskip 0pt\nopagebreak%
\textbf{#1}\vspace{3mm}\\\nopagebreak%
\textit{#2}\\\nopagebreak%
#3\\\nopagebreak%
\url{#4}\vspace{3mm}\\\nopagebreak%
\ifthenelse{\equal{#5}{}}{}{Соавторы: #5\vspace{3mm}\\\nopagebreak}%
\ifthenelse{\equal{#6}{}}{}{Секция: #6\quad \vspace{3mm}\\\nopagebreak}%
}

\pagestyle{empty}

\begin{document}
\begin{talk}
{О докладе Эйлера 1752 г. ``Principia motus fluidorum''} %
{Овсянников Владислав Михайлович} %
{Российский университет транспорта (РУТ-МИИТ)}%
{OvsyannikovVM@yandex.ru} %
{} %
{История математики} %

Обсуждается уравнение неразрывности Эйлера для несжимаемой жидкости в классической работе ``Principia motus fluidorum'' 1752 г. на латыни, выведенное при использовании линейного лагранжева закона движения жидкой частицы с учетом членов второго и третьего порядков малости по времени, содержащее квадратичный и кубичный инварианты тензора скоростей деформаций.  Его запись для сжимаемой среды, произведенная в 2006 г., показала генерацию волн давления и звука, генерируемого аналогичными членами высокого порядка малости волнового уравнения. Принцип построения математики, высказанный Гауссом в переписке с Шумахером, требует учета членов высокого порядка малости при решении задач механики, описывающихся дифференциальными волновыми уравнениями второго и третьего порядков по времени. Л.\,И. Седов в курсе ``Механика сплошной среды'' отмечал, что неучет квадратичного и кубичного инвариантов превращает уравнение неразрывности в приближенное равенство. 
\end{talk}
\end{document}
