\documentclass[12pt, a4paper, figuresright]{book}
\usepackage{mathtools, commath, amssymb, amsthm}
\usepackage{tabularx,graphicx,url,xcolor,rotating,multicol,epsfig}
\usepackage[T2A]{fontenc}
\usepackage[english,main=russian]{babel}

\setlength{\textheight}{25.2cm}
\setlength{\textwidth}{16.5cm}
\setlength{\voffset}{-1.6cm}
\setlength{\hoffset}{-0.3cm}
\setlength{\evensidemargin}{-0.3cm} 
\setlength{\oddsidemargin}{0.3cm}
\setlength{\parindent}{0cm} 
\setlength{\parskip}{0.3cm}

\newenvironment{talk}[6]{%
\vskip 0pt\nopagebreak%
\vskip 0pt\nopagebreak%
\textbf{#1}\vspace{3mm}\\\nopagebreak%
\textit{#2}\\\nopagebreak%
#3\\\nopagebreak%
\url{#4}\vspace{3mm}\\\nopagebreak%
\ifthenelse{\equal{#5}{}}{}{ #5\vspace{3mm}\\\nopagebreak}%
\ifthenelse{\equal{#6}{}}{}{Секция: #6\quad \vspace{3mm}\\\nopagebreak}%
}

\pagestyle{empty}

\begin{document}
\begin{talk}
{Семён Ефимович Белозёров: к 120-летию со дня рождения} %
{Пырков Вячеслав Евгеньевич} %
{Южный федеральный университет}%
{vepyrkov@sfedu.ru} %
{} %
{История математики} %

В докладе будут освещены уточненные и ранее неизвестные сведения о творческом пути и научном наследии С.Е. Белозёрова --- историка математики, ректора Ростовского университета (1938--1954). Эти сведения касаются его деятельности по созданию  кафедры Истории физико-математических наук и по руководству этой кафедрой, постановки курса ``История математики'', а также создания историко-математической школы в Ростовском университете. 

\medskip

\begin{enumerate}
\item[{[1]}] С.Е. Белозёров, {\it Математика в Ростовском университете}, Историко-математические исследования.  Вып.VI (1953), 247-352.
\item[{[2]}] С.Е. Белозёров, {\it Первые шаги в исследовательской работе по истории наук}, Ученые записки Ростовского-н/Д государственного университета им. В.М. Молотова.  Т.XXIV. Труды кафедры истории наук. Вып.1 (1955), 213-214.
\end{enumerate}
\end{talk}\end{document}