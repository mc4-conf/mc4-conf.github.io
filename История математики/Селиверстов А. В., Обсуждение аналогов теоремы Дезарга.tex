\documentclass[12pt, a4paper, figuresright]{book}
\usepackage{mathtools, commath, amssymb, amsthm}
\usepackage{tabularx,graphicx,url,xcolor,rotating,multicol,epsfig,colortbl,lipsum}
\usepackage[T2A]{fontenc}
\usepackage[english,main=russian]{babel}

\setlength{\textheight}{25.2cm}
\setlength{\textwidth}{16.5cm}
\setlength{\voffset}{-1.6cm}
\setlength{\hoffset}{-0.3cm}
\setlength{\evensidemargin}{-0.3cm} 
\setlength{\oddsidemargin}{0.3cm}
\setlength{\parindent}{0cm} 
\setlength{\parskip}{0.3cm}

\newenvironment{talk}[6]{%
\vskip 0pt\nopagebreak%
\vskip 0pt\nopagebreak%
\textbf{#1}\vspace{3mm}\\\nopagebreak%
\textit{#2}\\\nopagebreak%
#3\\\nopagebreak%
\url{#4}\vspace{3mm}\\\nopagebreak%
\ifthenelse{\equal{#5}{}}{}{Соавторы: #5\vspace{3mm}\\\nopagebreak}%
\ifthenelse{\equal{#6}{}}{}{Секция: #6\quad \vspace{3mm}\\\nopagebreak}%
}

\pagestyle{empty}

\begin{document}
\begin{talk}
{Обсуждение аналогов теоремы Дезарга} %
{Селиверстов Александр Владиславович} %
{Институт проблем передачи информации им. А.\,А. Харкевича РАН}%
{slvstv@iitp.ru} %
{Алексей А. Бойков (РТУ МИРЭА)} %
{История математики} %

Целью работы служит иллюстрация изменений в доказательствах теорем с развитием многомерной геометрии. Теорема Дезарга (Girard Desargues, 1591--1661) о перспективных треугольниках переносится на случай перспективных тетраэдров. Эту теорему о тетраэдрах впервые доказал Понселе (Jean-Victor Poncelet, 1788--1867). Теорема может быть доказана как в трёхмерном проективном пространстве, так и вовлекая многомерное проективное пространство. Ранние публикации упоминают только первое доказательство, а второе было найдено позже. По свидетельству Нины Васильевны Наумович, выход в пятимерное пространство использовал в 1913 г. Дмитрий Дмитриевич Мордухай-Болтовской (1876--1952). С другой стороны, в 1899 г. Гильберт (David Hilbert, 1862--1943) показал, что нельзя вывести теорему Дезарга из аксиом проективной плоскости. Астроном Моултон (Forest Ray Moulton, 1872--1952) упростил доказательство в 1902 г. Поэтому доказательство теоремы Дезарга о перспективных треугольниках на плоскости требует выхода в трёхмерное пространство. Начиная с работ Кэли (Arthur Cayley, 1821--1895) и Шлефли (Ludwig Schläfli, 1814--1895) в середине XIX века, многомерная геометрия быстро развивалась. Геометрический смысл алгебраических уравнений от многих переменных был осознан к 1844 г., прежде чем многомерная геометрия стала общепризнанной. Но даже в начале XX века доказательство теоремы о перспективных тетраэдрах, использующее выход в многомерное пространство, не было привлекательным из-за возможности провести доказательство в трёхмерном пространстве. Напротив, в середине XX века доказательство, вовлекающее многомерное пространство, стало восприниматься как естественное обобщение доказательства теоремы Дезарга. Многомерные пространства стали обычными. При этом снизился интерес к основаниям геометрии. Но расширение доступных методов позволяет поддерживать единство математики, чтобы видеть красоту взаимосвязей между разделами. 
\end{talk}
\end{document}

