\documentclass[12pt, a4paper, figuresright]{book}
\usepackage{mathtools, commath, amssymb, amsthm}
\usepackage{tabularx,graphicx,url,xcolor,rotating,multicol,epsfig,colortbl,lipsum}
\usepackage[T2A]{fontenc}
\usepackage[english,main=russian]{babel}

\setlength{\textheight}{25.2cm}
\setlength{\textwidth}{16.5cm}
\setlength{\voffset}{-1.6cm}
\setlength{\hoffset}{-0.3cm}
\setlength{\evensidemargin}{-0.3cm} 
\setlength{\oddsidemargin}{0.3cm}
\setlength{\parindent}{0cm} 
\setlength{\parskip}{0.3cm}

\newenvironment{talk}[6]{%
\vskip 0pt\nopagebreak%
\vskip 0pt\nopagebreak%
\textbf{#1}\vspace{3mm}\\\nopagebreak%
\textit{#2}\\\nopagebreak%
#3\\\nopagebreak%
\url{#4}\vspace{3mm}\\\nopagebreak%
\ifthenelse{\equal{#5}{}}{}{Соавторы: #5\vspace{3mm}\\\nopagebreak}%
\ifthenelse{\equal{#6}{}}{}{Секция: #6\quad \vspace{3mm}\\\nopagebreak}%
}

\pagestyle{empty}

\begin{document}
\begin{talk}{О тождестве Эйлера} % [1] название доклада
{Г.И. Синкевич} % [2] имя докладчика
{Санкт-Петербургский государственный архитектурно-строительный университет}% [3] аффилиация
{galina.sinkevich@gmail.com} % [4] адрес электронной почты (НЕОБЯЗАТЕЛЬНО)
{} % [5] соавторы (НЕОБЯЗАТЕЛЬНО)
{История математики} % [6] название секции

Об истории формул и тождества Эйлера много писали, но в ней остается немало противоречий и лакун. Еще когда Эйлер был
ребенком, равенство $\ln(\cos x+\sqrt{-1}\sin x)=x\sqrt{-1}$ было получено в словесной форме при расчете поверхности геоида Р. Коутсом
(R. Cotes) с помощью метода логарифмических пропорций. Некоторые комментаторы полагают, что у Коутса содержалась ошибка. Так
ли это?

В 1743--1748 гг. Эйлер определил показательную функцию через ряды синуса и косинуса и получил уравнение
$\cos\phi+\sqrt{-1}\sin\phi=e^{\sqrt{-1}\phi}$, а также выразил тригонометрические функции через экспоненту.
Но у Эйлера нет знаменитого тождества $e^{i\pi}=-1$,
или $ e^{i\pi}+1=0$. Нам удалось выяснить, когда и у кого оно появилось впервые. Мы покажем, как постепенно менялось мнение об
этой формуле как о незначительном частном случае и до признания ее красивейшей формулой математики.
\medskip
\begin{enumerate}
\item[{[1]}] Синкевич Г.И. История самой красивой формулы математики. Тождество Эйлера // История науки и техники, 2023, {\bf 3}.
-- с. 3--25.
\end{enumerate}
\end{talk}
\end{document}