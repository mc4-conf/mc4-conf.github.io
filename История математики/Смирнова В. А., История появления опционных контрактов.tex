\documentclass[12pt, a4paper, figuresright]{book}
\usepackage{mathtools, amssymb, amsthm}
\usepackage{tabularx,graphicx,url,xcolor,rotating,multicol,epsfig,colortbl,lipsum}
\usepackage[T2A]{fontenc}
\usepackage[english,main=russian]{babel}

\setlength{\textheight}{25.2cm}
\setlength{\textwidth}{16.5cm}
\setlength{\voffset}{-1.6cm}
\setlength{\hoffset}{-0.3cm}
\setlength{\evensidemargin}{-0.3cm} 
\setlength{\oddsidemargin}{0.3cm}
\setlength{\parindent}{0cm} 
\setlength{\parskip}{0.3cm}

\newenvironment{talk}[6]{%
\vskip 0pt\nopagebreak%
\vskip 0pt\nopagebreak%
\textbf{#1}\vspace{3mm}\\\nopagebreak%
\textit{#2}\\\nopagebreak%
#3\\\nopagebreak%
\url{#4}\vspace{3mm}\\\nopagebreak%
\ifthenelse{\equal{#5}{}}{}{Соавторы: #5\vspace{3mm}\\\nopagebreak}%
\ifthenelse{\equal{#6}{}}{}{Секция: #6\quad \vspace{3mm}\\\nopagebreak}%
}

\pagestyle{empty}

\begin{document}
\begin{talk}
{История появления опционных контрактов. Решение задачи о справедливой цене опциона со времен античности до наших времен} %
{Смирнова Вера Андреевна} %
{Санкт-Петербургский Государственный Электротехнический Университет}%
{vera-sm@yandex.ru} %
{} %
{История математики} %

\begin{enumerate}
\item Появление опционов (античные времена). Использование их математиком Фалесом.
\item Средние века. Опционные контракты для голландских торговцев тюльпанами.
\item Девятнадцатый век. Кризис торговли опционами. 
\item Двадцатый век. Создание фондовых бирж и плавающего валютного курса.

Задача о справедливой цене опциона превращается в корректно поставленную математическую.

\item Современное состояние задачи о вычислении справедливой цены опциона и ее связь с развитием теории случайных процессов.
\end{enumerate}
\end{talk}
\end{document}
