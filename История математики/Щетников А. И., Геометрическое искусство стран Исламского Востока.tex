\documentclass[12pt, a4paper, figuresright]{book}
\usepackage{mathtools, commath, amssymb, amsthm}
\usepackage{tabularx,graphicx,url,xcolor,rotating,multicol,epsfig}
\usepackage[T2A]{fontenc}
\usepackage[english,main=russian]{babel}

\setlength{\textheight}{25.2cm}
\setlength{\textwidth}{16.5cm}
\setlength{\voffset}{-1.6cm}
\setlength{\hoffset}{-0.3cm}
\setlength{\evensidemargin}{-0.3cm} 
\setlength{\oddsidemargin}{0.3cm}
\setlength{\parindent}{0cm} 
\setlength{\parskip}{0.3cm}

\newenvironment{talk}[6]{%
\vskip 0pt\nopagebreak%
\vskip 0pt\nopagebreak%
\textbf{#1}\vspace{3mm}\\\nopagebreak%
\textit{#2}\\\nopagebreak%
#3\\\nopagebreak%
\url{#4}\vspace{3mm}\\\nopagebreak%
\ifthenelse{\equal{#5}{}}{}{Соавторы: #5\vspace{3mm}\\\nopagebreak}%
\ifthenelse{\equal{#6}{}}{}{Секция: #6\quad \vspace{3mm}\\\nopagebreak}%
}

\pagestyle{empty}

\begin{document}
\begin{talk}
{Геометрическое искусство стран Исламского Востока} %
{Щетников Андрей Иванович} %
{ООО ``Новая школа'', Новосибирск}%
{a.schetnikov@gmail.com} %
{} %
{История математики} %

Исламский геометрический орнамент возникает в конце X века на территории государства Саманидов со столицей в Бухаре, а затем эта традиция развивается на обширной территории от Индии до Магриба, вплоть до наших дней. Будучи формой архитектурного декора, это искусство является в высокой степени математизированным, связанным с плоскими кристаллографическими группами и задачами замощения плоскости, с приближёнными построениями и вычислениями, а в современных публикациях прослеживается его связь с квазикристаллами и фрактальными структурами. Создание таких орнаментов невозможно без хорошего знания геометрии, и к нему приложили руку многие математики средневекового Востока, среди которых есть такие значимые для истории нашей науки фигуры, как Омар Хайям.
\end{talk}
\end{document}


