\documentclass[12pt, a4paper, figuresright]{book}
\usepackage{mathtools, commath, amssymb, amsthm}
\usepackage{tabularx,graphicx,url,xcolor,rotating,multicol,epsfig,colortbl,lipsum}
\usepackage[T2A]{fontenc}
\usepackage[english,main=russian]{babel}

\setlength{\textheight}{25.2cm}
\setlength{\textwidth}{16.5cm}
\setlength{\voffset}{-1.6cm}
\setlength{\hoffset}{-0.3cm}
\setlength{\evensidemargin}{-0.3cm} 
\setlength{\oddsidemargin}{0.3cm}
\setlength{\parindent}{0cm} 
\setlength{\parskip}{0.3cm}

\newenvironment{talk}[6]{%
	\vskip 0pt\nopagebreak%
	\vskip 0pt\nopagebreak%
	\textbf{#1}\vspace{3mm}\\\nopagebreak%
	\textit{#2}\\\nopagebreak%
	#3\\\nopagebreak%
	\url{#4}\vspace{3mm}\\\nopagebreak%
	\ifthenelse{\equal{#5}{}}{}{Соавторы: #5\vspace{3mm}\\\nopagebreak}%
	\ifthenelse{\equal{#6}{}}{}{Секция: #6\quad \vspace{3mm}\\\nopagebreak}%
}

\pagestyle{empty}

\begin{document}
\begin{talk}
{Элементы теории поля в работах академика О.\,И. Сомова} %
{Юлина Анна Олеговна} %
{Санкт-Петербургский государственный архитектурно-строительный университет}%
{parfenova19761976@mail.ru} %
{} %
{История математики} %

Петербургский математик и механик О.\,И. Сомов первый в России использовал аппарат векторного исчисления в  курсе теоретической механики.  Ему принадлежит
введение математического понятия градиента, годографа, векторного произведения, линии и поверхности уровня, потенциала и их геометрического и векторного смысла. Все вопросы механики Сомов рассматривает в тесной взаимосвязи с математикой. В его фундаментальных работах блестяще показано как математический анализ помогает раскрывать законы движения и действия сил природы с одной стороны, а с другой как механика помогает развитию аналитических и геометрических методов исследования.
Однако же работы академика Сомова незаслуженно забыты. Постараемся восполнить этот пробел в данном докладе.

\medskip

\begin{enumerate}
\item[{[1]}] Сомов О.И. Рациональная механика. Кинематика. С-Петербург. Типография Императорской Академии Наук. 1872г.- 491 с.
\item[{[2]}] Сомов О.И. О решении одного вопроса механики, предложенного Абелем // Записки Императорской Академии наук. Санкт-Петербург: Типография Императорской Академии Наук. Т.9, кн.1. т. 1866г. Раздельная пагинация.
491 с.
\item[{[3]}] А. О. Юлина.  Векторное исчисление в механике Сомова. // История науки и техники. 2023. №3. с. 26-33.	
\end{enumerate}
\end{talk}
\end{document}