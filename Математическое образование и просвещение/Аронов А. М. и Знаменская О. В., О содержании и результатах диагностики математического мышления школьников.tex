\documentclass[12pt, a4paper, figuresright]{book}
\usepackage{mathtools, commath, amssymb, amsthm}
\usepackage{tabularx,graphicx,url,xcolor,rotating,multicol,epsfig,colortbl,lipsum}
\usepackage[T2A]{fontenc}
\usepackage[english,main=russian]{babel}

\setlength{\textheight}{25.2cm}
\setlength{\textwidth}{16.5cm}
\setlength{\voffset}{-1.6cm}
\setlength{\hoffset}{-0.3cm}
\setlength{\evensidemargin}{-0.3cm} 
\setlength{\oddsidemargin}{0.3cm}
\setlength{\parindent}{0cm} 
\setlength{\parskip}{0.3cm}

\newenvironment{talk}[6]{%
\vskip 0pt\nopagebreak%
\vskip 0pt\nopagebreak%
\textbf{#1}\vspace{3mm}\\\nopagebreak%
\textit{#2}\\\nopagebreak%
#3\\\nopagebreak%
\url{#4}\vspace{3mm}\\\nopagebreak%
\ifthenelse{\equal{#5}{}}{}{Соавторы: #5\vspace{3mm}\\\nopagebreak}%
\ifthenelse{\equal{#6}{}}{}{Секция: #6\quad \vspace{3mm}\\\nopagebreak}%
}

\pagestyle{empty}

\begin{document}
\begin{talk}
{О содержании и результатах диагностики математического мышления школьников на переходе из начальных классов в основную школу}
{Аронов Александр Моисеевич и Знаменская Оксана Витальевна}
{Московский городской педагогический университет, Сибирский федеральный университет}
{aronovam@mgpu.ru; ovznamenskaya@sfu-kras.ru}
{}
{Математическое образование и просвещение}

Рассматривается подход к математическому мышлению как системе средств организации математической деятельности. В основе этого подхода лежит трехуровневая модель ``Дельта'', позволяющая диагносцировать понимание и рефлексию при решении школьниками математических задач. Будут представлены результаты диагностики школьников 4–6  классов и дан анализ этих результатов с точки зрения выявления мыслительных средств, необходимых для освоения математических способов решения. Указаны риски на переходе из начальных классов в основную школу, связанные с дефицитами мышления учеников в понятиях начальной школы. Полученные данные диагностики позволяют проектировать дифференцированное обучение, применять методы индивидуализации и персонализации.
\end{talk}
\end{document}