\documentclass[12pt, a4paper, figuresright]{book}
\usepackage{mathtools, commath, amssymb, amsthm}
\usepackage{tabularx,graphicx,url,xcolor,rotating,multicol,epsfig,colortbl,lipsum}
\usepackage[T2A]{fontenc}
\usepackage[english,main=russian]{babel}

\setlength{\textheight}{25.2cm}
\setlength{\textwidth}{16.5cm}
\setlength{\voffset}{-1.6cm}
\setlength{\hoffset}{-0.3cm}
\setlength{\evensidemargin}{-0.3cm} 
\setlength{\oddsidemargin}{0.3cm}
\setlength{\parindent}{0cm} 
\setlength{\parskip}{0.3cm}

\newenvironment{talk}[6]{%
\vskip 0pt\nopagebreak%
\vskip 0pt\nopagebreak%
\textbf{#1}\vspace{3mm}\\\nopagebreak%
\textit{#2}\\\nopagebreak%
#3\\\nopagebreak%
\url{#4}\vspace{3mm}\\\nopagebreak%
\ifthenelse{\equal{#5}{}}{}{Соавторы: #5\vspace{3mm}\\\nopagebreak}%
\ifthenelse{\equal{#6}{}}{}{Секция: #6\quad \vspace{3mm}\\\nopagebreak}%
}

\pagestyle{empty}

\begin{document}
\begin{talk}
{О развитии навыков коммуникации в области математики у обучающихся старших классов на занятии в игровом формате}
{Бегунц Александр Владимирович}
{МЦМУ ``Московский центр фундаментальной и прикладной математики''}
{alexander.begunts@math.msu.ru}
{}
{Математическое образование и просвещение}

Разработана методика проведения занятий с группой обучающихся старших классов, позволяющая за счет привлечения игрового формата развить у участников навыки коммуникации в области математики.

``Внутренний диалог'', являясь одной из основных форм мышления, в том числе математического, формируется посредством интериоризации коммуникативных форм межсубъектного взаимодействия. Известно, что совместное обсуждение и поиск  решений математических задач являются мощным средством развития интеллектуальных способностей старшеклассников и содействуют повышению их интереса к изучению предмета (см., например, [1]). В то же время зачастую организованная коммуникация в области математики обучающихся в школе практически отсутствует, а стихийная сводится к бездумному переписыванию у одноклассников заданных на дом заданий.

Помимо регулярных и системных мероприятий, направленных на задействование коммуникативного ресурса в течение учебного года, сотрудниками лаборатории ``Современные образовательные технологии в математике'' МЦМУ ``Московский центр фундаментальной и прикладной математики'' разработаны и апробированы в рамках внеурочных мероприятий (см. [2]) игровые форматы проведения учебных занятий, содействующие развитию навыков коммуникации школьников в области математики. 

Представим один из этих форматов. Школьники разбиваются на команды так, чтобы в каждой команде были представители нескольких последовательных классов (например, 8, 9 и 10). Всем командам выдаются одинаковые комплекты заданий, после чего команды одновременно приступают к их выполнению. Жюри принимает выполненное задание устно от одного из членов команды. Этот участник  имеет право взять с собой и опираться на написанный им текст решения задачи и обязан давать устные пояснения по требованию члена жюри. Система начисления баллов организована так, что команде выгоднее, чтобы решали задачу старшеклассники (это увеличивает количество решенных задач), а рассказывал ее решение жюри младшеклассник (это увеличивает количество баллов за каждую задачу). В результате возникает ситуация необходимости коммуникации между учащимися, не просто транслирующей текст решения, а обеспечивающей понимание самого решения. 

Об этом формате, методике его применения и результатах апробации  и пойдет речь в докладе.

\medskip

\begin{enumerate}
\item[{[1]}] Зильберберг Н.\,И. Приобщение к математическому творчеству. Уфа: Башкирское книжное издательство, 1988. 96 с.
\item[{[2]}] Бегунц А.\,В., Гашков С.\,Б., Татаринова Е.\,Я. Зимние и летние школы классов при мехмате МГУ // Потенциал. Математика. Физика. Информатика. 2023. \textnumero\,1. С.\,17--24.
\end{enumerate}
\end{talk}
\end{document}