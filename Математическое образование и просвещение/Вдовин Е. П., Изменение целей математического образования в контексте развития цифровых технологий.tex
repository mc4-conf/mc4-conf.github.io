\documentclass[12pt, a4paper, figuresright]{book}
\usepackage{mathtools, commath, amssymb, amsthm}
\usepackage{tabularx,graphicx,url,xcolor,rotating,multicol,epsfig,colortbl,lipsum}
\usepackage[T2A]{fontenc}
\usepackage[english,main=russian]{babel}

\setlength{\textheight}{25.2cm}
\setlength{\textwidth}{16.5cm}
\setlength{\voffset}{-1.6cm}
\setlength{\hoffset}{-0.3cm}
\setlength{\evensidemargin}{-0.3cm} 
\setlength{\oddsidemargin}{0.3cm}
\setlength{\parindent}{0cm} 
\setlength{\parskip}{0.3cm}

\newenvironment{talk}[6]{%
\vskip 0pt\nopagebreak%
\vskip 0pt\nopagebreak%
\textbf{#1}\vspace{3mm}\\\nopagebreak%
\textit{#2}\\\nopagebreak%
#3\\\nopagebreak%
\url{#4}\vspace{3mm}\\\nopagebreak%
\ifthenelse{\equal{#5}{}}{}{Соавторы: #5\vspace{3mm}\\\nopagebreak}%
\ifthenelse{\equal{#6}{}}{}{Секция: #6\quad \vspace{3mm}\\\nopagebreak}%
}

\pagestyle{empty}

\begin{document}
\begin{talk}
{Изменение целей математического образования в контексте развития цифровых технологий}
{Вдовин Евгений Петрович}
{Тюменский государственный университет}
{e.p.vdovin@utmn.ru}
{Математическое образование и просвещение}

В предлагаемом докладе мы различим на принципиальном уровне несколько тактов действий человека в ситуации достижения какой-либо цели (примеры таких различений приведены в [1] и [2]). На основании сформированного различения мы покажем, как выглядят текущие цели массового математического образования, сформулируем тезис о том, что в сложившейся технологической ситуации все эти цели сейчас закрывают цифровые технологии. После этого будут сформулированы те цели, которые более соответствуют сложившейся технологической ситуации. Завершим доклад примерами из практики автора и его команды по перестройке математического образования в соответствии с новыми целями.

\medskip

\begin{enumerate}
\item[{[1]}] Werner Blum, Rita Borromeo Ferri, {\it Mathematical Modelling: Can It Be Taught and Learnt?}, Journal of Mathematical Modelling and Application, 1 (2009), No. 1, 45--58.
\item[{[2]}] А.В. Боровских, {\it О понятии математической грамотности}, Бедагогика, 86 (2022), 3б 33--45.
\end{enumerate}
\end{talk}
\end{document}
