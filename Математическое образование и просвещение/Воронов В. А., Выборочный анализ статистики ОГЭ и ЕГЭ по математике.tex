\documentclass[12pt, a4paper, figuresright]{book}
\usepackage{mathtools, commath, amssymb, amsthm}
\usepackage{tabularx,graphicx,url,xcolor,rotating,multicol,epsfig,colortbl,lipsum}
\usepackage[T2A]{fontenc}
\usepackage[english,main=russian]{babel}

\setlength{\textheight}{25.2cm}
\setlength{\textwidth}{16.5cm}
\setlength{\voffset}{-1.6cm}
\setlength{\hoffset}{-0.3cm}
\setlength{\evensidemargin}{-0.3cm} 
\setlength{\oddsidemargin}{0.3cm}
\setlength{\parindent}{0cm} 
\setlength{\parskip}{0.3cm}

\newenvironment{talk}[6]{%
\vskip 0pt\nopagebreak%
\vskip 0pt\nopagebreak%
\textbf{#1}\vspace{3mm}\\\nopagebreak%
\textit{#2}\\\nopagebreak%
#3\\\nopagebreak%
\url{#4}\vspace{3mm}\\\nopagebreak%
\ifthenelse{\equal{#5}{}}{}{Соавторы: #5\vspace{3mm}\\\nopagebreak}%
\ifthenelse{\equal{#6}{}}{}{Секция: #6\quad \vspace{3mm}\\\nopagebreak}%
}

\pagestyle{empty}

\begin{document}
\begin{talk}
{Выборочный анализ статистики ОГЭ и ЕГЭ по математике}
{Воронов Всеволод Александрович}
{Кавказский математический центр Адыгейского государственного университета, Московский физико-технический институт (национальный исследовательский университет)}
{v-vor@yandex.ru}
{}
{Математическое образование и просвещение}

Статистические данные, которые обрабатывают Федеральный институт педагогических измерений и Рособрнадзор по результатам государственной итоговой аттестации в 9 и 11 классах школы, к сожалению, лишь частично являются открытыми. Далеко не все регионы публикуют статистические отчеты, и не всегда статистический отчет по единой форме доступен для страны в целом. Деперсонифицированная база результатов ОГЭ/ЕГЭ доступна лишь на информационных ресурсах отдельных регионов. Анализ статистики затрудняется, кроме того, запретом на публикацию контрольно-измерительных материалов, не имеющим срока давности. Тем не менее на основе неполных данных можно сделать ряд любопытных наблюдений.

\begin{enumerate}
\item Для Сибири и Дальнего Востока типичны сравнительно низкие результаты по профильной математике в сравнении с регионами Европейской части. Это не всегда может быть объяснено различным уровнем социально-экономического развития регионов.
\item Из тех регионов, для которых статистика доступна, наилучшие показатели по профильной математике имеет Татарстан.
\item В 2021-м году средний балл ЕГЭ по профильной математике в Москве был ниже, чем в среднем по России.
\item В нескольких регионах на графике распределения баллов ОГЭ по математике наблюдается резкое падение числа участников при переходе от 19 первичных баллов к 20 (граница первой части ОГЭ).
\item В статистических отчетах редко указывают процент двоек, полученных в основной период (без учета пересдач). 
\end{enumerate}
\end{talk}
\end{document}
