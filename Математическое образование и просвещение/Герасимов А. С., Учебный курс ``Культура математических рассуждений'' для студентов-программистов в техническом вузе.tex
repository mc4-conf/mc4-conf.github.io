\documentclass[12pt, a4paper, figuresright]{book}
\usepackage{mathtools, commath, amssymb, amsthm}
\usepackage{tabularx,graphicx,url,xcolor,rotating,multicol,epsfig,colortbl,lipsum}
\usepackage[T2A]{fontenc}
\usepackage[english,main=russian]{babel}

\setlength{\textheight}{25.2cm}
\setlength{\textwidth}{16.5cm}
\setlength{\voffset}{-1.6cm}
\setlength{\hoffset}{-0.3cm}
\setlength{\evensidemargin}{-0.3cm} 
\setlength{\oddsidemargin}{0.3cm}
\setlength{\parindent}{0cm} 
\setlength{\parskip}{0.3cm}

\newenvironment{talk}[6]{%
\vskip 0pt\nopagebreak%
\vskip 0pt\nopagebreak%
\textbf{#1}\vspace{3mm}\\\nopagebreak%
\textit{#2}\\\nopagebreak%
#3\\\nopagebreak%
\url{#4}\vspace{3mm}\\\nopagebreak%
\ifthenelse{\equal{#5}{}}{}{Соавторы: #5\vspace{3mm}\\\nopagebreak}%
\ifthenelse{\equal{#6}{}}{}{Секция: #6\quad \vspace{3mm}\\\nopagebreak}%
}

\pagestyle{empty}

\begin{document}
\begin{talk}
{Учебный курс ``Культура математических рассуждений''
для студентов-программистов в техническом вузе}
{Герасимов Александр Сергеевич}
{Санкт-Петербургский политехнический университет Петра Великого}
{alexander.s.gerasimov@ya.ru}
{}
{Математическое образование и просвещение}

Представляется учебный курс ``Культура математических рассуждений'', 
введенный автором доклада для студентов, обучающихся по направлению 
подготовки ``Фундаментальная информатика и информационные технологи'' 
в Санкт-Петербургском политехническом университете.
Целью этого курса является систематическое освоение базовых приемов 
математических рассуждений, что нужно по меньшей мере для доказательства
корректности алгоримов, которые изучаются в последующем курсе 
``Алгоритмы и анализ сложности'', также читаемом автором доклада.
На курс ``Культура математических рассуждений'' отведено 30 академических часов
практических занятий, включающих в себя элементы лекции.
Большая часть этого курса посвящена изучению исчисления натуральных выводов 
в стиле С. Яськовского, построению формальных доказательств (или выводов) 
в этом исчислении и построению неформальных (или содержательных) доказательств, 
близких по структуре к формальным.
Также в данном курсе систематизируются базовые понятия и факты теории множеств;
изучаются метод возвратной индукции (применяемый, в частности, 
для доказательства корректности рекурсивных алгоритмов) 
и метод инвариантов циклов для доказательства корректности алгоритмов,содержащих циклы.
\end{talk}
\end{document}