\documentclass[12pt, a4paper, figuresright]{book}
\usepackage{mathtools, commath, amssymb, amsthm}
\usepackage{tabularx,graphicx,url,xcolor,rotating,multicol,epsfig,colortbl,lipsum}
\usepackage[T2A]{fontenc}
\usepackage[english,main=russian]{babel}

\setlength{\textheight}{25.2cm}
\setlength{\textwidth}{16.5cm}
\setlength{\voffset}{-1.6cm}
\setlength{\hoffset}{-0.3cm}
\setlength{\evensidemargin}{-0.3cm} 
\setlength{\oddsidemargin}{0.3cm}
\setlength{\parindent}{0cm} 
\setlength{\parskip}{0.3cm}

\newenvironment{talk}[6]{%
	\vskip 0pt\nopagebreak%
	\vskip 0pt\nopagebreak%
	\textbf{#1}\vspace{3mm}\\\nopagebreak%
	\textit{#2}\\\nopagebreak%
	#3\\\nopagebreak%
	\url{#4}\vspace{3mm}\\\nopagebreak%
	\ifthenelse{\equal{#5}{}}{}{Соавторы: #5\vspace{3mm}\\\nopagebreak}%
	\ifthenelse{\equal{#6}{}}{}{Секция: #6\quad \vspace{3mm}\\\nopagebreak}%
}

\pagestyle{empty}

\begin{document}
\begin{talk}
{Использование балльно-рейтинговой системы для геймификации учебного процесса} 
{Даровская Ксения Александровна}
{Первый МГМУ имени И.\,М. Сеченова, Российский университет дружбы народов}
{k.darovsk@gmail.com}
{}
{Математическое образование и просвещение}

Геймификацию можно определить как ``использование игровых элементов в неигровых ситуациях'' (К. Вербах).  Основной целью применения игровых технологий в образовании мы будем считать рост вовлеченности обучающихся в учебную деятельность. 

Балльно-рейтинговую систему (БРС), часто сопряженную ECTS (European Credit Transfer and Accumulation System), можно рассматривать как инструмент формализации отметки.
В настоящем докладе планируется обсудить приемы использования БРС, позволяющие геймифицировать процесс обучения математике в вузе и школе.
\end{talk}
\end{document}