\documentclass[12pt, a4paper, figuresright]{book}
\usepackage{mathtools, commath, amssymb, amsthm}
\usepackage{tabularx,graphicx,url,xcolor,rotating,multicol,epsfig,colortbl,lipsum}
\usepackage[T2A]{fontenc}
\usepackage[english,main=russian]{babel}

\setlength{\textheight}{25.2cm}
\setlength{\textwidth}{16.5cm}
\setlength{\voffset}{-1.6cm}
\setlength{\hoffset}{-0.3cm}
\setlength{\evensidemargin}{-0.3cm} 
\setlength{\oddsidemargin}{0.3cm}
\setlength{\parindent}{0cm} 
\setlength{\parskip}{0.3cm}

\newenvironment{talk}[6]{%
\vskip 0pt\nopagebreak%
\vskip 0pt\nopagebreak%
\textbf{#1}\vspace{3mm}\\\nopagebreak%
\textit{#2}\\\nopagebreak%
#3\\\nopagebreak%
\url{#4}\vspace{3mm}\\\nopagebreak%
\ifthenelse{\equal{#5}{}}{}{Соавторы: #5\vspace{3mm}\\\nopagebreak}%
\ifthenelse{\equal{#6}{}}{}{Секция: #6\quad \vspace{3mm}\\\nopagebreak}%
}

\pagestyle{empty}

\begin{document}
\begin{talk}
{Что данные национальных оценочных процедур говорят о качестве математического образования в школе}
{Денисенко Илья Сергеевич}
{ФГБУ Федеральный институт оценки качества образования}
{ilya.denisenko@gmail.com}
{} 
{Математическое образование и просвещение}

Мероприятия по оценке качества образования (МОКО) позволяют получать данные о качестве математической подготовки в российских школах. Анализ данных способствует выявлению ограничений, с которыми российские школьники из разных образовательных организаций сталкиваются при изучении математики; позволяет определить факторы, способствующие развитию математических навыков, оценивать влияние углубленного изучения математики на уровень обученности, находить практики преодоления контекстных вызовов, и изучать связь результатов освоения школьной программы по математике с дальнейшей образовательной траекторией обучающихся. Изучение получаемых от МОКО данных позволяет формулировать и уточнять запрос на развитие математического просвещения в общеобразовательной школе.
\end{talk}
\end{document}
