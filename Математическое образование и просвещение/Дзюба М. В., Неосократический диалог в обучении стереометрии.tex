\documentclass[12pt, a4paper, figuresright]{book}
\usepackage{mathtools, commath, amssymb, amsthm}
\usepackage{tabularx,graphicx,url,xcolor,rotating,multicol,epsfig,colortbl,lipsum}
\usepackage[T2A]{fontenc}
\usepackage[english,main=russian]{babel}

\setlength{\textheight}{25.2cm}
\setlength{\textwidth}{16.5cm}
\setlength{\voffset}{-1.6cm}
\setlength{\hoffset}{-0.3cm}
\setlength{\evensidemargin}{-0.3cm} 
\setlength{\oddsidemargin}{0.3cm}
\setlength{\parindent}{0cm} 
\setlength{\parskip}{0.3cm}

\newenvironment{talk}[6]{%
\vskip 0pt\nopagebreak%
\vskip 0pt\nopagebreak%
\textbf{#1}\vspace{3mm}\\\nopagebreak%
\textit{#2}\\\nopagebreak%
#3\\\nopagebreak%
\url{#4}\vspace{3mm}\\\nopagebreak%
\ifthenelse{\equal{#5}{}}{}{Соавторы: #5\vspace{3mm}\\\nopagebreak}%
\ifthenelse{\equal{#6}{}}{}{Секция: #6\quad \vspace{3mm}\\\nopagebreak}%
}

\pagestyle{empty}

\begin{document}
\begin{talk}
{Неосократический диалог в обучении стереометрии}
{Дзюба Марина Витальевна}
{Российский государственный педагогический университет им. А.\,И. Герцена}
{marinafrolova25@mail.ru}
{}
{Математическое образование и просвещение}

Во время доклада будут описаны основные принципы метода неосократического диалога при обучении стереометрии. Отдельное внимание будет уделено отличиям этого метода, от известного проблемного обучения в педагогике. Особый акцент будет сделан на проблемно-диалогической форме работы с учащимися, разработанной в трудах Е.\,Л. Мельниковой. 

Центральным вопросом станет описание методики применение метода неосократического диалога на заключительном этапе решения стереометрических задач с использованием  опорных конструкций и метода варьирования задач, который поможет создать живое исследование на уроках геометрии. 
\end{talk}
\end{document}
