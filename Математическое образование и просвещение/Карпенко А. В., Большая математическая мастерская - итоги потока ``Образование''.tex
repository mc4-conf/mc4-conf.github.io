\documentclass[12pt, a4paper, figuresright]{book}
\usepackage{mathtools, commath, amssymb, amsthm}
\usepackage{tabularx,graphicx,url,xcolor,rotating,multicol,epsfig,colortbl,lipsum}
\usepackage[T2A]{fontenc}
\usepackage[english,main=russian]{babel}

\setlength{\textheight}{25.2cm}
\setlength{\textwidth}{16.5cm}
\setlength{\voffset}{-1.6cm}
\setlength{\hoffset}{-0.3cm}
\setlength{\evensidemargin}{-0.3cm} 
\setlength{\oddsidemargin}{0.3cm}
\setlength{\parindent}{0cm} 
\setlength{\parskip}{0.3cm}

\newenvironment{talk}[6]{%
\vskip 0pt\nopagebreak%
\vskip 0pt\nopagebreak%
\textbf{#1}\vspace{3mm}\\\nopagebreak%
\textit{#2}\\\nopagebreak%
#3\\\nopagebreak%
\url{#4}\vspace{3mm}\\\nopagebreak%
\ifthenelse{\equal{#5}{}}{}{Соавторы: #5\vspace{3mm}\\\nopagebreak}%
\ifthenelse{\equal{#6}{}}{}{Секция: #6\quad \vspace{3mm}\\\nopagebreak}%
}

\pagestyle{empty}

\begin{document}
\begin{talk}
{Большая математическая мастерская: итоги потока ``Образование''}
{Карпенко Анастасия Валерьевна}
{Новосибирский государственный университет, НГУ}
{anastasia.v.karpenko@gmail.com}
{Абдыкеров Жанат Сергеевич}
{Математическое образование и просвещение}

Большая математическая мастерская (БММ) --- научно-образовательное мероприятие, в рамках которого команды школьников, студентов и педагогов при сопровождении кураторов в интенсивном формате работают над решением реальных задач, имеющих математическую составляющую. 

БММ реализуется с 2020 года. В 2024 году площадками для проведения Мастерской выступили: Математический центр в Академгородке, Омский филиал Института математики им. С.Л. Соболева СО РАН, Региональный научно-образовательный математический центр Томского государственного университета, Региональный научно-образова-тельный математический центр Адыгейского государственного университета — ``Кавказский математический центр'' и Школа компьютернаых наук Тюменского государственного университета.

С момента создания БММ приросла не только количественно, но и качественно: в 2021 появился школьный поток, а с 2022 года Мастерская набирает проекты в области образования. Реализация в параллели потоков для школьников и педагогов позволяет педагогам за время Мастерской планировать и проводить полноценные эксперименты, а также обобщать полученный опыт. 

Так, например, команда проекта <<Исследовательские блуждания по стереометрической задаче>> в рамках первого модуля проработала концепт применения метода работы с задачной ситуацией, дважды аппробировали его на школьниках, а затем обобщила результаты в виде заметки об использовании метода при решении стереометрических задач.

Другим примером результата проекта является полноценная потенциально коммерциализуемая игра ``Геоформы'', геймифицирующая изучение школьного курса геометрии 7-9 классов. Игра разработана помандой проекта ``Использование игровых инструментов в преподавании математики'', аппробирована на БММ и представлена на Августовской конференции в г. Томске в 2023 году.  

В рамках доклада будет представлен концепт реализации потока ``Образование'' на Большой математической мастерской, а также результаты, которых удалось добиться командам проектов.
\end{talk}
\end{document}
