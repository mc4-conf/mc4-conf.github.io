\documentclass[12pt, a4paper, figuresright]{book}
\usepackage{mathtools, commath, amssymb, amsthm}
\usepackage{tabularx,graphicx,url,xcolor,rotating,multicol,epsfig,colortbl,lipsum}
\usepackage[T2A]{fontenc}
\usepackage[english,main=russian]{babel}

\setlength{\textheight}{25.2cm}
\setlength{\textwidth}{16.5cm}
\setlength{\voffset}{-1.6cm}
\setlength{\hoffset}{-0.3cm}
\setlength{\evensidemargin}{-0.3cm} 
\setlength{\oddsidemargin}{0.3cm}
\setlength{\parindent}{0cm} 
\setlength{\parskip}{0.3cm}

\newenvironment{talk}[6]{%
	\vskip 0pt\nopagebreak%
	\vskip 0pt\nopagebreak%
	\textbf{#1}\vspace{3mm}\\\nopagebreak%
	\textit{#2}\\\nopagebreak%
	#3\\\nopagebreak%
	\url{#4}\vspace{3mm}\\\nopagebreak%
	\ifthenelse{\equal{#5}{}}{}{Соавторы: #5\vspace{3mm}\\\nopagebreak}%
	\ifthenelse{\equal{#6}{}}{}{Секция: #6\quad \vspace{3mm}\\\nopagebreak}%
}

\pagestyle{empty}

\begin{document}
\begin{talk}
{Вариативный виток спиральной модели как средство развития математического мышления школьников}
{Картвелишвили Татьяна Александровна}
{МГУ им. М.\,В.Ломоносова}
{tgs497@gmail.com}
{Сергеев Игорь Николаевич}
{Математическое образование и просвещение}

В последнее время все большее развитие получают учебные программы по математике, построенные на основе дидактической спирали. Как известно, спиральная модель включает в себя семь основных витков. Но особый интерес представляет собой финальный --- вариативный виток. Именно на нем происходит настоящее развитие и формирование математического мышления школьников. Переходя на вариативный виток, обучающийся сталкивается с нетривиальной для него задачей --- выбором наилучшего подхода к решению
сложных и нестандартных задач. И именно тут они окончательно уходят от действий по ``алгоритму'', испытывая на себе акт дифференциации и получая средства для решения необходимых математических проблем.

Как правило, рассматриваются четыре основных подхода: алгебраический, логический, функциональный и графический, но если подумать, то можно добавить еще и арифметический, комбинаторный, вероятностный и топологический подходы. В данном докладе, помимо анализа всего вариантивного витка, мы особо остановимся на логическом подходе и его преиуществах. Ведь, согласно Пиаже, именно логика является единственным и главным критерием мышления, таким образом, развитие логики и математического мышления неотделимо связаны друг с другом. 
\end{talk}
\end{document}