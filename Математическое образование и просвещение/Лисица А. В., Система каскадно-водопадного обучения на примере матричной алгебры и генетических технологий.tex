\documentclass[12pt, a4paper, figuresright]{book}
\usepackage{mathtools, commath, amssymb, amsthm}
\usepackage{tabularx,graphicx,url,xcolor,rotating,multicol,epsfig,colortbl,lipsum}
\usepackage[T2A]{fontenc}
\usepackage[english,main=russian]{babel}

\setlength{\textheight}{25.2cm}
\setlength{\textwidth}{16.5cm}
\setlength{\voffset}{-1.6cm}
\setlength{\hoffset}{-0.3cm}
\setlength{\evensidemargin}{-0.3cm} 
\setlength{\oddsidemargin}{0.3cm}
\setlength{\parindent}{0cm} 
\setlength{\parskip}{0.3cm}

\newenvironment{talk}[6]{%
\vskip 0pt\nopagebreak%
\vskip 0pt\nopagebreak%
\textbf{#1}\vspace{3mm}\\\nopagebreak%
\textit{#2}\\\nopagebreak%
#3\\\nopagebreak%
\url{#4}\vspace{3mm}\\\nopagebreak%
\ifthenelse{\equal{#5}{}}{}{Соавторы: #5\vspace{3mm}\\\nopagebreak}%
\ifthenelse{\equal{#6}{}}{}{Секция: #6\quad \vspace{3mm}\\\nopagebreak}%
}

\pagestyle{empty}

\begin{document}
\begin{talk}
{Система каскадно-водопадного обучения на примере матричной алгебры и генетических технологий}
{Лисица Андрей Валерьевич}
{ФГАОУ ВО ``Тюменский государственный университет''}
{lisitsa052@gmail.com}
{Андреюк Денис Сергеевич, Российская ассоциация содействия науке (РАСН); Козлова Анна Сергеена, ФГБНУ ``Научно-исследовательский институт биомедицинской химии имени В.\,Н. Ореховича'' (ИБМХ)}
{Математическое образование и просвещение}

Рассматриваются результаты применения системы ``Таблекс'' (разработчик --- ООО ``КуБ'') на базе Центра научно-практического образования ИБМХ с 2021 по 2024 г. Система [1] предоставляет возможность организации научных кружков для школьников и студентов. Проведены следующие курсы: генетические технологии (сборка геномов), анализ широкомасштабных протеомных и метаболомных данных, 3D моделирование белков, с применением облачных технологий Яндекс-Клауд. Математическое мышление формируется с использованием стандартных модулей Питона — а именно гистограмм, диаграмм Венна, методов распознавания образов, анализ главных компонент, дисперсионный анализ. 

Особенностью методологии преподавания в системе ``Таблекс'' является парное программирование, где есть роли пилота, штурмана и инструктора. Пилоты и штурманы, достигая уровня инструктора, вовлекаются в систему монетизации и школьник может заработать до 500 рублей в день. Результатом такого подхода являются команды, которые обновляют кадровый состав при реализации федеральных программ, связанных с генетическими технологиями и применением искусственного интеллекта. Общая логика построения кадрового резерва предложена в рамках системы ``Кванториум''.

\medskip

\begin{enumerate}
\item[{[1]}] http://oookub.ru/tablex-main.html ``Таблекс - каскадно-водопадная система обучения'' св. о рег. программы для ЭВМ №2022685715 от 27.12.2022.
\end{enumerate}
\end{talk}
\end{document}
