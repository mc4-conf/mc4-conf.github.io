\documentclass[12pt, a4paper, figuresright]{book}
\usepackage{mathtools, commath, amssymb, amsthm}
\usepackage{tabularx,graphicx,url,xcolor,rotating,multicol,epsfig,colortbl,lipsum}
\usepackage[T2A]{fontenc}
\usepackage[english,main=russian]{babel}

\setlength{\textheight}{25.2cm}
\setlength{\textwidth}{16.5cm}
\setlength{\voffset}{-1.6cm}
\setlength{\hoffset}{-0.3cm}
\setlength{\evensidemargin}{-0.3cm} 
\setlength{\oddsidemargin}{0.3cm}
\setlength{\parindent}{0cm} 
\setlength{\parskip}{0.3cm}

\newenvironment{talk}[6]{%
\vskip 0pt\nopagebreak%
\vskip 0pt\nopagebreak%
\textbf{#1}\vspace{3mm}\\\nopagebreak%
\textit{#2}\\\nopagebreak%
#3\\\nopagebreak%
\url{#4}\vspace{3mm}\\\nopagebreak%
\ifthenelse{\equal{#5}{}}{}{Соавторы: #5\vspace{3mm}\\\nopagebreak}%
\ifthenelse{\equal{#6}{}}{}{Секция: #6\quad \vspace{3mm}\\\nopagebreak}%
}

\pagestyle{empty}

\begin{document}
\begin{talk}
{О развитии математического мышления преподавателей Малого мехмата}
{Мелешкина Анна Владимировна}
{МЦМУ ``Московский центр фундаментальной и прикладной математики''}
{anna.meleshkina@math.msu.ru}
{Лисицын Михаил Денисович}
{Математическое образование и просвещение}

Малый мехмат МГУ --- это система математических кружков для школьников 5-11 классов при механико-математическом факультете МГУ имени М. В. Ломоносова. В рамках тематических занятий ученики знакомятся с новыми математическими понятиями, идеями, конструкциями, выполняют решение соответствующих задач и демонстрируют результат преподавателям.

Старшими преподавателями на Малом мехмате являются сотрудники и выпускники разных кафедр механико-математического факультета, но значимую помощь им оказывают студенты и аспиранты, которые в ходе своей работы заблаговременно знакомятся с подготовленным материалом по новой теме (теоретической частью и задачами) и осуществляют во время занятий коммуникацию со школьниками по поводу решений, а также фактов из математики, которые могут не относиться прямым образом к теме занятия.

Замечено, что такая деятельность приводит к изменению мышления преподавателей, в частности, связанному с математическими мыслительными средствами и способами их применения. Например, анализируя предварительное авторское решение некоторой задачи, студент может не только познакомиться с новым способом использования какого-то отношения между математическими объектами, но и задуматься о том, какие изменения, сохраняющие общую правильность, допускает предложенное решение, где проходят границы данного способа. Ведь если преподаватель, учитывая, что ему предстоит оценивать потенциально разнообразные решения школьников, захочет заранее подготовиться к разным вариантам, то ему следует увидеть в предложенном тексте не просто последовательность действий, нацеленных на определённый результат, а лишь одну из таких возможных последовательностей.

В докладе, подготовленном исполнительным директором и руководителями параллелей Малого мехмата, будут представлены результаты анализа соответствующих изменений, полученные на основании работы в течение учебного года.
\end{talk}
\end{document}
