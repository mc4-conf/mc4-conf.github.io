\documentclass[12pt, a4paper, figuresright]{book}
\usepackage{mathtools, commath, amssymb, amsthm}
\usepackage{tabularx,graphicx,url,xcolor,rotating,multicol,epsfig,colortbl,lipsum}
\usepackage[T2A]{fontenc}
\usepackage[english,main=russian]{babel}

\setlength{\textheight}{25.2cm}
\setlength{\textwidth}{16.5cm}
\setlength{\voffset}{-1.6cm}
\setlength{\hoffset}{-0.3cm}
\setlength{\evensidemargin}{-0.3cm} 
\setlength{\oddsidemargin}{0.3cm}
\setlength{\parindent}{0cm} 
\setlength{\parskip}{0.3cm}

\newenvironment{talk}[6]{%
\vskip 0pt\nopagebreak%
\vskip 0pt\nopagebreak%
\textbf{#1}\vspace{3mm}\\\nopagebreak%
\textit{#2}\\\nopagebreak%
#3\\\nopagebreak%
\url{#4}\vspace{3mm}\\\nopagebreak%
\ifthenelse{\equal{#5}{}}{}{Соавторы: #5\vspace{3mm}\\\nopagebreak}%
\ifthenelse{\equal{#6}{}}{}{Секция: #6\quad \vspace{3mm}\\\nopagebreak}%
}

\pagestyle{empty}

\begin{document}
\begin{talk}
{Проблема систематических ошибок в освоении темы ``Площадь'' по результатам мониторинга на платформе ``Учи.ру''}
{Москаленко Ольга Борисовна}
{ООО ``Учи.ру''}
{Olga.b.moskalenko@yandex.ru}
{}
{Математическое образование и просвещение}

В начальной школе учащиеся знакомятся с рядом величин, одной из
которых является площадь. В рамках этой темы изучаются способы сравнения и
измерения площадей фигур, понятие квадратного сантиметра и другие единицы
измерения площади, а также правило нахождения площади прямоугольника по
известным длинам его сторон. В~2021-2022 учебном году задачи на тему
``Площадь'' были предложены ученикам~4 и~6 классов в рамках мониторинга на
платформе ``Учи.ру''. Каждую из этих задач решали более 30 тысяч учащихся.
Анализ результатов мониторинга, полученных на столь большой выборке,
позволяет, во-первых, отличить систематические ошибки учащихся от
случайных, а во-вторых, рассматривать в качестве основного фактора влияния
методику изложения темы ``Площадь'' в учебниках для начальной школы. Это
ставит проблему о выявлении тех дефектов методики обучения, которые
привели к совершению учащимися систематических ошибок в задачах
мониторинга. Цель исследования состояла в том, чтобы выяснить,
представлены ли в учебниках для начальной школы все составляющие,
необходимые для успешного решения задач мониторинга, и если представлены,
то в какой мере. Рассмотрены три УМК из Федерального перечня, и для каждого из них обоснована необходимость применения
учителем дополнительных упражнений к тем, что представлены в учебнике, а
также предложены источники, в которых содержатся недостающие упражнения.

\medskip

\begin{enumerate}
\item[{[1]}] Боровских А.В. О понятии математической грамотности // Педагогика. 2022, Т.~86. № 3. С.
33-45.
\item[{[2]}] Вертгеймер М. Продуктивное мышление. М.: Прогресс, 1987.
\item[{[3]}] Эльконин Б.Д. Строение действия и периодизация Д.Б. Эльконина // Деятельностный подход
в образовании: монография. Книга 3 / Составитель В.А. Львовский.
М.: Некоммерческое партнерство ``Авторский Клуб'', 2020. С. 104-117.
\end{enumerate}
\end{talk}
\end{document}
