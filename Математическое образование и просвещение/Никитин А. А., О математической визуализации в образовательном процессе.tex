\documentclass[12pt, a4paper, figuresright]{book}
\usepackage{mathtools, commath, amssymb, amsthm}
\usepackage{tabularx,graphicx,url,xcolor,rotating,multicol,epsfig,colortbl,lipsum}
\usepackage[T2A]{fontenc}
\usepackage[english,main=russian]{babel}

\setlength{\textheight}{25.2cm}
\setlength{\textwidth}{16.5cm}
\setlength{\voffset}{-1.6cm}
\setlength{\hoffset}{-0.3cm}
\setlength{\evensidemargin}{-0.3cm} 
\setlength{\oddsidemargin}{0.3cm}
\setlength{\parindent}{0cm} 
\setlength{\parskip}{0.3cm}

\newenvironment{talk}[6]{%
\vskip 0pt\nopagebreak%
\vskip 0pt\nopagebreak%
\textbf{#1}\vspace{3mm}\\\nopagebreak%
\textit{#2}\\\nopagebreak%
#3\\\nopagebreak%
\url{#4}\vspace{3mm}\\\nopagebreak%
\ifthenelse{\equal{#5}{}}{}{Соавторы: #5\vspace{3mm}\\\nopagebreak}%
\ifthenelse{\equal{#6}{}}{}{Секция: #6\quad \vspace{3mm}\\\nopagebreak}%
}

\pagestyle{empty}

\begin{document}
\begin{talk}
{О математической визуализации в образовательном процессе}
{Никитин Алексей Антонович}
{МГУ им. М.\,В. Ломоносова, факультет ВМК}
{nikitin@cs.msu.ru}
{}
{Математическое образование и просвещение}

Настоящая работа посвящена вопросу использования современных информационных технологий в аудиторном образовательном процессе. В ней подчёркивается необходимость объединения символьной и визуальной математики, описываются проблемы, связанные с этим вопросом, делается обзор существующих систем и определяются требования, которым должна удовлетворять современная система визуализации. В работе обсуждаются существующие наработки, созданные командой авторов. Описывается работа библиотеки визуализаций visualmath.ru. Этот ресурс содержит объёмный архив текстовых и визуальных модулей,  на основе которых преподаватели смогут создавать свои лекции-презентации, снабжённые большим количеством визуальных материалов. Другой важнейшей частью доклада является описание работы быстрых и мощных графических JavaScript-библиотек: Skeleton и Grafar. Первая из этих библиотек предназначена для отображения двумерных графиков и способна обрабатывать очень большие массивы элементов за исключительно короткое время, а вторая позволяет визуализировать красивейшие трёхмерные объекты,  прорабатывать их освещённость, прозрачность и т.п. В заключении приводится ряд примеров использования вышеописанных библиотек. Демонстрируются уже созданные визуализации для курсов математического анализа и аналитической геометрии.

\medskip

\begin{enumerate}
\item[{[1]}] Karpov A. D., Klepov V. Y., Nikitin A. A. On mathematical visualization in education // Communications in Computer and Information Science. — 2020. — Vol. 1140, no.~1. --- P. 11–27. 
\end{enumerate}
\end{talk}
\end{document}
