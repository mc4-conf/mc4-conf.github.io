\documentclass[12pt, a4paper, figuresright]{book}
\usepackage{mathtools, commath, amssymb, amsthm}
\usepackage{tabularx,graphicx,url,xcolor,rotating,multicol,epsfig,colortbl,lipsum}
\usepackage[T2A]{fontenc}
\usepackage[english,main=russian]{babel}

\setlength{\textheight}{25.2cm}
\setlength{\textwidth}{16.5cm}
\setlength{\voffset}{-1.6cm}
\setlength{\hoffset}{-0.3cm}
\setlength{\evensidemargin}{-0.3cm} 
\setlength{\oddsidemargin}{0.3cm}
\setlength{\parindent}{0cm} 
\setlength{\parskip}{0.3cm}

\newenvironment{talk}[6]{%
\vskip 0pt\nopagebreak%
\vskip 0pt\nopagebreak%
\textbf{#1}\vspace{3mm}\\\nopagebreak%
\textit{#2}\\\nopagebreak%
#3\\\nopagebreak%
\url{#4}\vspace{3mm}\\\nopagebreak%
\ifthenelse{\equal{#5}{}}{}{Соавторы: #5\vspace{3mm}\\\nopagebreak}%
\ifthenelse{\equal{#6}{}}{}{Секция: #6\quad \vspace{3mm}\\\nopagebreak}%
}

\pagestyle{empty}

\begin{document}
\begin{talk}
{Математическое образование: Тревожность или Восторженность?}
{Семенов Алексей Львович}
{Московский государственный университет им. М.\,В. Ломоносова, Москва, РФ; Российский государственный педагогический университет им. А.\,И. Герцена}
{alsemenov@math.msu.ru}
{}
{Математическое образование и просвещение}

Во всем мире, в том числе, и в России, в частности, на заседании Президиума РАН, обсуждается вопрос о математической тревожности. Конечно, имеется в виду прежде всего тревожность обучающихся. Притом обсуждаемая тревожность отличается от филологической, исторической, химической и пр.

В докладе предлагается гипотеза о том, что (школьная) математическая тревожность не вытекает непосредственно из природы математического знания, оставим в стороне предположение о ее генетической предопределенности. Она связана с тем, каким образом формулируются цели математического образования и какими путями они достигаются.

Перестанем требовать от ребенка скорость и безошибочность в решении задач по заученному алгоритму. Предложим ему решать неожиданную задачу -- задачу, которую неизвестно, как решать [1]. Такие задачи человечество изобретало и предлагало детям и взрослым в течение тысячелетий. Сегодня они составляют основу олимпиады «Кенгуру», во многом сформированную Марком Ивановичем Башмаковым. 

Предложение ученику интересных задач, которые неизвестно, как решать, естественно сочетается с расширением круга рассматриваемых математических объектов и конструкций. Они возникают в контексте «традиционных занимательных», «олимпиадных» задач. 

Такое построение математического образования восходит к школе Лузина, математическим олимпиадам и кружкам матклассам системы Н.\,Н. Константинова [2], 

С конца 1980-х гг. автор настоящего доклада вместе со своими коллегами реализует данный подход в учебниках по математике и информатике для начальной школы, используемых в сотнях российских школ.

\medskip

\begin{enumerate}
\item[{[1]}] E.J.\,Barbeau,  P.J.\,Taylor (eds.). Challenging Mathematics In and Beyond the Classroom. {\it The 16th ICMI Study,} New ICMI Study Series, v. 12 (2009). Springer Science + Business Media, LLC, 336 p. DOI: 10.1007/978-0-387-09603-2.
\item[{[3]}] Н.Н.\,Константинов, А.Л.\,Семенов Результативное образование в математической школе. {\it Чебышёвский сборник,} 22(1) (2021), 413–446.
\end{enumerate}
\end{talk}
\end{document}