\documentclass[12pt, a4paper, figuresright]{book}
\usepackage{mathtools, amssymb, amsthm}
\usepackage{tabularx,graphicx,url,xcolor,rotating,multicol,epsfig,colortbl,lipsum}
\usepackage[T2A]{fontenc}
\usepackage[english,main=russian]{babel}

\setlength{\textheight}{25.2cm}
\setlength{\textwidth}{16.5cm}
\setlength{\voffset}{-1.6cm}
\setlength{\hoffset}{-0.3cm}
\setlength{\evensidemargin}{-0.3cm} 
\setlength{\oddsidemargin}{0.3cm}
\setlength{\parindent}{0cm} 
\setlength{\parskip}{0.3cm}

\newenvironment{talk}[6]{%
\vskip 0pt\nopagebreak%
\vskip 0pt\nopagebreak%
\textbf{#1}\vspace{3mm}\\\nopagebreak%
\textit{#2}\\\nopagebreak%
#3\\\nopagebreak%
\url{#4}\vspace{3mm}\\\nopagebreak%
\ifthenelse{\equal{#5}{}}{}{Соавторы: #5\vspace{3mm}\\\nopagebreak}%
\ifthenelse{\equal{#6}{}}{}{Секция: #6\quad \vspace{3mm}\\\nopagebreak}%
}

\pagestyle{empty}

\begin{document}
\begin{talk}
{Создание благоприятной психологической обстановки на занятиях в группах}
{Смирнова Вера Андреевна}
{}%
{} %
{} %
{Математическое образование и просвещение}

В докладе рассматривается методика комфортного общения со студентами, направленная на повышения среднего балла группы на экзамене. Она разработана на основе 30-летнего педагогического опыта автора и анализа отзывов студентов разных вузов в интернете.
\begin{enumerate}
\item Постановка задачи.
\item Приоритеты.
\item Чего нельзя делать.
\item Как добиться сдачи индивидуальных заданий вовремя.
\item Организация успешного написания контрольных.
\item Взаимоотношения преподавателя и студентов.
\item О чем говорить не стоит.
\item Об организации экзамена.
\item Заключение.
\end{enumerate}
\end{talk}
\end{document}
