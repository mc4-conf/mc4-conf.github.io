\documentclass[12pt, a4paper, figuresright]{book}
\usepackage{mathtools, commath, amssymb, amsthm}
\usepackage{tabularx,graphicx,url,xcolor,rotating,multicol,epsfig,colortbl,lipsum}
\usepackage[T2A]{fontenc}
\usepackage[english,main=russian]{babel}

\setlength{\textheight}{25.2cm}
\setlength{\textwidth}{16.5cm}
\setlength{\voffset}{-1.6cm}
\setlength{\hoffset}{-0.3cm}
\setlength{\evensidemargin}{-0.3cm} 
\setlength{\oddsidemargin}{0.3cm}
\setlength{\parindent}{0cm} 
\setlength{\parskip}{0.3cm}

\newenvironment{talk}[6]{%
	\vskip 0pt\nopagebreak%
	\vskip 0pt\nopagebreak%
	\textbf{#1}\vspace{3mm}\\\nopagebreak%
	\textit{#2}\\\nopagebreak%
	#3\\\nopagebreak%
	\url{#4}\vspace{3mm}\\\nopagebreak%
	%\ifthenelse{\equal{#5}{}}{}{Соавторы: #5\vspace{3mm}\\\nopagebreak}%
	\ifthenelse{\equal{#6}{}}{}{Секция: #6\quad \vspace{3mm}\\\nopagebreak}%
}

\pagestyle{empty}

\begin{document}
\begin{talk}
{Явные разностные методы решения задачи Коши для интегро-дифференциальных уравнений Вольтерра с разностным ядром}
{Алиханов Анатолий Алиевич}
{Северо-Кавказский федеральный университет}
{aaalikhanov@gmail.com}
{}
{Пленарный доклад}

В докладе рассматривается задача Коши для интегро-дифференциальных уравнений Вольтерра с разностным ядром. При численной аппроксимации задачи на определенном временном слое свойства памяти требуют учета решения во всех предыдущих временных слоях. Это делает численное решение таких задач достаточно ресурсоемкими даже в одномерном случае, а при переходе к двух- или трехмерным задачам вычислительные затраты значительно увеличиваются. С помощью аппроксимации разностного ядра суммой экспонент исходная нелокальная задача сводится к системе локальных задач, что позволяет строить численные методы с минимальным учетом памяти. Для локальных задач хорошо изучены неявные разностные схемы. Построение явных разностных схем второго порядка аппроксимации требует введения дополнительной сетки связанной с исходной. Получены критерии устойчивости построенных явных разностных схем. Предложенные методы могут применяться для численного решения диффузионно-волновых уравнений дробного порядка по времени. Так же будут обсуждаться некоторые открытые вопросы и перспективы дальнейших исследований в данной области.    
\end{talk}
\end{document}
