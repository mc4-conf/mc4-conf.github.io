\documentclass[12pt, a4paper, figuresright]{book}
\usepackage{mathtools, commath, amssymb, amsthm}
\usepackage{tabularx,graphicx,url,xcolor,rotating,multicol,epsfig,colortbl,lipsum}
\usepackage[T2A]{fontenc}
\usepackage[english,main=russian]{babel}

\setlength{\textheight}{25.2cm}
\setlength{\textwidth}{16.5cm}
\setlength{\voffset}{-1.6cm}
\setlength{\hoffset}{-0.3cm}
\setlength{\evensidemargin}{-0.3cm}
\setlength{\oddsidemargin}{0.3cm}
\setlength{\parindent}{0cm}
\setlength{\parskip}{0.3cm}

\newenvironment{talk}[6]{%
\vskip 0pt\nopagebreak%
\vskip 0pt\nopagebreak%
\textbf{#1}\vspace{3mm}\\\nopagebreak%
\textit{#2}\\\nopagebreak%
#3\\\nopagebreak%
\url{#4}\vspace{3mm}\\\nopagebreak%
\ifthenelse{\equal{#5}{}}{}{Соавторы: #5\vspace{3mm}\\\nopagebreak}%
\ifthenelse{\equal{#6}{}}{}{Секция: #6\quad \vspace{3mm}\\\nopagebreak}%
}

\pagestyle{empty}

\begin{document}
\begin{talk}
{Инвариантные алгебраические многообразия и интегрируемость обыкновенных дифференциальных уравнений}
{Белова Мария Владимировна}
{Национальный исследовательский университет ``Высшая школа экономики''}
{mvbelova@hse.ru}
{}
{Пленарный доклад}

Доклад посвящен проблеме нахождения инвариантных многообразий для обыкновенных дифференциальных уравнений и систем обыкновенных дифференциальных уравнений. Будет дано детальное описание метода, позволяющего находить все инвариантные алгебраические многообразия для широких классов полиномиальных обыкновенных дифференциальных уравнений. Планируется рассмотреть приложения теории инвариантных многообразий при исследовании интегрируемости и разрешимости дифференциальных систем. В частности, будет рассматриваться вопрос построения первых интегралов, принадлежащих расширению Лиувилля поля рациональных функций. В качестве иллюстрации будет представлено решение проблемы интегрируемости по Лиувиллю для полиномиальных систем Льенара.
\end{talk}
\end{document}