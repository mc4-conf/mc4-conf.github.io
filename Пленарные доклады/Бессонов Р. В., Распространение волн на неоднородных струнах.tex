\documentclass[12pt, a4paper, figuresright]{book}
\usepackage{mathtools, amssymb, amsthm}
\usepackage{tabularx,graphicx,url,xcolor,rotating,multicol,epsfig,colortbl,lipsum}
\usepackage[T2A]{fontenc}
\usepackage[english,main=russian]{babel}

\setlength{\textheight}{25.2cm}
\setlength{\textwidth}{16.5cm}
\setlength{\voffset}{-1.6cm}
\setlength{\hoffset}{-0.3cm}
\setlength{\evensidemargin}{-0.3cm} 
\setlength{\oddsidemargin}{0.3cm}
\setlength{\parindent}{0cm} 
\setlength{\parskip}{0.3cm}

\newenvironment{talk}[6]{%
\vskip 0pt\nopagebreak%
\vskip 0pt\nopagebreak%
\textbf{#1}\vspace{3mm}\\\nopagebreak%
\textit{#2}\\\nopagebreak%
#3\\\nopagebreak%
\url{#4}\vspace{3mm}\\\nopagebreak%
\ifthenelse{\equal{#5}{}}{}{Соавторы: #5\vspace{3mm}\\\nopagebreak}%
\ifthenelse{\equal{#6}{}}{}{Секция: #6\quad \vspace{3mm}\\\nopagebreak}%
}

\pagestyle{empty}

\begin{document}
\begin{talk}
{Распространение волн на неоднородных струнах} %
{Бессонов Роман Викторович} %
{Санкт-Петербургский государственный университет Санкт-Петербургское отделение математического института им. В.\,А. Стеклова РАН}%
{r.bessonov@gmail.com} %
{С.\,А. Денисов} %
{Пленарный доклад} %

Рассматривается распространение волн по неоднородной полубесконечной струне общего вида. В терминах динамики волн описывается условие Крейна-Винера конечности логарифмического интеграла спектральной функции струны:
\[\int_{0}^{\infty}\frac{\log v_{\rm ac}(\lambda)}{\sqrt{\lambda}(1+\lambda)}\,d\lambda > -\infty.\]
Указанное условие играет ключевую роль в спектральной теории стационарных гауссовских процессов. Оказывается, что оно равносильно наличию ``асимптотически бегущих'' волн, распространяющихся по струне. 

Помимо динамического описания струн Крейна-Винера, приводится их полное описание в терминах функций плотности. В частности, струны, составленные из участков двух разных материалов принадлежат классу Крейна-Винера тогда и только тогда, когда общая длина одного из материалов конечна. 

Задача о распространении волн на неоднородной струне допускает интерпретацию в теории рассеяния. Доказывается, что условие Крейна-Винера равносильно существованию и полноте модифицированных волновых операторов для рассматриваемой струны на фоне однородной струны. 

Доклад содержит результаты цикла работ автора и С.\,А. Денисова (University of Wisconsin--Madyson).
\end{talk}
\end{document}
