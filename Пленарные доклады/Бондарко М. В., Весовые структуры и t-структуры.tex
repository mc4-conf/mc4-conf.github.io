\documentclass[12pt, a4paper, figuresright]{book}
\usepackage{mathtools, amssymb, amsthm} %commath,
\usepackage{tabularx,graphicx,url,rotating,multicol,epsfig}%xcolor,colortbl,,lipsum
\usepackage[T2A]{fontenc}
\usepackage[english,main=russian]{babel}

\setlength{\textheight}{25.2cm}
\setlength{\textwidth}{16.5cm}
\setlength{\voffset}{-1.6cm}
\setlength{\hoffset}{-0.3cm}
\setlength{\evensidemargin}{-0.3cm} 
\setlength{\oddsidemargin}{0.3cm}
\setlength{\parindent}{0cm} 
\setlength{\parskip}{0.3cm}

\newenvironment{talk}[6]{%
\vskip 0pt\nopagebreak%
\vskip 0pt\nopagebreak%
\textbf{#1}\vspace{3mm}\\\nopagebreak%
\textit{#2}\\\nopagebreak%
#3\\\nopagebreak%
\url{#4}\vspace{3mm}\\\nopagebreak%
\ifthenelse{\equal{#5}{}}{}{Соавторы: #5\vspace{3mm}\\\nopagebreak}%
\ifthenelse{\equal{#6}{}}{}{Секция: #6\quad \vspace{3mm}\\\nopagebreak}%
}

\pagestyle{empty}

\begin{document}
\begin{talk}
{Весовые структуры и \(t\)-структуры}
{Бондарко Михаил Владимирович}
{Санкт-Петербургский Государственный Университет}
{mbond77@mail.ru}
{}
{Пленарный доклад}

Часто важные (и функториальные) инварианты определяются или вычисляются при помощи неканонических конструкций.

Производные функторы вычисляются при помощи проективных и инъективных резольвент (модулей, объектов и комплексов); (ко)гомологии многообразий --- при помощи разбиений на симплексы; (ко)гомологии спектров --- при помощи клеточных фильтрации. Смешанные структуры Ходжа для когомологий непроективного (или негладкого) комплексного многообразия определяются при помощи хороших компактификаций (соотв., гладких гиперпокрытий). При этом, две разных резольвенты можно соединить морфизмом; для компактификаций получается только выбрать третью, ``мажорирующую'' первые две.

Иногда получается ``вложить исходные объекты'' в триангулированную категорию \(\underline{C}\) и связать искомый ``инвариант'' со срезками, соответствующими весовым структурам. Хоть весовые срезки и не каноничны, они позволяют строить канонические инварианты. В частности, при некоторых условиях (выполненных, например, для производных категорий регулярных собственных схем) весовые структуры позволяют строить \(t\)-структуры, срезки по которым (всегда) каноничны.
\end{talk}
\end{document}