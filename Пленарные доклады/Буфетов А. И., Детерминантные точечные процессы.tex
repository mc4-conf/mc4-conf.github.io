\documentclass[12pt, a4paper, figuresright]{book}
\usepackage{mathtools, commath, amssymb, amsthm}
\usepackage{tabularx,graphicx,url,xcolor,rotating,multicol,epsfig,colortbl,lipsum}
\usepackage[T2A]{fontenc}
\usepackage[english,main=russian]{babel}

\setlength{\textheight}{25.2cm}
\setlength{\textwidth}{16.5cm}
\setlength{\voffset}{-1.6cm}
\setlength{\hoffset}{-0.3cm}
\setlength{\evensidemargin}{-0.3cm} 
\setlength{\oddsidemargin}{0.3cm}
\setlength{\parindent}{0cm} 
\setlength{\parskip}{0.3cm}

\newenvironment{talk}[6]{%
	\vskip 0pt\nopagebreak%
	\vskip 0pt\nopagebreak%
	\textbf{#1}\vspace{3mm}\\\nopagebreak%
	\textit{#2}\\\nopagebreak%
	#3\\\nopagebreak%
	\url{#4}\vspace{3mm}\\\nopagebreak%
	\ifthenelse{\equal{#5}{}}{}{Соавторы: #5\vspace{3mm}\\\nopagebreak}%
	\ifthenelse{\equal{#6}{}}{}{Секция: #6\quad \vspace{3mm}\\\nopagebreak}%
}

\pagestyle{empty}

\begin{document}
\begin{talk}
{Детерминантные точечные процессы} %
{Буфетов Александр Игоревич} %
{Математический институт им. В.\,,А.,Стеклова РАН, Санкт-Петербургский государственный университет}%
{bufetov@mi-ras.ru} %
{} %
{Пленарный доклад} %

Деметрий Фалерей, основатель Александрийской библиотеки, провел перепись населения в Афинах в конце IV в. до Р.Х. В математических задачах демографии рождается теория точечных процессов на прямой --- последовательностей неразличимых событий, происходящих в случайные моменты времени. В 1915~г.  работа Р. Фишера~[1] открыла новую главу теории точечных процессов --- изучение собственных чисел матриц, задаваемых случаем. 

Синус-процесс Дайсона~[2] --- скейлинговый предел радиальных частей мер Хаара на унитарных группах растущей размерности. Корреляционные функции синус-процесса задаются детерминантами синус-ядра --- ядра проектора на пространство Пэли--Винера. Точечные процессы, чьи корреляционные функции задаются детерминантами, с одной стороны, возникают в самых разных конкретных задачах --- асимптотической комбинаторики, теории представлений бесконечномерных групп, теории гауссовских аналитических функций --- а, с другой, допускают богатую общую теорию.

В совместной работе с Янци Цью (Тулуза, Пекин) и А. Шамовым (Харьков, Реховот) доказано, что реализация детермиантного точечного процесса почти наверное есть множество единственности для гильбертова пространства, образа нашего проектора.

Минимально ли это множество единственности? Оказывается --- нет:
почти наверное реализация синус-процесса имеет \emph{избыток~1} для пространства Пэли--Винера, то есть, становится полным и минимальным множеством после удаления одной частицы. Дело в~том, что случайные целые функции, обобщённые произведения Эйлера, сопоставляемые синус-процессу, сходятся
при скейлинге по распределению к 
восходящему, на физическом уровне строгости, к работам А.\,Н. Колмогорова и его школы по теории однородной изотропной  турбулентности 	гауссову мультипликативному хаосу.

Доказательство сходимости к гауссову мультипликативному хаосу опирается на квази-инвариантность синус-процесса под действием диффеоморфизмов прямой с компактным носителем, а также на оценки остаточного члена в скейлинговом пределе формулы Бородина--Окунькова--Джеронимо--Кейса, обобщающей Сильную Теорему Сегё в~форме И.\,А. Ибрагимова. Для детерминантного процесса с ядром Бесселя точные оценки недавно получил С.\,М. Горбунов~[3].

\medskip

\begin{enumerate}
\item[{[1]}]R. A. Fisher. Frequency Distribution of the Values of the Correlation Coefficient in Samples from an Indefinitely Large Population. \textit{Biometrika}, \textbf{10}:4 (1915), 507--521.
\item[{[2]}]F. J. Dyson. Statistical Theory of the Energy Levels of Complex Systems. I. \textit{J. Math. Phys.}, \textbf{3}:1 (1962), 140--156.
\item[{[3]}]S. M. Gorbunov. Speed of convergence in the Central Limit Theorem for the determin\-an\-tal point process with the Bessel kernel. Preprint arXiv:2403.16219 (2024), 24 pp.
\end{enumerate}
\end{talk}
\end{document}
