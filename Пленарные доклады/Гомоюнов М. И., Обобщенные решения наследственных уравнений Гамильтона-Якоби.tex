\documentclass[12pt, a4paper, figuresright]{book}
\usepackage{mathtools, commath, amssymb, amsthm}
\usepackage{tabularx,graphicx,url,xcolor,rotating,multicol,epsfig,colortbl,lipsum}
\usepackage[T2A]{fontenc}
\usepackage[english,main=russian]{babel}

\setlength{\textheight}{25.2cm}
\setlength{\textwidth}{16.5cm}
\setlength{\voffset}{-1.6cm}
\setlength{\hoffset}{-0.3cm}
\setlength{\evensidemargin}{-0.3cm}
\setlength{\oddsidemargin}{0.3cm}
\setlength{\parindent}{0cm}
\setlength{\parskip}{0.3cm}

\newenvironment{talk}[6]{%
\vskip 0pt\nopagebreak%
\vskip 0pt\nopagebreak%
\textbf{#1}\vspace{3mm}\\\nopagebreak%
\textit{#2}\\\nopagebreak%
#3\\\nopagebreak%
\url{#4}\vspace{3mm}\\\nopagebreak%
\ifthenelse{\equal{#5}{}}{}{Соавторы: #5\vspace{3mm}\\\nopagebreak}%
\ifthenelse{\equal{#6}{}}{}{Секция: #6\quad \vspace{3mm}\\\nopagebreak}%
}

\pagestyle{empty}

\begin{document}
\begin{talk}
{Обобщенные решения наследственных уравнений Гамильтона--Якоби}
{Гомоюнов Михаил Игоревич}
{ИММ УрО РАН}
{m.i.gomoyunov@gmail.com} %
{} %
{} %

В рамках доклада будут представлены результаты по развитию теории минимаксных и вязкостных (обобщенных) решений для наследственных уравнений Гамильтона--Якоби с коинвариантными производными над пространством непрерывных функций.
Уравнения такого типа возникают в задачах динамической оптимизации систем с запаздыванием [1, 2].
Основной результат [3] состоит в доказательстве эквивалентности определений минимаксного решения (в форме пары неравенств для нижних и верхних производных по многозначным направлениям) и вязкостного решения (в форме пары неравенств для коинвариантных суб- и суперградиентов).
Одним из следствий этого результата является теорема о единственности вязкостного решения задачи Коши для рассматриваемого класса уравнений Гамильтона--Якоби.
Ключевую роль в доказательстве играет специальное свойство коинвариантного субдифференциала, обоснование которого в свою очередь потребовало развития техники, восходящей к доказательствам многомерных негладких обобщений формулы конечных приращений [4, 5] и использующей подходящие гладкие вариационные принципы [6].

\medskip

\begin{enumerate}
\item[{[1]}]
Н.~Ю.~Лукоянов, {\it Функциональные уравнения Гамильтона--Якоби и задачи управления с наследственной информацией}, УрФУ, Екатеринбург, 2011.
\item[{[2]}]
М.~И.~Гомоюнов, Н.~Ю.~Лукоянов, {\it Минимаксные решения уравнений Гамильто\-на--Яко\-би в задачах динамической оптимизации наследственных систем}, Успехи математических наук, 79:2(476) (2024), 43--144.
\item[{[3]}]
M.~I.~Gomoyunov, A.~R.~Plaksin, {\it Equivalence of minimax and viscosity solutions of path-dependent Hamilton--Jacobi equations}, Journal of Functional Analysis, 285:11 (2023), 110155, 41 pp.
\item[{[4]}]
А.~И.~Субботин, {\it Об одном свойстве субдифференциала}, Математический сборник, 182:9 (1991), 1315--1330.
\item[{[5]}]
F.~H.~Clarke, Yu.~S.~Ledyaev, {\it Mean value inequalities in Hilbert space}, Transactions of the American Mathematical Society, 344:1 (1994), 307--324.
\item[{[6]}]
J.~M.~Borwein, Q.~J.~Zhu, {\it Techniques of variational analysis}, Springer, 2005.
\end{enumerate}
\end{talk}
\end{document}