\documentclass[12pt, a4paper, figuresright]{book}
\usepackage{mathtools, commath, amssymb, amsthm}
\usepackage{tabularx,graphicx,url,xcolor,rotating,multicol,epsfig,colortbl,lipsum}
\usepackage[T2A]{fontenc}
\usepackage[english,main=russian]{babel}

\setlength{\textheight}{25.2cm}
\setlength{\textwidth}{16.5cm}
\setlength{\voffset}{-1.6cm}
\setlength{\hoffset}{-0.3cm}
\setlength{\evensidemargin}{-0.3cm} 
\setlength{\oddsidemargin}{0.3cm}
\setlength{\parindent}{0cm} 
\setlength{\parskip}{0.3cm}

\newenvironment{talk}[6]{%
\vskip 0pt\nopagebreak%
\vskip 0pt\nopagebreak%
\textbf{#1}\vspace{3mm}\\\nopagebreak%
\textit{#2}\\\nopagebreak%
#3\\\nopagebreak%
\url{#4}\vspace{3mm}\\\nopagebreak%
\ifthenelse{\equal{#5}{}}{}{Соавторы: #5\vspace{3mm}\\\nopagebreak}%
\ifthenelse{\equal{#6}{}}{}{Секция: #6\quad \vspace{3mm}\\\nopagebreak}%
}

\pagestyle{empty}

\begin{document}
\begin{talk}
{Порядки элементов конечных почти простых групп}
{Гречкосеева Мария Александровна}
{Институт математики им. С.\,Л. Соболева СО РАН}
{grechkoseeva@gmail.com}
{}
{Пленарный доклад}

Конечная группа \(G\) называется почти простой, если она удовлетворяет условию \(S\leq G\leq \operatorname{Aut} S\) для некоторой конечной неабелевой простой группы  \(S\); при этом  \(S\) является цоколем группы \(G\). Многие вопросы теории конечных групп сводятся не к простым, а к почти простым группам, поскольку важна информация не только о том, каковы композиционные факторы данной конечной группы, но и каково действие группы  на этих композиционных факторах.

Доклад посящен задаче вычисления множеств порядков элементов конечных почти простых групп. Эта задача легко решается для групп со знакопеременным цоколем и 
давно решена для групп со спорадическим цоколем, поэтому речь пойдет о группах с цоколем лиева типа и, как легко понять, можно ограничиться группами вида \(\langle S, \alpha\rangle\), где~\(\alpha\in \operatorname{Aut} S\). 

Будет дан обзор известных результатов, в том числе,  будет рассказано, как были найдены множества порядков элементов самих простых групп лиева типа и как случай внешнего автоморфизма \(\alpha\) был сведен к случаю, когда \(\alpha\) --- диагонально-графовый автоморфизм. Также будут представлены недавние результаты о диагонально-графовых автоморфизмах. 
\end{talk}
\end{document}
