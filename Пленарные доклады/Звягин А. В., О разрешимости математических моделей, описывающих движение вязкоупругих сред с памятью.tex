\documentclass[12pt, a4paper, figuresright]{book}
\usepackage{mathtools, commath, amssymb, amsthm}
\usepackage{tabularx,graphicx,url,xcolor,rotating,multicol,epsfig,colortbl,lipsum}
\usepackage[T2A]{fontenc}
\usepackage[english,main=russian]{babel}

\setlength{\textheight}{25.2cm}
\setlength{\textwidth}{16.5cm}
\setlength{\voffset}{-1.6cm}
\setlength{\hoffset}{-0.3cm}
\setlength{\evensidemargin}{-0.3cm} 
\setlength{\oddsidemargin}{0.3cm}
\setlength{\parindent}{0cm} 
\setlength{\parskip}{0.3cm}

\newenvironment{talk}[6]{%
\vskip 0pt\nopagebreak%
\vskip 0pt\nopagebreak%
\textbf{#1}\vspace{3mm}\\\nopagebreak%
\textit{#2}\\\nopagebreak%
#3\\\nopagebreak%
\url{#4}\vspace{3mm}\\\nopagebreak%
\ifthenelse{\equal{#5}{}}{}{Соавторы: #5\vspace{3mm}\\\nopagebreak}%
\ifthenelse{\equal{#6}{}}{}{Секция: #6\quad \vspace{3mm}\\\nopagebreak}%
}

\pagestyle{empty}

\begin{document}
\begin{talk}
{О разрешимости математических моделей, описывающих движение вязкоупругих сред с памятью}
{Звягин Андрей Викторович}
{Воронежский государственный университет}
{zvyagin.a@mail.ru}
{}
{Пленарный доклад}

Математические вопросы, возникающие при изучении гидродинамики, являются актуальной и быстро развивающейся областью исследований последние сто пятьдесят лет. При этом основное внимание математиков было уделено системе уравнений Эйлера, описывающей движение идеальной среды, и системе уравнений Навье--Стокса, описывающей движение вязкой ньютоновской жидкости. Однако было замечено, что многие реальные среды (например, полимерные растворы, суспензии и др.) не подчиняются моделям классической гидродинамики. Такие модели называются ``неньютоновскими средами''. Данный доклад посвящен математическому исследованию начально--краевых задач для одного класса моделей неньютоновской гидродинамики, а именно, моделей движения вязкоупругих сред. Такие среды, как следует из названия, сочетают в себе свойства вязкости и упругости. 

При изучение большого класса полимеров, в которых необходимо учитывать эффекты ползучести и релаксации, в последние годы появились математические модели с дробными производными. В силу своей сложности математические постановки задач для таких моделей неньютоновской гидродинамики на сегодняшний день не столь подробно изучены и существующие математические методы зачастую оказываются не столь эффективными для них. Именно о слабой разрешимости для таких модей в докладе пойдет речь. 

\medskip

\begin{enumerate}
\item[{[1]}] А. В. Звягин, {\it О слабой разрешимости и сходимости решений дробной альфа--модели Фойгта движения вязкоупругой среды}, Успехи математических наук, 74:3 (2019), 189–190.
\item[{[2]}] А. В. Звягин, {\it Исследование слабой разрешимости дробной альфа--модели Фойгта}, Известия Академии Наук. Серия математическая, 85:1 (2021), 66–97.
\item[{[3]}] А. В. Звягин, {\it О существовании слабых решений дробной модели Кельвина--Фойгта}, Математические заметки, 116:1 (2024), 152–157.
\end{enumerate}
\end{talk}
\end{document}
