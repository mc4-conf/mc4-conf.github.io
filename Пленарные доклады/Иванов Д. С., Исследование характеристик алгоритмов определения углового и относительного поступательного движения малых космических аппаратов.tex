\documentclass[12pt, a4paper, figuresright]{book}
\usepackage{mathtools, amssymb, amsthm}
\usepackage{tabularx,graphicx,url,xcolor,rotating,multicol,epsfig,colortbl,lipsum}
\usepackage[T2A]{fontenc}
\usepackage[english,main=russian]{babel}

\setlength{\textheight}{25.2cm}
\setlength{\textwidth}{16.5cm}
\setlength{\voffset}{-1.6cm}
\setlength{\hoffset}{-0.3cm}
\setlength{\evensidemargin}{-0.3cm} 
\setlength{\oddsidemargin}{0.3cm}
\setlength{\parindent}{0cm} 
\setlength{\parskip}{0.3cm}

\newenvironment{talk}[6]{%
\vskip 0pt\nopagebreak%
\vskip 0pt\nopagebreak%
\textbf{#1}\vspace{3mm}\\\nopagebreak%
\textit{#2}\\\nopagebreak%
#3\\\nopagebreak%
\url{#4}\vspace{3mm}\\\nopagebreak%
\ifthenelse{\equal{#5}{}}{}{Соавторы: #5\vspace{3mm}\\\nopagebreak}%
\ifthenelse{\equal{#6}{}}{}{Секция: #6\quad \vspace{3mm}\\\nopagebreak}%
}

\pagestyle{empty}

\begin{document}
\begin{talk}
{Исследование характеристик алгоритмов определения углового и относительного поступательного движения малых космических аппаратов}
{Иванов Данил Сергеевич}
{Институт прикладной математики им. М.\,В. Келдыша РАН}
{danilivanovs@gmail.com}
{}
{Пленарный доклад}

Бурное развитие малых космических аппаратов привело к возникновению новых задач в области определения углового движения в одиночных миссиях и определения относительного поступательного движения в миссиях группового полёта. В настоящей работе предложена аналитическая методика исследования точностных характеристик алгоритмов определения движения на основе расширенного фильтра Калмана для космических аппаратов с активной системой управления ориентации и набором бортовых измерительных средств. Результаты аналитического исследования сравниваются с результатами математического моделирования работы алгоритмов и с результатами лабораторных исследований характеристик алгоритмов на стендах полунатурного моделирования движения макетов малых спутников с аэродинамическим подвесом, позволяющим имитировать условия орбитального полёта. Разработанные алгоритмы определения движения были реализованы на более 30-ти отечественных малых космических аппаратах, лётные испытания показали адекватность полученных аналитических оценок и надёжность предложенных алгоритмов для решения поставленных задач.
\end{talk}
\end{document}
