\documentclass[12pt, a4paper, figuresright]{book}
\usepackage{mathtools, commath, amssymb, amsthm}
\usepackage{tabularx,graphicx,url,xcolor,rotating,multicol,epsfig,colortbl,lipsum}
\usepackage[T2A]{fontenc}
\usepackage[english,main=russian]{babel}

\setlength{\textheight}{25.2cm}
\setlength{\textwidth}{16.5cm}
\setlength{\voffset}{-1.6cm}
\setlength{\hoffset}{-0.3cm}
\setlength{\evensidemargin}{-0.3cm} 
\setlength{\oddsidemargin}{0.3cm}
\setlength{\parindent}{0cm} 
\setlength{\parskip}{0.3cm}

\newenvironment{talk}[6]{%
\vskip 0pt\nopagebreak%
\vskip 0pt\nopagebreak%
\textbf{#1}\vspace{3mm}\\\nopagebreak%
\textit{#2}\\\nopagebreak%
#3\\\nopagebreak%
\url{#4}\vspace{3mm}\\\nopagebreak%
\ifthenelse{\equal{#5}{}}{}{Соавторы: #5\vspace{3mm}\\\nopagebreak}%
\ifthenelse{\equal{#6}{}}{}{Секция: #6\quad \vspace{3mm}\\\nopagebreak}%
}

\pagestyle{empty}

\begin{document}
\begin{talk}
{Об аттракторе Лоренца и псевдогиперболических аттракторах нового типа}
{Казаков Алексей Олегович}
{Национальный исследовательский университет ``Высшая школа экономики''}
{kazakovdz@yandex.ru}
{}
{Пленарный доклад}

Аттрактор Лоренца является первым примером негрубого, но при этом робастного хаотического поведения. Его негрубость обусловлена тем, что при малых возмущениях в нем возникают бифуркации гомоклинических траекторий к седловому состоянию равновесия. Робастность аттрактора Лоренца заключается в том, что любая его траектория характеризуется положительным максимальным показателем Ляпунова, и это свойство сохраняется при малых возмущениях. В работе [1] выдвинута гипотеза о том, что робастность хаотического аттрактора эквивалентна его псевдогиперболичности. Из этого следует, что установив псевдогиперболичность аттрактора, исследователь может быть уверен, что наблюдаемый в эксперименте динамический режим действительно является хаотическим. 

В наших недавних работах были разработаны методы проверки псевдогиперболичности, а также обнаружен ряд новых негрубых псевдогиперболических аттракторов лоренцевского типа. В докладе будут представлены недавние результаты по данной тематике.

Работа подготовлена в рамках Программы фундаментальных исследований Национального исследовательского университета ``Высшая школа экономики''.

\medskip

\begin{enumerate}
\item[{[1]}] Gonchenko S., Kazakov A., Turaev D. Wild pseudohyperbolic attractor in a four-dimensional Lorenz system //Nonlinearity. – 2021. – Т. 34. – №. 4. – С. 2018.
\end{enumerate}
\end{talk}
\end{document}
