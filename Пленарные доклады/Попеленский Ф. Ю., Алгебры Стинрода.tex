\documentclass[12pt, a4paper, figuresright]{book}
\usepackage{mathtools, commath, amssymb, amsthm}
\usepackage{tabularx,graphicx,url,xcolor,rotating,multicol,epsfig,colortbl,lipsum}
\usepackage[T2A]{fontenc}
\usepackage[english,main=russian]{babel}

\setlength{\textheight}{25.2cm}
\setlength{\textwidth}{16.5cm}
\setlength{\voffset}{-1.6cm}
\setlength{\hoffset}{-0.3cm}
\setlength{\evensidemargin}{-0.3cm} 
\setlength{\oddsidemargin}{0.3cm}
\setlength{\parindent}{0cm} 
\setlength{\parskip}{0.3cm}

\newenvironment{talk}[6]{%
\vskip 0pt\nopagebreak%
\vskip 0pt\nopagebreak%
\textbf{#1}\vspace{3mm}\\\nopagebreak%
\textit{#2}\\\nopagebreak%
#3\\\nopagebreak%
\url{#4}\vspace{3mm}\\\nopagebreak%
\ifthenelse{\equal{#5}{}}{}{Соавторы: #5\vspace{3mm}\\\nopagebreak}%
\ifthenelse{\equal{#6}{}}{}{Секция: #6\quad \vspace{3mm}\\\nopagebreak}%
}

\pagestyle{empty}

\begin{document}
\begin{talk}
  {Алгебры Стинрода} % [1] название доклада
  {Попеленский Фёдор Юрьевич} % [2] имя докладчика
  {МГУ им. М.В.Ломоносова, механико-математический факультет}% [3] аффилиация
  {} % [4] адрес электронной почты (НЕОБЯЗАТЕЛЬНО)
  {} % [5] соавторы (НЕОБЯЗАТЕЛЬНО)
  {Пленарный доклад} % [6] название секции

В алгебрах Стинрода $A_p$ стабильных когомологических операций $\mod p$ имеются 
сложные соотношения между мультипликативными образующими --- соотношения Адема.
В докладе пойдет речь о наборах элементов, образующих аддитивные базисы алгебр Стинрода.

Для произвольного простого $p$ хорошо известны базис Милнора и базис допустимых мономов.
Кроме того, имеются элементы $P_t^s$, из которых тоже можно сформировать серии аддитивных базисов. Для $p=2$  довольно давно были известны базисы, открытые Уоллом, Арноном, Вудом и др.,
они нашли приложения в исследовании действий алгебры  $A_2$.
Нам удалось получить аналоги этих результатов, а в некоторых случаях --- более сильные утверждения, для алгебр $A_p$, где $p>2$.

 \end{talk}
\end{document}
