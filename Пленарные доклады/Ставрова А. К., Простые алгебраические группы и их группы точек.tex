\documentclass[12pt, a4paper, figuresright]{book}
\usepackage{mathtools, commath, amssymb, amsthm}
\usepackage{tabularx,graphicx,url,xcolor,rotating,multicol,epsfig,colortbl,lipsum}
\usepackage[T2A]{fontenc}
\usepackage[english,main=russian]{babel}

\setlength{\textheight}{25.2cm}
\setlength{\textwidth}{16.5cm}
\setlength{\voffset}{-1.6cm}
\setlength{\hoffset}{-0.3cm}
\setlength{\evensidemargin}{-0.3cm} 
\setlength{\oddsidemargin}{0.3cm}
\setlength{\parindent}{0cm} 
\setlength{\parskip}{0.3cm}

\newenvironment{talk}[6]{%
\vskip 0pt\nopagebreak%
\vskip 0pt\nopagebreak%
\textbf{#1}\vspace{3mm}\\\nopagebreak%
\textit{#2}\\\nopagebreak%
#3\\\nopagebreak%
\url{#4}\vspace{3mm}\\\nopagebreak%
\ifthenelse{\equal{#5}{}}{}{Соавторы: #5\vspace{3mm}\\\nopagebreak}%
\ifthenelse{\equal{#6}{}}{}{Секция: #6\quad \vspace{3mm}\\\nopagebreak}%
}

\pagestyle{empty}

\begin{document}
\begin{talk}
{Простые алгебраические группы и их группы точек} %
{Ставрова Анастасия Константиновна} %
{Санкт-Петербургское отделение Математического института им. В.\,А. Стеклова РАН}%
{anastasia.stavrova@gmail.com} %
{} %
{Пленарный доклад} %

Простые алгебраические группы над полем \(K\) являются аналогами в алгебраической геометрии простых групп Ли 
в геометрии дифференциальной. Будучи подмногообразием аффинного пространства, простая алгебраическая группа 
\(G\) задается системой полиномиальных уравнений от нескольких переменных,
и множество решений \(G(L)\) этой системы уравнений в произвольном расширении \(L\) поля \(K\)
является группой в обычном смысле и называется группой \(L\)-точек алгебраической группы \(G\). Группа \(G(L)\), 
вообще говоря, не является 
простой, однако, если \(G\) изотропна (условие, соответствующее не-компактности для простых групп Ли),
то по теореме Ж. Титса (1964) она содержит ``большую'' нормальную подгруппу \(EG(L)\), которая проективно проста.
В. П. Платонов (1975) привел первый пример, показывающий, что фактор-группа \(G(L)/EG(L)\) может быть 
нетривиальной, и в настоящее время проблема ее вычисления называется проблемой Кнезера-Титса. 
Мы обсудим некоторые результаты по этой проблеме и ее обобщения.
\end{talk}
\end{document}
