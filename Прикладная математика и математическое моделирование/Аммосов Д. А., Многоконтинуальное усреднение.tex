\documentclass[12pt, a4paper, figuresright]{book}
\usepackage{mathtools, commath, amssymb, amsthm}
\usepackage{tabularx,graphicx,url,xcolor,rotating,multicol,epsfig,colortbl,lipsum}
\usepackage[T2A]{fontenc}
\usepackage[english,main=russian]{babel}

\setlength{\textheight}{25.2cm}
\setlength{\textwidth}{16.5cm}
\setlength{\voffset}{-1.6cm}
\setlength{\hoffset}{-0.3cm}
\setlength{\evensidemargin}{-0.3cm} 
\setlength{\oddsidemargin}{0.3cm}
\setlength{\parindent}{0cm} 
\setlength{\parskip}{0.3cm}

\newenvironment{talk}[6]{%
\vskip 0pt\nopagebreak%
\vskip 0pt\nopagebreak%
\textbf{#1}\vspace{3mm}\\\nopagebreak%
\textit{#2}\\\nopagebreak%
#3\\\nopagebreak%
\url{#4}\vspace{3mm}\\\nopagebreak%
\ifthenelse{\equal{#5}{}}{}{Соавторы: #5\vspace{3mm}\\\nopagebreak}%
\ifthenelse{\equal{#6}{}}{}{Секция: #6\quad \vspace{3mm}\\\nopagebreak}%
}

\pagestyle{empty}

\begin{document}
\begin{talk}
{Многоконтинуальное усреднение}
{Аммосов Дмитрий Андреевич}
{Институт математики и информатики, Северо-Восточный федеральный университет им. М.\,К. Аммосова}
{dmitryammosov@gmail.com}
{}
{Прикладная математика и математическое моделирование}

Многие прикладные задачи являются многомасштабными по своей природе, с неоднородностями на различных масштабах и высоким контрастом. Для решения таких задач часто применяют методы усреднения, в которых вычисляются эффективные свойства в каждой макромасштабной точке. Однако в ряде случаев, особенно при высоком контрасте, этого оказывается недостаточно, поскольку требуется больше усредненных коэффициентов для точного моделирования.

В данном докладе рассматривается применение нового метода многоконтинуального усреднения для задач с высоким контрастом и без разделения масштабов. Данный метод представляет собой гибкий и в то же время строгий подход для вывода многоконтинуальных моделей. Основная идея метода заключается в построении специальных задач на ячейках в расширенных представительных элементах с ограничениями для учета различных эффектов. Решение задач на ячейках позволяет получить разложение искомой функции по континуумам, а затем с помощью некоторых операций вычислить эффективные свойства и вывести соответствующую многоконтинуальную модель.
\end{talk}
\end{document}