\documentclass[12pt, a4paper, figuresright]{book}
\usepackage{mathtools, commath, amssymb, amsthm}
\usepackage{tabularx,graphicx,url,xcolor,rotating,multicol,epsfig,colortbl,lipsum}
\usepackage[T2A]{fontenc}
\usepackage[english,main=russian]{babel}

\setlength{\textheight}{25.2cm}
\setlength{\textwidth}{16.5cm}
\setlength{\voffset}{-1.6cm}
\setlength{\hoffset}{-0.3cm}
\setlength{\evensidemargin}{-0.3cm} 
\setlength{\oddsidemargin}{0.3cm}
\setlength{\parindent}{0cm} 
\setlength{\parskip}{0.3cm}

\newenvironment{talk}[6]{%
\vskip 0pt\nopagebreak%
\vskip 0pt\nopagebreak%
\textbf{#1}\vspace{3mm}\\\nopagebreak%
\textit{#2}\\\nopagebreak%
#3\\\nopagebreak%
\url{#4}\vspace{3mm}\\\nopagebreak%
\ifthenelse{\equal{#5}{}}{}{Соавторы: #5\vspace{3mm}\\\nopagebreak}%
\ifthenelse{\equal{#6}{}}{}{Секция: #6\quad \vspace{3mm}\\\nopagebreak}%
}

\pagestyle{empty}

\begin{document}
\begin{talk}
{Моделирование воздействия вихревых структур на сверхзвуковое обтекания крыла}
{Борисов Виталий Евгеньевич}
{ИПМ им. М.\,В. Келдыша РАН}
{borisov@keldysh.ru}
{Константиновская Т.\,В., Луцкий А.\,Е.}
{Прикладная математика и математическое моделирование}

В работе представлены результаты численного исследования сверхзвукового обтекания тандема крыльев (пара: крыло--генератор и основное крыло) с углом атаки 20\(^\circ\), актуальность которого связана, в частности, со сложностями проведения натурных экспериментов [1,2]. Расчеты проводились для прямоугольных в плане крыльев с острыми кромками и ромбовидным основанием. Рассматривались две конфигурации тандема, отличающиеся полуразмахом первого по потоку крыла-генератора, которое составляло половину либо равнялось полуразмаху основного крыла. Для численного расчета использовалась система осредненных по Рейнольдсу и Фавру нестационарных уравнений Навье--Стокса (URANS) с моделью турбулентности Спаларта--Аллмараса. Расчеты проводились на гибридной суперкомпьютерной системе К-60 [3] с помощью авторского программного комплекса ARES для расчета трехмерных турбулентных течений вязкого сжимаемого газа. Показано развитие взаимодействия вихревых структур в зависимости от полуразмаха крыла-генератора, а также изменение зоны обратного течения на подветренной стороне основного крыла. Проведено сравнение с результатами обтекания крыла в невозмущенном набегающем потоке.

\medskip

\begin{enumerate}
\item[{[1]}] Гайфуллин А.М. Вихревые течения. --М.: Наука, 2015. 319 с.
\item[{[2]}] Borisov~V.E., Davydov~A.A., Konstantinovskaya~T.V., Lutsky~A.E., Shevchenko~A.M., Shmakov~A.S. Numerical and experimental investigation of a supersonic vortex wake at a wide distance from the wing // AIP Conference Proceedings. 2018. 2027, 030120.
\item[{[3]}] Вычислительный комплекс K-60. [Электронный ресурс]. \\ URL: https://www.kiam.ru/MVS/resourses/k60.html 
\end{enumerate}
\end{talk}
\end{document}
