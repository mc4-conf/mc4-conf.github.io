\documentclass[12pt, a4paper, figuresright]{book}
\usepackage{mathtools, commath, amssymb, amsthm}
\usepackage{tabularx,graphicx,url,xcolor,rotating,multicol,epsfig,colortbl,lipsum}
\usepackage[T2A]{fontenc}
\usepackage[english,main=russian]{babel}

\setlength{\textheight}{25.2cm}
\setlength{\textwidth}{16.5cm}
\setlength{\voffset}{-1.6cm}
\setlength{\hoffset}{-0.3cm}
\setlength{\evensidemargin}{-0.3cm} 
\setlength{\oddsidemargin}{0.3cm}
\setlength{\parindent}{0cm} 
\setlength{\parskip}{0.3cm}

\newenvironment{talk}[6]{%
\vskip 0pt\nopagebreak%
\vskip 0pt\nopagebreak%
\textbf{#1}\vspace{3mm}\\\nopagebreak%
\textit{#2}\\\nopagebreak%
#3\\\nopagebreak%
\url{#4}\vspace{3mm}\\\nopagebreak%
\ifthenelse{\equal{#5}{}}{}{Соавторы: #5\vspace{3mm}\\\nopagebreak}%
\ifthenelse{\equal{#6}{}}{}{Секция: #6\quad \vspace{3mm}\\\nopagebreak}%
}

\pagestyle{empty}

\begin{document}
\begin{talk}
{Cтруктура внутренних волн в расчетах движения неоднородной жидкости с использованием численной модели ROMS}
{Володько Ольга Станиславовна}
{Институт вычислительного моделирования СО РАН, обособленное подразделение ФИЦ КНЦ СО РАН}
{olga.pitalskaya@gmail.com}
{Мальцев Е.\,Д.}
{Прикладная математика и математическое моделирование}

Течения и внутренние волны в озерах в основном вызываются ветровыми воздействиями. Понимание пространственной структуры обеспечивает основу для понимания последующих физических, химических и биологических процессов.
Но, как правило, натурные измерения гидрофизических характеристик (скорости течения, температуры и солености воды), могут быть проведены только в нескольких конкретных точках. При проведении численных расчетов мы имеем значения гидрофизических характеристик в каждой точке разностной сетки и можем на основании этих данных определить горизонтальную структуру внутренних волн. В настоящей работе на основе данных численных расчетов, полученных с использованием численной модели ROMS (Regional Oceanic Modeling System), определены время возникновения внутренних волн в зависимости от направления и силы ветра, характер изменения возвышения свободной поверхности и изоповерхностей температуры. Для интерпретации полученных в расчетах результатов был проведен переход от \(\sigma\)-координат к декартовым, что позволило идентифицировать наиболее длинные волны как одноузловые сейши. С применением линейной модели трехмерного течения двухслойной жидкости проведена оценка длины вращающейся сейши.
\end{talk}
\end{document}
