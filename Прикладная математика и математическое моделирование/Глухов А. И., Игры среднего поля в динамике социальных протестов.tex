\documentclass[12pt, a4paper, figuresright]{book}
\usepackage{mathtools, commath, amssymb, amsthm}
\usepackage{tabularx,graphicx,url,xcolor,rotating,multicol,epsfig,colortbl,lipsum}
\usepackage[T2A]{fontenc}
\usepackage[english,main=russian]{babel}

\setlength{\textheight}{25.2cm}
\setlength{\textwidth}{16.5cm}
\setlength{\voffset}{-1.6cm}
\setlength{\hoffset}{-0.3cm}
\setlength{\evensidemargin}{-0.3cm} 
\setlength{\oddsidemargin}{0.3cm}
\setlength{\parindent}{0cm} 
\setlength{\parskip}{0.3cm}

\newenvironment{talk}[6]{%
\vskip 0pt\nopagebreak%
\vskip 0pt\nopagebreak%
\textbf{#1}\vspace{3mm}\\\nopagebreak%
\textit{#2}\\\nopagebreak%
#3\\\nopagebreak%
\url{#4}\vspace{3mm}\\\nopagebreak%
\ifthenelse{\equal{#5}{}}{}{Соавторы: #5\vspace{3mm}\\\nopagebreak}%
\ifthenelse{\equal{#6}{}}{}{Секция: #6\quad \vspace{3mm}\\\nopagebreak}%
}

\pagestyle{empty}

\begin{document}
\begin{talk}
{Игры среднего поля в динамике социальных протестов}
{Глухов Антон Иосифович}
{Международный математический центр ИМ СО РАН}
{a.glukhov@g.nsu.ru}
{Шишленин Максим Александрович}
{Прикладная математика и математическое моделирование}

Доклад посвящен математическому моделированию социальной динамики общества и решению обратных задач на основе концепции ``игр среднего поля'' [1]. В последние годы во всем мире наблюдается рост социальной напряженности общества, которая проявляется в виде социальных протестов. Понимание динамики уличных протестов и изучение факторов, которые могут повлиять на их возникновение, продолжительность, а также интенсивность, принципиально важно для стабильного и устойчивого развития общества.

Разработана комбинированная математическая модель, для изучения динамики социальных протестов с учетом индивидуальности поведения агентов в разных группах, на основе подхода “игр среднего поля” и модели социальных протестов, основанной на динамическом моделировании. При исследовании прямой задачи реализован разностный метод для решения сопряженной системы дифференциальных уравнений в частных производных Колмогорова-Фоккера-Планка и Гамильтона-Якоби-Беллмана, а поиск оптимального управления сведен к итерационному алгоритму [2]. Алгоритм решения обратной задачи состоит в минимизации целевого функционала методом дифференциальной эволюции. Разработанные алгоритмы апробированы на статистических данных социального движения
во Франции (2018–2019 гг.) [3].

\medskip

\begin{enumerate}
\item[{[1]}] M. Lasry and P.-L. Lions, Mean field games, Jpn. J. Math., vol. 2 (1), 2007, pp. 229-260.
\item[{[2]}] Lachapelle, J. Salomon, G. Turinici. Computation of mean field equilibria in economics// Mathematical Models and Methods in Applied Sciences. – 2010. Vol. 20. – P. 567-588.
\item[{[3]}] URL: \url{https://www.interieur.gouv.fr}.
\end{enumerate}
\end{talk}
\end{document}