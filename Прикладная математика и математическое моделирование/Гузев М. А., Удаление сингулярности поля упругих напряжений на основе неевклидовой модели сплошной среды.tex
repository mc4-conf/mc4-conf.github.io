\documentclass[12pt, a4paper, figuresright]{book}
\usepackage{mathtools, commath, amssymb, amsthm}
\usepackage{tabularx,graphicx,url,xcolor,rotating,multicol,epsfig,colortbl,lipsum}
\usepackage[T2A]{fontenc}
\usepackage[english,main=russian]{babel}

\setlength{\textheight}{25.2cm}
\setlength{\textwidth}{16.5cm}
\setlength{\voffset}{-1.6cm}
\setlength{\hoffset}{-0.3cm}
\setlength{\evensidemargin}{-0.3cm} 
\setlength{\oddsidemargin}{0.3cm}
\setlength{\parindent}{0cm} 
\setlength{\parskip}{0.3cm}

\newenvironment{talk}[6]{%
\vskip 0pt\nopagebreak%
\vskip 0pt\nopagebreak%
\textbf{#1}\vspace{3mm}\\\nopagebreak%
\textit{#2}\\\nopagebreak%
#3\\\nopagebreak%
\url{#4}\vspace{3mm}\\\nopagebreak%
%\ifthenelse{\equal{#5}{}}{}{Соавторы: #5\vspace{3mm}\\\nopagebreak}%
\ifthenelse{\equal{#6}{}}{}{Секция: #6\quad \vspace{3mm}\\\nopagebreak}%
}

\pagestyle{empty}

\begin{document}
\begin{talk}
{Удаление сингулярности поля упругих напряжений на основе неевклидовой модели сплошной среды}
{Гузев Михаил Александрович}
{Институт прикладной математики ДВО РАН, Владивосток}
{guzev@iam.dvo.ru}
{}
{Прикладная математика и математическое моделирование}

Рассматривается сингулярные решения для поля упругих напряжений плоскодеформи-рованного состояния сплошной среды. Построена схема минимального расширения классической модели упругой сплошной среды на пути отказа от условия совместности Сен-Венана для деформаций, что приводит к неевклидовой модели сплошной среды. В рамках этой модели показано, что поле полных напряжений не содержит сингулярностидеформированного состояния сплошной среды. 

\medskip

Исследование выполнено в рамках государственного задания Института прикладной математики ДВО РАН (тема No.~075-00459-24-00).
\end{talk}
\end{document}