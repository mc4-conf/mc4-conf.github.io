\documentclass[12pt, a4paper, figuresright]{book}
\usepackage{mathtools, commath, amssymb, amsthm}
\usepackage{tabularx,graphicx,url,xcolor,rotating,multicol,epsfig,colortbl,lipsum}
\usepackage[T2A]{fontenc}
\usepackage[english,main=russian]{babel}

\setlength{\textheight}{25.2cm}
\setlength{\textwidth}{16.5cm}
\setlength{\voffset}{-1.6cm}
\setlength{\hoffset}{-0.3cm}
\setlength{\evensidemargin}{-0.3cm} 
\setlength{\oddsidemargin}{0.3cm}
\setlength{\parindent}{0cm} 
\setlength{\parskip}{0.3cm}

\newenvironment{talk}[6]{%
\vskip 0pt\nopagebreak%
\vskip 0pt\nopagebreak%
\textbf{#1}\vspace{3mm}\\\nopagebreak%
\textit{#2}\\\nopagebreak%
#3\\\nopagebreak%
\url{#4}\vspace{3mm}\\\nopagebreak%
\ifthenelse{\equal{#5}{}}{}{Соавторы: #5\vspace{3mm}\\\nopagebreak}%
\ifthenelse{\equal{#6}{}}{}{Секция: #6\quad \vspace{3mm}\\\nopagebreak}%
}

\pagestyle{empty}

\begin{document}
\begin{talk}
{Создание программного комплекса квантово-химических расчетов, обладающих повышенной точностью и быстродействием}
{Даньшин Артем Александрович}
{НИЦ ``Курчатовский институт''}
{danshin_aa@nrcki.ru}
{А.\,А. Ковалишин}
{Прикладная математика и математическое моделирование}

Существующие методы квантовой химии обладают большой вычислительной сложностью в большинстве приложений в химии, физике конденсированного состояния, биохимии и фармакологии, что ограничивает их область применимости. Поэтому необходимо развивать новые численные методы и математические модели, которые позволят на порядки ускорить вычисления, не теряя в точности. В докладе рассматриваются методы квантового Монте-Карло, Хартри-Фока, пост-Хартри-Фока и теории функционала плотности как с точки зрения численной реализации [1], так и методологических аспектов [2, 3]. Представленные результаты легли в основу программного комплекса, предназначенного для расчета структуры и свойств многоэлектронных систем.

\medskip

\begin{enumerate}
\item[{[1]}] A. A. Danshin, A. A. Kovalishin, {\it Operator Spectrum Transformation in Hartree–Fock and Kohn–Sham Equations}, Doklady Mathematics, 107 (2023), 17–20.
\item[{[2]}] A. A. Danshin, A. A. Kovalishin, M. I. Gurevich, {\it Approach to determine nodal surfaces of some \(s\)-electron systems}, Physical Review E, 108 (2023), 015305.
\item[{[3]}] A. A. Danshin, A. A. Kovalishin, {\it High-Performance Computing in Solving the Electron Correlation Problem}, Lecture Notes in Computer Science, 13708 (2022), 140-151.
\end{enumerate}
\end{talk}
\end{document}