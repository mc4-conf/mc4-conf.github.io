\documentclass[12pt, a4paper, figuresright]{book}
\usepackage{mathtools, commath, amssymb, amsthm}
\usepackage{tabularx,graphicx,url,xcolor,rotating,multicol,epsfig,colortbl,lipsum}
\usepackage[T2A]{fontenc}
\usepackage[english,main=russian]{babel}

\setlength{\textheight}{25.2cm}
\setlength{\textwidth}{16.5cm}
\setlength{\voffset}{-1.6cm}
\setlength{\hoffset}{-0.3cm}
\setlength{\evensidemargin}{-0.3cm} 
\setlength{\oddsidemargin}{0.3cm}
\setlength{\parindent}{0cm} 
\setlength{\parskip}{0.3cm}

\newenvironment{talk}[6]{%
\vskip 0pt\nopagebreak%
\vskip 0pt\nopagebreak%
\textbf{#1}\vspace{3mm}\\\nopagebreak%
\textit{#2}\\\nopagebreak%
#3\\\nopagebreak%
\url{#4}\vspace{3mm}\\\nopagebreak%
\ifthenelse{\equal{#5}{}}{}{Соавторы: #5\vspace{3mm}\\\nopagebreak}%
\ifthenelse{\equal{#6}{}}{}{Секция: #6\quad \vspace{3mm}\\\nopagebreak}%
}

\pagestyle{empty}

\begin{document}
\begin{talk}
{Моделирование динамики эпидемий в зависимости от социально-экономических процессов с применением искусственного интеллекта}
{Криворотько Ольга Игоревна}
{Институт математики им. С.Л. Соболева СО РАН}
{o.i.krivorotko@math.nsc.ru}
{Н.\,Ю. Зятьков, А.\,В. Неверов, С.\,И. Кабанихин}
{Прикладная математика и математическое моделирование}

В работе формулируются и анализируются математические модели распространения инфекционных заболеваний в регионах Российской Федерации, основанные на системах дифференциальных уравнений и законе действующих масс с учетом возрастных особенностей с привлечением моделей нейронных сетей~[1]. Анализ идентифицируемости моделей основывается на методе чувствительности Соболя~[2].

Моделирование эпидемий COVID-19 и туберкулеза с учетом социально-экономических характеристик регионов РФ проводилось с применением генеративно-состязательных и рекуррентных нейронных сетей~[2] в комбинации с дифференциальными моделями. В работе проведен статистический анализ реальных данных, приведены области применения математических моделей и показаны результаты моделирования и прогнозирования эпидемической ситуации COVID-19 и туберкулеза в регионах РФ.

\medskip

Работа выполнена при поддержке Математического Центра в Академгородке, соглашение с Министерством науки и высшего образования Российской Федерации №~075-15-2022-281.

\begin{enumerate}
\item[{[1]}] O. Krivorotko, S Kabanikhin, {\it Artificial intelligence for COVID-19 spread modeling}, J. Inverse Ill-Posed Probl. (2024).
\item[{[2]}] О.И. Криворотько, С.И. Кабанихин, В.С. Петракова, {\it Идентифицируемость математических моделей эпидемиологии: туберкулез, ВИЧ, COVID-19}, Математическая биология и биоинформатика, 18(1) (2023), 177-214.  
\item[{[3]}] О.И. Криворотько, Н.Ю. Зятьков, С.И. Кабанихин, {\it Моделирований эпидемий: нейросеть на основе данных и SIR-модели}, Журнал вычислительной математики и математической физики, 63(10) (2023), 1733-1746.
\end{enumerate}
\end{talk}
\end{document}