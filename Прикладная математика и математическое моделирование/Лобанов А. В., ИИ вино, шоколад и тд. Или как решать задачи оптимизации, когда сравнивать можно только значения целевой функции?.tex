\documentclass[12pt, a4paper, figuresright]{book}
\usepackage{mathtools, commath, amssymb, amsthm}
\usepackage{tabularx,graphicx,url,xcolor,rotating,multicol,epsfig,colortbl,lipsum}
\usepackage[T2A]{fontenc}
\usepackage[english,main=russian]{babel}

\setlength{\textheight}{25.2cm}
\setlength{\textwidth}{16.5cm}
\setlength{\voffset}{-1.6cm}
\setlength{\hoffset}{-0.3cm}
\setlength{\evensidemargin}{-0.3cm} 
\setlength{\oddsidemargin}{0.3cm}
\setlength{\parindent}{0cm} 
\setlength{\parskip}{0.3cm}

\newenvironment{talk}[6]{%
\vskip 0pt\nopagebreak%
\vskip 0pt\nopagebreak%
\textbf{#1}\vspace{3mm}\\\nopagebreak%
\textit{#2}\\\nopagebreak%
#3\\\nopagebreak%
\url{#4}\vspace{3mm}\\\nopagebreak%
\ifthenelse{\equal{#5}{}}{}{Соавторы: #5\vspace{3mm}\\\nopagebreak}%
\ifthenelse{\equal{#6}{}}{}{Секция: #6\quad \vspace{3mm}\\\nopagebreak}%
}

\pagestyle{empty}

\begin{document}
\begin{talk}
{ИИ вино, шоколад и тд. Или как решать задачи оптимизации, когда сравнивать можно только значения целевой функции?}
{Лобанов Александр Владимирович}
{МФТИ, Сколтех, ИСП РАН}
{lobbsasha@mail.ru}
{Александр Гасников, Андрей Краснов}
{Прикладная математика и математическое моделирование}

Часто растущая область оптимизации «черного ящика» сталкивается с проблемами из-за ограниченного понимания механизмов целевой функции. Чтобы решить такие проблемы, в этой работе мы сосредотачиваемся на детерминистской концепции Order Oracle, которая использует только порядковый доступ между значениями функций (возможно, с некоторым ограниченным шумом), но не предполагает доступа к их значениям. В качестве теоретических результатов мы предлагаем новый подход к созданию неускоренных алгоритмов оптимизации (полученный путем интеграции Order Oracle в существующие ``инструменты'' оптимизации) в невыпуклых, выпуклых и сильно выпуклых настройках, который не уступает как SOTA алгоритмам координатного спуска с оракулом первого порядка, так и SOTA алгоритмам с Order Oracle с точностью до логарифмического коэффициента. Более того, используя предложенный подход, мы предоставляем первый ускоренный алгоритм оптимизации с использованием Order Oracle. Наконец, наши теоретические результаты демонстрируют эффективность предложенных алгоритмов посредством численных экспериментов.
\end{talk}
\end{document}
