\documentclass[12pt, a4paper, figuresright]{book}
\usepackage{mathtools, commath, amssymb, amsthm}
\usepackage{tabularx,graphicx,url,xcolor,rotating,multicol,epsfig,colortbl,lipsum}
\usepackage[T2A]{fontenc}
\usepackage[english,main=russian]{babel}

\setlength{\textheight}{25.2cm}
\setlength{\textwidth}{16.5cm}
\setlength{\voffset}{-1.6cm}
\setlength{\hoffset}{-0.3cm}
\setlength{\evensidemargin}{-0.3cm} 
\setlength{\oddsidemargin}{0.3cm}
\setlength{\parindent}{0cm} 
\setlength{\parskip}{0.3cm}

\newenvironment{talk}[6]{%
\vskip 0pt\nopagebreak%
\vskip 0pt\nopagebreak%
\textbf{#1}\vspace{3mm}\\\nopagebreak%
\textit{#2}\\\nopagebreak%
#3\\\nopagebreak%
\url{#4}\vspace{3mm}\\\nopagebreak%
\ifthenelse{\equal{#5}{}}{}{Соавторы: #5\vspace{3mm}\\\nopagebreak}%
\ifthenelse{\equal{#6}{}}{}{Секция: #6\quad \vspace{3mm}\\\nopagebreak}%
}

\pagestyle{empty}

\begin{document}
\begin{talk}
{Критические переходы в моделях процессов агрегации и дробления вещества}
{Матвеев Сергей Александрович}
{ИВМ РАН и МГУ имени М.\,В. Ломоносова}
{matseralex@gmail.com}
{Роман Дьяченко и Павел Крапивский}
{Прикладная математика и математическое моделирование}

В настоящем докладе будут разобраны несколько примеров критических переходов в математических моделях процессов агрегации и дробления вещества. Эти процессы широко распространены в природе и часто описываются при помощи больших или даже формально бесконечных систем нелинейных дифференциальных уравнений. Эти системы можно исследовать при помощи численных методов различной природы, в том числе методов Монте Карло и классических разностных схем типа Рунге-Кутты. В данном докладе мы покажем, что использование конечных выборок частиц в методах Монте Карло может приводить к принципиальным различиям между ``разыгрываемым'' случайным процессом и решением исходной системы нелинейных дифференциальных уравнений. Такое различие мы будем называть эффектом конечной выборки при моделировании процессов агрегации. Эти эффекты можно наблюдать с ростом объёмов используемой выборки в методах Монте Карло вплоть до миллиардов частиц [1]. 

Вторым видом критических переходов в моделях агрегации вещества является явление золь-гель перехода в случае быстро растущих коэффициентов агрегации. Для задач данного типа с источником мономеров мы покажем, что поведение решения после критической точки существенно зависит от формы записи системы исследуемых дифференциальных уравнений. Эффекты такого типа не могут наблюдаться для задач кинетики агрегации с ``медленно'' растущими коэффициентами, удовлетворяющими закону сохранения массы. Полученные наблюдения напоминают нам знаменитую дискуссию о природе явления золь-гель перехода между химиками П. Флори и У. Стокмайером, а также имеют прямое отношение к явлению перколяции при росте случайных графов в модели Эрдёша-Реньи [2]. Обе работы [1],[2], обсуждаемые в докладе, выполнены при поддержке отделения Московского центра фундаментальной и прикладной математики в ИВМ РАН.

\medskip

\begin{enumerate}
\item[{[1]}] Dyachenko, R. R., Matveev, S. A., Krapivsky, P. L. (2023). Finite-size effects in addition and chipping processes. Physical Review E, 108(4), 044119.
\item[{[2]}] Krapivsky, P. L., Matveev, S. A. (2024). Gelation in input-driven aggregation. arXiv preprint arXiv:2404.01032.
\end{enumerate}
\end{talk}
\end{document}