\documentclass[12pt, a4paper, figuresright]{book}
\usepackage{mathtools, commath, amssymb, amsthm}
\usepackage{tabularx,graphicx,url,xcolor,rotating,multicol,epsfig,colortbl,lipsum}
\usepackage[T2A]{fontenc}
\usepackage[english,main=russian]{babel}

\setlength{\textheight}{25.2cm}
\setlength{\textwidth}{16.5cm}
\setlength{\voffset}{-1.6cm}
\setlength{\hoffset}{-0.3cm}
\setlength{\evensidemargin}{-0.3cm} 
\setlength{\oddsidemargin}{0.3cm}
\setlength{\parindent}{0cm} 
\setlength{\parskip}{0.3cm}

\newenvironment{talk}[6]{%
\vskip 0pt\nopagebreak%
\vskip 0pt\nopagebreak%
\textbf{#1}\vspace{3mm}\\\nopagebreak%
\textit{#2}\\\nopagebreak%
#3\\\nopagebreak%
\url{#4}\vspace{3mm}\\\nopagebreak%
\ifthenelse{\equal{#5}{}}{}{Соавторы: #5\vspace{3mm}\\\nopagebreak}%
\ifthenelse{\equal{#6}{}}{}{Секция: #6\quad \vspace{3mm}\\\nopagebreak}%
}

\pagestyle{empty}

\begin{document}
\begin{talk}
{Дискретные динамические системы, обратные задачи и связанные вопросы численного моделирования} % 
{Михайлов Александр Сергеевич и Михайлов Виктор Сергеевич} % 
{ПОМИ РАН, СПбГУ}% 
{mikhaylov@pdmi.ras.ru} % 
{} %
{Прикладная математика и математическое моделирование} % 

Мы дадим обзор по применению метода Граничного управления к одномерным динамическим дискретным задачам. Этот подход применялся авторами 
к решению классических проблем моментов, построению функции Вейля для матриц Якоби, пространствами де Бранжа, струнам Стильтьема-Крейна. 
В докладе мы сфокусируемся на применеии этого подхода к численному моделированию задач управления и обратных задач для волнового уравнения 
на графах-деревьях и к решению одномерной обратной задачи для параболического уравнения. 
\end{talk}
\end{document}