\documentclass[12pt, a4paper, figuresright]{book}
\usepackage{mathtools, commath, amssymb, amsthm}
\usepackage{tabularx,graphicx,url,xcolor,rotating,multicol,epsfig,colortbl,lipsum}
\usepackage[T2A]{fontenc}
\usepackage[english,main=russian]{babel}

\setlength{\textheight}{25.2cm}
\setlength{\textwidth}{16.5cm}
\setlength{\voffset}{-1.6cm}
\setlength{\hoffset}{-0.3cm}
\setlength{\evensidemargin}{-0.3cm} 
\setlength{\oddsidemargin}{0.3cm}
\setlength{\parindent}{0cm} 
\setlength{\parskip}{0.3cm}

\newenvironment{talk}[6]{%
\vskip 0pt\nopagebreak%
\vskip 0pt\nopagebreak%
\textbf{#1}\vspace{3mm}\\\nopagebreak%
\textit{#2}\\\nopagebreak%
#3\\\nopagebreak%
\url{#4}\vspace{3mm}\\\nopagebreak%
\ifthenelse{\equal{#5}{}}{}{Соавторы: #5\vspace{3mm}\\\nopagebreak}%
\ifthenelse{\equal{#6}{}}{}{Секция: #6\quad \vspace{3mm}\\\nopagebreak}%
}

\pagestyle{empty}

\begin{document}
\begin{talk}
{Применение PINN в SIR модели игры среднего поля}
{Неверов Андрей Вячеславович}
{Институт математики им. Соболева СО РАН}
{a.neverov@g.nsu.ru}
{Криворотько Ольга Игоревна}
{Прикладная математика и математическое моделирование}

Рассматривается пространственная эпидемиологическая SIR модель, в которой люди распределены в некотором населенном пункте и стремятся не стать инфицированными. Для реализации взаимодействия большого населения в условиях эпидемии применен подход игр среднего поля~[1], характеризующийся совместным решением систем уравнений в частных производных типа Колмогорова-Фоккера-Планка и Гамильтона-Якоби-Беллмана.

Для численной реализации математического моделирования распространения эпидемии в популяции с учетом оптимального управления применяется метод машинного обучения, а именно физически информированные нейронные сети (PINN) с различными модификациями~[2]. Рассматривается возможность решения коэффициентных обратных задач, где информация вводится в виде дополнительных уравнений.

\medskip

Работа выполнена в рамках государственного задания Института математики им. С.\,Л. Соболева СО РАН, проект FWNF-2024-0002 ``Обратные некорректные задачи и машинное обучение в биологических, социально-экономических и экологических процессах''.

\begin{enumerate}
\item[{[1]}] V. Petrakova, O. Krivorotko, Mean Field Optimal Control Problem for Predicting the Spread of Viral Infections, {\it 19th International Asian School-Seminar on Optimization Problems of Complex Systems (OPCS)}, (2023), 79-84.
\item[{[2]}] M. Raissi, P. Perdikaris, G.E. Karniadakis,
Physics-informed neural networks: A deep learning framework for solving forward and inverse problems involving nonlinear partial differential equations,
{\it Journal of Computational Physics},
378
(2019), 
686-707.
\end{enumerate}
\end{talk}
\end{document}
