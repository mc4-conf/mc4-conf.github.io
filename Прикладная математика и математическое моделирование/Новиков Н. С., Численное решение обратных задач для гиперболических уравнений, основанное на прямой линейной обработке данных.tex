\documentclass[12pt, a4paper, figuresright]{book}
\usepackage{mathtools, commath, amssymb, amsthm}
\usepackage{tabularx,graphicx,url,xcolor,rotating,multicol,epsfig,colortbl,lipsum}
\usepackage[T2A]{fontenc}
\usepackage[english,main=russian]{babel}

\setlength{\textheight}{25.2cm}
\setlength{\textwidth}{16.5cm}
\setlength{\voffset}{-1.6cm}
\setlength{\hoffset}{-0.3cm}
\setlength{\evensidemargin}{-0.3cm} 
\setlength{\oddsidemargin}{0.3cm}
\setlength{\parindent}{0cm} 
\setlength{\parskip}{0.3cm}

\newenvironment{talk}[6]{%
	\vskip 0pt\nopagebreak%
	\vskip 0pt\nopagebreak%
	\textbf{#1}\vspace{3mm}\\\nopagebreak%
	\textit{#2}\\\nopagebreak%
	#3\\\nopagebreak%
	\url{#4}\vspace{3mm}\\\nopagebreak%
	\ifthenelse{\equal{#5}{}}{}{Соавторы: #5\vspace{3mm}\\\nopagebreak}%
	\ifthenelse{\equal{#6}{}}{}{Секция: #6\quad \vspace{3mm}\\\nopagebreak}%
}

\pagestyle{empty}
\begin{document}
\begin{talk}
{Численное решение обратных задач для гиперболических уравнений, основанное на прямой линейной обработке данных}
{Новиков Никита Сергеевич}
{ИМ СО РАН}
{novikov-1989@yandex.ru}
{}
{Прикладная математика и математическое моделирование}

В докладе мы построим многомерный аналог уравнения Гельфанда--Левитана--Крейна для задачи определения плотности из уравнения акустики по измеренным на поверхности откликам на систему зондирующих сигналов. 
Рассматривается переопределённая система задач, в которой используется отклик от (бесконечного) семейства источников.Тем самым метод использует больше информации, чем это обычно требуется для доказательства теоремы единственности. Преимуществом же данного подхода является сведение нелинейной задачи к системе линейных уравнений, а также возможность вычисления неизвестного коэффициента без многократного решения прямой задачи.
В докладе рассматриваются аналоги метода для других обратных задач для гиперболических уравнений, приводятся результаты численного решения обратной задачи, а также обсуждается вопрос использования подходов, основанных на глубоком обучении для решения обратной задачи в рамках рассматриваемого подхода.
\end{talk}

\end{document}