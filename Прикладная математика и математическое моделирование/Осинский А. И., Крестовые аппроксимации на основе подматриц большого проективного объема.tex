\documentclass[12pt, a4paper, figuresright]{book}
\usepackage{mathtools, commath, amssymb, amsthm}
\usepackage{tabularx,graphicx,url,xcolor,rotating,multicol,epsfig,colortbl,lipsum}
\usepackage[T2A]{fontenc}
\usepackage[english,main=russian]{babel}

\setlength{\textheight}{25.2cm}
\setlength{\textwidth}{16.5cm}
\setlength{\voffset}{-1.6cm}
\setlength{\hoffset}{-0.3cm}
\setlength{\evensidemargin}{-0.3cm} 
\setlength{\oddsidemargin}{0.3cm}
\setlength{\parindent}{0cm} 
\setlength{\parskip}{0.3cm}

\newenvironment{talk}[6]{%
\vskip 0pt\nopagebreak%
\vskip 0pt\nopagebreak%
\textbf{#1}\vspace{3mm}\\\nopagebreak%
\textit{#2}\\\nopagebreak%
#3\\\nopagebreak%
\url{#4}\vspace{3mm}\\\nopagebreak%
\ifthenelse{\equal{#5}{}}{}{Соавторы: #5\vspace{3mm}\\\nopagebreak}%
\ifthenelse{\equal{#6}{}}{}{Секция: #6\quad \vspace{3mm}\\\nopagebreak}%
}

\pagestyle{empty}

\begin{document}
\begin{talk}
{Крестовые аппроксимации на основе подматриц большого проективного объема}
{Осинский Александр Игоревич}
{Сколковский институт науки и технологий; Институт вычислительной математики имени Г.\,И.~Марчука РАН}
{osinskiy1189@gmail.com}
{}
{Прикладная математика и математическое моделирование}

Крестовые аппроксимации часто используются в вычислительной математике в качестве замены сингулярного разложения. Основная причина состоит в существенно более высокой скорости их построения, а также в отсутствии необходимости знать все элементы приближаемой матрицы. В данном докладе будет показано, что часто можно быстро строить крестовые аппроксимации с точностью по норме Фробениуса сколь угодно близкой к сингулярному разложению. Это позволяет использовать крестовую аппроксимацию в качестве приближенного проектора на множество матриц фиксированного ранга. Таким образом можно существенно ускорить методы переменных проекций и проекций градиента на пространство малоранговых матриц. В частности, в алгоритмах восстановления матриц и построения неотрицательных аппроксимаций.
\end{talk}
\end{document}
