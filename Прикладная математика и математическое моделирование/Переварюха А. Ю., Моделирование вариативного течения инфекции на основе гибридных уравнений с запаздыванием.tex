\documentclass[12pt, a4paper, figuresright]{book}
\usepackage{mathtools, commath, amssymb, amsthm}
\usepackage{tabularx,graphicx,url,xcolor,rotating,multicol,epsfig,colortbl,lipsum}
\usepackage[T2A]{fontenc}
\usepackage[english,main=russian]{babel}

\setlength{\textheight}{25.2cm}
\setlength{\textwidth}{16.5cm}
\setlength{\voffset}{-1.6cm}
\setlength{\hoffset}{-0.3cm}
\setlength{\evensidemargin}{-0.3cm} 
\setlength{\oddsidemargin}{0.3cm}
\setlength{\parindent}{0cm} 
\setlength{\parskip}{0.3cm}

\newenvironment{talk}[6]{%
\vskip 0pt\nopagebreak%
\vskip 0pt\nopagebreak%
\textbf{#1}\vspace{3mm}\\\nopagebreak%
\textit{#2}\\\nopagebreak%
#3\\\nopagebreak%
\url{#4}\vspace{3mm}\\\nopagebreak%
\ifthenelse{\equal{#5}{}}{}{Соавторы: #5\vspace{3mm}\\\nopagebreak}%
\ifthenelse{\equal{#6}{}}{}{Секция: #6\quad \vspace{3mm}\\\nopagebreak}%
}

\pagestyle{empty}

\begin{document}
\begin{talk}
{Моделирование вариативного течения инфекции на основе гибридных уравнений с запаздыванием} %
{Переварюха Андрей Юрьевич} % [2] имя докладчика
  {Санкт-Петербургский Федеральный исследовательский центр РАН}% [3] аффилиация
  {madelf@rambler.ru} % [4] адрес электронной почты (НЕОБЯЗАТЕЛЬНО)
   {}  
  {Прикладная математика и математическое моделирование} % [6] название секции

Обсудим гибридные модели с вероятностной компонентой для сценариев развития ситуации инвазионного процесса в биосистеме с адаптивным сопротивлением. Представим несколько аспектов запаздывания. Частный случай инвазии с неопределенно запаздывающим ответом это иммунный ответ на коронавирус, который может быть или сильным или медленно возникающим по целому ряду не полностью детерменированных факторов.
Нами предложено включение в модель возмущенного равномерно распределенной на $[0,1]$ $\sigma$ репродуктивного запаздывания $x(t-\tau\times\sigma)$ c целью получить варианты поведения траектории, которые соответствуют динамике концентрации вирионов при различных сценариях развития инфекции в организме. Варианты развития отличаются от быстрого выздоровления, до летального варианта. Наиболее сложный для моделирования сценарий хронизации после острой фазы. В предложенной нами модели получен вариант хронизации без необходимости дальнейшего увеличения $r$, $H=1/3K$:
$$
\frac{dN}{dt}=rN\left(1-\frac{N(t-\tau\times\sigma)}{\mathcal{K}}\right)\left(H-N(t-\gamma)\right), \gamma<\tau. \eqno(1)
$$
Используем в новой форме модели вместо квадратичной зависимости логарифмическую форму регуляции: 
$$
\frac{dN}{dt}=r\ln\left(\frac{\mathcal{K}}{N(t-\tau)}\right) - \mathcal{Q}N(t-\nu), \eqno(2)
$$
B таком варианте уравнения с внешним воздействием биотической среды дополнение модели фактором противодействия с отдельным запаздыванием изменит качественный характер решения для описания острой фазы инфекции.
\end{talk}
\end{document}
