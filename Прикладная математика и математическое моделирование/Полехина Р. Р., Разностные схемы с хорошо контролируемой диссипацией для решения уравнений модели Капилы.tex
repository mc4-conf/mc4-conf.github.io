\documentclass[12pt, a4paper, figuresright]{book}
\usepackage{mathtools, commath, amssymb, amsthm}
\usepackage{tabularx,graphicx,url,xcolor,rotating,multicol,epsfig,colortbl,lipsum}
\usepackage[T2A]{fontenc}
\usepackage[english,main=russian]{babel}

\setlength{\textheight}{25.2cm}
\setlength{\textwidth}{16.5cm}
\setlength{\voffset}{-1.6cm}
\setlength{\hoffset}{-0.3cm}
\setlength{\evensidemargin}{-0.3cm} 
\setlength{\oddsidemargin}{0.3cm}
\setlength{\parindent}{0cm} 
\setlength{\parskip}{0.3cm}

\newenvironment{talk}[6]{%
\vskip 0pt\nopagebreak%
\vskip 0pt\nopagebreak%
\textbf{#1}\vspace{3mm}\\\nopagebreak%
\textit{#2}\\\nopagebreak%
#3\\\nopagebreak%
\url{#4}\vspace{3mm}\\\nopagebreak%
\ifthenelse{\equal{#5}{}}{}{Соавторы: #5\vspace{3mm}\\\nopagebreak}%
\ifthenelse{\equal{#6}{}}{}{Секция: #6\quad \vspace{3mm}\\\nopagebreak}%
}

\pagestyle{empty}

\begin{document}
\begin{talk}
{Разностные схемы с хорошо контролируемой диссипацией для решения уравнений модели Капилы}
{Полехина Рузана Рамилевна}
{ИПМ им. М.\,В. Келдыша РАН}
{tukhvatullinarr@gmail.com}
{Савенков Е.\,Б.}
{Прикладная математика и математическое моделирование}

Работа посвящена применению разностной схемы с хорошо контролируемой
диссипацией для решения уравнений модели Капилы, описывающей
двухфазные течения. Последняя является неконсервативной системой
гиперболических уравнений первого порядка и, таким образом, требует
указания конкретного вида регуляризующего диссипативного оператора,
выделяющего единственное решение задачи. Суть схем с хорошо
контролируемой диссипацией заключается в том, что диссипативный
оператор, который определяется видом их первого дифференциального
приближения, совпадает с точностью до малых высшего порядка с
заданным, использованным при определении обобщенного решения в
континуальной постановке. В~результате ожидается сходимость
численного решения схемы к заданному решению. Численные
эксперименты, представленные в работе, демонстрируют эффективность
такого подхода. В качестве точных решений
использованы численные решения типа бегущей волны, полученные другим
методом.
\end{talk}
\end{document}