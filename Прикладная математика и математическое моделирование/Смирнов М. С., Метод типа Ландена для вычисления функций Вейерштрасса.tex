\documentclass[12pt, a4paper, figuresright]{book}
\usepackage{mathtools, commath, amssymb, amsthm}
\usepackage{tabularx,graphicx,url,xcolor,rotating,multicol,epsfig,colortbl,lipsum}
\usepackage[T2A]{fontenc}
\usepackage[english,main=russian]{babel}

\setlength{\textheight}{25.2cm}
\setlength{\textwidth}{16.5cm}
\setlength{\voffset}{-1.6cm}
\setlength{\hoffset}{-0.3cm}
\setlength{\evensidemargin}{-0.3cm} 
\setlength{\oddsidemargin}{0.3cm}
\setlength{\parindent}{0cm} 
\setlength{\parskip}{0.3cm}

\newenvironment{talk}[6]{%
\vskip 0pt\nopagebreak%
\vskip 0pt\nopagebreak%
\textbf{#1}\vspace{3mm}\\\nopagebreak%
\textit{#2}\\\nopagebreak%
#3\\\nopagebreak%
\url{#4}\vspace{3mm}\\\nopagebreak%
\ifthenelse{\equal{#5}{}}{}{Соавторы: #5\vspace{3mm}\\\nopagebreak}%
\ifthenelse{\equal{#6}{}}{}{Секция: #6\quad \vspace{3mm}\\\nopagebreak}%
}

\pagestyle{empty}

\begin{document}
\begin{talk}
{Метод типа Ландена для вычисления функций Вейерштрасса}
{Смирнов Матвей Станиславович}
{Институт вычислительной математики им. Г.\,И. Марчука РАН}
{matsmir98@gmail.com}
{}
{Прикладная математика и математическое моделирование}

Метод Ландена широко известен в контексте эллиптических интегралов и функций Якоби. Его суть состоит в замене модулярных параметров, соответствующих удвоению одного из периодов. Повторение этой процедуры приводит к эффективному вычислительному методу, в особенности, потому что эллиптический модуль при этом сходится квадратично быстро. В отличие от функций Якоби, функции Вейерштрасса традиционно мало применяются в вычислительной практике. Данный доклад будет посвящен новому методу вычисления функций Вейерштрасса, основанному на идеях, аналогичных методу Ландена. Полученный метод также устойчив и имеет квадратичную сходимость. Будет показано как на этом пути можно получить эффективный метод вычисления периодов и отображения Абеля для данных инвариантов Вейерштрасса эллиптической кривой.
\end{talk}
\end{document}
