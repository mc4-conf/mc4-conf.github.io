\documentclass[12pt, a4paper, figuresright]{book}
\usepackage{mathtools, commath, amssymb, amsthm}
\usepackage{tabularx,graphicx,url,xcolor,rotating,multicol,epsfig,colortbl,lipsum}
\usepackage[T2A]{fontenc}
\usepackage[english,main=russian]{babel}

\setlength{\textheight}{25.2cm}
\setlength{\textwidth}{16.5cm}
\setlength{\voffset}{-1.6cm}
\setlength{\hoffset}{-0.3cm}
\setlength{\evensidemargin}{-0.3cm} 
\setlength{\oddsidemargin}{0.3cm}
\setlength{\parindent}{0cm} 
\setlength{\parskip}{0.3cm}

\newenvironment{talk}[6]{%
\vskip 0pt\nopagebreak%
\vskip 0pt\nopagebreak%
\textbf{#1}\vspace{3mm}\\\nopagebreak%
\textit{#2}\\\nopagebreak%
#3\\\nopagebreak%
\url{#4}\vspace{3mm}\\\nopagebreak%
\ifthenelse{\equal{#5}{}}{}{Соавторы: #5\vspace{3mm}\\\nopagebreak}%
\ifthenelse{\equal{#6}{}}{}{Секция: #6\quad \vspace{3mm}\\\nopagebreak}%
}

\pagestyle{empty}

\begin{document}
\begin{talk}
{Применение упрощенной модели жидкого кристалла в акустическом приближении для анализа эффекта ориентационной термоупругости}
{Смолехо Ирина Владимировна}
{Институт вычислительного моделирования СО РАН}
{ismol@icm.krasn.ru}
{}
{Прикладная математика и математическое моделирование}

В работе проведен анализ эффекта ориентационной термоупругости в слое нематического жидкого кристалла, возникающий при нагревании части границы слоя. При этом использовалась упрощенная двумерная динамическая модель жидкого кристалла в акустическом приближении [1]. В основе решения уравнений модели лежит метод двуциклического расщепления по пространственным переменным с применением конечно-разностной схемы распада разрыва Годунова при решении уравнений акустики и схемы Иванова с контролируемой диссипацией энергии, при решении уравнения теплопроводности. Проведена серия расчетов, отображающая невозможность  измения ориентации молекул жидкого кристалла только за счет теплового воздействия на границе. Выдвинута гипотеза, что эффект ориентационной термоупругости будет наблюдаться при учете сил поверхностного натяжения.

\medskip

Работа поддержана Красноярским математическим центром, финансируемым Минобрнауки РФ в рамках мероприятий по созданию и развитию региональных НОМЦ (Соглашение 075-02-2024-1378).

\begin{enumerate}
\item[{[1]}] {\it Sadovskii V., Sadovskaya O.} Acoustic approximation of the gover\-ning
equations of liquid crystals under weak thermomechanical and
electrostatic perturbations // Advances in Mechanics of Micro\-structured
Media and Structures. Ser.: Advanced Structured Mater\-ials,
V. 87. Cham: Springer, 2018. Chapt. 17. P. 297–341.
\end{enumerate}
\end{talk}
\end{document}
