\documentclass[12pt, a4paper, figuresright]{book}
\usepackage{mathtools, commath, amssymb, amsthm}
\usepackage{tabularx,graphicx,url,xcolor,rotating,multicol,epsfig,colortbl,lipsum}
\usepackage[T2A]{fontenc}
\usepackage[english,main=russian]{babel}

\setlength{\textheight}{25.2cm}
\setlength{\textwidth}{16.5cm}
\setlength{\voffset}{-1.6cm}
\setlength{\hoffset}{-0.3cm}
\setlength{\evensidemargin}{-0.3cm} 
\setlength{\oddsidemargin}{0.3cm}
\setlength{\parindent}{0cm} 
\setlength{\parskip}{0.3cm}

\newenvironment{talk}[6]{%
\vskip 0pt\nopagebreak%
\vskip 0pt\nopagebreak%
\textbf{#1}\vspace{3mm}\\\nopagebreak%
\textit{#2}\\\nopagebreak%
#3\\\nopagebreak%
\url{#4}\vspace{3mm}\\\nopagebreak%
\ifthenelse{\equal{#5}{}}{}{Соавторы: #5\vspace{3mm}\\\nopagebreak}%
\ifthenelse{\equal{#6}{}}{}{Секция: #6\quad \vspace{3mm}\\\nopagebreak}%
}

\pagestyle{empty}

\begin{document}
\begin{talk}
{Молекулярно-динамическое моделирование взаимодействия газовой смеси с графеновыми мембранами}
{Челнокова Анна Сергеевна}
{Региональный научно-образовательный математический центр Томского государственного университета}
{smolina-nyuta@mail.ru}
{Бубенчиков Алексей Михайлович}
{Прикладная математика и математическое моделирование}

Наноструктуры на основе углерода, такие как графен, углеродные нанотрубки и фуллерены, привлекли широкое внимание исследователей по всему миру благодаря своим уникальным свойствам. В наномасштабе одним из устоявшихся подходов к изучению подобных структур является молекулярно-динамическое моделирование. Оно особенно полезно для количественной оценки основных взаимодействий и динамических процессов, определяющих коэффициенты адсорбции или диффузии.

В настоящее время в качестве перспективного фильтрующего материала рассматриваются графеноподобные мембраны, и существует необходимость в разработке теоретических подходов для изучения диффузии и сорбции, которые включают межчастичные взаимодействия для предоставления точной информации о массопереносе.

В докладе будет представлена математическая модель взаимодействия компонент газовой смеси He, Ar и Xe с графеновыми листами, в том числе имеющими дефекты. Силы взаимодействия описаны с использованием потенциалов Леннарда-Джонса и Бреннера второго рода. Приведено сравнение коэффициентов проницаемости различных газовых компонент через графеновые листы с дефектами с применением вышеуказанных потенциалов. Представлены оценки температуры газовой смеси и графеновой мембраны.
\end{talk}
\end{document}
