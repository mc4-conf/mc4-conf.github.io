\documentclass[12pt, a4paper, figuresright]{book}
\usepackage{mathtools, commath, amssymb, amsthm}
\usepackage{tabularx,graphicx,url,xcolor,rotating,multicol,epsfig,colortbl,lipsum}
\usepackage[T2A]{fontenc}
\usepackage[english,main=russian]{babel}

\setlength{\textheight}{25.2cm}
\setlength{\textwidth}{16.5cm}
\setlength{\voffset}{-1.6cm}
\setlength{\hoffset}{-0.3cm}
\setlength{\evensidemargin}{-0.3cm} 
\setlength{\oddsidemargin}{0.3cm}
\setlength{\parindent}{0cm} 
\setlength{\parskip}{0.3cm}

\newenvironment{talk}[6]{%
\vskip 0pt\nopagebreak%
\vskip 0pt\nopagebreak%
\textbf{#1}\vspace{3mm}\\\nopagebreak%
\textit{#2}\\\nopagebreak%
#3\\\nopagebreak%
\url{#4}\vspace{3mm}\\\nopagebreak%
\ifthenelse{\equal{#5}{}}{}{Соавторы: #5\vspace{3mm}\\\nopagebreak}%
\ifthenelse{\equal{#6}{}}{}{Секция: #6\quad \vspace{3mm}\\\nopagebreak}%
}

\pagestyle{empty}

\begin{document}
	
\begin{talk}
{Производная дифференциальная геометрия} 
{Айвазьян Аршак Владимирович}
{НИУ ВШЭ}
{aivazian.arshak@gmail.com}
{}
{Топология}

Кольцо гладких функций гладкого многообразия имеет естественную дополнительную структуру: к набору элементов $a_1, .., a_n$ можно применить не только целочисленный многочлен $p \in \mathbb{Z}[x_1, .., x_n]$ (что в точности составляет структуру кольца), а любую гладкую функцию $f \in C^\infty(\mathbb{R}^n)$. Категория $C^\infty$-колец (то есть, множеств, снабженных согласованным действием гладких функций $\mathbb{R}^n \to \mathbb{R}$ для всех $n$) — это обычная алгебраическая категория со всеми их общими местами и хорошими свойствами. Дульная ей категория \textit{гладких локусов} $\mathrm{Locus}$ --- это естественный сеттинг для дифференциальной геометрии, аналогичный схемной революции в алгебраической геометрии. В качестве некоторых ярких преимуществ, на ряду с гладкими многообразиями, она единообразно включает, например, гладкие многообразия с углами или стратификациями, инфинитезимальные пространства и др. Со всеми релевантными геометрическими понятиями и структурами для всех её объектов (например, расслоения струй являются просто пространствами отображений из соответствующих инфинитезимальных пространств). Также она имеет все пределы и копределы, что, например, позволяет прямо говорить о пространствах струй бесконечного порядка (подпространства которых суть дифференциальные уравнения на исходном многообразии, как это развито в подходе Виноградова, Красильщика, Вербовецкого и других). За подробностями отсылаем к классическому тексту I. Moerdijk, G. E. Reyes, ``Models for smooth infinitesimal analysis'', (1991) и другой литературе. 

При этом, ровно как и в алгебраической геометрии, такая настройка все ещё несет такие изъяны, как геометрически неправильное поведение нетрансверсальных пересечений. И так же естественный контекст исправляющий это --- производная дифференциальная геометрия --- реализуется естественным переходом от $C^\infty$ колец к симплициальным $C^\infty$-кольцам, локализованным в слабых эквивалентностях. Альтернативный эквивалентный язык (в духе соответствия Дольда-Кана): $C^\infty$ дифференциально-градуированные алгебры, локализованные в слабых эквивалентностях. Классические локусы вкладываются в производные как полная подкатегория и включение сохраняет трансверсальные пересечения, но не все пределы. В соответствии с D. I. Spivak, ``Derived Smooth Manifolds'', (2010), производные пересечения в точности ведут себя как форма пересечениях на гомологических классах (что дуально по Пуанкаре умножению в когомологиях). А, например, производный локус нулей (вместо усеченного классического) возникает в BV-BRST формализме. О роли производной геометрии в квантовой теории поля см., например, L. Alfonsi, ``Higher geometry in physics'', (2023), L. Alfonsi, C. A. S. Young, ``Towards non-perturbative BV-theory via derived differential geometry'', (2023), G. Giotopoulos, H. Sati, ``Field Theory via Higher Geometry I: Smooth Sets of Fields'', (2023), J. P. Pridham, ``An outline of shifted Poisson structures and deformation quantisation in derived differential geometry'', (2020). Производная геометрия помимо этого имеет также серьезные приложения в классической гладкой геометрии как обсуждается, например, в D. Joyce, ``Algebraic Geometry over $C^\infty$-rings'', (2016).

В этом докладе будет дан краткий обзор области и некоторые новые результаты. 
\end{talk}
\end{document}
