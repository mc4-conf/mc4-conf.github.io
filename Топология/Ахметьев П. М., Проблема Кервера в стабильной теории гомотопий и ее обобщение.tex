\documentclass[12pt, a4paper, figuresright]{book}
\usepackage{mathtools, commath, amssymb, amsthm}
\usepackage{tabularx,graphicx,url,xcolor,rotating,multicol,epsfig,colortbl,lipsum}
\usepackage[T2A]{fontenc}
\usepackage[english,main=russian]{babel}

\setlength{\textheight}{25.2cm}
\setlength{\textwidth}{16.5cm}
\setlength{\voffset}{-1.6cm}
\setlength{\hoffset}{-0.3cm}
\setlength{\evensidemargin}{-0.3cm} 
\setlength{\oddsidemargin}{0.3cm}
\setlength{\parindent}{0cm} 
\setlength{\parskip}{0.3cm}

\newenvironment{talk}[6]{%
	\vskip 0pt\nopagebreak%
	\vskip 0pt\nopagebreak%
	\textbf{#1}\vspace{3mm}\\\nopagebreak%
	\textit{#2}\\\nopagebreak%
	#3\\\nopagebreak%
	\url{#4}\vspace{3mm}\\\nopagebreak%
	\ifthenelse{\equal{#5}{}}{}{Соавторы: #5\vspace{3mm}\\\nopagebreak}%
	\ifthenelse{\equal{#6}{}}{}{Секция: #6\quad \vspace{3mm}\\\nopagebreak}%
}

\pagestyle{empty}

\begin{document}
\begin{talk}
{Проблема Кервера в стабильной теории гомотопий и ее обобщение} %
{Ахметьев Петр Михайлович} %
{ИЗМИРАН}%
{pmakhmet@mail.ru} %
{} %
{Топология} %

Проблема Кервера в стабильной теории гомотопий состоит в перечислении списка размерностей, в которых существует погружение коразмерности \(1\) замкнутого ориентированного многообразия с Арф-инвариантом \(1\).
Первый интересный пример имеется в размерности \(30\). Цель доклада --- геометрическая конструкция указанного многообразия как в работе \([1]\). 

Далее мы обобщим конструкцию и построим бесконечную серию погружений замкнутых многообразий размерностей 
\(2^l-2\) в коразмерности  \(2^{l-1}-1\), \(l\ge 5\), которые оснащены с коразмерности \(2^{l-1}\) (\(1\)-стабильно-оснащенные погружения) со скрученным Арф-инвариантом \(1\).

Кроме того, мы докажем, что список размерностей, в которых существует оснащенные многообразия с Арф-инвариантом \(1\) конечен, а также представим доказательство того, что в размерности \(126\) не существует оснащенного многообразия, представляющего элемент стабильной гомотопической группы сфер \(\pi_{126+128}(S^{128})\), если такой элемент денадстраивается в группу \(\pi_{126+128-9}(S^{128-9})\). 

\medskip

\begin{enumerate}
\item[{[1]}]
J.D.S. Jones, {\it The Kervaire invariant of extended power manifolds} Topology
17 (1978) 249-266.	
\end{enumerate}
\end{talk}
\end{document}