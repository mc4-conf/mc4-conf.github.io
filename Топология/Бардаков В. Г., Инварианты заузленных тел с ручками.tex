\documentclass[12pt, a4paper, figuresright]{book}
\usepackage{mathtools, commath, amssymb, amsthm}
\usepackage{tabularx,graphicx,url,xcolor,rotating,multicol,epsfig,colortbl,lipsum}
\usepackage[T2A]{fontenc}
\usepackage[english,main=russian]{babel}

\setlength{\textheight}{25.2cm}
\setlength{\textwidth}{16.5cm}
\setlength{\voffset}{-1.6cm}
\setlength{\hoffset}{-0.3cm}
\setlength{\evensidemargin}{-0.3cm} 
\setlength{\oddsidemargin}{0.3cm}
\setlength{\parindent}{0cm} 
\setlength{\parskip}{0.3cm}

\newenvironment{talk}[6]{%
\vskip 0pt\nopagebreak%
\vskip 0pt\nopagebreak%
\textbf{#1}\vspace{3mm}\\\nopagebreak%
\textit{#2}\\\nopagebreak%
#3\\\nopagebreak%
\url{#4}\vspace{3mm}\\\nopagebreak%
\ifthenelse{\equal{#5}{}}{}{Соавторы: #5\vspace{3mm}\\\nopagebreak}%
\ifthenelse{\equal{#6}{}}{}{Секция: #6\quad \vspace{3mm}\\\nopagebreak}%
}

\pagestyle{empty}

\begin{document}
	
\begin{talk}
{Инварианты заузленных тел с ручками} %
{Бардаков Валерий Георгиевич} %
{ведущий научный сотрудник Регионального научно-образовательного математического центра Томского государственного университета}%
{bardakova@math.nsc.ru} %
{Федосеев Денис Александрович} %
{Топология} %

Одним из обобщений теории узлов является теория пространственных графов, 
в которой под пространственным графом понимается вложение графа в 
трехмерное пространство. Два пространственных графа называются 
эквивалентными если существует сохраняющий ориентацию гомеоморфизм 
объемлющего пространства, переводящий один граф в другой. Если 
рассмотреть граф, состоящий из одной вершины и одного ребра, то 
полученная теория будет совпадать с теорией узлов. Та же теория 
получится если заменить такой граф его регулярной окрестностью 
(полноторием). Если же взять тело с двумя ручками и изучать его 
вложения, то получим теорию, отличную от теории вложения 
соответствующего графа.

Разницу между вложениями тел с ручками и вложениями соответствующих 
графов хорошо видно на примере топологического человечка [1, рисунок 
306], который может распутать пальцы. Если же взять соответствующий 
пространственный граф, то он не эквивалентен плоскому графу. Что 
произойдет если у человечка есть часы? Рисунок 307 показывает, что трюк, 
используемый ранее не годится. Тем не менее остается такой вопрос: может 
ли человечек с часами распутать пальцы и снять часы?

В докладе мы расскажем о так называемых G-системах квандлов, введенных в 
[2] для построения инвариантов заузленных тел с ручками, введем 
алгебраическую систему, дающую инвариант пространственного графа и дадим 
ответ на вопрос, сформулированный выше.

\medskip

\begin{enumerate}

\item[{[1]}] С. В. Матвеев, А. Т. Фоменко,
{\it Алгоритмические и компьютерные методы в трехмерной топологии}, Изд-во МГУ, 1991.
\item[{[2]}] A.Ishii, {\it Moves and invariants for knotted 
handlebodies}, Algebr. Geom. Topol. 8 (2008), 1403-1418.
\end{enumerate}
\end{talk}
\end{document}
