\documentclass[12pt, a4paper, figuresright]{book}
\usepackage{mathtools, commath, amssymb, amsthm}
\usepackage{tabularx,graphicx,url,xcolor,rotating,multicol,epsfig,colortbl,lipsum}
\usepackage[T2A]{fontenc}
\usepackage[english,main=russian]{babel}

\setlength{\textheight}{25.2cm}
\setlength{\textwidth}{16.5cm}
\setlength{\voffset}{-1.6cm}
\setlength{\hoffset}{-0.3cm}
\setlength{\evensidemargin}{-0.3cm}
\setlength{\oddsidemargin}{0.3cm}
\setlength{\parindent}{0cm}
\setlength{\parskip}{0.3cm}

\newenvironment{talk}[6]{%
\vskip 0pt\nopagebreak%
\vskip 0pt\nopagebreak%
\textbf{#1}\vspace{3mm}\\\nopagebreak%
\textit{#2}\\\nopagebreak%
#3\\\nopagebreak%
\url{#4}\vspace{3mm}\\\nopagebreak%
\ifthenelse{\equal{#5}{}}{}{Соавторы: #5\vspace{3mm}\\\nopagebreak}%
\ifthenelse{\equal{#6}{}}{}{Секция: #6\quad \vspace{3mm}\\\nopagebreak}%
}

\pagestyle{empty}

\begin{document}
\begin{talk}
{О типичности гиперболических узлов} %
{Белоусов Юрий Станиславович} %
{Международный математический институт имени Л. Эйлера}%
{bus99@yandex.ru} %
{А.\,В.~Малютин} %
{Топология} %

Знаменитая теорема Терстона 1978 года о классификации узлов утверждает, что каждый узел либо торический, либо сателлитный, либо гиперболический. До недавнего времени существовала гипотеза (известная как гипотеза Адамса) утверждающая, что доля гиперболических узлов среди всех простых узлов с \(n\) или менее пересечениями стремится к 1 при стремлении \(n\) к бесконечности. В 2017 году А.~Малютин показал, что это утверждение противоречит нескольким другим правдоподобным гипотезам. Наконец, в 2019 году в совместной работе с А.~Малютиным было установлено, что гипотеза Адамса неверна. В докладе мы обсудим ключевые компоненты её опровержения.

\medskip

\begin{enumerate}
\item[{[1]}] Y.~Belousov, A.~Malyutin {\it Hyperbolic knots are not generic}, arXiv:1908.06187, 2019.
\end{enumerate}
\end{talk}
\end{document}