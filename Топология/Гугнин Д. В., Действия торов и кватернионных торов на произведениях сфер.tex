\documentclass[12pt, a4paper, figuresright]{book}
\usepackage{mathtools, commath, amssymb, amsthm}
\usepackage{tabularx,graphicx,url,xcolor,rotating,multicol,epsfig,colortbl,lipsum}
\usepackage[T2A]{fontenc}
\usepackage[english,main=russian]{babel}

\setlength{\textheight}{25.2cm}
\setlength{\textwidth}{16.5cm}
\setlength{\voffset}{-1.6cm}
\setlength{\hoffset}{-0.3cm}
\setlength{\evensidemargin}{-0.3cm} 
\setlength{\oddsidemargin}{0.3cm}
\setlength{\parindent}{0cm} 
\setlength{\parskip}{0.3cm}

\newenvironment{talk}[6]{%
\vskip 0pt\nopagebreak%
\vskip 0pt\nopagebreak%
\textbf{#1}\vspace{3mm}\\\nopagebreak%
\textit{#2}\\\nopagebreak%
#3\\\nopagebreak%
\url{#4}\vspace{3mm}\\\nopagebreak%
\ifthenelse{\equal{#5}{}}{}{Соавторы: #5\vspace{3mm}\\\nopagebreak}%
\ifthenelse{\equal{#6}{}}{}{Секция: #6\quad \vspace{3mm}\\\nopagebreak}%
}

\pagestyle{empty}

\begin{document}
	
\begin{talk}
{Действия торов и кватернионных торов на произведениях сфер} %
{Гугнин Дмитрий Владимирович} %
{Математический институт им. В.\,А. Стеклова РАН \\ 
Механико-математический факультет МГУ имени М.\,В. Ломоносова}%
{dmitry-gugnin@yandex.ru} %
{} %
{Топология} %

Речь пойдет о действиях торов (стандартных компактных торов, а также их кватернионных аналогов) на произведениях сфер [1]. Мы покажем, что пространство орбит некоторого специального действия тора на произведении произвольного конечного набора сфер гомеоморфно сфере (размерность каждой сферы набора в случае тора должна быть не меньше 2, в случае кватернионного тора --- не меньше 4). Для \(k\ge 2\) сфер в наборе действует тор \(T^{k-1}\), в кватернионном случае --- тор \(\mathrm{Sp}(1)^{k-1}\). Аналогичное утверждение для вещественного тора \(\mathbb{Z}_2^{k-1}\) было доказано автором в 2019 году [2]. Основным содержанием данных теорем является явно выписываемая вещественно аналитическая формула канонической проекции на пространство орбит (стандартную круглую сферу). Также будет сформулирована естественная гипотеза о неуменьшаемости ранга тора \(k-1\), доказанная автором для случая вещественного тора в 2023 году [3]. 

\medskip 

\begin{enumerate}
\item[{[1]}] А. А. Айзенберг, Д. В. Гугнин, {\it О действиях торов и кватернионных торов на произведениях сфер}, Топология, геометрия, комбинаторика и математическая физика, Сборник статей. К 80-летию члена-корреспондента РАН Виктора Матвеевича Бухштабера, Труды МИАН, 326, МИАН, М., 2024 (в печати).
\item[{[2]}] Д. В. Гугнин, {\it Разветвленные накрытия многообразий и \(nH\)-пространства}, Функц. анализ и его прил., 53:2 (2019), 68–71.
\item[{[3]}] Д. В. Гугнин, {\it О несвободных действиях коммутирующих инволюций на многообразиях}, Матем. заметки, 113:6 (2023), 820–826.
\end{enumerate}
\end{talk}
\end{document} 