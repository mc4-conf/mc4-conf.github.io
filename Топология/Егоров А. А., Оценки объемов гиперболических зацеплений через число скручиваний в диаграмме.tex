\documentclass[12pt, a4paper, figuresright]{book}
\usepackage{mathtools, commath, amssymb, amsthm}
\usepackage{tabularx,graphicx,url,xcolor,rotating,multicol,epsfig,colortbl,lipsum}
\usepackage[T2A]{fontenc}
\usepackage[english,main=russian]{babel}

\setlength{\textheight}{25.2cm}
\setlength{\textwidth}{16.5cm}
\setlength{\voffset}{-1.6cm}
\setlength{\hoffset}{-0.3cm}
\setlength{\evensidemargin}{-0.3cm} 
\setlength{\oddsidemargin}{0.3cm}
\setlength{\parindent}{0cm} 
\setlength{\parskip}{0.3cm}

\newenvironment{talk}[6]{%
\vskip 0pt\nopagebreak%
\vskip 0pt\nopagebreak%
\textbf{#1}\vspace{3mm}\\\nopagebreak%
\textit{#2}\\\nopagebreak%
#3\\\nopagebreak%
\url{#4}\vspace{3mm}\\\nopagebreak%
\ifthenelse{\equal{#5}{}}{}{Соавторы: #5\vspace{3mm}\\\nopagebreak}%
\ifthenelse{\equal{#6}{}}{}{Секция: #6\quad \vspace{3mm}\\\nopagebreak}%
}

\pagestyle{empty}

\begin{document}
	
\begin{talk}
{Оценки объемов гиперболических зацеплений через число скручиваний в диаграмме} %
{Егоров Андрей Александрович} %
{ИМ СО РАН, НГУ, НОМЦ ТГУ}%
{a.egorov2@g.nsu.ru} %
{Веснин А.Ю.} %
{Топология} %

В приложении к работе [1] приведена верхняя оценка на объём гиперболического зацепления через число скручиваний в его диаграмме. В этом докладе я расскажу о новой верхней оценке для объёмов гиперболических зацеплений, которая улучшает оценку из [1] в случае, если диаграмма зацепления имеет более восьми скручиваний.

\medskip

\begin{enumerate}
\item[{[1]}] M.~Lackenby , {\it The volume of hyperbolic alternating link complements. With an appendix by I.~Agol and D.~Thurston}, Proceedings of the London Mathematical Society, \textbf{88} (2004),  204--224.
\end{enumerate}
\end{talk}
\end{document}