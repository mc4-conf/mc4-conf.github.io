\documentclass[12pt, a4paper, figuresright]{book}
\usepackage{mathtools, commath, amssymb, amsthm}
\usepackage{tabularx,graphicx,url,xcolor,rotating,multicol,epsfig,colortbl,lipsum}
\usepackage[T2A]{fontenc}
\usepackage[english,main=russian]{babel}

\setlength{\textheight}{25.2cm}
\setlength{\textwidth}{16.5cm}
\setlength{\voffset}{-1.6cm}
\setlength{\hoffset}{-0.3cm}
\setlength{\evensidemargin}{-0.3cm} 
\setlength{\oddsidemargin}{0.3cm}
\setlength{\parindent}{0cm} 
\setlength{\parskip}{0.3cm}

\newenvironment{talk}[6]{%
\vskip 0pt\nopagebreak%
\vskip 0pt\nopagebreak%
\textbf{#1}\vspace{3mm}\\\nopagebreak%
\textit{#2}\\\nopagebreak%
#3\\\nopagebreak%
\url{#4}\vspace{3mm}\\\nopagebreak%
\ifthenelse{\equal{#5}{}}{}{Соавторы: #5\vspace{3mm}\\\nopagebreak}%
\ifthenelse{\equal{#6}{}}{}{Секция: #6\quad \vspace{3mm}\\\nopagebreak}%
}

\pagestyle{empty}

\begin{document}
	
\begin{talk}
{Циклическая упорядочиваемость и группы виртуальных узлов} %
{Иванов Максим Эдуардович} %
{Институт математики им. С.\,Л. Соболева Сибирского отделения Российской академии наук}%
{} %
{} %
{Топология} %

Узлом называется вложение окружности в трёхмерную сферу, рассматриваемое с точностью до объемлющей изотопии. Классическим инвариантом узла является его группа. Группа $G$ называется левоупорядочиваемой, если существует порядок в $G$, инвариантный относительно умножения слева, т.е. для любых $g,h,k$ из $G$ верно, что из $g < h$ следует $kg < kh$. Понятие циклической упорядочиваемости группы является естественным обобщением левоупорядочиваемости. Неформально, оно говорит о том, что элементы группы $G$ можно расположить на окружности таким образом, чтобы относительное положение элементов оставалось неизменным при умножении слева на любой элемент из группы. Известно, что все группы узлов являются левоупорядочиваемыми. Луис Кауфман, в конце 90-х, ввел теорию виртуальных узлов как обобщение классической. В докладе мы обсудим группы виртуальных узлов и свойство циклической упорядочиваемости.
\end{talk}
\end{document}