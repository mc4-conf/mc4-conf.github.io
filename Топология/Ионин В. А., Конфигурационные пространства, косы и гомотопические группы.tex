\documentclass[12pt, a4paper, figuresright]{book}
\usepackage{mathtools, commath, amssymb, amsthm}
\usepackage{tabularx,graphicx,url,xcolor,rotating,multicol,epsfig,colortbl,lipsum}
\usepackage[T2A]{fontenc}
\usepackage[english,main=russian]{babel}

\setlength{\textheight}{25.2cm}
\setlength{\textwidth}{16.5cm}
\setlength{\voffset}{-1.6cm}
\setlength{\hoffset}{-0.3cm}
\setlength{\evensidemargin}{-0.3cm} 
\setlength{\oddsidemargin}{0.3cm}
\setlength{\parindent}{0cm} 
\setlength{\parskip}{0.3cm}

\newenvironment{talk}[6]{%
\vskip 0pt\nopagebreak%
\vskip 0pt\nopagebreak%
\textbf{#1}\vspace{3mm}\\\nopagebreak%
\textit{#2}\\\nopagebreak%
#3\\\nopagebreak%
\url{#4}\vspace{3mm}\\\nopagebreak%
\ifthenelse{\equal{#5}{}}{}{Соавторы: #5\vspace{3mm}\\\nopagebreak}%
\ifthenelse{\equal{#6}{}}{}{Секция: #6\quad \vspace{3mm}\\\nopagebreak}%
}

\pagestyle{empty}

\begin{document}
	
\begin{talk}
{Конфигурационные пространства, косы и гомотопические группы} %
{Ионин Василий Андреевич} %
{Лаборатория Чебышева, СПбГУ}%
{ionin.code@gmail.com} %
{} %
{Топология} %

Доклад посвящен некоторым примечательным связям между косами и гомотопическими группами сфер. Мы приведем обзор ключевых аспектов этой темы, а также некоторые новые результаты. В частности, мы построим симплициальную группу на коммутантах \([P_n, P_n]\) групп крашеных кос. Анализируя коммутант конструкции Милнора, мы сможем показать, что эта симплициальная группа гомотопически эквивалентна трехмерной сфере \(S^3\). В качестве приложения мы покажем, как эта экономная модель для трехмерной сферы приводит к некоторым интересным формулам типа Ву для групп~\(\pi_n(S^3)\).
\end{talk}
\end{document}

