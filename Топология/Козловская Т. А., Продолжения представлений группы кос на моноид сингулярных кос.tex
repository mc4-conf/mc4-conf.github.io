\documentclass[12pt, a4paper, figuresright]{book}
\usepackage{mathtools, commath, amssymb, amsthm}
\usepackage{tabularx,graphicx,url,xcolor,rotating,multicol,epsfig,colortbl,lipsum}
\usepackage[T2A]{fontenc}
\usepackage[english,main=russian]{babel}

\setlength{\textheight}{25.2cm}
\setlength{\textwidth}{16.5cm}
\setlength{\voffset}{-1.6cm}
\setlength{\hoffset}{-0.3cm}
\setlength{\evensidemargin}{-0.3cm} 
\setlength{\oddsidemargin}{0.3cm}
\setlength{\parindent}{0cm} 
\setlength{\parskip}{0.3cm}

\newenvironment{talk}[6]{%
\vskip 0pt\nopagebreak%
\vskip 0pt\nopagebreak%
\textbf{#1}\vspace{3mm}\\\nopagebreak%
\textit{#2}\\\nopagebreak%
#3\\\nopagebreak%
\url{#4}\vspace{3mm}\\\nopagebreak%
\ifthenelse{\equal{#5}{}}{}{Соавторы: #5\vspace{3mm}\\\nopagebreak}%
\ifthenelse{\equal{#6}{}}{}{Секция: #6\quad \vspace{3mm}\\\nopagebreak}%
}

\pagestyle{empty}

\begin{document}
	
\begin{talk}
{Продолжения представлений группы кос на моноид сингулярных кос} %
{Козловская Татьяна Анатольевна} %
{старший научный сотрудник Регионального научно-образовательного математического центра Томского государственного университета}%
{t.kozlovskaya@math.tsu.ru} %
{Бардаков Валерий Георгиевич, Нафаа Чбили.} %
{Топология} %

Сингулярные узлы были определены для изучения инвариантов конечного
порядка (инвариантов Васильева-Гусарова) классических узлов. Для изучения сингулярных узлов были введены моноид сингулярных кос \(SM_n\) и группа сингулярных кос \(SG_n\). Группы сингулярных крашеных кос изучались в работах [1]-[4].

Естественным подходом к построению инвариантов узлов является построение различных представлений группы кос. Используя представление Артина автоморфизмами свободной группы, можно найти группу соответствующего зацепления. Используя приведенное представление Бурау, можно найти полином Александера соответствующего узла.

О.~Дашбах и Б.~Гемейн изучали представления моноида сингулярных кос эндоморфизмами свободной группы. Также они построили линейное представление моноида сингулярных кос \(SM_n\) и доказали, что при \(n=3\) это представление точно.

В работе [5] построены линейные представления и представления эндоморфизмами свободной группы \(F_n\) моноида и группы сингулярных кос,
продолжающие представления группы кос Артина. Доказано, что построенное продолжение представления Бурау группы сингулярных кос приводимо.

Одним из известных точных линейных представлений группы кос \(B_n\) является
представление Лоуренса-Краммера-Бигелоу (ЛКБ). Возникает естественный вопросу о линейности группы \(SB_n\).

В работе [5] построено линейное представление группы сингулярных кос,
которое является продолжением представления ЛКБ, и вычислен дефект этого продолжения по отношению к внешнему произведению двух продолжений представления Бурау.

\medskip

\begin{enumerate}
\item[{[1]}] V.G. Bardakov, T.A. Kozlovskaya,
{\it On 3-strand singular pure braid group}, J. Knot Theory Ramif.,  29(10), (2020), 2042001 (20 pages).
\item[{[2]}] V.G. Bardakov, T.A. Kozlovskaya,
{\it Singular braids, singular links and subgroups of camomile type}. (2023), arXiv:2212.08267.
\item[{[3]}] K. Gongopadhyay,  T. Kozlovskaya, O. Mamonov, {\it On some decompositions of the 3-strand singular braid group}, Topology Appl. 283(1), (2020), Article 107394.
\item[{[4]}] T.A. Kozlovskaya,
{\it Structure of 4-strand singular pure braid group}, Siberian Electronic Mathematical Reports, 19(1), (2022), 18--33, 
\item[{[5]}] V.G. Bardakov, N. Chbili, T.A. Kozlovskaya,
{\it Extensions of braid group representations to the monoid of singular braids}. (2024), arXiv:2403.00516.
\end{enumerate}
\end{talk}
\end{document}

