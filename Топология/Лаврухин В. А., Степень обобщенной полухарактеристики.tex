\documentclass[12pt, a4paper, figuresright]{book}
\usepackage{mathtools, commath, amssymb, amsthm}
\usepackage{tabularx,graphicx,url,xcolor,rotating,multicol,epsfig,colortbl,lipsum}
\usepackage[T2A]{fontenc}
\usepackage[english,main=russian]{babel}

\setlength{\textheight}{25.2cm}
\setlength{\textwidth}{16.5cm}
\setlength{\voffset}{-1.6cm}
\setlength{\hoffset}{-0.3cm}
\setlength{\evensidemargin}{-0.3cm} 
\setlength{\oddsidemargin}{0.3cm}
\setlength{\parindent}{0cm} 
\setlength{\parskip}{0.3cm}

\newenvironment{talk}[6]{%
\vskip 0pt\nopagebreak%
\vskip 0pt\nopagebreak%
\textbf{#1}\vspace{3mm}\\\nopagebreak%
\textit{#2}\\\nopagebreak%
#3\\\nopagebreak%
\url{#4}\vspace{3mm}\\\nopagebreak%
\ifthenelse{\equal{#5}{}}{}{Соавторы: #5\vspace{3mm}\\\nopagebreak}%
\ifthenelse{\equal{#6}{}}{}{Секция: #6\quad \vspace{3mm}\\\nopagebreak}%
}

\pagestyle{empty}

\begin{document}
	
\begin{talk}
{Степень обобщенной полухарактеристики} %
{Лаврухин Виктор Александрович} %
{ФМКН СПбГУ}%
{} %
{} %
{Топология} %

Понятие полухарактеристики замкнутого многообразия было введено Кервером в 1956 году [3] и неоднократно использовалось в работах по дифференциальной топологии и теории кобордизмов [4], [5], [6]. В статьях [1] и [2] С. С. Подкорытовым было доказано, что полухарактеристика вложенного подмногообразия квадратично зависит от характеристической функции множества его ростков как подмножества множества всех ростков подмногообразий. 

С каждым соотношением на числа Штифеля---Уитни \((n+1)\)-мерных многообразий можно связать ``вторичный'' инвариант \(\lambda\) замкнутых \(n\)-мерных многообразий с дополнительной структурой на касательном расслоении, который принимает значения в \(\mathbb{Z}/4\). Полухарактеристика Кервера (принимающая значения в \(\mathbb{Z}/2\subseteq\mathbb{Z}/4\)) соответствует соотношению \(w_{n+1}+v^2_{\frac{n+1}{2}}\). При кобордизмах с разумной дополнительной структурой инвариант \(\lambda\), либо сохраняется, либо, подобно полухарактеристике, меняется на относительную эйлерову характеристику кобордизма. Аналогично полухарактеристике Кервера, инвариант \(\lambda\) обладает квадратичным свойством.

\medskip

\begin{enumerate}
\item[{[1]}] Подкорытов С. С. О числах Штифеля–Уитни и полухарактеристике //Алгебра и анализ. – 2002. – Т. 14. – №. 5. – С. 171-187.
\item[{[2]}] Подкорытов С. С. Квадратичное свойство рациональной полухарактеристики //Записки научных семинаров ПОМИ. – 2000. – Т. 267. – №. 0. – С. 241-259.
\item[{[3]}] Kervaire M. A. Courbure intégrale généralisée et homotopie : дис. – ETH Zurich, 1956. 
\item[{[4]}]  Lusztig G., Milnor J., Peterson F. P. Semi-characteristics and cobordism //Topology. – 1969. – Т. 8. – №. 4. – С. 357-359.
\item[{[5]}] Stong R. E. Semi-characteristics and free group actions //Compositio Mathematica. – 1974. – Т. 29. – №. 3. – С. 223-248.
\item[{[6]}] Tang Z. Bordism theory and the Kervaire semi-characteristic //Science in China Series A: Mathematics. – 2002. – Т. 45. – С. 716-720.
\end{enumerate}
\end{talk}
\end{document}

