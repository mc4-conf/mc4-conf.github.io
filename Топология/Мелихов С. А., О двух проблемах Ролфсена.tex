\documentclass[12pt, a4paper, figuresright]{book}
\usepackage{mathtools, commath, amssymb, amsthm}
\usepackage{tabularx,graphicx,url,xcolor,rotating,multicol,epsfig,colortbl,lipsum}
\usepackage[T2A]{fontenc}
\usepackage[english,main=russian]{babel}

\setlength{\textheight}{25.2cm}
\setlength{\textwidth}{16.5cm}
\setlength{\voffset}{-1.6cm}
\setlength{\hoffset}{-0.3cm}
\setlength{\evensidemargin}{-0.3cm} 
\setlength{\oddsidemargin}{0.3cm}
\setlength{\parindent}{0cm} 
\setlength{\parskip}{0.3cm}

\newenvironment{talk}[6]{%
\vskip 0pt\nopagebreak%
\vskip 0pt\nopagebreak%
\textbf{#1}\vspace{3mm}\\\nopagebreak%
\textit{#2}\\\nopagebreak%
#3\\\nopagebreak%
\url{#4}\vspace{3mm}\\\nopagebreak%
\ifthenelse{\equal{#5}{}}{}{Соавторы: #5\vspace{3mm}\\\nopagebreak}%
\ifthenelse{\equal{#6}{}}{}{Секция: #6\quad \vspace{3mm}\\\nopagebreak}%
}

\pagestyle{empty}

\begin{document}
	
\begin{talk}
{О двух проблемах Ролфсена} %
{Мелихов Сергей Александрович} %
{МИАН}%
{melikhov@mi-ras.ru} %
{} %
{Топология} %

50 лет назад Д. Ролфсен поставил две проблемы [1]: (а) Всякий ли узел в \(S^3\) изотопен (=гомотопен в классе вложений) кусочно-линейному или, эквивалентно,
тривиальному узлу? В частности, изотопен ли кусочно-линейному узлу слинг Бинга? (б) Если два кусочно-линейных зацепления в \(S^3\) изотопны, 
будут ли они кусочно-линейно изотопны?

Ответ на вопрос (б) утвердителен, если инварианты конечного порядка дают полную классификацию кусочно-линейных зацеплений [2].
Cлинг Бинга не изотопен никакому кусочно-линейному узлу: (i) изотопией, продолжающейся до изотопии двухкомпонентного зацепления с
коэффициентом зацепления 1; (ii) в классе узлов, являющихся пересечениями вложенных цепочек полноториев [4].
Причём результат (i) сохраняет силу, если дополнительной компоненте разрешить самопересекаться и даже заменяться на новую, если она представляет
тот же класс сопряжённости в \(G/[G',G'']\), где \(G\) --- фундаментальная группа дополнения к исходной компоненте [4]. 
Доказательства основаны на прояснении геометрического смысла определяемых с помощью полинома Конвея от двух переменных формальных аналогов 
производных инвариантов Кохрана для двухкомпонентных зацеплений с коэффициентом зацепления 1 [3]. 

\medskip

\begin{enumerate}
\item[{[1]}] D. Rolfsen, {\it Some counterexamples in link theory}, Canadian J. Math. 26 (1974)
\item[{[2]}] S. A. Melikhov, {\it Topological isotopy and finite type invariants}, arXiv:2406.09331
\item[{[3]}] S. A. Melikhov, {\it Two-variable Conway polynomial and Cochran's derived invariants}, arXiv:math/0312007v3 (2024).
\item[{[4]}] S. A. Melikhov, {\it Is every knot isotopic to the unknot?}, arXiv:2406.09365
\end{enumerate}
\end{talk}
\end{document}