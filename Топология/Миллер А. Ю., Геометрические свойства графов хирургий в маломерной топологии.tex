\documentclass[12pt, a4paper, figuresright]{book}
\usepackage{mathtools, commath, amssymb, amsthm}
\usepackage{tabularx,graphicx,url,xcolor,rotating,multicol,epsfig,colortbl,lipsum}
\usepackage[T2A]{fontenc}
\usepackage[english,main=russian]{babel}

\setlength{\textheight}{25.2cm}
\setlength{\textwidth}{16.5cm}
\setlength{\voffset}{-1.6cm}
\setlength{\hoffset}{-0.3cm}
\setlength{\evensidemargin}{-0.3cm} 
\setlength{\oddsidemargin}{0.3cm}
\setlength{\parindent}{0cm} 
\setlength{\parskip}{0.3cm}

\newenvironment{talk}[6]{%
\vskip 0pt\nopagebreak%
\vskip 0pt\nopagebreak%
\textbf{#1}\vspace{3mm}\\\nopagebreak%
\textit{#2}\\\nopagebreak%
#3\\\nopagebreak%
\url{#4}\vspace{3mm}\\\nopagebreak%
\ifthenelse{\equal{#5}{}}{}{Соавторы: #5\vspace{3mm}\\\nopagebreak}%
\ifthenelse{\equal{#6}{}}{}{Секция: #6\quad \vspace{3mm}\\\nopagebreak}%
}

\pagestyle{empty}

\begin{document}
	
\begin{talk}
{Геометрические свойства графов хирургий в маломерной топологии} %
{Миллер Алексей Юрьевич} %
{ПОМИ РАН}%
{miller.m2@mail.ru} %
{} %
{Топология} %

Последние двадцать лет вопрос о локальном и глобальном геометрическом поведении графов преобразований различных маломерных объектов получает особенное внимание. В этом докладе мы обсудим ряд продвижений в этом направлении, а именно: доказательство гипотезы Жиса--Гамбаду о поведении гордиевых графов на бесконечности, техники, позволяющие обнаружить исключительный локальный геометрический паттерн (в духе Баадера и Хирасавы--Учиды) в гордиевом графе переключения перекрёстков и расположение в Большом графе хирургий Дена одного примечательного класса трехмерных многообразий.
\end{talk}
\end{document}
