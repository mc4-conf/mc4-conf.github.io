\documentclass[12pt, a4paper, figuresright]{book}
\usepackage{mathtools, commath, amssymb, amsthm}
\usepackage{tabularx,graphicx,url,xcolor,rotating,multicol,epsfig,colortbl,lipsum}
\usepackage[T2A]{fontenc}
\usepackage[english,main=russian]{babel}

\setlength{\textheight}{25.2cm}
\setlength{\textwidth}{16.5cm}
\setlength{\voffset}{-1.6cm}
\setlength{\hoffset}{-0.3cm}
\setlength{\evensidemargin}{-0.3cm} 
\setlength{\oddsidemargin}{0.3cm}
\setlength{\parindent}{0cm} 
\setlength{\parskip}{0.3cm}

\newenvironment{talk}[6]{%
\vskip 0pt\nopagebreak%
\vskip 0pt\nopagebreak%
\textbf{#1}\vspace{3mm}\\\nopagebreak%
\textit{#2}\\\nopagebreak%
#3\\\nopagebreak%
\url{#4}\vspace{3mm}\\\nopagebreak%
\ifthenelse{\equal{#5}{}}{}{Соавторы: #5\vspace{3mm}\\\nopagebreak}%
\ifthenelse{\equal{#6}{}}{}{Секция: #6\quad \vspace{3mm}\\\nopagebreak}%
}

\pagestyle{empty}

\begin{document}
	
\begin{talk}
{Пополнения Боусфилда-Кана подстягиваемых копредставлений и ациклические разложения Дрора} %
{Михович Андрей Михайлович} %
{Московский центр фундаментальной и прикладной математики}%
{amikhovich@gmail.com} %
{}
{Топология} %

Как показали Беррик и Хилман, для любого стягиваемого копредставления его конечное копредставление асферично тогда и только тогда, когда верна гипотеза асферичности Уайтхеда.
При этом, как известно, если гипотеза Уайтхеда не верна, то накрытие, соответствующее радикалу Адамса, является ацикличным 2-комплексом.
В начале 70-х Дрор показал, как можно исследовать ациклические пространства с помощью ациклических разложений и их алгебраических инвариантов.
Для подстягиваемых копредставлений удобно использовать относительные ациклические разложения, которые строятся с использованием целочисленного пополнения Боусфилда-Кана. 
Мы показываем, что целочисленное пополнение Боусфилда-Кана конечного подстягиваемого копредставления асферично и проводим вычисления в его разложении Дрора.

\medskip

\begin{enumerate}
\item[{[1]}] Andrey M. Mikhovich. Bousfield-Kan completions of subcontractible presentations, doi:10.13140/rg.2.2.14640.78088/2
\item[{[2]}] A. J. Berrick and J. A. Hillman. Whitehead’s asphericity question and its relation to other open problems. In Algebraic
topology and related topics, Trends Math., pages 27–49. Birkhauser/Springer, Singapore, 2019.
\item[{[3]}] Emmanuel Dror. Homology spheres. Isr. J. Math., 15:115–129, 1973.
\end{enumerate}
\end{talk}
\end{document}