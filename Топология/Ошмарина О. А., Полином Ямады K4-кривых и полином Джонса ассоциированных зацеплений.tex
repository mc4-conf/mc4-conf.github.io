\documentclass[12pt, a4paper, figuresright]{book}
\usepackage{mathtools, commath, amssymb, amsthm}
\usepackage{tabularx,graphicx,url,xcolor,rotating,multicol,epsfig,colortbl,lipsum}
\usepackage[T2A]{fontenc}
\usepackage[english,main=russian]{babel}

\setlength{\textheight}{25.2cm}
\setlength{\textwidth}{16.5cm}
\setlength{\voffset}{-1.6cm}
\setlength{\hoffset}{-0.3cm}
\setlength{\evensidemargin}{-0.3cm} 
\setlength{\oddsidemargin}{0.3cm}
\setlength{\parindent}{0cm} 
\setlength{\parskip}{0.3cm}

\newenvironment{talk}[6]{%
\vskip 0pt\nopagebreak%
\vskip 0pt\nopagebreak%
\textbf{#1}\vspace{3mm}\\\nopagebreak%
\textit{#2}\\\nopagebreak%
#3\\\nopagebreak%
\url{#4}\vspace{3mm}\\\nopagebreak%
\ifthenelse{\equal{#5}{}}{}{Соавторы: #5\vspace{3mm}\\\nopagebreak}%
\ifthenelse{\equal{#6}{}}{}{Секция: #6\quad \vspace{3mm}\\\nopagebreak}%
}

\pagestyle{empty}

\begin{document}
\begin{talk}
{Полином Ямады K4-кривых и полином Джонса ассоциированных зацеплений} %
{Ошмарина Ольга Андреевна} %
{ТГУ, НГУ}%
{o.oshmarina@g.nsu.ru} %
{Веснин А.\,Ю.} %
{Топология} %

В теории заузленных графов нередко используются методы, пришедшие из теории узлов. Так, для графов строятся полиномиальные инварианты, наиболее известными из которых являются полином Ямады [1] и полином Егера [2]. 

В работе [3] была доказана эквивалентность полиномов Ямады и Егера для планарных графов, а также была изучена связь, возникающая между полиномом Ямады тета-кривой и полиномом Джонса зацепления, однозначно строящегося по заузленному тета-графу. В данном докладе мы представим аналогичные результаты для заузленных $\mathbb{K}_4$-графов [4].

\medskip

\begin{enumerate}
\item[{[1]}] S. Yamada, {\it An invariant of spatial graphs}, Graph Theory, 13 (1989), 537–551.
\item[{[2]}] F. Jaeger, {\it On some graph invariants related to the Kauffman polynomial}, Progress in knot theory and related topics, 56 (1997), 69–82.
\item[{[3]}] Y. Huh, {\it Yamada polynomial and associated link of theta-curves}, Discrete Mathematics, 347 (2024).
\item[{[4]}]  O. Oshmarina, A. Vesnin, {\it Polynomials of complete spatial graphs and Jones polynomial of related links}, 2024, preprint arXiv:2404.12264.
\end{enumerate}
\end{talk}
\end{document}

