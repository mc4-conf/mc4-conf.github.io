\documentclass[12pt, a4paper, figuresright]{book}
\usepackage{mathtools, commath, amssymb, amsthm}
\usepackage{tabularx,graphicx,url,xcolor,rotating,multicol,epsfig,colortbl,lipsum}
\usepackage[T2A]{fontenc}
\usepackage[english,main=russian]{babel}

\setlength{\textheight}{25.2cm}
\setlength{\textwidth}{16.5cm}
\setlength{\voffset}{-1.6cm}
\setlength{\hoffset}{-0.3cm}
\setlength{\evensidemargin}{-0.3cm} 
\setlength{\oddsidemargin}{0.3cm}
\setlength{\parindent}{0cm} 
\setlength{\parskip}{0.3cm}

\newenvironment{talk}[6]{%
\vskip 0pt\nopagebreak%
\vskip 0pt\nopagebreak%
\textbf{#1}\vspace{3mm}\\\nopagebreak%
\textit{#2}\\\nopagebreak%
#3\\\nopagebreak%
\url{#4}\vspace{3mm}\\\nopagebreak%
\ifthenelse{\equal{#5}{}}{}{Соавторы: #5\vspace{3mm}\\\nopagebreak}%
\ifthenelse{\equal{#6}{}}{}{Секция: #6\quad \vspace{3mm}\\\nopagebreak}%
}

\pagestyle{empty}

\begin{document}
\begin{talk}
{Сильное сходство отображений} %
{Подкорытов Семён Сергеевич} %
{ПОМИ РАН}%
{ssp@pdmi.ras.ru} %
{} %
{Топология} %

На множестве гомотопических классов отображений \([X,Y]\)
для каждого \(r=0,1,\dotso\)
есть отношение \(r\)"=сходства,
эквивалентность.
Гипотеза:
если
отображение \(a:X\to Y\) \ \((p-1)\)"=сходно с постоянным,
а отображение \(b:Y\to Z\) \ \((q-1)\)"=сходно с постоянным,
то их композиция \(b\circ a\) \ \((pq-1)\)"=сходна с постоянным отображением.
Утверждение гипотезы верно, если
отображение \(b\) {\it сильно \((q-1)\)"=сходно\/} с постоянным.
Гипотетически,
\(r\)"=сходство и сильное \(r\)"=сходство равносильны.
Мы доказываем, что
произведение Уайтхеда \(q\) сомножителей сильно \((q-1)\)"=сходно с постоянным отображением.
\end{talk}
\end{document}
