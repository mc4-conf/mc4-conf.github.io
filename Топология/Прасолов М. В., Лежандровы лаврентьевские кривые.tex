\documentclass[12pt, a4paper, figuresright]{book}
\usepackage{mathtools, commath, amssymb, amsthm}
\usepackage{tabularx,graphicx,url,xcolor,rotating,multicol,epsfig,colortbl,lipsum}
\usepackage[T2A]{fontenc}
\usepackage[english,main=russian]{babel}

\setlength{\textheight}{25.2cm}
\setlength{\textwidth}{16.5cm}
\setlength{\voffset}{-1.6cm}
\setlength{\hoffset}{-0.3cm}
\setlength{\evensidemargin}{-0.3cm} 
\setlength{\oddsidemargin}{0.3cm}
\setlength{\parindent}{0cm} 
\setlength{\parskip}{0.3cm}

\newenvironment{talk}[6]{%
\vskip 0pt\nopagebreak%
\vskip 0pt\nopagebreak%
\textbf{#1}\vspace{3mm}\\\nopagebreak%
\textit{#2}\\\nopagebreak%
#3\\\nopagebreak%
\url{#4}\vspace{3mm}\\\nopagebreak%
\ifthenelse{\equal{#5}{}}{}{Соавторы: #5\vspace{3mm}\\\nopagebreak}%
\ifthenelse{\equal{#6}{}}{}{Секция: #6\quad \vspace{3mm}\\\nopagebreak}%
}

\pagestyle{empty}

\begin{document}
	
\begin{talk}
{Лежандровы лаврентьевские кривые}
{Прасолов Максим Вячеславович}
{Московский государственный университет им. М.\,В. Ломоносова}
{0x00002A@gmail.com}
{}
{Топология}

Контактной структурой на гладком трёхмерном многообразии называется распределение плоскостей, которое в окрестности любой точки может быть задано ядром такой 1-формы \(\alpha\), что \(\alpha\wedge d\alpha\neq0\) всюду в этой окрестности. Мы рассматриваем задачу сглаживания кусочно-гладких объектов на трёхмерном многообразии в присутствии контактной структуры. Основная трудность здесь --- определить, в каком смысле сглаженный объект эквивалентен исходному. Мы расскажем о том, что удалось сделать для лежандровых подмногообразий --- таких одномерных подмногообразий, которые касаются контактной структуры в каждой точке.

\medskip

Работа выполнена за счёт гранта Российского научного фонда № 22-11-00299.

\begin{enumerate}
\item[{[1]}] M.\,Prasolov, Legendrian Lavrentiev Links, {\it Preprint.} arXiv:2404.13473.
\end{enumerate}
\end{talk}
\end{document}
