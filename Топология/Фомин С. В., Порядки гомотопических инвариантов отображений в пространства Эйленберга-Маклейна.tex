\documentclass[12pt, a4paper, figuresright]{book}
\usepackage{mathtools, commath, amssymb, amsthm}
\usepackage{tabularx,graphicx,url,xcolor,rotating,multicol,epsfig,colortbl,lipsum}
\usepackage[T2A]{fontenc}
\usepackage[english,main=russian]{babel}

\setlength{\textheight}{25.2cm}
\setlength{\textwidth}{16.5cm}
\setlength{\voffset}{-1.6cm}
\setlength{\hoffset}{-0.3cm}
\setlength{\evensidemargin}{-0.3cm} 
\setlength{\oddsidemargin}{0.3cm}
\setlength{\parindent}{0cm} 
\setlength{\parskip}{0.3cm}

\newenvironment{talk}[6]{%
\vskip 0pt\nopagebreak%
\vskip 0pt\nopagebreak%
\textbf{#1}\vspace{3mm}\\\nopagebreak%
\textit{#2}\\\nopagebreak%
#3\\\nopagebreak%
\url{#4}\vspace{3mm}\\\nopagebreak%
\ifthenelse{\equal{#5}{}}{}{Соавторы: #5\vspace{3mm}\\\nopagebreak}%
\ifthenelse{\equal{#6}{}}{}{Секция: #6\quad \vspace{3mm}\\\nopagebreak}%
}

\pagestyle{empty}

\begin{document}
\begin{talk}
{Порядки гомотопических инвариантов отображений в пространства Эйленберга--Маклейна} %
{Фомин Сергей Вадимович} %
{СПбГУ}%
{sf2902@mail.ru} %
{}
{Топология} %

Пусть \(X, Y\) --- топологические пространства, \(A\) --- абелева группа, тогда на множестве функций \([X,Y]\rightarrow A\) (гомотопических инвариантов) можно определить меру сложности, называемую порядком. Инварианты конечного порядка можно понимать как гомотопические аналоги инвариантов Васильева узлов (см. предложение 2 в [1]). Пусть \(A, B\) --- абелевы группы, тогда у функции из \(A\) в \(B\) можно определить её степень. Это непосредственный аналог степени многочлена.

Если \(Y\) --- это \(H\)-пространство, то множество \([X,Y]\) --- это абелева группа. В статье [2] доказано, что, если \(Y=S^1\), порядок гомотопического инварианта равен его степени как отображения между абелевыми группами. В дипломной работе докладчика доказано двойное неравенство на порядок в терминах степени, если \(X\) --- конечный CW-комплекс, \(Y\) --- \(K(G,n)\)-пространство (\(G\) абелева), и исследован вопрос достижения верхнего и нижнего пределов в этом неравенстве.  Доклад будет посвящён результатам этой работы.

\medskip

\begin{enumerate}
\item[{[1]}]  Подкорытов С. С. Об отображениях сферы в односвязное пространство //Записки научных семинаров ПОМИ. – 2005. – Т. 329. – №. 0. – С. 159-194.
\item[{[2]}]  Подкорытов С. С. О гомотопических инвариантах отображений в окружность //Записки научных семинаров ПОМИ. – 2009. – Т. 372. – С. 187-202.
\end{enumerate}
\end{talk}
\end{document}