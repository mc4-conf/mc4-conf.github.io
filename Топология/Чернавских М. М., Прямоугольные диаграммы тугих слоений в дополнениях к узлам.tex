\documentclass[12pt, a4paper, figuresright]{book}
\usepackage{mathtools, commath, amssymb, amsthm}
\usepackage{tabularx,graphicx,url,xcolor,rotating,multicol,epsfig,colortbl,lipsum}
\usepackage[T2A]{fontenc}
\usepackage[english,main=russian]{babel}

\setlength{\textheight}{25.2cm}
\setlength{\textwidth}{16.5cm}
\setlength{\voffset}{-1.6cm}
\setlength{\hoffset}{-0.3cm}
\setlength{\evensidemargin}{-0.3cm} 
\setlength{\oddsidemargin}{0.3cm}
\setlength{\parindent}{0cm} 
\setlength{\parskip}{0.3cm}

\newenvironment{talk}[6]{%
\vskip 0pt\nopagebreak%
\vskip 0pt\nopagebreak%
\textbf{#1}\vspace{3mm}\\\nopagebreak%
\textit{#2}\\\nopagebreak%
#3\\\nopagebreak%
\url{#4}\vspace{3mm}\\\nopagebreak%
\ifthenelse{\equal{#5}{}}{}{Соавторы: #5\vspace{3mm}\\\nopagebreak}%
\ifthenelse{\equal{#6}{}}{}{Секция: #6\quad \vspace{3mm}\\\nopagebreak}%
}

\pagestyle{empty}

\begin{document}
	
\begin{talk}
{Прямоугольные диаграммы тугих слоений в дополнениях к узлам} %
{Чернавских Михаил Михайлович} %
{МИАН им. Стеклова}%
{mike.chernavskikh.at.gmail.com} %
{Иван Алексеевич Дынников} %
{Топология} %

Тугие слоения являются важным инструментом маломерной топологии. А именно, следуя работам Д.~Габая [2] и У.~Тёрстона [3], тугие слоения можно использовать для сертификации рода узла. Развивая формализм прямоугольных диаграмм поверхностей~[1], совместно с И.\,А.~Дынниковым мы предложили универсальный способ представления слоений в дополнениях к узлам в трёхмерной сфере и показали, что любое тугое слоение конечной глубины может быть представлено таким образом.

\medskip

Исследование выполнено за счет гранта Российского научного фонда №	22-11-00299, https://rscf.ru/project/22-11-00299/

\begin{enumerate}
\item[{[1]}] Dynnikov~I., Prasolov~M. Rectangular diagrams of surfaces: representability, Matem. Sb. 208 (2017), no. 6, 55–108;
translation in Sb. Math. 208 (2017), no. 6, 781–841, arXiv:1606.03497.
\item[{[2]}] D.~Gabai, Foliations and the topology of 3-manifolds. III., J. Differential Geom.26 (1987), no.3, 479–536.
\item[{[3]}] W.~P.~Thurston, A norm for the homology of 3-manifolds.
Mem. Amer. Math. Soc.59 (1986), no.339, i–vi and 99--130.
\end{enumerate}
\end{talk}
\end{document}
