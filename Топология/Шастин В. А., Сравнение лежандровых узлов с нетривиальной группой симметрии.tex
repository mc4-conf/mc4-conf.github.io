\documentclass[12pt, a4paper, figuresright]{book}
\usepackage{mathtools, commath, amssymb, amsthm}
\usepackage{tabularx,graphicx,url,xcolor,rotating,multicol,epsfig,colortbl,lipsum}
\usepackage[T2A]{fontenc}
\usepackage[english,main=russian]{babel}

\setlength{\textheight}{25.2cm}
\setlength{\textwidth}{16.5cm}
\setlength{\voffset}{-1.6cm}
\setlength{\hoffset}{-0.3cm}
\setlength{\evensidemargin}{-0.3cm} 
\setlength{\oddsidemargin}{0.3cm}
\setlength{\parindent}{0cm} 
\setlength{\parskip}{0.3cm}

\newenvironment{talk}[6]{%
\vskip 0pt\nopagebreak%
\vskip 0pt\nopagebreak%
\textbf{#1}\vspace{3mm}\\\nopagebreak%
\textit{#2}\\\nopagebreak%
#3\\\nopagebreak%
\url{#4}\vspace{3mm}\\\nopagebreak%
\ifthenelse{\equal{#5}{}}{}{Соавторы: #5\vspace{3mm}\\\nopagebreak}%
\ifthenelse{\equal{#6}{}}{}{Секция: #6\quad \vspace{3mm}\\\nopagebreak}%
}

\pagestyle{empty}

\begin{document}
\begin{talk}
{Сравнение лежандровых узлов с нетривиальной группой симметрии} %
{Шастин Владимир Алексеевич} %
{МГУ им. М.\,В. Ломоносова}%
{vashast@gmail.com}
{М.\,В. Прасолов}
{Топология} %

В работе [1] был построен алгоритм сравнения лежандровых узлов. Если группа симметрий узла тривиальна, соответствующий алгоритм значительно упрощается (см. [2]).  В случае нетривиальной группы симметрии возникают дополнительные трудности: нужно проанализировать подгруппу группы симметрий, порождённую лежандровыми изотопиями. В докладе будут предъявлены порождающие лежандровы изотопии в случае узлов \(7_4\), \(9_{48}\), \(10_{136}\), что позволяет завершить классификацию лежандровых узлов сложности не выше 9. Доклад основан на совместной работе с М.\,В. Прасоловым [3]. 

\medskip

\begin{enumerate}
\item[{[1]}] I.\,Dynnikov, M.\,Prasolov.
An algorithm for comparing Legendrian knots. \emph{Preprint} \\ arXiv:2309.05087
\item[{[2]}] I.\,Dynnikov, V.\,Shastin. Distinguishing Legendrian knots with trivial orientation-\\-preserving symmetry group. \emph{Algebraic \& Geometric Topology} {\bf 23-4} (2023), 1849--1889. \\ arXiv:1810.06460.
\item[{[3]}] M.\,Prasolov, V.\,Shastin Distinguishing Legendrian Knots in Topological Types \(7_4\), \(9_{48}\), \(10_{136}\) with maximal Thurston-Benequin number, \emph{Journal of Knot Theory and Its Ramifications}, {\bf 33-01} (2024). arXiv:2306.15461
\end{enumerate}
\end{talk}
\end{document}
