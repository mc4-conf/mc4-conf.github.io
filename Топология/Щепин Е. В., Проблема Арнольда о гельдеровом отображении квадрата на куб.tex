\documentclass[12pt, a4paper, figuresright]{book}
\usepackage{mathtools, commath, amssymb, amsthm}
\usepackage{tabularx,graphicx,url,xcolor,rotating,multicol,epsfig,colortbl,lipsum}
\usepackage[T2A]{fontenc}
\usepackage[english,main=russian]{babel}

\setlength{\textheight}{25.2cm}
\setlength{\textwidth}{16.5cm}
\setlength{\voffset}{-1.6cm}
\setlength{\hoffset}{-0.3cm}
\setlength{\evensidemargin}{-0.3cm}
\setlength{\oddsidemargin}{0.3cm}
\setlength{\parindent}{0cm}
\setlength{\parskip}{0.3cm}

\newenvironment{talk}[6]{%
\vskip 0pt\nopagebreak%
\vskip 0pt\nopagebreak%
\textbf{#1}\vspace{3mm}\\\nopagebreak%
\textit{#2}\\\nopagebreak%
#3\\\nopagebreak%
\url{#4}\vspace{3mm}\\\nopagebreak%
\ifthenelse{\equal{#5}{}}{}{Соавторы: #5\vspace{3mm}\\\nopagebreak}%
\ifthenelse{\equal{#6}{}}{}{Секция: #6\quad \vspace{3mm}\\\nopagebreak}%
}

\pagestyle{empty}

\begin{document}
\begin{talk}
{Проблема Арнольда о гельдеровом отображении квадрата на куб} %
{Щепин Евгений Витальевич} %
{Математический институт им. В.\,А. Стеклова РАН}%
{scepin@mi-ras.ru} %
{} %
{Топология} %

В докладе представлено построение отображения \(\alpha\colon I^2\to I^3\) двумерного квадрата на трехмерный куб,
которое для некоторой константы \(C\)  при любых точках квадрата \(x,y\in I^2\) удовлетворяет неравенству 
\[ |\alpha(x)-\alpha(y)|^3\le C|x-y|^2\]  
Отображение \(\alpha\) решает задачу Арнольда 1988-5 из книги ``Задачи Арнольда'' Фазис, Москва 2000. 
\end{talk}
\end{document}